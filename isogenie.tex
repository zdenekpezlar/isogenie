\documentclass[12pt]{report}
\usepackage{multicol}
\usepackage[czech]{babel}
\usepackage{multicol}
\usepackage[dvipsnames]{xcolor}
\usepackage{a4wide}
\usepackage{booktabs}
\usepackage{longtable}
\usepackage{algorithm}
\usepackage{algorithmic}
\usepackage{amsthm}
\usepackage{array}
\usepackage{thmtools}
\usepackage{mathtools}
\usepackage{amsmath, amssymb, amsfonts}
\usepackage[all]{xypic}
\usepackage{graphicx}
\usepackage{fancyhdr}
\usepackage{caption}
\usepackage{mathalfa}
\usepackage{verse}
\usepackage{extarrows}
\usepackage{enumerate}
\renewcommand{\labelenumi}{(\roman{enumi})} 
\usepackage{tabularx}
\usepackage{wrapfig}
\usepackage{color}
\usepackage{array}
\usepackage{textcomp}
\usepackage{siunitx}
\usepackage{tikz-cd}
\usetikzlibrary{arrows}
\usepackage{pgfplots}
\usetikzlibrary{babel}
\usepackage{epsfig}
\PassOptionsToPackage{hyphens}{url}\usepackage{hyperref}
\hypersetup{colorlinks,
citecolor=RoyalBlue
}
\usepackage{url}
\usepackage{tabularx}
\usetikzlibrary{intersections}
\usepackage[metapost, mplabels,truebbox,clip]{mfpic}
\allowdisplaybreaks

\makeatletter
\newcommand\figcaption{\def\@captype{figure}\caption}
\makeatother

\renewcommand{\sectionmark}[1]{ \markright{-- \thesection.\ #1}{}}

\fancypagestyle{plain}{%
\fancyhf{}
\lhead[L]{}
\chead{}
\rhead[R]{\nouppercase\leftmark}
\lfoot{}
\cfoot{}
\rfoot{\thepage}
\renewcommand{\headrulewidth}{0.4pt}}

\pagestyle{fancy}
\fancyhf{}
\lhead[L]{}
\chead{}
\rhead[R]{\nouppercase\leftmark}
\lfoot{}
\cfoot{}
\rfoot{\thepage}
\begin{document}
\newcommand{\ZZ}{{\mathbb{Z}}}
\newcommand{\cyc}[1]{{\langle #1 \rangle}}


%\newtheorem{veta}{Věta}[section]
%\newtheorem{definice}{Definice}[section]
%\newtheorem{dusledek}{Důsledek}[section]
%\newtheorem{lemma}{Lemma}[section]


\theoremstyle{de}
\newtheorem{de}{Definition}[section]

\newtheorem{veta}{Věta}[section]
\newtheorem{dusledek}[veta]{Důsledek}
\newtheorem{lemma}[veta]{Lemma}
\newtheorem*{lemma*}{Lemma}


\theoremstyle{definition}
\newtheorem{priklad}[veta]{Příklad}
\newtheorem{definice}[veta]{Definice}
\newtheorem{problem}[veta]{Problém}
\newtheorem{znaceni}[veta]{Značení}
\newtheorem*{umluva}{Úmluva}
\newtheorem*{poznamka}{Poznámka}
\newtheorem{dfn}[veta]{Definition}

\floatname{algorithm}{Algoritmus}

\setlength{\parindent}{2ex}

\def\Tr{\operatorname{Tr}}
\def\N{\operatorname{N}}
\def\SL{\operatorname{SL}}
\def\nsd{\operatorname{nsd}}
\def\End{\operatorname{End}}
\def\id{\operatorname{id}}
\def\char{\operatorname{char}}
\def\ker{\operatorname{ker}}
\def\Aut{\operatorname{Aut}}
\def\Gal{\operatorname{Gal}}
\def\Fix{\operatorname{Fix}}
\def\c{\operatorname{\mathbb{C}}}
\def\R{\operatorname{\mathbb{R}}}
\def\q{\operatorname{\mathbb{Q}}}
\def\e{\operatorname{\mathcal{V}}}
\def\z{\operatorname{\mathbb{Z}}}
\def\n{\operatorname{\mathbb{N}}}
\def\s{\operatorname{\subseteq}}
\def\w{\operatorname{\zeta}}
\def\fii{\operatorname{\varphi}}
\def\o{\operatorname{\mathcal{O}}}
\def\I{\operatorname{\mathcal{I}}}
\def\J{\operatorname{\mathcal{J}}}
\def\P{\operatorname{\mathcal{P}}}
\def\pn{\operatorname{\mathfrak{P}}}
\def\pn{\operatorname{\mathfrak{p}}}
\def\res{\operatorname{res}}

\setlength{\emergencystretch}{3em}


\begin{titlepage}
{
\centering
\LARGE \textbf{STŘEDOŠKOLSKÁ ODBORNÁ ČINNOST}\\
\Large\textbf{Obor č. 1: Matematika a statistika}\\
\vspace{6cm}
\LARGE\textbf{Isogenie v kryptografii}\\
}
\vspace{10cm}
{\noindent\large\bfseries Zdeněk Pezlar\\ 
	\large\bfseries Jihomoravský kraj\\ }
\center\large Brno 2021
	
\end{titlepage}

\begin{titlepage}
{
\centering
\LARGE \textbf{STŘEDOŠKOLSKÁ ODBORNÁ ČINNOST}\\
\Large\textbf{Obor č. 1: Matematika a statistika}\\
\vspace{6cm}
\LARGE\textbf{Isogenie v kryptografii}\\
\vspace{1cm}
\LARGE\textbf{Isogeny Based Cryptography}\\
}
\vspace{6cm}
{\noindent\large\bfseries Autor: Zdeněk Pezlar\\ 
	\large\bfseries Škola: Gymnázium Brno, třída Kapitána Jaroše, p. o.\\
    \large\bfseries Kraj: Jihomoravský \\
	\large\bfseries Konzultant: Mgr. Vojtěch Suchánek\\}

\end{titlepage}

\newpage
\thispagestyle{empty}
\vspace*{14cm}
\subsubsection*{Prohlášení}

Prohlašuji, že jsem svou práci SOČ vypracoval samostatně a použil jsem pouze prameny a literaturu uvedené v seznamu bibliografických záznamů.
Prohlašuji, že tištěná verze a elektronická verze soutěžní práce SOČ jsou shodné. 
Nemám závažný důvod proti zpřístupňování této práce v souladu se zákonem č. 121/2000 Sb., o právu autorském, o~právech souvisejících s právem autorským a o změně některých zákonů (autorský zákon) v~platném znění. \\[1cm]
V Brně dne: \dotfill \ \ \ \ \ \  Podpis: \dotfill

\newpage
\thispagestyle{empty}
\begin{center}
\includegraphics[width=0.35\textwidth]{podpora_soc-horizontalni.png}
\end{center}
\vspace*{1.5cm}
\begin{center}
\includegraphics[width=0.45\textwidth]{logo_JMK_pruhledne.png}
\end{center}
\vspace*{2.2cm}
\begin{center}
\includegraphics[width=0.35\textwidth]{jcmm-logotype-positive1.png}
\end{center}
\vspace*{6.5cm}
\subsection*{Poděkování}
Chtěl bych mnohokrát poděkovat mému vedoucímu, Vojtovi Suchánkovi, který mne kousek po kousku prováděl po obtížné teorii, kterou jsme studovali, a opravoval mé nesčetné chyby. Tato práce byla vypracována za finanční podpory JMK.


\newpage
\thispagestyle{empty}
\subsection*{Abstrakt}
V naší práci podáme lehký úvod do studia isogenií eliptických křivek bez předchozího studia algebraické geometrie. V práci rovněž diskutujeme několik vybraných protokolů a~poskytujeme úvod do studia algebraické teorie čísel. Pomocí jejího studia pak podrobněji studujeme okruhy endomorfismů supersingulárních křivek. Práce je obohacena o implementace některých zmíněných protokolů, přičemž poskytujeme první implementaci velmi slibného protokolu SITH.


\subsection*{Klíčová slova}
isogenie; eliptická křivka; okruh endomorfismů; grupa tříd ideálů; kvantový počítač; Diffie-Hellman; SIDH; CSIDH; SITH


\vspace*{4cm}

\subsection*{Abstract}
We provide a gentle introduction to the study of elliptic curve isogenies without any assumed knowledge in algebraic geometry. We then discuss several chosen protocols and give a~brief introduction to algebraic number theory. After that, we apply the gained knowledge on the study of endomorphism rings of supersingular curves. The thesis is accompanied by a couple of implemented protocols, providing the first ever implementation of the very promising protocol SITH.

\subsection*{Key words}
isogeny; elliptic curve; endomorphism ring; ideal class group; quantum computer; Diffie-Hellman; SIDH; CSIDH; SITH





{
\hypersetup{linkcolor=black}
\tableofcontents
}
\thispagestyle{empty}

\chapter*{Úvod}
\addcontentsline{toc}{chapter}{Úvod}
\markboth{Úvod}{}

Za počátek užití eliptických křivek v kryptografii lze považovat kryptosystémy nezávisle navržené Koblitzem \cite{Koblitz} a Millerem \cite{Miller} v 80. letech minulého století založené na principu Diffie-Hellmanovy \cite{Diffie} výměny. V dnešních dobách jsou autentizační protokoly pracující s eliptickými křivkami hojně užíváné, jak protokol TLS zabezpečující naši každodenní komunikaci na internetu, tak transakce kryptoměny Bitcoin jsou chráněné pevnou rukou protokolu ECDSA \cite{ECDSA}.

Zmíněné protokoly, založené na obtížnosti problému diskrétního logaritmu v konečné grupě či eliptické křivce, stejně jako další prominentní systémy navrženy ke konci minulého století, včetně  známého RSA \cite{RSA}, ovšem všechny padají v polynomiálním čase pod rukou algoritmu navrženého v 90. letech Shorem \cite{Shor}. Ne však na klasickém počítači, ale ve světě kvantovém. Cílem odborníků pracujících v post-kvantové kryptografii je pak hledat kryptografická primitiva, která pod hrozbou kvantovou obstojí.

Přímočará adaptace Diffie-Hellmanovy výměny navržená Koblitzem a Millerem však zdaleka není vše, co eliptické křivky nabízejí. Konkrétně zobrazení mezi nimi, která zachovávají jejich grupovou strukuru, tzv. \textit{isogenie}, skýtají bohatou strukturu a nabízejí potenciální těžko prolomitelné protokoly.

Kořeny studia isogenií sahají hluboko do světa algebraické geometrie a studium tohoto oboru poskytuje mnohem lepší pohled \uv{pod kapotu} teorie s nimi spojené. Naše práce volí jiný směr, nabízíme totiž úvod do studia těchto struktur přístupný pro studenta bakalářského studia matematiky či informatiky na vysoké škole. \uv{Elementární} pohled na věc nás donutí jistá klíčová tvrzení předpokládat a priori za platné, práce ale díky tomuto rozhodnutí tvoří dobrý úvod pro čtenáře nezasvěceného do studia eliptických křivek.

Ke konci první kapitoly budeme mít dostatek znalostí k diskuzi protokolu SIDH a na něm založeném systému SIKE \cite{DeFeo3}, který je jedním z nejslibnějších doposud navržených kandidátů na kvantově bezpečný protokol. Tento protokol mimo samotné diskuze implentujeme pro čtenáře k vyzkoušení a podíváme se na některé jeho následníky. Nejnovětší z nich, SITH, který je rezistentní aktivním útokům na SIDH, implementujeme též. Obě implementace se nachází na \url{https://github.com/zdenekpezlar/isogenie/tree/Implementation-Web}. Poté se vydáme na cestu ke hlubšímu studiu endomorfismů na eliptické křivce, tedy isogenií zobrazujících křivku samu na sebe. K této příležitosti odbočíme do světa algebraické teorie čísel, jejíž poznatky nám budou ve studiu endomorfismů velmi přínosné. Konečně, podíváme se na akci tzv. \textit{grupy tříd ideálů} na množinu isomorfismů našich křivek, která dává vzniku následníku SIDH pod názvem CSIDH, který stručně probereme. 

 



\chapter*{Použitá značení}
\begin{flalign*}
&a \mid b  &&a \text{ dělí } b\\
&\frac{1}{a}  &&\text{multiplikativní inverze } a,\text{ tj. } a^{-1} \\
&\nu_p(n) &&p\text{-adická valuace } n\\
&\genfrac{(}{)}{}{}{a}{p} && \text{Legendreův symbol } a \text{ vzhledem k } p\\
&\mathbb{N},\mathbb{Z},\mathbb{Q},\mathbb{R},\mathbb{C} &&\text{množina přirozených, celých, racionálních, reálných, komplexních čísel} \\
&\mathbb{Z}_d &&\text{okruh zbytků modulo } d \\
&\mathbb{F}_q &&\text{konečné těleso s } q \text{ prvky}\\
&\overline{K} &&\text{algebraický uzávěr tělesa } K\\
&K^{\times} &&\text{multiplikativní podrupa tělesa } K\\
&\mathbb{P}^{n}(K) &&\text{projektivní prostor nad } K \text{ o dimenzi } n+1\\
&E(K) &&\text{množina bodů křivky } E \text{ nad } K\\
&\# E(K) &&\text{počet bodů na křivce } E \text{ nad konečným tělesem } K\\
&\mathcal{O},O &&\text{bod v nekonečnu křivky } E\\
&[n]_E,[n] &&\text{násobení } n \text{ na křivce } E\\
&\pi,\pi_E &&\text{Frobeniův endomorfismus}\\
&\widehat{\phi} &&\text{isogenie duální k } \phi\\
&\deg \phi &&\text{stupeň isogenie } \phi\\
&\ker \phi &&\text{jádro isogenie } \phi\\
&\# \ker \phi &&\text{velikost jádra isogenie } \phi\\
&\langle G\rangle &&\text{podgrupa generovaná množinou } G\\
&E/G &&\text{obraz } E \text{ v separabilní isogenii s jádrem } G\\
&E/\mathfrak{a} &&\text{obraz } E \text{ v isogenii generované ideálem } \mathfrak{a}\\
&E[n] &&n\text{-torze křivky } E\\
&\End(E) &&\text{okruh endomorfismů } E\\
&\mathrm{Ell}_{\mathcal{O}} &&\text{množina eliptických křivek nad } \mathbb{F}_p \text{ s okruhem endomorfismů } \End(E) \cong \mathcal{O}\\
& M \otimes_{R} N &&\text{tenzorový součin } R\text{-modulů } M \text{ a } N\\
&\End ^0(E) &&\text{algebra endomorfismů } E\\
&\Tr \phi, \Tr \alpha && \text{stopa endomorfismu } \phi \text{, stopa } \alpha \in \End^0 (E)\\
&\N \alpha && \text{norma } \alpha \in \End^0 (E)\\
&\widehat{\alpha} && \text{Rosatiho involuce } \alpha \in \End^0 (E)\\
&j(E) &&j\text{-invariant křivky } E\\
&G_{\ell} (\overline{\mathbb{F}}_p) &&\text{graf supersingulárních } j \text{-invariantů nad } \overline{\mathbb{F}}_p \text{ spojených isogeniemi stupně } \ell\\
&R[x] &&\text{okruh polynomů s koeficienty nad okruhem } R\\
&K(a_1,\dots, a_n) &&\text{nejmenší nadtěleso } K \text{ obsahující prvky } a_1, \dots, a_n\\  
&[K:L] &&\text{stupeň rozšíření tělesa } K \text{ nad } L\\
&\alpha(x) &&\text{lineární transformace } x \mapsto \alpha x \text{ působící na } \mathbb{Q}(\theta)\\ 
&M_{\alpha} &&\text{matice odpovídající } \alpha(x)\\ 
&\Tr M &&\text{stopa matice } M\\ 
&\det M &&\text{determinant matice } M\\
&Tr_K(\alpha) &&\text{stopa prvku } \alpha \text{ v } K\\ 
&N_K(\alpha) &&\text{norma prvku } \alpha \text{ v } K\\ 
&\mathcal{O}_K &&\text{okruh celých algebraických čísel tělesa } K\\
&Cl(\mathcal{O}) &&\text{grupa tříd ideálů pořádku } \mathcal{O}\\
&h_{\mathcal{O}} &&\text{řád grupy } Cl(\mathcal{O})\\
&(a) &&\text{hlavní ideál generovaný prvkem } a\\
&\frac{\mathfrak{a}}{m} &&\text{lomený ideál } \frac{\mathfrak{a}}{m}\\
&N_{\mathcal{O}}(\mathfrak{a}) &&\text{norma ideálu } \mathfrak{a} \subseteq \mathcal{O}, \text{ tj. } \vert \mathcal{O}/\mathfrak{a} \vert\\
&\mathfrak{a} + \mathfrak{b} &&\text{součet ideálů } \mathfrak{a} \text{ a } \mathfrak{b}\\
&\mathfrak{a} \mathfrak{b}, \mathfrak{a} \cdot \mathfrak{b} &&\text{součin ideálů } \mathfrak{a} \text{ a } \mathfrak{b}\\
&\mathfrak{a} \vert \mathfrak{b} &&\text{ideál } \mathfrak{a} \text{ dělí ideál } \mathfrak{b}\\
&\mathsf{G} / \mathsf{H} &&\text{faktorgrupa } \mathsf{G} \text{ podle } \mathsf{H}\\
&\deg f&&\text{stupeň polynomu, lomené funkce } f\\
&f^{\prime}&&\text{derivace } f\\
&f \vert_{M} && \text{zúžení } f \text{ na množinu } M\\
&\phi \vert_{\ell} && \text{zúžení isogenie } \phi \text{ na } \ell\text{-torzi}\\
&f \in O(g) &&f \text{ roste asymptoticky nejvýše stejně rychle jako } g
\end{flalign*}

\chapter{Eliptické křivky}


\begin{center}
\begin{verse}
\setverselinenums{1}{3}
\textit{It is possible to write endlessly on elliptic curves (This is not a threat.)}
\end{verse}
\hfill \textit{Serge Lang}
\end{center}

V naší první kapitole se budeme procházet světem isogenií eliptických křivek a učit se s~nimi pracovat. Kořeny této teorie sahají hluboko do algebraické geometrie, pro porozumění této kapitoly její znalost ale nevyžadujeme, čtenář si bohatě vystačí se znalostmi abstraktní algebry, viz například \cite{Rosicky}. Budeme postupovat volně dle \cite{Sutherland}, nicméně další vhodný úvodní materiál se nachází na \cite{DeFeo2}. Ne vždy budeme uvádět důkazy tvrzení, neboť jsou mnohdy příliš pokročilé či technické, v takových případech se odkážeme na relevantní literaturu. 

\section{Základy}


Po celou dobu budeme pracovat nad projektivním prostorem nad uzávěrem tělesa $K$, což je zjednodušeně řečeno množina bodů v $\overline{K}^n$, kde dva body považujeme za ekvivalentní, pokud leží v přímce s~počátkem, můžeme proto místo jednotlivých bodů pracovat s přímkami procházejícími skrz počátek.
\begin{definice}
Buďte $K$ těleso a $n$ přirozené číslo. \textit{Projektivní prostor} $\mathbb{P}^n (\overline{K})$ definujeme jako množinu tříd nenulových vektorů $(a_0, \dots, a_n) \in \overline{K}^{n+1}$ s relací ekvivalence $(a_0, \dots, a_n) \sim (b_0, \dots, b_n)$, pokud existuje $\lambda \in \overline{K}$, že $(a_0, \dots, a_n) = \lambda (b_0, \dots, b_n)$. Tyto třídy ekvivalence budeme značit $(a_0 : \dots : a_n)$ a nazývat \textit{body}.
\end{definice}

Představme si v $\mathbb{R}^3$ množinu $M$ všech přímek procházejících počátkem a množinu $N$ všech rovin procházejících počátkem. Každé dvě různé přímky z $M$ určují jedinou rovinu z~$N$ a naopak každé dvě různé roviny se protínají v jedné přímce z $M$. Nyní uvažme rovinu například $z=1$, každá přímka z $M$, která s ní není rovnoběžná, ji protíná v jednom bodě a každá rovina z $N$, která s ní není rovnoběžná, ji protíná v jedné přímce. Abychom každé přímce přiřadili právě jeden bod, můžeme v každém směru přiřadit rovnoběžným přímkám bod v nekonečnu v příslušném směru. Body dané průsečíky přímek z $M$ s rovinou $z=1$, i~v případě průsečíku v nekonečnu, tvoří tzv. \textit{projektivní rovinu} $\mathbb{P}^2 (\mathbb{R})$,

\begin{poznamka}
Je zajímavé uvážit souvislosti projektivních prostorů a barycentrických souřadnic, kde je každý bod vyjádřen jako vážený průměr vrcholů referenčního simplexu. Tyto souřadnice jsou též homogenní a každé dvě přímky se protínají, byť některé v~nekonečnu, takové body mají součet vah roven $0$. Můžeme o~barycentrických souřadnicích tedy přemýšlet jako o~projektivním prostoru s jiným základem.
\end{poznamka}

Připomeňme si pak definici eliptické křivky. Čtenář je možná obeznámen s \textit{Weirstrassovým tvarem} eliptické křivky $y^2 = x^3+ax+b$ pro $x,y \in K$, ten však nekreslí celou situaci. Často se eliptické křivky definují jako nesingulární projektivní křivky genu $1$ v $\overline{K}^3$, tj. jako množinu bodů $(X:Y:Z)$ splňujících:
\begin{equation*}
Y^2 Z + a_1 XYZ + a_3 Y Z^2 = X^3 + a_2 X^2 Z + a_4 X Z^2 + a_6 Z^3
\end{equation*}
s koeficienty $a_i \in K$. Pro naše účely si definici zúžíme a práci podstatně zjednodušíme. Konkrétně se budeme pohybovat nad tělesy, jejichž charakteristika není $2$ ani $3$. Tato tělesa často nabízí praktické výhody, my je však vynecháme. Nejprve totiž můžeme substitucí $Y \mapsto Y - \frac{a_1 X +a_3 Z}{2}$ zapsat naši křivku jako:
\begin{align*}
Y^2 Z -  \left(\frac{a_1 X + a_3 Z}{2}\right)^2 Z &=  X^3 + a_2 X^2 Z + a_4 X Z^2 + a_6 Z^3,\\
Y^2 Z &= X^3 + \frac{b_2}{4} X^2 Z + \frac{b_4}{2} X Z^2 + \frac{b_6}{4} Z^3,
\end{align*}
kde $b_2 = a_1 ^2 + 4 a_2, b_4 = a_1 a_3 + 2 a_4$ a $b_6 = a_3^2 + 4 a_6$. Substituce $X \mapsto X - \frac{b_2}{12} Z$ dále zjednodušuje naši křivku:
\begin{align*}
Y^2 Z &= \left( X - \frac{b_2}{12} Z\right)^3 + \frac{b_2}{4} \left( X - \frac{b_2}{12} Z\right)^2 Z + \frac{b_4}{2} \left( X - \frac{b_2}{12} Z\right) Z^2 + \frac{b_6}{4} Z^3,\\
Y^2 Z &= X^3 + \left( \frac{24 b_4 - b_2 ^2}{48} \right) X Z^2 + \left( \frac{b_2 ^2 + 216 b_6 - 36 b_2 b_4}{864} \right) Z^3.
\end{align*}
Naši křivku proto můžeme zapsat ve tvaru:
\begin{equation*}
Y^2 Z = X^3 +a X Z^2 + b Z^3,
\end{equation*}
kde $a,b \in K$. Libovolná taková rovnice udává eliptickou křivku za podmínky, že takzvaný \textit{diskriminant} této křivky, $4a^3 + 27 b^2$, je nenulový. Tato skutečnost je ekvivalentní s faktem, že eliptická křivka je nesingulární, přičemž lineární transformace proměnných zachovají (ne)singularitu křivky. Geometricky lze tuto podmínku interpretovat tak, že křivka sama sebe neprotíná, tedy nemá \uv{hrot}.

\begin{definice}
Mějme $K$ těleso charakteristiky různé od $2$ a $3$. Pro $a,b \in K$ taková, že\\ $4a^2+27b^3 \neq 0$, definujeme v $\mathbb{P}^2 (\overline{K})$ \textit{eliptickou křivku} jako množinu bodů $(X:Y:Z)~\in~\overline{K}^3$ splňující:
\begin{equation*}
Y^2 Z = X^3 + a X Z^2 + b Z^3. 
\end{equation*}
\end{definice}


\begin{znaceni}
Pokud všechny koeficienty eliptické křivky $E$ náleží do tělesa $K$, značíme ji $E/K$.
\end{znaceni}


Průsečíky naší křivky s přímkou $Z=0$ nutně mají i $X$-ovou souřadnici nulovou, všechny jsou proto reprezentovány třídou $(0:1:0)$. V opačném případě můžeme přejít na proměnné $x := X/Z, y:= Y/Z$, tedy bod $(x:y:1)$, čímž získáme křivku ve známém \textit{afinním}, v~literatuře  často uváděneném i~jako Weierstrassově, tvaru:
\begin{align*}
y^2 = x^3+ax+b.
\end{align*} 
Množina bodů na naší křivce tedy sestává z bodů $(x,y) \in K^2$ na naší afinní křivce spolu s~bodem v nekonečnu $\mathcal{O} = (0:1:0)$, jenž je exklusivní její projektivní variantě. 

\begin{znaceni}
Množinu všech bodů $E$ se souřadnicemi nad $K$ (společně s $\mathcal{O}$) budeme značit $E(K)$ a pokud $K$ je konečné těleso, počet prvků $E(K)$ budeme značit $\# E(K)$.
\end{znaceni}

Počet bodů na $E$ nad konečným tělesem $\mathbb{F}_q$ je shora ohraničen číslem $2q+1$, protože pro každé $x \in \mathbb{F}_q$ existují v $\mathbb{F}_q$ nejvýše $2$ odmocniny z $x^3+ax+b$, a poslední bod do počtu je~$\mathcal{O}$. V~$\mathbb{F}_q$ leží právě $\frac{q+1}{2}$ čtverců, tudíž za předpokladu, že $x^3+ax+b$ pokrývá $\mathbb{F}_q$ rovnoměrně, bychom na $E$ očekávali okolo $q$ bodů, společně s bodem v nekonečnu $q+1$. Roku 1933 tento odhad Helmut Hasse dokázal, tedy skutečně se $\# E(\mathbb{F}_q)$ nepříliš liší od $q+1$.
\begin{veta}(Hasse)\label{Hasse}
Nechť $E/\mathbb{F}_q$ je eliptická křivka. Pak:
\begin{equation*}
\vert q+1 - \# E(\mathbb{F}_q) \vert \leqslant 2\sqrt{q}.
\end{equation*}
\end{veta}
 Důkaz je k nalezení v \cite[Thm. V.1.1]{Silverman}. Již zde, ještě na začátku naší poutě, musíme a~priori brát jako platný jeden z nejdůležitějších výsledků ohledně eliptických křivek, ne však bez důvodu. Většina učebních textu jej dokáže v průběhu studia algebraické geometrie, v našem případě bychom potřebovali udělat poměrně velkou odbočku. Na naší cestě se přesto setkáme s místy, kde uvidíme taký či onaký způsob pohledu na problém poskytující řešení.
 
Všimněme si, že rozlišujeme body na eliptické křivce definované nad daným tělesem. Jako body na samotné křivce $E/K$ nebudeme, jak by se na první pohled mohlo zdát, brát pouze body definované nad $K$, nýbrž nad celým uzávěrem, abychom zachytili celou její strukturu.

\begin{figure}[h]
\begin{center}
\begin{tikzpicture}
        \begin{axis}[
            xmin=-2,
            xmax=4,
            ymin=-3,
            ymax=3,
            scale only axis,
            axis lines=middle,
            % set the minimum value to the minimum x value
            % which in this case is $-\sqrt[3]{7}$
            domain=-1.3247:2.1663,      % <-- works for pdfLaTeX and LuaLaTeX
%            domain=-1.91293118:3,   % <-- would also work for LuaLaTeX
            samples=200,
            smooth,
            % to avoid that the "plot node" is clipped (partially)
            clip=false,
            % use same unit vectors on the axis
        ]
            \addplot [red] {sqrt(x^3-x+1)};
            \addplot [red] {-sqrt(x^3-x+1)};
        \end{axis}
         \draw[->] (-0.5,3.63) -- (-0.5,6);
         \draw[->] (-0.5,3.63) -- (-0.5,1.35);
         \node[above right, outer sep=2pt] at (-0.5,6) {$\mathcal{O}$};
        \end{tikzpicture}
\end{center}
\caption{Eliptická křivka $y^2 = x^3-x+1$ nad $\mathbb{R}$.}
\end{figure}


\begin{znaceni}
Pod bodem $P \in E$ rozumíme $P = (x,y) \in E(\overline{K})$.
\end{znaceni}

Podívejme se nyní na eliptickou křivku $E$ geometricky, tedy v rovině vyznačme všechny body, které na ní leží. Je zjevné, že $E$ je symetrická podle osy $x$, definujme proto k $P \in E$ opačný bod $-P \in E$ jako obraz $P$ podle osy $x$. Pokud bychom na bodech naší křivky definovali součet, přirozeně bychom chtěli, aby součet $P$ a $-P$ byl $\mathcal{O}$.

\begin{figure}[h]
\begin{center}
\begin{tikzpicture}
        \begin{axis}[
            xmin=-2,
            xmax=4,
            ymin=-3,
            ymax=3,
            scale only axis,
            axis lines=middle,
            % set the minimum value to the minimum x value
            % which in this case is $-\sqrt[3]{7}$
            domain=-1.3247:2.1663,      % <-- works for pdfLaTeX and LuaLaTeX
%            domain=-1.91293118:3,   % <-- would also work for LuaLaTeX
            samples=200,
            smooth,
            % to avoid that the "plot node" is clipped (partially)
            clip=false,
            % use same unit vectors on the axis
        ]
            \addplot [red] {sqrt(x^3-x+1)};
            \addplot [red] {-sqrt(x^3-x+1)};
        \end{axis}
        \draw[black!30!green] (4.78,5.485) -- (0.975,4);
         \fill[blue] (0.975,4) circle (1.25 pt);
           \node[left, outer sep=2pt] at (0.975,4) {$P$};
           \fill[blue] (2.975,4.78) circle (1.25 pt);
           \node[above right, outer sep=2pt] at (2.975,4.78) {$Q$};
          \fill[blue] (4.78,5.485) circle (1.25 pt);
           \node[right, outer sep=2pt] at (4.78,5.485) {$-(P+Q)$};
             \fill[blue] (4.78,1.785) circle (1.25 pt);
           \node[right, outer sep=2pt] at (4.78,1.785) {$P+Q$};
           \draw[dashed] (4.78,5.485) -- (4.78,1.785);
             
         \draw[->] (-0.5,3.63) -- (-0.5,6);
         \draw[->] (-0.5,3.63) -- (-0.5,1.35);
         \node[above right, outer sep=2pt] at (-0.5,6) {$\mathcal{O}$};
        \end{tikzpicture}
\end{center}
\caption{Sčítání na eliptické křivce.}
\end{figure}

Řekneme-li, že tečna k $E$ ji protíná ve dvou identických bodech, pak každá přímka protíná $E$ v právě třech bodech včetně multiplicity. Průsečíky lineární rovnice s kubickou křivkou budou i s případným bodem v nekonečnu tři. Speciálně tečna v bodě s $y=0$ tento bod protíná dvakrát a ten třetí je bod v nekonečnu $E$. Přichází tedy na mysl definice součtu na $E$ taková, že součet každých tří bodů v přímce je $\mathcal{O}$. Pokud přímka procházející $P,Q \in E$ protíná $E$ potřetí v $R$, definujeme tedy $P+ Q = -R$. Pro součet bodů $P,Q \in E$ můžeme poté odvodit několik klíčových vlastností:
\begin{enumerate}
\item $P + Q = Q + P$,
\item $(P + Q) + R =P + ( Q + R)$,
\item $P + \mathcal{O} = P$,
\item $P + (-P) = \mathcal{O}$.
\end{enumerate} 

Rovnosti (i),(iii) a (iv) jsou dle naší definice sčítání intuitivně jasné, potíže však nastanou s bodem (ii), který je notoricky obtížné dokázat. Jeho klasický důkaz užívá pokročilejších metod algebraické geometrie, konkrétně Riemann-Rochovu větu, či větu Cayley-Bacharacha, která u dvou kubických křivek protínajících se v $9$ bodech zaručuje, že každá jiná kubická křivka procházející osmi z nich obsahuje i ten poslední. Tato poslední věta má aplikace i~mimo eliptické křivky, klasické výsledky projektivní geometrie jako Pappova či Pascalova věta z ní totiž snadno plynou. Poměrně elementární, byť výpočetně zdlouhavý důkaz Cayley-Bacharovy věty i jejích zmíněných důsledků se dá najít v \cite[Sec. 2.3]{Washington}.

Při takto definovaném součtu můžeme s body na $E$ pracovat jako s abelovskou grupou se sčítáním $+$ a~neutrálním prvkem $\mathcal{O}$. Samozřejmě součet dvou bodů dokážeme za pomocí analytické geometrie přímo spočíst. Přímka procházející dvěma různými body $P = (x_1,y_1)$ a $Q = (x_2,y_2)$ v rovině je daná rovnicí $y = \frac{y_2-y_1}{x_2-x_1} (x-x_1) + y_1$. Známe-li dva průsečíky této přímky s $E$, tedy $P$ a $Q$, dosazením do rovnice $E$ jsme schopni spočíst jejich třetí průsečík, bod $-(P+Q)$. 

Jediné, co nám chybí ke spokojenosti, je najít dvojnásobek bodu $P$, omezme se na případ $P$ neležící na ose $x$. Tečna k $E$ v bodě $P$ je přímka $PQ$, když se $Q$ limitně blíží k $P$. Sklon této přímky je tedy dán implicitní derivací $y^2 = x^3+ax+b$ v bodě $P = (x_1,y_1)$, tedy $2 y_1 y^\prime =3x_1 ^2 + a$. Tečna k $E$ v $P$ je pak určena vztahem $2y_1(y-y_1) = (3x_1 ^2+a)(x-x_1)$. Z~této rovnosti vyjádříme $y$ a dosadíme do rovnice přímky $E$, kde je $x_1$ dvojnásobný kořen. Můžeme proto vyfaktorizovat člen $(x-x_1)^2$ a~jako třetí lineární člen získat řešení pro $-(P+P)$.

Předchozí úvahy shrnuje následující tvrzení: 

\begin{veta}\label{sum}
Buďte $P = (x_1,y_1), Q = (x_2,y_2)$ afinní body na křivce $E : y^2 = x^3+ax+b$, přičemž $P \neq -Q$. Pak $P+Q = (x_3,y_3)$ je daný:
\begin{align*}
x_3 &= \lambda^2 - x_1 - x_2,\\
y_3 &= - \lambda x_3 - y_1 + \lambda x_1,
\end{align*}
kde:
\begin{equation*}
\lambda = \begin{cases}
\frac{y_2 - y_1}{x_2-x_1}, \text{ } \mathrm{ pokud } \text{ } x_1 \neq x_2,\\
\frac{3x_1 ^2 + a}{2y_1}, \text{ } \mathrm{ pokud } \text{ }  x_1 = x_2.
\end{cases}
\end{equation*}

\end{veta}
Úplný výpočet neuvádíme. Je možné dokázat asociativitu sčítání i tím, že pro body $P = (x_1,y_1), Q = (x_2,y_2)$ a $R = (x_3,y_3)$ spočteme bod $(P+Q)+R$ a ukážeme, že je symetrický ve dvojicích $(x_1,x_3)$ a $(y_1,y_3)$, případně že je přímo roven $P+(Q+R)$. Tyto výpočty nejsou prakticky proveditelné bez výpočetních přístrojů, nicméně za pomocí například programu Wolfram Mathematica se můžeme přesvědčit, že asociativita platí.

Pro zkrácení zápisu píšeme skálární násobky bodů, jinak řečeno $P+\cdots+P$, následovně:

\begin{definice}
Mějme bod $P \in E$. Pak pro $n$ přirozené definujeme jeho $n$-násobek:
\begin{equation*}
[n]_E P = \underbrace{P+ \cdots + P}_{n},
\end{equation*}
přičemž definujeme $[0]_E P = \mathcal{O}$ a pro $n < 0: [n]_E P = [-n]_E (-P)$.
\end{definice}

Díky asociativitě sčítání je bod $[n]_EP$ dobře definovaný. Pokud bude z kontextu jasná eliptická křivka, nad kterou pracujeme, budeme značit násobení skalárem pouze $[n]P$. Pojďme se pokusit $n$-násobek bodu spočíst co nejrychleji, zjevně se stačí omezit na případ $n > 0$.

Naivní postup výpočtu $[n] P$ jímá $n-1$ sčítání, to jistě dokážeme vylepšit. Analogickým postupem jako při rychlém umocňování využijeme zápis $n$ v binární soustavě. Inicializujeme $Q = \mathcal{O}$ a v $k$-tém kroku si budeme pamatovat bod $[2^k] P$, který ke $Q$ přičteme jen pokud $k$-tý bit v binárním zápisu $n$ je $1$, přičemž postupujeme od nejméně významného bitu. Spočteme si pak $[2][2^k] P = [2^{k+1}] P$ a celý proces opakujeme znovu.

\begin{priklad}
Spočtěme padesátinásobek nějakého bodu $P$. Binární zápis $50$ je $110010$. Počítejme pak:
\begin{equation*}
\begin{tikzcd}[arrows=to]
& \mathcal{O} \rar & P \rar & {[2]}P \rar & {[4]}P \rar & {[8]}P \rar & {[16]}P \rar & {[32]} P   \\
Q: &   \mathcal{O} \rar & \mathcal{O} \rar & {[2]}P \rar & {[2]}P \rar & {[2]}P \rar & {[18]}P \rar & {[50]} P   \\
 &  &  & + {[2]}P  & & & + {[16]}P  & +{[32]}P 
    \end{tikzcd} 
\end{equation*}
Užijeme tedy pouze $10$ operací sčítání.
\end{priklad}
 

Dohromady při výpočtu užijeme nejvýše $\lfloor \log_2(n) \rfloor -1 \leqslant \log_2(n)-1$ operací sčítání i~dvojnásobení. Dvojnásobek prvků spočteme alespoň tak rychle jako součet dvou bodů, tedy tímto postupem spočteme $[n]P$ v nejvýše $2( \log_2(n)-1)$ sčítáních.

\begin{priklad}\label{priklad2}
Určeme dvojnásobek bodu $P = (x,y)$ na $E : y^2 = x^3 + ax + b$. V duchu značení věty \ref{sum} máme pro $[2] P = (x_1,y_1)$:
\begin{align*}
x_1 &= \lambda^2 - 2x = \frac{(3x ^2 + a)^2 - 8 y^2 x }{4 y ^2} = \frac{(3x ^2 + a)^2 - 8 (x^3+ax+b) x }{4 (x^3+ax+b)} = \frac{x^4-2a x^2 - 8bx + a^2}{4 (x^3+ax+b)},\\
y_1 &=  - \lambda x_1 - y + \lambda x = \frac{(3x^2 + a)[-(3x ^2 + a)^2 + 12 y^2 x] - 8y^4}{8y^4} y \\
&=  \frac{(3x^2 + a)[-(3x ^2 + a)^2 + 12 (x^3 + ax + b) x] - 8 (x^3 + ax + b)^2}{8(x^3 + ax + b)^2} y\\
&= \frac{x^6 + 5 a x^4 + 20 b x^3 - 5 a^2 x^2 - 4ab x - a^3 - 8 b^2}{8(x^3 + ax + b)^2} y.  \\
\end{align*}
\end{priklad}


Všimněme si, že pro $P = (x,y)$ na eliptické křivce s $y=0$ je $[2] P = \mathcal{O}$. Pro bod $Q = (6,27) := (x_0,y_0)$ na křivce:
\begin{equation*}
y^2 = x^3 +54x+189
\end{equation*}
nad $\mathbb{Q}$ zase ověříme, že platí:
\begin{align*}
x_0^6 + 5 a x_0^4 + 20 b x_0^3 - 5 a^2 x_0^2 - 4ab x_0 - a^3 - 8 b^2 = 0,
\end{align*}
tedy $[4]Q = \mathcal{O}$. Obecně by nás mohlo zajímat, které body zobrazí násobení $n$ do nekonečna.


\begin{definice}
Buď $n$ celé číslo. O množině všech $P \in E$, že $[n] P = \mathcal{O}$, řekneme, že tvoří $n$-\textit{torzi} na $E$, a tuto množinu budeme značit $E[n]$.
\end{definice}

\begin{definice}
Buď $P$ bod na $E$. Pokud $n$ je nejmenší kladné číslo, že $[n]P = \mathcal{O}$, nazveme $n$ \textit{řádem} $P$. Pokud takové $n$ neexistuje, tak řekneme, že $P$ má nekonečný řád.
\end{definice}

Torze na eliptické křivce $E$ tvoří podgrupu $E(\overline{K})$, neboť pokud $[n]P = \mathcal{O} = [n]Q$, tak $[n](P+Q) =[n]P+[n]Q=\mathcal{O}$. Torzní grupy nám pomáhají hlouběji studovat eliptické křivky v~mnohých směrech. Zprvu si můžeme všimnout, že $E(\overline{\mathbb{F}_q})$ je sjednocením všech torzních grup, tedy že každý bod má konečný řád.
\begin{veta}
Každý bod $P$ na eliptické křivce $E$ nad konečným tělesem má konečný řád.
\end{veta}
\noindent \textit{Důkaz. } Mějme bod $P \in E(\overline{\mathbb{F}_q})$. Bod $P$ leží v konečném rozšíření $E\left(\mathbb{F}_q\right)$, neboli pro nějaké přirozené $k$ platí $P \in E\left(\mathbb{F}_{q^k}\right)$. V konečné grupě má každý prvek konečný řád, přičemž neutrální prvek grupy $E\left(\mathbb{F}_{q^k}\right)$ je $\mathcal{O}$, tedy $P$ má na $E$ konečný řád. \hfill $\square$\\

Zatímco $E(\mathbb{F}_q)$ je konečná grupa, množina bodů na racionální křivce $E(\mathbb{Q})$ obecně není a~existují na ní i body nekonečného řádu. Příkladem mřížového bodu nekonečného řádu na křivce je bod $(17,70)$ na křivce:
\begin{equation*}
E : y^2 = x^3 - 13,
\end{equation*}
tedy jeho násobením můžeme získat nekonečně mnoho racionálních bodů na $E$. Body nekonečného řádu jsou obecně těžko spočitatelné, nicméně body s řádem konečným dokážeme všechny najít za pomocí věty Lutz-Nagella \cite[Thm. 8.7]{Washington}, dle které všechny takové racionální body $(x,y)$ jsou mřížové a buď $2$-torzní, či $y^2$ dělí diskriminant naší křivky.

\section{Zobrazení mezi eliptickými křivkami}

Když studujeme algebraické struktury, často nás zajímají zobrazení mezi nimi. Násobení bodů na $E$ skalárem určuje homomorfismus grup $E(\overline{K}) \longrightarrow E(\overline{K})$, definuje proto endomorfismus na $E(\overline{K})$ daný lomenou funkcí nad $K$. Nyní se trochu obecněji podíváme na zobrazení mezi jednotlivými eliptickými křivkami, opět homomorfismy grup $E_1(\overline{K})~\longrightarrow~E_2(\overline{K})$.

Nejprve studujme zobrazení invertibilní, tedy lineární změny souřadnic $x,y$. Pokud zobrazení $(x,y) \mapsto (ax+by+c,dx+ey+f)$ převádí eliptické křivky ve Weierstrassově tvaru, snadno porovnáním koeficientů, například $xy$ a $x^2 y$, dojdeme k~nulovosti členů $b,c,d$ i $f$. Následně, aby členy při $y^2$ i $x^3$ byly po krácení oba rovny jedné, musí být $a = u^2, b = u^3$ pro nějaké nenulové číslo $u \in \overline{K}$. Taková zobrazení, $(x,y) \mapsto (u^2 x, u^3 y)$, převádí křivky:
\begin{equation*}
E_1 : y^2 = x^3 + u^4 a x + u^6 b \quad \longrightarrow \quad E_2 :  y^2 = x^3 + ax + b 
\end{equation*}
pro nenulové $u \in \overline{K}$. Jako lineární zobrazení mezi $E_1(\overline{K})$ a $E_2(\overline{K})$ jistě naše zobrazení zachovává přímky a tedy i součet bodů na našich křivkách, definuje proto homomorfismus z $E_1(\overline{K})$ do $E_2(\overline{K})$. Díky jeho invertibilitě definuje mezi těmito grupami dokonce isomorfismus.

\begin{definice}
Buďte $E_1/K : y^2 = x^3 + ax + b, E_2/K : y^2 = x^3 + cx + d$ eliptické křivky. Pak řekneme, že $E_1$ a $E_2$ jsou isomorfní, pokud existují $u \in \overline{K}$ splňující $a = u^4 c$ a $b = u^6 d$.
\end{definice}

Isomorfismy nemusí nutně být definované $K$, ale nad jeho rozšířením. Aby byl nad $\overline{K}$ definovaný, musí být díky předpisu $(x,y) \mapsto (u^2 x, u^3 y)$ psán nad rozšířením $K$ stupně dělícího $12$.
\begin{definice}
Buďte $E,E^\prime$ křivky isomorfní nad rozšířením $K$, ale ne nad $K$. Pak řekneme, že $E^\prime$ je \textit{twistem} $E$ nad $K$.
\end{definice}
Zobrazení z  $E:~y^2~=~x^3~+~ax~+~b$ dané $(x,y) \mapsto \left(\frac{x}{d}, \frac{y}{\sqrt{d^3}}\right) $ pro $\sqrt{d} \not \in K, d \in K$, nám dává isomorfismus do:
\begin{equation*}
E_d : y^2 = x^3 + d^2 a x + d^3 b,
\end{equation*}
avšak ne nad $K$, ale nad jeho kvadratickým rozšířením $K(\sqrt{d})$. Křivku $E_d$ nazveme \textit{kvadratickým twistem} $E$.

Pro křivky s $a=0$, resp. $b=0$, můžeme analogicky najít \textit{kubický} a \textit{sextický}, resp. \textit{kvartický} twist:
\begin{align*}
y^2 = x^3 + b \quad &\longrightarrow \quad y^2 = x^3 + d^2 b ,\\
y^2 = x^3 + b \quad &\longrightarrow \quad y^2 = x^3 + d b, \\
y^2 = x^3 +ax \; &\longrightarrow \quad y^2 = x^3 + d ax ,
\end{align*}
dané po řadě $(x,y)  \mapsto \left(\frac{x}{\sqrt[3]{d^2}}, \frac{y}{d}\right)$ a $(x,y)  \mapsto \left(\frac{x}{\sqrt[3]{d}}, \frac{y}{\sqrt{d}}\right)$, resp. $(x,y)  \mapsto \left(\frac{x}{\sqrt{d}}, \frac{y}{\sqrt[4]{d^3}}\right)$. Vidíme, že poslední dvě zmíněné křivky jsou navíc kvadratickými twisty po řadě kubického a kvadratického twistu $E$.

Chtěli bychom říci, kdy mezi dvěma eliptickými křivkami existuje isomorfismus, tedy najít nějaký invariant, který isomorfní křivky sdílí. Takovou funkci splňuje právě $j$-invariant, jehož definice se táhne hluboko do komplexní analýzy.

\begin{definice}
Pro eliptickou křivku $E: y^2 = x^3 + ax + b$ definujeme její $j$-\textit{invariant} jako:
\begin{equation*}
j(E) = 1728 \frac{4a^3}{4a^3+27b^2}.
\end{equation*}
\end{definice}
Poznamenejme, že ten je vždy nad $K$ definovaný, neboť eliptické křivky mají nenulový diskriminant.
\begin{veta}
Dvě křivky definované nad $K$ jsou isomorfní nad $\overline{K}$, právě pokud mají stejný $j$-invariant.
\end{veta}

\noindent \textit{Důkaz. } Nejprve předpokládejme, že křivky $E_1: y^2 = x^3+a_1x+b_1$ a $E_2 : y^2 = x^3+a_2 x + b_2$ jsou nad $\overline{K}$ isomorfní. Máme pak $a_2 = u^2 a_1$ a $b_2 = u^3 b_1$ pro nějaké $b \in \overline{K}$. Spočtěme $j$-invariant obou křivek:
\begin{equation*}
j(E_2) = 1728 \frac{4 u^6 a_1^3}{4 u^6 a_1^3 + 27 u^6 b_1^2} = 1728 \frac{4 a_1^3}{4a_1^3 + 27 b_1^2} = j(E_1),
\end{equation*}
$j$-invarianty isomorfních křivek se proto rovnají.

Nyní předpokládejme, že $j(E_1) = j(E_2)$. Počítejme:
\begin{align*}
1728 \frac{4 a_1^3}{4 a_1^3 + 27 b_1^2} &= 1728 \frac{4 a_2^3}{4a_2^3 + 27 b_2^2},\\
a_1^3 (4 a_2^3 + 27b_2^2) &= a_2^3 (4a_1 ^3 + 27 b_1^2),\\
a_1^3 b_2^2 &= a_2^3 b_1^2.
\end{align*}
Pokud by například $a_1$ bylo nulové, je z nesingularity $E_1$ nutně $b_1$ nenulové, tudíž $a_2 = 0$. Proto ani $b_2$ není rovno nule, tedy pro $u \in \overline{K}$ s $u^3 = \frac{b_1}{b_2}$ máme $(0,b_1) = (0,u^3 b_2)$. Analogicky pokud $b_i$ jsou nulová, máme $(a_1,0) = (u^2 a_2,0)$ pro $u$ s $u^2  = \frac{a_1}{a_2} \in \overline{K}$.

Konečně v případě, že $a_1 a_2 b_1 b_2 \neq 0$, máme $\frac{a_1^3}{a_2^3} = \frac{b_1^2}{b_2^2}$, což je druhou i třetí mocninou, tedy i šestou mocninou nějakého $u \in \overline{K}$. Toto číslo je tak šestou mocninou i~všech šestých odmocnin $u^6$ v~$\overline{K}$, pro tato $u$ je tak $\frac{a_1}{a_2}$ rovno $u^2$ násobeno třetí odmocninou z $1$ (ne nutně primitivní) a~$\frac{b_1}{b_2}$ rovno $u^3$ násobeno odmocninou z $1$. Pro nějaké z těchto šesti $u$ se obě odmocniny rovnají $1$, čili $a_1 = u^2 a_2$ a $b_1 = u^3 b_2$. \hfill $\square$\\

\begin{poznamka}
Čtenáře by mohla zarazit konstanta $1728 = 12^3$, kterou $j$-invariant násobíme. Koncept $j$-invariantu se definuje nejen pro eliptické křivky, ale i pro tzv. \textit{mřížky} (lattice), viz \cite[Def. 16.2]{Sutherland}. $j$-invariant těchto struktur je spojen s tzv. \textit{Laurentovou expanzí}, která je po násobení $1728$ vždy celočíselná, viz \cite[Ch. 11]{Cox}. Poznamenejme též, že Weierstrassův tvar není jediný možný vyjadřující eliptickou křivku, existují rodiny křivek vyjádřitelné v~tzv. \textit{Legendreově} či \textit{Edwardsově} tvaru, každá z nich mající svou vlastní formu $j$-invariantu. 
\end{poznamka}
\begin{priklad}
Vezměme si následujících pět křivek nad $\mathbb{F}_{101}$:
\begin{align*}
E_1 : y^2 &= x^3+x+1,\\
E_2 : y^2 &= x^3+5x+23,\\
E_3 : y^2 &= x^3+x-1,\\
E_4 : y^2 &= x^3+2x,\\
E_5 : y^2 &= x^3+2,
\end{align*}
a spočtěme si jejich $j$-invarianty (což jsou čísla v $\mathbb{F}_{101}$):
\begin{align*}
j(E_1) &= 1728 \frac{4}{31},\\
j(E_2) &= 1728 \frac{4 \cdot 5^3}{4 \cdot 5^3+27 \cdot 23^2} = 1728 \frac{4 \cdot 24}{4 \cdot 24 + 27 \cdot 24} = 1728 \frac{4}{31},\\
j(E_3) &= 1728 \frac{4}{31},\\
j(E_4) &= 1728,\\
j(E_5) &= 0.
\end{align*}
Vidíme, že $j$-invarianty $E_1$ a $E_2$ se shodují, přičemž v $\mathbb{F}_{101}$ se oba rovnají $1728 \cdot 4 \cdot 88$, nutně mezi nimi nad $\overline{\mathbb{F}}_{101}$ existuje isomorfismus. Snadno ověříme, že zobrazení:
\begin{equation*}
(x,y) \longmapsto (3^2 x, 3^3 y) = (9x,27y) 
\end{equation*}
převádí:
\begin{alignat*}{3}
y^2 &= x^3+x+1 \qquad \longrightarrow \qquad 27^2 y^2&&= 9^3 x^3 + 9x + 1,\\
&\qquad \hspace*{4.225cm}  22y^2&&= 22 x^3 + 9x + 1,\\
&\qquad \hspace*{4.225cm} 22y^2&&= 22 x^3 + 110x + 506,\\
&\qquad \hspace*{4.65cm} y^2&&= x^3 + 5x + 23.
\end{alignat*}
Inverzní isomorfismus $E_2 \longrightarrow E_1$ je pak daný $(x,y) \mapsto (34^2 x, 34^3 y) = (45x,15y)$, neboť multiplikativní inverze $3$ v $\mathbb{F}_{101}$ je $34$.

Křivka $E_3$ má stejný $j$-invariant jako $E_1$ a $E_2$, nad $\mathbb{F}_{101}$ mezi nimi a $E_3$ přesto isomorfismus neexistuje. $E_3$ je kvadratickým twistem $E_1$ nad $\mathbb{F}_{101^2} = \mathbb{F}_{101}[i]$, jakožto zobrazení $(x,y) \mapsto \left(\frac{x}{i^2}, \frac{y}{i^3}\right) = (-x,iy)$ převádí:
\begin{align*}
y^2 = x^3+x+1 \quad \longrightarrow \quad -y^2 &= -x^3-x+1,\\
 y^2 &=  x^3 + x - 1.
\end{align*}
Obdobně můžeme najít isomorfismus definovaný nad $\mathbb{F}_{101^2}$ mezi $E_1$ a $E_3$.\end{priklad}

Dvě speciální hodnoty $j$-invariantu jsou $0$ a $1728$, kterých nabývají křivky, které mají po řadě lineární, resp. konstantní člen roven $0$. Právě křivky s $j$-invariantem $0$ mají kubický (a~sextický) twist, ty s $j$-invariantem $1728$ zase kvartický.

Na propojení twistů křivek a počtu bodů na křivce poukazuje následující věta:

\begin{veta}\label{twister}
Uvažme křivku $E/\mathbb{F}_q : y^2 = x^3+ax+b$ a  $ \tilde{E}/\mathbb{F}_q : y^2 = x^3 + g^2 ax + g^3 b $ její kvadratický twist. Pak $\#E(\mathbb{F}_q) + \# \tilde{E}(\mathbb{F}_q) = 2(q+1)$.
\end{veta}
\noindent \textit{Důkaz.} Jistě je $g \in \mathbb{F}^\times_q$ kvadratický nezbytek. Ukážeme, že každé $x_1 \in \mathbb{F}_q$ přispívá právě dvěma body na obou křivkách. Pokud platí $x_1^3 + ax_1 + b = 0$, číslo $x_1$ dává po jednom bodu na obou křivkách, $(x_1,0)$, resp. $(g x_1,0)$. Pro zbylé body tvrdíme, že je právě jedno z~tvrzení pravdivé: 
\begin{itemize}
\item Existují dva body na $E(\mathbb{F}_q)$ s $x$-ovou souřadnicí $x_1$,
\item Existují dva body na $\tilde{E}(\mathbb{F}_q)$ s $x$-ovou souřadnicí $gx_1$.
\end{itemize}
Druhá odrážka je ekvivalentní s faktem, že: 
\begin{equation*}
(gx_1)^3 + g^2 a (gx_1) + g^3 b = g \cdot g^2 (x_1^3+ax_1+b)
\end{equation*}
je nenulový čtverec. Připomeňme, že součin dvou kvadratických nezbytků je kvadratický zbytek a~součin kvadratického zbytku a~nezbytku je nezbytek. Protože $g$ není čtverec v~$\mathbb{F}_q$, je právě jedno z čísel $x_1 ^3 + ax_1 + b, g(x_1 ^3 + ax_1 + b)$ (nenulovým) čtvercem, tedy v $\mathbb{F}_q$ má dvě odmocniny. Afinních bodů na obou křivkách je tak dohromady $2q$. Poslední dva jsou příslušné body v~nekonečnu. \hfill $\square$\\

Počet různých $j$-invariantů v $K$ určuje počet tříd isomorfismů křivek nad $\overline{K}$, případně kterých hodnot $j$-invariant nikdy nenabude. Jak si nyní ukážeme, tento počet je nejvyšší možný.

\begin{veta}\label{jjjj}
Pro každé $s \in K$ existuje eliptická křivka $E$ nad $K$ s $j(E) = s$.
\end{veta}
\noindent\textit{Důkaz.} Pro $s \in \lbrace 0,1728 \rbrace$ poslouží jako příklady po řadě křivky $y^2 = x^3+1, y^2 = x^3+x$. Pro zbylá $s \in K$ uvažme křivku:
\begin{equation*}
E: y^2 = x^3 +3s(1728-s)x + 2s(1728-s)^2.
\end{equation*}
Za předpokladu $\char K \not\in \lbrace 2,3 \rbrace$ je $E$ vskutku eliptická, můžeme tedy definovat $j$-invariant. Ten je roven:
\begin{align*}
j(E) &= 1728 \frac{4 [3s(1728-s)]^3}{4 [3s(1728-s)]^3 + 27[2s(1728-s)^2]^2 }\\
 &= 1728 s \frac{4 \cdot 27 s^2(1728-s)^3}{4 \cdot 27 s^2(1728-s)^3(s+1728-s)}=\frac{1728}{1728} s = s.
\end{align*}
Křivka $E$ proto má $j$-invariant roven $s$. \hfill $\square$\\

\begin{veta}
Pro každé $s \in \overline{K}$ existuje eliptická křivka $E$ nad $K(s)$, že $j(E) = s$.
\end{veta}
\noindent \textit{Důkaz.} Opět si rozmyslíme, že křivka $y^2 = x^3 +3s(1728-s)x + 2s(1728-s)^2$ je definovaná nad $K(s)$, tedy může posloužit jako řešení. \hfill $\square$\\

Jak násobení bodů $E$ skalárem, tak isomorfismy křivek, jsou homomorfismy bodů křivek nad tělesem $K$, resp. jeho rozšířením. Spadají tak pod rodinu zobrazení eliptických křivek zvaných \textit{isogenie}, o kterých se budeme dále bavit.

\section{Isogenie}

Podívejme se trochu obecněji na zobrazení mezi křivkami. Hlavní vlastnost, kterou bychom chtěli na takových zobrazeních vynutit, by bylo zachování grupové struktury bodů na křivce. Ukáže se, že taková zobrazení mají několik velmi dobrých vlastností.

\begin{definice}
Ať $E_1,E_2$ jsou eliptické křivky nad tělesem $K$. Surjektivní homorfismus grup $\phi: E_1(\overline{K}) \longrightarrow E_2(\overline{K})$ tvaru $\phi : (x:y:z) \longmapsto (u(x,y,z):v(x,y,z):w(x,y,z))$ pro polynomy $u,v,w \in K[x]$ nazveme \textit{isogenií}. 
\end{definice}

Dá se ukázat, viz \cite[II.6.8.]{Hartshorne} a \cite[III.4.8.]{Silverman}, že nekonstantní zobrazení mezi eliptickými křivkami dané polynomy nad $K$ je surjektivní homomorfismus mezi grupami $E_1 (\overline{K}) \longrightarrow~E_2(\overline{K})$, definice výše je tedy příliš silná. Zachycuje nicméně všechny důležité vlastnosti, které v~isogeniích hledáme. 

Isogenie ale můžeme obecně zapsat mnohem kompaktněji:
\begin{veta}
Buďte $E_1,E_2$ eliptické křivky nad $K$ a $\phi : E_1 \longrightarrow E_2 $ isogenie. Pak ji můžeme zapsat ve tvaru:
\begin{equation*}
\phi(x,y) = \left(u(x), v(x) y \right)
\end{equation*}
pro $u,v$ lomené funkce nad $K$.  
\end{veta}

\noindent \textit{Důkaz.} Víme, že isogenii můžeme vyjádřit jako $\phi: (x,y) \mapsto (u(x,y),v(x,y))$ pro $u,v$ lomené funkce nad $K$. Z rovnice eliptické křivky $E_1 : y^2 = x^3 + ax +b$ můžeme $y$ v sudé mocnině nahradit polynomem v $x$, čímž zajistíme, že $u$ i $v$ dokážeme vyjádřit jako funkce $r,s$, jejichž stupeň v $y$ je nejvýše $1$. Speciálně mějme $u(x,y) = \frac{f_1(x)+f_2(x)y}{f_3(x)+f_4(x) y}$ pro $f_i \in K[x]$. Pokud  tento zlomek rozšíříme o $f_3(x)-f_4(x)y$, vyruší se nám všechny liché mocniny $y$ ve jmenovateli a~sudé dokážeme nahradit polynomem v $x$. Můžeme proto předpokládat $u(x,y) = \frac{f_1(x)+f_2(x)y}{f_3(x)}$.

Protože $\phi$ je homomorfismem mezi grupami $E_1(\overline{K}) \longrightarrow E_2(\overline{K})$, platí rovnost $\phi(x,y) = -\phi(x,-y)$, tedy $f_2$ je identicky nulový polynom a $u$ je lomená funkce v $x$. Pokud obdobně vyjádříme $v(x,y) = \frac{g_1(x)+g_2(x)y}{g_3(x)}$, získáme $g_1 \equiv 0$ a $v(x,y) = \frac{g_2(x)}{g_3(x)}y$ pro $g_2,g_3 \in K[x]$. \hfill $\square$\\

\begin{definice}\label{covfefe}
Buď $\phi : E_1 \longrightarrow E_2$ isogenie. Pod \textit{standardním tvarem} $\phi$ rozumíme vyjádření $ \phi(x,y) = \left(\frac{u(x)}{v(x)}, \frac{r(x)}{s(x)} y \right)$, kde $u,v \in K[x]$ a $r,s \in K[x]$ jsou dvojice nesoudělných polynomů.
\end{definice}

Díky této charakterizaci můžeme začít s isogeniemi pořádně pracovat. Nyní již nebude překvapením se zabývat otázkou, které body se zobrazí do nekonečna. Zprvu vidíme, že do nekonečna se zobrazí body s $x$-ovou souřadnicí kořenem $v,s$. Tyto polynomy mají navíc stejnou množinu kořenů, právě protože bod $\mathcal{O}$ je isogenií zachován. 

Nás zajímají pouze eliptické křivky nad konečnými tělesy a každý polynom nad konečným tělesem má pouze konečně mnoho kořenů, množina bodů zobrazených do nekonečna isogenií je konečná. Tyto body opět tvoří podgrupu $E_1 (\overline{K})$, protože isogenie jsou homomorfismy grup bodů na křivkách.

\begin{definice}
Pod \textit{jádrem} isogenie $\phi$ rozumíme jádro $\phi$ ve smyslu homomorfismu grup $E_1 (\overline{K})\longrightarrow E_2(\overline{K})$. Značíme $\ker \phi$ a počet jeho prvků $\# \ker \phi$. 
\end{definice}

Propůjčme si i další terminologii zaobývající se lomenými funkcemi, abychom isogenie mohli přesněji popisovat.

\begin{definice}
Pod \textit{stupněm} isogenie $\phi$ budeme rozumět $\max (\deg u, \deg v)$, kde $u,v$ jsou definované v definici \ref{covfefe}, a značit $\deg \phi$. Definujeme $\deg [0] = 0$. 
\end{definice}
 
\begin{znaceni}
Skládání, resp. sčítání isogenií definujeme následovně: pro libovolné isogenie $\phi : E \longrightarrow E_1$ a $\psi : E_1 \longrightarrow E_2$ definujeme $(\psi \circ \phi) P  := \psi(\phi( P))$ pro každý bod $P \in E$ a pro isogenie $\phi,\psi : E \longrightarrow E_1$ zase $(\phi + \psi)P := \phi(P)+\psi(P)$ pro každý $P \in E$. Značme též isogenii opačnou jako $- \phi := [-1] \circ \phi$.
\end{znaceni}

Všimněme si, že složení dvou isogenií je zjevně opět isogenií. Všechny vlastnosti stupňů racionálních funkcí jsou u stupňů isogenií zachovány, zejména jejich multiplikativita.


S isogeniemi jsme se již na naší (prozatím) krátké cestě hned několikrát setkali, jak násobení (nenulovým) skalárem, tak isomorfismy zmíněné v předchozí kapitole, jsou isogeniemi, druhý případ dokonce dává jediné invertibilní. Násobení $[n]$ má jádro $E[n]$ a za chvíli si ukážeme, že má coby isogenie stupeň $n^2$. Zobrazení $[0]$ není surjektivní a proto není isogenií. Isomorfismy jsou isogenie lineární a mají pouze triviální jádro. Zobrazení:
\begin{equation*}
\phi : y^2 = x^3+x \quad \longrightarrow \quad y^2 =  x^3 + 11x + 62
\end{equation*}
mezi křivkami nad $\mathbb{F}_{101}$ dané $(x,y) \mapsto \left(\frac{x^2 + 10x - 2}{x+10},\frac{x^2  + 20x + 1}{x^2 + 20x - 1} y\right)$ je též isogenií, tentokrát stupně dva. Jádrem $\phi$ je množina $\lbrace \mathcal{O},(-10,0) \rbrace$, protože $x^2 + 20x - 1 = (x+10)^2$ v~$\mathbb{F}_{101}$.

Jedním z nejdůležitějších zobrazení na $\overline{\mathbb{F}}_p$ je tzv. \textit{Frobeniův morfismus}, pojmenovaný po Ferdinandu Frobeniovi, jemuž diktuje předpis $\pi: x \mapsto x^p$. Pevné body Frobeniova morfismu jsou přesně prvky $\mathbb{F}_p$, tudíž pro lomenou funkci $f$ nad $\mathbb{F}_p$ a $x_i \in \overline{\mathbb{F}}_p$ platí $f(x_1^p,\dots,x_n^p)=f(x_1,\dots,x_n)^p$. Speciálně platí vztahy $0^p = 0, 1^p = 1, a^p + b^p = (a+b)^p$ a~$a^p \cdot b^p = (ab)^p$ pro libovolné $a,b \in \overline{\mathbb{F}}_p$. Navíc toto zobrazení je nad $\overline{\mathbb{F}}_p$ prosté, pokud $a^p = b^p$:
\begin{equation*}
0 = a^p - b^p =  (a-b)^p,
\end{equation*} tedy $a=b$. Frobeniův morfismus je proto nad $\overline{\mathbb{F}}_p$ automorfismem.

Mocninu Frobeniova automorfismu definujeme jako $\pi^n : x \mapsto x^{p^n}$, neboli složení $n$ interací $\pi$. Rozkladové těleso polynomu $x^{p^n}-x$ je $\mathbb{F}_{p^n}$, což znamená, že $\pi^n$ se chová jako identita právě na konečných tělesech $\mathbb{F}_q$, kde $q =p^k$ s $k \leqslant n$. 

Zobrazení s podobným předpisem převádějící eliptické křivky též nese jméno po Frobeniovi.
\begin{definice}
Buď $E/\mathbb{F}_q: y^2 = x^3+ax+b$ eliptická křivka. Zobrazení $\pi_E : E \longrightarrow E$ dané $(x,y) \longmapsto (x^q,y^q)$ se nazývá \textit{Frobeniovým endomorfismem}.
\end{definice}

Díky vlastnostem $\pi$ definuje $\pi_E$ homomorfismus mezi grupami křivek, tedy je vskutku isogenií. Frobeniův endomorfismus fixuje právě $E(\mathbb{F}_q)$ a~má pouze triviální jádro. Dále komutuje s libovolnou isogenií nad $\mathbb{F}_q$, tj.:
\begin{equation*}
\pi_{E^\prime} \circ \phi = \phi \circ \pi_E,
\end{equation*}
kde $\phi : E \longrightarrow E^\prime$ je isogenie. Mocninu Frobeniova endomorfismu analogicky definujeme jako ${\pi^n}_E := \underbrace{\pi_E \circ \pi_E \circ \cdots \circ \pi_E}_{n}$ a má vlastnosti analogické k $\pi$. Pokud bude jasné, kdy mluvíme o~isogenii a ne o zobrazení na $\mathbb{F}_q$, zneužitím notace budeme $\pi_E$ značit pro jednoduchost též $\pi$.

Můžeme též definovat $p$-Frobeniův morfismus $\pi_p : (x,y) \mapsto (x^p,y^p)$ na $E$ nad $\mathbb{F}_q$ pro $q \neq p$, který je opět homomorfismem grup bodů eliptických křivek, ale již ne nutně definuje endomorfismus.

Když již máme solidní představu pojmu isogenie, pojďme se nyní pobavit o několika jejich základních vlastnostech. Jedním z nejdůležitějších výsledků ohledně isogenií mluví o~jejich duálu.

\begin{veta}
Buď $\phi: E \longrightarrow E_1$ isogenie stupně $n$. Pak existuje jediná isogenie $\hat{\phi}:~E_1~\longrightarrow~E$ splňující $\phi \circ \hat{\phi} = [n]_E$. Tuto isogenie nazýváme k $\phi$ \textit{duální}. Definujeme též $\hat{[0]} = [0]$.
\end{veta}
Důkaz existence duální isogenie je poměrně zdlouhavý a vyžaduje rozebírání mnoha případů, zde jej proto vynecháme. Čtenář jej však může najít v \cite[Thm. III.6.1.]{Silverman}, trochu elementárnější přístup se nachází v \cite[Thm. 7.8.]{Sutherland}.

Duální isogenie konečně opodstatňuje fakt, který na první pohled není vůbec jasný, že \uv{být isogenní} je relace ekvivalence. Několik základních vlastností duální isogenie stanovuje následující věta:

\begin{veta}\label{dual}
Buďte $E/K,E^\prime/K$ eliptické křivky a $\phi: E \longrightarrow E^\prime$ isogenie stupně $n$. Pak její duální isogenie pro každou jinou isogenii $\psi:E^\prime \longrightarrow E_1, \chi : E \longrightarrow E^\prime$ splňuje:
\begin{enumerate}
\item $\phi \circ \hat{\phi} = [n]_E$,
\item $\hat{\phi} \circ \phi = [n]_{E^\prime}$,
\item $\widehat{\phi \circ \psi} = \hat{\psi} \circ \hat{\phi}$,
\item $\widehat{\phi + \chi} = \hat{\phi} + \hat{\chi}$,
\item $\hat{\hat{\phi} } = \phi $.
\end{enumerate} 
\end{veta}

\noindent \textit{Důkaz.} Dokážeme vlastnosti $(ii)$, $(iii)$ a $(v)$. Platí:
\begin{equation*}
(\hat{\phi} \circ \phi) \circ \hat{\phi} = \hat{\phi} \circ (\phi \circ \hat{\phi}) = \hat{\phi} \circ [n]_{E} = [n]_{E^\prime} \circ \hat{\phi},
\end{equation*}
kde poslední rovnost platí, protože isogenie jsou jsou homomorfismy grup. Protože isogenie jsou surjektivní, musí platit $\hat{\phi} \circ \phi = [n]_{E^\prime}$. Dále, protože isogenie jsou homomorfismy grup bodů na křivkách, platí:
\begin{align*}
(\hat{\psi} \circ \hat{\phi}) \circ (\phi \circ \psi) &= \hat{\psi} \circ (\hat{\phi} \circ \phi) \circ \psi = \hat{\psi} \circ [\deg \phi] \circ \psi = \hat{\psi} \circ \psi \circ [\deg \phi]= [\deg \psi] \circ [\deg \phi]\\
&=[\deg \psi \circ \phi] = [\deg \phi \circ \psi] = (\widehat{\phi \circ \psi}) \circ ( \phi \circ \psi),
\end{align*}
tedy protože isogenie $\phi \circ \psi$ je surjektivní, platí $\hat{\psi} \circ \hat{\phi} =  \hat{\phi \circ \psi}$. Konečně, bod $(v)$ plyne z~$(i)$ a $(ii)$, platí totiž $\hat{ \hat{\phi}} \circ \hat{\phi}  = [n]_{E} = \phi \circ \hat{\phi}$, tedy $\hat{\hat{\phi}} = \phi$. Čtvrtá vlastnost je ukázána v \cite[Thm. III.6.1, Exc. 3.31]{Silverman} pro $\char K = 0$ a důkaz je naznačen pro tělesa konečná. \hfill $\square$\\

\begin{lemma}\label{deg}
Platí:
\begin{equation*}
\widehat{[n]} = [n] \qquad \text{ a } \qquad \deg [n] = n^2.
\end{equation*}
\end{lemma}
\noindent \textit{Důkaz. } Zjevně $\widehat{[0]} = [0]$ a $\widehat{[1]} = [1]$. Za pomocí věty $\ref{dual}, (iv)$, máme pro každé celé $n$:
\begin{equation*}
\widehat{[n+1]} =  \widehat{[n]} + \widehat{[1]} = [n]+[1] = [n+1],
\end{equation*} 
standardní oboustranný indukční argument pak dokončí první část. Z definice sčítání máme $[m] \circ [n] = [mn]$, tudíž $[n] \circ \widehat{[n]} = [n^2]$. Dle věty $\ref{dual} (ii)$, je $[n]$ isogenií stupně $n^2$. \hfill $\square$\\

\begin{poznamka}
V literatuře se vlastnosti duální isogenie dokazují tak, že se elementárnějšími úvahami, například o tzv. \textit{division polynomials}, ukáže $\deg [n] = n^2$, kde pak jednoduše plynou odrážky $(ii)$, $(iii)$ a $(v)$. Čtvrtý bod je obzvláště těžké dokázat a~jeho nejvíce přímočarý důkaz užívá \textit{Weilových párování}, kterým se v naší práci nevěnujeme.
\end{poznamka}

Je důležité si uvědomit, co nám předchozí charakterizace vlastně říkají o duální isogenii. Duální isogenie k $\phi$ je z našeho lemmatu též isogenií stupně $n$, která má velmi pěkné vlastnosti. Navíc pro libovolnou isogenii $\phi$ z $E$ stupně $n$ je $\ker \phi \subseteq E[n]$, neboť libovolný prvek v jádře $\phi$ se skrz $\hat{\phi}$ zobrazí do nekonečna $E$.

Když víme, že \uv{být isogenní} je relace ekvivalence, dalším krokem je jistě hledat způsob, jak klasifikovat třídy isogenních křivek. V minulé sekci jsme si ukázali, že na kvadratickém twistu křivky leží pouze určitý počet bodů. I případ isogenií definovaných nad tělesem $\mathbb{F}_q$ úzce souvisí s počtem bodů ležících na křivce. Samo kritérium zní až překvapivě jednoduše:

\begin{veta}\label{satotate} (Sato-Tate)
Buďte $E,E^\prime$ eliptické křivky nad $\mathbb{F}_q$. Pak tyto křivky jsou nad $\mathbb{F}_q$ isogenní, právě pokud platí $\#E (\mathbb{F}_q) = \#E^\prime (\mathbb{F}_q)$.
\end{veta}
\noindent \textit{Důkaz.} Isogenie jsou surjektivní, přičemž isogenie nad $\mathbb{F}_q$ zobrazuje $E(\mathbb{F}_q)$ samu na sebe. Pokud jsou $E$ a $E^\prime$ isogenní, platí pak $\#E (\mathbb{F}_q) \geqslant \#E^\prime (\mathbb{F}_q)$ a $\#E^\prime (\mathbb{F}_q) \geqslant \#E (\mathbb{F}_q)$, což dává jednu polovinu věty. Druhá část již tak jednoduše nepřichází a její důkaz dokonce není ani zdaleka přístupný z pohledu algebraické geometrie. Poprvé byla druhá implikace (resp. tvrzení jí ekvivalentní) zveřejněno v jedné z nejvlivnějších publikací Johna Tate, \cite{Tate}. \hfill $\square$\\

Body, které se nachází v jádru isogenie, tvoří podgrupu $E(\overline{K})$, přičemž její velikost je shora omezena stupněm isogenie. Limitní případ v tomto smyslu má zajímavé vlastnosti.

\section{Separabilní isogenie}

\begin{definice}
Mějme $E,E^{\prime}$ křivky nad $K$ a $\phi: E \longrightarrow  E^\prime$ isogenii stupně $n$. Pokud je $\# \ker \phi = n$, pak o $\phi$ řekneme, že je \textit{separabilní}. V opačném případě řekneme, že $\phi$ je \textit{neseparabilní}. V~případě, že je $\deg \phi $ roven mocnině $\char K$, mluvíme o $\phi$ jako o \textit{čistě neseparabilní}.
\end{definice}

Pozoruhodné na tomto pojmenování je fakt, že separabilita a čistá neseparabilita se ne nutně vylučují. Každý isomorfismus je isogenií stupně $1$ s jádrem velikosti $1$, tedy separabilní, přičemž $p^0 = 1$, takže isomorfismy jsou čistě neseparabilní. Naopak Frobeniův endomorfismus je isogenie neseparabilní i čistě neseparabilní. Charakterizujme dále separabilní isogenie.
\begin{veta}
Ať $E,E^\prime$ jsou eliptické křivky nad $K$ a $\phi : E \longrightarrow  E^\prime$ je isogenie daná standardní formou $(x,y) \mapsto \left( \frac{u(x)}{v(x)}, \frac{r(x)}{s(x)} y \right)$. Pak $\big(\frac{u}{v} \big)^\prime \neq 0$ nastane právě pokud $\phi$ je separabilní.
\end{veta}
\noindent \textit{Důkaz.} Položme $p = \char K$. Rovnost $0 = \left(\frac{u}{v} \right)^\prime = \frac{u^\prime v - v^\prime u}{v^2}$ v $K$ nastane, právě pokud $u^\prime v = v^\prime u$. Protože je $\phi$ isogenie, jsou $u,v$ nenulové polynomy nad $K$. Předpokládejme, že $u^\prime$, a tedy i $v^\prime$ nejsou nulové. Z nesoudělnosti polynomů $u,v$ nutně každý kořen $u$ je kořenem $u^\prime$ s nejméně stejnou násobností. Nicméně pro $u^\prime \neq 0$ je $\deg u > \deg u^\prime$, což je spor. Rovnost $u^\prime v = v^\prime u$ proto můžeme relaxovat na $u^\prime = v^\prime = 0$, tedy každý nenulový jednočlen $u,v$ má exponent dělitelný $p$ a tak $u = f(x^p)$ a $v = g(x^p)$ pro nějaké polynomy $f,g \in K[x]$. Pak ale $\frac{u(x)}{v(x)} = \frac{f(x^p)}{g(x^p)} = \left( \frac{f(x)}{g(x)} \right)^p$ a $v$ jistě nemá jádro velikosti $\deg u/v$, ať už $p > 0$ či ne.

Uvažme nyní $(a,b)$ bod v obrazu $E(\overline{K})$ ve $\phi$ takový, že $ab \neq 0$ a $a$ není podílem vedoucích koeficientů $u$ a $v$. Takový bod jistě existuje, protože obraz $\phi(E(\overline{K}))$ je nekonečná množina. Uvažme nyní množinu $\mathsf{M}$ všech předobrazů $(a,b)$ ve $\phi$, neboli bodů $(x,y) \in E$ s $\phi(x,y) = (a,b)$. Protože $\phi$ je homomorfismus grup, počet prvků $\mathsf{M}$ je přesně roven velikosti jádra $\phi$.

Pro každé $(x,y) \in \mathsf{M}$ dále platí:
\begin{equation*}
\frac{u(x)}{v(x)} = a, \qquad \frac{r(x)}{s(x)}y = b.
\end{equation*}
Díky předpokladu $b \neq 0$ je každé vyhovující $y$ jednoznačně určeno daným $x$ jako $b \frac{s(x)}{r(x)}$, což znamená, že velikost $\mathsf{M}$ je rovna počtu $x$ splňujích první naši rovnost, tedy počtu různých kořenů polynomu $h:= u - av$, který má díky podmínkám na $a$ stupeň $\deg \phi$. Dejme tomu, že $x_0$ je vícenásobný kořen $h$, pak platí:
\begin{align*}
u(x_0) &= a v(x_0),\\
u^{\prime} (x_0)  &=  a v^{\prime} (x_0).
\end{align*}  
Násobení protějších stran těchto rovností dává $u^{\prime} (x_0) v (x_0) = u (x_0) v^{\prime} (x_0)$, $x_0$ je tedy kořenem (nenulového) polynomu $u^\prime v - u v^\prime$, který má v $\overline{K}$ pouze konečně mnoho kořenů. Protože $\phi(E(\overline{K}))$ je nekonečná a $\mathsf{M}$ konečná množina, můžeme si zvolit $(a,b)$ bod takový, že $h$ žádný násobný kořen nemá. Pak $\# \ker \phi = \vert \mathsf{M}\vert = \deg h = \deg \phi$. \hfill $\square$\\

Speciálně nad tělesem s nulovou charakteristikou jsou všechny isogenie neseparabilní. Zaměřme se na konečný případ, kde musí pro $\phi$ ve standardním tvaru platit $(u/v)^\prime =0$, tedy jak jsme si ukázali v důkazu předchozí věty, $u/v$ je složením racionální funkce nad $\mathbb{F}_q$ a $p$-Frobeniova morfismu na $\mathbb{F}_q$. Vypadá to tedy, že i $y$-ová souřadnice se bude chovat podobně, bohužel dokázat tento fakt je poměrně ošklivější, než ho konstatovat. 

\begin{dusledek}\label{separ2}
Buď $\phi$ isogenie nad $\mathbb{F}_q$. Pak existuje separabilní isogenie $\psi$ a $n \in \mathbb{N}_0$, že:
\begin{equation*}
\phi = \psi \circ \pi_p ^n.
\end{equation*}
\end{dusledek}

\noindent \textit{Důkaz.} Stačí ukázat, že pro každou neseparabilní isogenii $\phi$ existuje separabilní isogenie $\psi$ s~$\phi = \psi \circ \pi$. Pro $x$-ovou souřadnici tento výsledek známe, zbytek důkazu se dá najít na \cite[Lemma 6.3.]{Sutherland}. Tento důkaz není nijak zvlášť instruktivní, zde ho proto vynecháváme. Iterací tohoto faktu a skutečností, že Frobenius komutuje s libovolnou isogenií nad $\mathbb{F}_q$, pak získáme výsledek. \hfill $\square$\\

Separabilní isogenie, jako takové, zatím nevypadají příliš zajímavě. Mají ale jednu vlastnost úzce spojenou s jejich jádrem, která je pro naši práci natolik stěžejní, že bez jejího zmínění by text byl poloviční.

\begin{veta}\label{isomor}
Buďte $E$ eliptická křivka a $\phi : E \longrightarrow E^\prime$ libovolná separabilní isogenie s~jádrem $G \subseteq E(\overline{K})$. Pak všechny křivky $E^\prime$ jsou spolu isomorfní.
\end{veta}
Důkaz tvrzení je uveden v \cite[Prop. 12.12]{Washington}, nicméně autor jej zde podává s notnou dávkou Galoisovy teorie, jejíž znalost od čtenáře nepředpokládáme.

\begin{znaceni}
Buď $G \subseteq E(\overline{\mathbb{F}}_q)$ konečná grupa. Značme $E/G$ až na isomorfismus unikátní křivku, která pro každou separabilní isogenii $\phi : E \longrightarrow E^\prime$ s jádrem $G$ splňuje $E^\prime \cong E/G$.
\end{znaceni}

\begin{poznamka}
Ač $E/G$ je pouze značení pro křivku a nesmí být naivně bráno ve smyslu faktorizace, není zcela nepodložené. Ve zkratce zde načrtněme důvod. Každá konečná podgrupa $G \subseteq E(\overline{K})$ definuje surjektivní homomorfismus grup $\phi : E \longrightarrow E/G$ s jádrem $G$, kde $E/G$ je isomorfní faktorgrupě $E(\overline{K})/G$. Není naprosto vůbec zjevné, že $E/G$ je eliptickou křivkou, ani že $\phi$ je isogenií, detaily faktorizace $E/G$ též vyžadují náramnou péči. Čtenář obeznámen s teorií tělesových vnoření a obecně Galoisovou teorií nalezne podrobnější náznak důkazu na \cite[Thm. 6.10.]{Sutherland}.
\end{poznamka}

Jednoznačnost (až na isomorfismus) cílové křivky separabilní isogenie má kolosální dopady na naše pochopení isogenií. Říká nám totiž, že separabilní isogenie můžeme uvažovat ne mezi přímo eliptickými křivkami, ale mezi jejich $j$-invarianty, což je jedna z klíčových vlastností vedoucí na praktické prokoly užívající isogenií.

Separabilní isogenie z $E \longrightarrow E^\prime$ je daná lomenými funkcemi nad $K$ a známe-li její jádro, dokážeme ji explicitně spočíst, přičemž libovolná konečná podgrupa $E(\overline{K})$ je jádrem separabilní isogenie. Vzorce udávající (až na isomorfismus) přesný tvar separabilní isogenie z $E \longrightarrow E^\prime$ s daným jádrem se nazývají \textit{Véluovy} po Jeanu Véluovy, který je první publikoval roku 1971 ve \cite{Velu}. Jejich zápis je obecně velice nezáživný a~pro nás nepodstatný, stačí nám mít v povědomí, že separabilní isogenie s daným jádrem můžeme explicitně vyjádřit. Jejich přesnou formu a~důkaz správnosti jsou k uvedeny v \cite[Ch.~8.2]{DeFeo}. V Sage $9.0$ jsou Véluovy vzorce implementovány pro isogenii z $E$ s jádrem $G$ s časovou složitostí $O(\# G)$ příkazem:
\begin{equation*}
\texttt{EllipticCurveIsogeny(E,ker G)}.
\end{equation*}

\begin{priklad}
Spočtěme separabilní isogenii $\phi$ s doménou eliptickou křivkou $E/\mathbb{F}_{101} : y^2 = x^3+8x+23$ a~jádrém cyklickou grupou generovanou bodem $P =(68,9) \in E$. Bod $P$ má řád $4$ a grupa $\langle P \rangle = \lbrace P,[2]P,[3]P,\mathcal{O}\rbrace = \lbrace (68,9),(29,0), (68,92),\mathcal{O} \rbrace$ je tedy jádrem $\phi$. Příkaz $\texttt{phi = EllipticCurveIsogeny(E,P)}$ v Sage 9.0 vygeneruje isogenii $\phi$ a~tu určíme s pomocí Véluových formulí příkazem $\texttt{phi.rational\_maps()}$:
\begin{equation*}
\phi : (x,y) \longmapsto \left(\frac{x^4 + 37 x^3 - 26 x^2 - 15x - 21}{x^3 + 37 x^2 - 17 x + 32},\frac{x^5 + 41 x^4 - 9 x^3 + 5 x^2 + 3x - 7}{x^5 + 41 x^4 - 18 x^3 + 6 x^2 + 35x - 21} y  \right). 
\end{equation*}
Příkaz $\texttt{phi.codomain()}$ dává cílovou křivku $\phi$ spočtenou pomocí Vélouvých formulí a je to $E^\prime/\mathbb{F}_{101} : y^2 = x^3 + 53 x + 41$, samozřejmě všechny křivky s ní isomorfní jsou doménou eliptické křivky s jádrem $\langle P \rangle$. Kořeny polynomu $x^5 + 41 x^4 - 18 x^3 + 6 x^2 + 35x - 21$ nad $\mathbb{F}_{101}$ jsou pouze $29$ a $68$, přičemž $29$ dvojnásobný a $68$ trojnásobný, což odpovídá faktu, že grupa $\langle P \rangle$ se zobrazí do nekonečna. V době psaní této práce je Sage schopen spočíst pouze isogenie s cyklickým jádrem a pomocí Véluových formulí.
\end{priklad}

Jistě složením neseparabilní isogenie s libovolnou jinou získáme opět neseparabilní isogenii. Podobné vlastnosti má ale i součet isogenii.
\begin{veta}\label{separsum}
Buďte $\phi,\psi : E \longrightarrow E_1$ isogenie, přičemž $\phi$ je neseparabilní. Pak $\phi+\psi$ je neseparabilní, právě pokud $\psi$ je neseparabilní.
\end{veta}
\noindent \textit{Důkaz.} Označme $\pi_p : (x,y) \rightarrow (x^p,y^p)$ $p$-Frobeniův endomorfismus na $E$, ten komutuje s~libovolnou isogenií, a~navíc isogenie $\pi$ je nějakou jeho mocninou. Podle věty \ref{isomor} existují separabilní isogenie $\eta,\vartheta : E \longrightarrow E_1$ splňující $\phi =  \eta \circ \pi_p ^a$ a $\psi = \vartheta\circ \pi_p ^b$, kde $a > 0$. Pokud $\psi$ je neseparabilní, je exponent $b$ kladný, tedy součet $\phi+\psi$ je roven:
\begin{equation*}
\phi+\psi = \eta \circ \pi_p ^a +\vartheta\circ \pi_p ^b = (\eta \circ \pi_p ^{a-1} + \vartheta \circ \pi_p ^{b-1}) \circ \pi_p, 
\end{equation*}
neseparabilní isogenii. Naopak je-li isogenie $\phi+\psi$ neseparabilní, je $\psi = (\phi+\psi) - \phi$ součtem neseparabilních isogenií $\phi+\psi$ a $-\phi$, o kterém jsme právě ukázali, že je neseparabilní. \hfill $\square$\\

\begin{poznamka}
Tato věta má hned několik důležitých aplikací, jednu z nich si ukážeme hned o dvě sekce dále. Je ale též jednou z klíčových ingrediencí důkazu Hasseho věty \ref{Hasse}. Konkrétně z ní plyne, že $[1]-\pi$ je separabilní isogenie, tedy $\deg [1]-\pi = \# \ker [1]-\pi = \# E(\mathbb{F}_q)$, k tomuto fakt se ještě vrátíme. Stačí si pak všimnout, že příslušné členy v Hasseho větě jsou po řadě $\deg [1] - \pi, \deg [1], \deg -\pi$ a užít jednu speciální formu Cauchy-Schwarzovy nerovnosti, na detaily čtenáře odkazujeme na \cite[Thm. V.1.1.]{Silverman}.
\end{poznamka}

Konečně, pojďme se pokusit spočítat separabilní isogenie efektivněji. Je-li velikost jádra této isogenie prvočíselné (a tedy jádro cyklické), nespočteme ji jistě v čase lepším než lineárním vzhledem k velikosti jádra. Pokud ale pracujeme jádrem \textit{hladké} velikosti, tedy dělitelné pouze prvočísly do dané hranice, můžeme postupovat mnohem rychleji. 

\begin{veta}\label{prvoo}
Každou isogenii $\phi$ složeného stupně můžeme rozložit na kompozici isogenií prvočíselných stupňů.
\end{veta}
\noindent \textit{Důkaz.} Dejme tomu, že  $\phi$ převádí křivky $E \longrightarrow E_1$. Protože $\pi$ má prvočíselný stupeň charakteristiky našeho tělesa, stačí nám díky větě \ref{separ2} uvažovat $\phi$ isogenii separabilní. Postupujme nyní silnou indukcí vzhledem k počtu dělitelů $\deg \phi$. Pokud $G = \ker \phi$ je triviální či má prvočíselný řád, jsme hotovi. V opačném případě dejme tomu, že všechny isogenie s jádrem nižšího počtu dělitelů než $\# \ker \phi$ jsou rozložitelné. Víme, že $G$ obsahuje podgrupu $H$ prvočíselného řádu (tzv. \textit{Sylowova podgrupa}), která určuje separabilní isogenii $\psi : E \longrightarrow E_2 \cong E/H$. Pak obraz $G$ v $\psi$ je konečná podgrupa $E_1 (\overline{K})$, která je isomorfní $G/H$, a definuje isogenii $\chi : E_2 \longrightarrow E_3 \cong E_2/\psi(G)$. Jádro $\chi \circ \psi$ je právě $G$, tedy podle věty \ref{isomor} existuje isomorfismus $\iota : E_3 \longrightarrow E_2$ splňující $\phi = \iota \circ \chi \circ \psi$. Podle předpokladu $\iota \circ \chi$ je buďto isomorfismus, nebo je rozložitelná na kompozici separabilních isogenií prvočíselných stupňů. \hfill $\square$\\

Tato věta zní hezky z čiře teoretického pohledu studia křivek, je ale hlavní ingrediencí v~rychlejším počítání (separabilních) isogenií. Označíme $\langle G \rangle$ podgrupu $E(\overline{K})$ generovanou množinou $G = \lbrace P,Q,R,\dots \rbrace$ a pojďme se pokusit efektivně spočíst isogenii $\phi : E \longrightarrow E/ \langle G \rangle$. Postačí nám spočíst separabilní isogenii $\psi : E \longrightarrow E/\langle P \rangle$, kde $P$ má prvočíselný řád, pro podgrupy generované $Q,R, \dots$ spočteme analogicky separabilní isogenie převádějící $E/\langle P \rangle := E^\prime \longrightarrow E^\prime/\langle Q \rangle := E^{\prime \prime} \longrightarrow E^{\prime \prime}/\langle R \rangle \dots$. Věta \ref{isomor} nám zaručí, že složení všech takových isogenií bude mít jádro $\langle G \rangle$.

Ale isogenii $E \longrightarrow E/\langle P \rangle$ spočteme jednoduše:
\begin{equation*}
E \stackrel{\phi_1}{\longrightarrow} E/\langle \ell^{a-1} P \rangle \stackrel{\phi_2}{\longrightarrow}  E/\langle \ell^{a-2} \phi_1 (P) \rangle \stackrel{\phi_3}{\longrightarrow} \cdots \stackrel{\phi_{a}}{\longrightarrow} E/\langle \phi_{a-1} \circ \phi_{a-2} \circ \cdots \circ \phi_1 (P) \rangle,
\end{equation*}
kde řád $P$ je $\ell^a$. Snadno nahlédneme, že jádro $\phi_i \circ \dots \circ \phi_1$ je $\langle \ell^{a-i} P \rangle$ a tedy separabilní isogenie daná složením všech $\phi_i$ má jádro přesně $\langle P \rangle$.

Véluovy formule nám umožní každou $\phi_i$ spočíst v $O(\ell)$ operacích a tedy celý proces je hotov pouze v $O(\ell a)$ operacích. Celou isogenii $E \longrightarrow E/\langle G \rangle$ takto spočteme v~logaritmickém čase vzhledem k velikosti jádra.

\section{Torzní body}

Vraťme se k operaci násobení bodů. Za pomocí vlastností isogenií vyvinutých v předchozích částech budeme konečně schopni přijít na kloub struktuře torzních grup a na základě toho i samotné grupě $E(\mathbb{F}_q)$. Začněme tedy směrem k tomuto cíli dělat první krůčky.

Charakterizovat $E[2]$ je jednoduché. Spolu s bodem v nekonečnu jsou násobením dvěma anihilované právě tři další body, jejich $x$-ové souřadnice jsou jednotlivými (různými!) kořeny $x^3+ax+b$. Protože torze tvoří grupu a na naší $2$-torzi má každý afinní bod řád $2$, musí nutně být $E[2] \cong \mathbb{Z}_2 \times \mathbb{Z}_2$.

$3$-torze jsme též schopni diskutovat. Body na ní splňují $[2]P = -P$, speciálně se $x$-ové souřadnice obou stran rovnají. To znamená, že:
\begin{equation*}
\left(\frac{3x^2+a}{2y}\right)^2 -2x = x,
\end{equation*}
neboli díky rovnosti $y^2 = x^3+ax+b$:
\begin{equation*}
(3x^2+a)^2 = 12x(x^3+ax+b),
\end{equation*}
což je kvartická rovnice, která se snadno ověří jako s nenulovým diskriminantem. Každému ze čtyř různých vyhovujících $x$ přísluší právě dvě hodnoty $y$ (krom $\mathcal{O}$ se $2$ a $3$-torze neprotínají) a body $(x,y)$ mají všechny řád $3$. Spolu s $\mathcal{O}$ náleží $3$-torzi právě $9$ bodů. Snadno pak dojdeme k závěru $E[3] \cong \mathbb{Z}_3 \times \mathbb{Z}_3$.

V obou případech argumenty implicitně závisí na faktu, že $q$ není mocnina $2$ ani $3$, jinak naše eliptická křivka nemá tvar, který jí připisujeme. Tento případ je podrobněji rozebírán v \cite[Ch. 3.1]{Washington}.

Mohli bychom se tedy dovtípit, že $n$-torze pro $n$ nesoudělné s $q$ je isomorfní $\mathbb{Z}_n \times \mathbb{Z}_n$. Tato skutečnost je díky existenci duální isogenie velmi úzce spjata se separabilitou $n$-násobící isogenie.
\begin{lemma}\label{nasobsepar}
Buď $E/K$ eliptická křivka s $p = \char K$ a $n$ celé číslo. Pak $[n]$ je neseparabilní, právě pokud $p \mid n$.
\end{lemma}
\textit{Důkaz.} Dejme tomu, že $[n]$ je neseparabilní, pak díky důsledku \ref{separ2} je $[n] = \pi \circ \phi$ pro nějakou isogenii $\phi$ a tedy $p \mid \deg \pi \cdot \deg \phi = \deg \pi \circ \phi = \deg [n] = n^2$, neboli $p \mid n$. Mějme naopak $p \mid n$, můžeme pak psát $[n] = [p] [n/p]$. Víme, že $[p]$ je neseparabilní, protože $\pi \circ \widehat{\pi} = [\deg \pi] = [p]$. Definice separability pomocí velikosti jádra jistě implikuje, že složení neseparabilní isogenie, zde $[p]$, s libovolnou jinou vyprodukuje isogenii neseparabilní, tedy $[n]$ je neseparabilní sama. \hfill $\square$\\

Nejprve se zaměříme na prvočísla a jejich mocniny.

\begin{veta}\label{prvotorze}
Buď $E/K$ eliptická křivka s $p = \char K$ a  $\ell \neq p$ prvočíslo. Pak:
\begin{equation*}
E[\ell^e] \cong \mathbb{Z}_{\ell^e} \times \mathbb{Z}_{\ell^e}
\end{equation*}
pro každé $e \geqslant 1$.
\end{veta}
\noindent \textit{Důkaz.} Postupujme silnou indukcí podle $e$. Isogenie $[\ell]$ je pro prvočísla $\ell \neq p$ separabilní, tedy $\# E[\ell] = \# \ker [\ell]= \ell^2$. Každý afinní prvek $E[\ell]$ má řád $\ell$, tedy platí $E[\ell] \cong \mathbb{Z}_{\ell} \times \mathbb{Z}_{\ell}$. Nyní již uvažme abelovskou grupu $E[\ell^e]$ pro nějaké $e > 1$ a předpokládejme, že věta platí pro všechna kladná $a < e$. Opět víme, že $\# E[\ell^e] = \# \ker [\ell^e] = \ell^{2e}$ a každý afinní prvek $E[\ell^e]$ nemá řád vyšší než $\ell^e$. Navíc pro každé $a < e$ existuje na $E[\ell^{e}]$ právě $\ell^{2a}$ prvků řádu $\ell^a$, tedy $E[\ell^e]$ má shodnou strukturu jako $\mathbb{Z}_{\ell^e} \times \mathbb{Z}_{\ell^e}$. \hfill $\square$\\
% Fundamentální věta konečně generovaných abelovských grup tvrdí, že $E[\ell^e]$ je vyjádřitelná jako direktní součet několika kopií $\mathbb{Z}$ a grup $\mathbb{Z}_{n_i}$ s $n_i \mid n_{i+1}$, jejichž řády musí být všechny mocniny $\ell$, protože $E[\ell^e]$ obsahuje $\ell^{2e}$ prvků. Speciálně $n_i = \ell^{a_i}$ a $a_i$ je neklesající posloupnost.

\begin{dusledek}\label{nesoudtorze}
Buď $E/K$ eliptická křivka s $p = \char K$ a $p \nmid m$ přirozené číslo. Pak $E[m] \cong \mathbb{Z}_m \times \mathbb{Z}_m$.
\end{dusledek}
\noindent \textit{Důkaz.} Pokud $m,n$ jsou nesoudělná čísla, jistě platí $E[m] \times E[n] \cong E[mn]$. Čínská zbytková věta pro taková $m,n$ tvrdí $(\mathbb{Z}_m \times \mathbb{Z}_m) \times (\mathbb{Z}_n \times \mathbb{Z}_n) \cong \mathbb{Z}_{mn} \times \mathbb{Z}_{mn}$, tedy pokud $m = p_1^{a_1} \cdots p_k ^{a_k}$ rozložíme na součin prvočíselných mocnin, s pomocí předchozí věty platí:
\begin{equation*}
E[m] \cong E[p_1^{a_1}] \times \cdots \times E[p_k^{a_k}] \cong \left(\mathbb{Z}_{p_1 ^{a_1}} \times \mathbb{Z}_{p_1 ^{a_1}}\right) \times \cdots \times \left(\mathbb{Z}_{p_k ^{a_k}} \times \mathbb{Z}_{p_k ^{a_k}}\right) \cong \mathbb{Z}_m \times \mathbb{Z}_m,
\end{equation*} což jsme chtěli dokázat. \hfill $\square$\\

Zásadní rozdíl nastává při násobení mocninou charakteristiky našeho tělesa, isogenie $[p]$ je totiž (čistě) neseparabilní. Případ $\char K = 0$ je triviální, podíváme se proto opět pouze na konečný případ.

\begin{veta}\label{soudtorze}
Buď $E/\mathbb{F}_q$ s  $q = p^k$ eliptická křivka. Pak platí:
\begin{equation*}
E[p^e] \cong  \begin{cases}
      \lbrace \mathcal{O} \rbrace, & \text{pro každé nezáporné } e, \\
      \mathbb{Z}_{p^e}, & \text{pro každé nezáporné } e.
    \end{cases}
\end{equation*}
\end{veta}
\noindent \textit{Důkaz.} Isogenie $[p]$ je neseparabilní a její jádro má tedy řád ostře nižší než $\deg [p] = p^2$. Každý prvek $E[p]$ má ale řád dělící $p$,  platí tedy buď $ E[p] \cong \lbrace \mathcal{O} \rbrace$, či $\mathbb{Z}_p$. První případ jistě znamená $E[p^e] \cong \lbrace \mathcal{O} \rbrace$ pro každé $e \geqslant 0$, nyní tedy předpokládeme $E[p] \cong \mathbb{Z}_p$.

Dále postupujme silnou indukcí podle $e \geqslant 1$, $e=0,1$ je dáno. Dále ať dané tvrzení platí pro všechna nezáporná čísla nepřevyšující $e$. Isogenie $[p]$ je surjektivní, tedy pro každé $f \leqslant e$ a $P$ bod řádu $p^f$ existuje bod $Q$ splňující $[p]Q = P$, jehož řád je $p^{f+1}$. Speciálně existuje bod $P_0 \in E[p^{e+1}]$ řádu $p^{e+1}$. Takový bod ale existuje díky $E[p^{e}] \cong \mathbb{Z}_{p^{e}}$ pouze jeden a $E[p^{e}] \cong \mathbb{Z}_{p^{e}}$. \hfill $\square$\\

Předchozí věta ukazuje, že existují dvě rodiny křivek s drasticky odlišnými $p$-torzemi. Abychom si je mohli vložit do správných přihrádek, zavedeme nové názvosloví:

\begin{definice}
Pokud máme $E[p] \cong \lbrace \mathcal{O} \rbrace $, nazveme $E$ \textit{supersingulární}. Jinak $E$ budeme říkat \textit{obyčejná}.
\end{definice}

Znalost struktury $\ell$-torzí pro $\ell$ prvočíslo nám pomůže spočítat, kolik separabilních isogenií prvočíselného stupně vychází z dané křivky. K tomu si nejprve pochopitelně musíme spočítat podgrupy na $E$ řádu $\ell$. Ty musí být generované bodem řádu $\ell$, tedy celá pogrupa leží v $E[\ell]$. Přirozeně tedy chceme spočítat podgrupy $\ell$-torze řádu $\ell$. Případ $\ell = p$ dává buď žádnou či jednu podgrupu, v závislosti na supersingularitě křivky, dále tento případ neuvažujme.
\begin{lemma}\label{bigl+1}
Buď $E/\mathbb{F}_{q}$ křivka s $q = p^k$ a $\ell \neq p$ prvočíslo. Pak $E[\ell ^e]$ obsahuje právě $\ell^{e-1} (\ell+1)$ podgrup řádu $\ell^e$.
\end{lemma}
\noindent \textit{Důkaz.} Díky větě \ref{prvotorze} platí $E[\ell^e] \cong \mathbb{Z}_{\ell ^e} \times \mathbb{Z}_{\ell ^e}$, mějme $P,Q$ její generátory. Každá podgrupa $E[\ell ^e]$ řádu $\ell^e$ je cyklická generovaná prvkem $R = [a]P+[b]Q$. Ten musí mít řád $\ell^e$, tedy právě jeden z $a,b$ není dělitelný $\ell^e$. Počet takových $a$ je pak roven $\ell^{e-1} (\ell -1)$ a počet $b$ je $\ell^e$, dohromady dávají $\ell^{2e-1} (\ell -1)$ možných bodů. Započítali jsme ale případy, kdy oba mají řád $\ell^e$ dvakrát, takových případů je $(\ell^{e-1} (\ell - 1))^2$. Konečně, zde počítáme každou podgrupu $\ell^{e-1}(\ell-1)$-krát, jednou pro každý její bod řádu $\ell^e$, tedy hledaný počet je roven:
\begin{equation*}
\frac{2 \ell^{2e-1} (\ell - 1) - (\ell^{e-1} (\ell - 1))^2}{\ell^{e-1}(\ell-1)} = \ell^{e-1} (\ell+1).
\end{equation*}
 \hfill $\square$\\
%Řád každého prvku dělí počet prvků na $E[\ell]$, což je $\ell^2$, neboli

\begin{dusledek}\label{l+1}
Buď $E/\mathbb{F}_{q}$ křivka s $q = p^k$ a $\ell \neq p$ prvočíslo. Pak existuje přesně $\ell+1$ až na isomorfismus různých separabilních isogenií stupně $\ell$ vycházejících z $E$ definovaných nad $\overline{\mathbb{F}}_{q}$.
\end{dusledek}
\noindent \textit{Důkaz.} Podle věty \ref{isomor} je počet separabilních isogenií vycházejících z $E$ stupně $\ell$ dán počtem podgrup $E$ řádu $\ell$. Všechny takové grupy musí být obsaženy v $E[\ell]$ a předchozí lemma pak tvrdí, že hledaný počet je právě $\ell+1$. \hfill $\square$\\

Jak jsme zmínili před chvíli, pro nesoudělná $m,n$ platí $E[m] \times E[n] \cong E[mn]$, tedy pomocí před chvílí zmíněného páru vět jsme schopni kompletně charakterizovat libovolnou torzní podgrupu $E$. Speciálně toho můžeme říci mnoho o samotné grupě bodů nad konečným tělesem $E(\mathbb{F}_q)$:

\begin{veta}
Buď $E/\mathbb{F}_q$ eliptická křivka s $q = p^k$. Pak:
\begin{equation*}
E(\mathbb{F}_q) \cong \mathbb{Z}_m \times \mathbb{Z}_n
\end{equation*}
pro $p \nmid m \mid n$ přirozená čísla.
\end{veta}
\noindent \textit{Důkaz.} Pokud $p$ nedělí řád $E(\mathbb{F}_q)$, který označme $m$, pak $E(\mathbb{F}_q) \subseteq E[m] \cong \mathbb{Z}_m \times \mathbb{Z}_m$ je podgrupa řádu nejvýše $2$, lze ji proto zapsat jako direktní součin $\mathbb{Z}_m \times \mathbb{Z}_n$ s $m \mid n$ a~$p \nmid mn$, kde umožňujeme případ $m=1$. Jinak existuje podgrupa $G \subseteq E(\mathbb{F}_q)$ řádu nejvyšší mocniny $p$, kde $E(\mathbb{F}_q) \cong G \times H$ a~$H \cong \mathbb{Z}_m \times \mathbb{Z}_m$ nemá řád dělitelný $p$. Grupu $E(\mathbb{F}_q)$ tedy můžeme zapsat jako direktní součin nejvýše dvou cyklických grup a pouze jedna z nich má řád dělitelný $p$. \hfill $\square$\\

Působení isogenie na libovolnou $m$-torzi či samotnou grupu $E(\mathbb{F}_q)$ je jednoznačně určeno jejím chováním na (nejvýše dvou) generátorech těchto grup. Isogenie jsou totiž homomorfismy grup bodů na křivkách, pro příslušné generátory $G_1,G_2$ a bod $P = [m] G_1 + [n] G_2$ platí:
\begin{equation*}
\phi([m]G_1+[n]G_2) = [m] \phi (G_1 )+ [n] \phi (G_2).
\end{equation*}
Isogenie na dané křivce, tedy $\phi : E \longrightarrow E$, působí na $E(\mathbb{F}_q)$ i na její torzní podgrupy jako $2 \times 2$ celočíselné matice, v~případě $m$-torzní grupy dokonce jako matice modulo $m$. Jak se chovají takové isogenie na torzích budeme podrobněji studovat ve čtvrté kapitole.

Před chvíli jsme ale eliptické křivky rozlišily na dvě třídy podle jejich $p$-torze. Ty \uv{neobyčejné} z nich, supersingulární, jsou více než zajímavé.

\section{Supersingulární křivky}

Slovo supersingulární napovídá, že na křivky takto pojmenované nenarazíme příliš často, že jsou mezi všemi eliptickými křivkami vzácné. Tato malá větev křivek se od obyčejných fundamentálně liší, přičemž jejich četné rozdíly jsou spolu mnohdy těsně provázané. Ve~skutečnosti se některé vlastnosti, o kterých se zmíníme, berou jako ekvivalentní definice supersingularity, každá vhodná v jistém úhlu pohledu. Jejich vlastnosti ve všech směrech, které jsme prozatím studovali, dopodrobna prozkoumáme, počínaje definicí pomocí torze.

Počítání celé $p$-torze je pro velká prvočísla výpočetně náročné, chtěli bychom najít vhodnější kritéria supersingularity. Ukáže se, že supersingulární eliptické křivky nesou pouze specifické počty bodů.

\begin{veta}\label{super}
Nechť $E$ je křivka nad $\mathbb{F}_q$, kde $q = p^r$ je mocnina prvočísla $p > 3$. Pak: $$\# E(\mathbb{F}_q) \equiv 1 \pmod{p}$$ nastane právě pokud $E$ je supersingulární.
\end{veta}

\noindent \textit{Důkaz.} Věta \ref{dual} říká:
\begin{equation*}
[\deg([1]-\pi)] = ([1]-\pi) \circ \widehat{([1]-\pi)} = ([1]-\pi) \circ (\widehat{[1]}-\widehat{\pi}) = ([1]-\pi) \circ ([1]-\widehat{\pi}),
\end{equation*} 
neboli, protože isogenie jsou homomorfismy grup, isogenie:
\begin{equation*}
\pi+\widehat{\pi} = [1] - [\deg([1]-\pi)]+\pi \circ \widehat{\pi} = [1]-[\deg([1]-\pi)]+[p]
\end{equation*}
působí jako skalární násobení na $E$. Isogenie $[1]-\pi = [1] - \pi_p ^r$ má jádro $E(\mathbb{F}_q)$, protože tato množina je pod Frobeniovým endomorfismem invariantní. Navíc $-\pi$ je neseparabilní a $[1]$ zase separabilní, tedy věta \ref{separsum} tvrdí, že $[1]-\pi$ je isogenií separabilní se stupněm rovným velikosti jádra, $\#E(\mathbb{F}_q)$. Pak tedy platí:
\begin{equation*}
\pi+\widehat{\pi}  = [1]-[\deg([1]-\pi)]+[p] = [1-\deg ([1]-\pi)+p] =  [p+1-\#E(\mathbb{F}_q)].
\end{equation*}
Pokud $E$ je supersingulární, je $\ker \pi \circ \widehat{\pi} = \ker [p] \cong \lbrace \mathcal{O} \rbrace$, neboli $\widehat{\pi}$ má triviální jádro a~je neseparabilní. Podle věty \ref{separsum} je $\pi+\widehat{\pi}$ neseparabilní, $[p+1-\#E(\mathbb{F}_q)]$ je proto nesparabilní též.  Konečně, díky lemmatu \ref{nasobsepar} $p$ dělí $p+1-\#E(\mathbb{F}_q)$.

Naopak pokud platí $E(\mathbb{F}_q) \equiv 1 \pmod{p}$, isogenie: $$\pi+\widehat{\pi} = [p+1-\#E(\mathbb{F}_q)]$$ je neseparabilní. Víme, že $\pi$ je neseparabilní isogenie a $\pi+\widehat{\pi}$ taky, opět utilizujeme větu \ref{separsum}, dle které i $\widehat{\pi}$ není separabilní. Protože stupeň $\widehat{\pi}$ je prvočíselný, $\widehat{\pi}$ má nutně triviální jádro, kompozice $[p] = \widehat{\pi} \circ \pi$ jej proto má též a $E$ je supersingulární. \hfill $\square$\\

\begin{poznamka}
Fakt, že $\phi+\widehat{\phi}$ je rovno skalární isogenii $[m]_E$ pro nějaké $m$ zřejmě není unikátní pro Frobeniův endomorfismus. Stejný postup můžeme replikovat pro každou jinou isogenii $E \longrightarrow E$. My si však tento fakt \uv{připomeneme} na vhodnějším místě ve čtvrté kapitole. 
\end{poznamka}

\begin{poznamka}
Pozorování, že $\pi+\widehat{\pi} = [p+1-\#E(\mathbb{F}_q)]$ a že isogenie $\phi : E \longrightarrow E$ působí na torzní grupy jako $2 \times 2$ matice, navádí na důkaz Hasseho věty s pomocí znalostí, které nyní máme, spolu s trochu hlubším studiem působení isogenií $\phi : E \longrightarrow E$ na torzní podgrupy. Naznačme jej tu rychle, plný důkaz se nachází na \cite[Thm. 8.1, Thm. 7.17]{Sutherland}. Pokud $M$ je $2 \times 2$ matice udávající akci $\pi$ na nějakou fixní torzi $E[n]$, pro libovolná celá $r,s$ lze fakt $\deg ([r] \circ \pi -[s]) \geqslant 0$ pro dostatečně velké $n$ převést na nezápornost determinantu matice $r M - I s$, což lze upravit na nezápornost kvadratického polynomu. Konečně se ukáže, že nekladnost jeho diskriminantu je jen jiná forma Hasseho věty.
\end{poznamka}

\begin{dusledek}\label{super2}
Ať $E$ je křivka nad $\mathbb{F}_p$ s $p > 3$. Pak: $$\# E(\mathbb{F}_p) = p+1$$  nastane, právě pokud $E$ je supersingulární.
\end{dusledek}

\noindent \textit{Důkaz.} Pokud $\# E(\mathbb{F}_p) = p+1$, tak dle předchozí věty je $E$ supersingulární. Pro $E$ supersingulární je $\# E(\mathbb{F}_p) \equiv 1 \pmod{p}$, tedy jestli $\# E(\mathbb{F}_p) \neq p+1$, je číslo $p+1 - \# E(\mathbb{F}_p)$ v~absolutní hodnotě alespoň $p$. Dle Hasseho věty \ref{Hasse}, kterou a priori bereme za platnou, toto číslo v absolutní hodnotě nepřesahuje $2\sqrt{p}$, neboli:
\begin{equation*}
2\sqrt{p} \geqslant \vert p+1 - \# E(\mathbb{F}_p)\vert \geqslant p,
\end{equation*}
což je spor s $p > 3$. \hfill $\square$\\

Při zkoumání počtu bodů na supersingulárních křivek jsme narazili na číslo $t = q+1 - \# E(\mathbb{F}_q)$, které je úzce spojené s Frobeniovým endomorfismem. Tento pár spolu rozhodně nevidíme naposledy, kapitola zaměřena na okruhy endomorfismů jejich pouto prohloubí.

Samotné počítání bodů na eliptické křivce je pro nás zatím obtížný úkon, pro $\mathbb{F}_p$ s malým $p$ můžeme jednoduše projít všechny možné hodnoty $x$, jak můžeme vidět na následujícím příkladu:
\begin{priklad}
Ukažme, že křivka:
$$E: y^2 = x^3+10x+7$$
nad $\mathbb{F}_{13}$ je supersingulární.
\end{priklad}
\noindent \textit{Řešení.} Mějme $(x,y) \in E(\mathbb{F}_{13})$. Pokud je číslo $x^3+10x+7$ v $\mathbb{F}_{13}$ nenulový čtverec, existují dvě vyhovující $y$, jedno, pokud je rovno nule, a~jinak žádné. Můžeme si proto vypsat hodnoty pravé strany ve všech možných hodnotách a za pomocí Eulerova kritéria snadno určit, zda je výraz čtvercem, viz následující tabulka:

\begin{longtable}[H]{l>{\centering}p{0.2\linewidth}>{\raggedleft}p{0.1\linewidth}>{\centering\arraybackslash}p{0.2\linewidth}}
\toprule
$x$ & $x^3+10x+7$ & $\genfrac{(}{)}{}{}{x^3+10x+7}{13}$ & počet řešení\\
\midrule
$0$ & $7$  & $-1\quad\;$  & $0$\\
$1$ & $5$  & $-1\quad\;$  & $0$\\
$2$ & $9$  & $1\quad\;$  & $2$\\
$3$ & $12$  & $1\quad\;$  & $2$\\
$4$ & $7$  & $-1\quad\;$  & $0$\\
$5$ & $0$  & $0\quad\;$  & $1$\\
$6$ & $10$  & $1\quad\;$  & $2$\\
$7$ & $4$  & $1\quad\;$  & $2$\\
$8$ & $1$  & $1\quad\;$  & $2$\\
$9$ & $7$  & $-1\quad\;$  & $0$\\
$10$ & $2$  & $-1\quad\;$  & $0$\\
$11$ & $5$  & $-1\quad\;$  & $0$\\
$12$ & $9$  & $1\quad\;$  & $2$\\
\bottomrule 
\end{longtable}


%\begin{center}
%\begin{table}[ht]
%\centering
%\begin{tabular}[t]{l>{\centering}p{0.2\linewidth}>{\centering}p{0.2\linewidth}>{\centering%\arraybackslash}p{0.2\linewidth}}
%\toprule
%$x$ & $x^3+10x+7$ & $\genfrac{(}{)}{}{}{x^3+10x+7}{13}$ & počet řešení\\
%\midrule
%$0$ & $7$  & $-1$  & $0$\\
%$1$ & $5$  & $-1$  & $0$\\
%$2$ & $9$  & $1$  & $2$\\
%$3$ & $12$  & $1$  & $2$\\
%$4$ & $7$  & $-1$  & $0$\\
%$5$ & $0$  & $0$  & $1$\\
%$6$ & $10$  & $1$  & $2$\\
%$7$ & $4$  & $1$  & $2$\\
%$8$ & $1$  & $1$  & $2$\\
%$9$ & $7$  & $-1$  & $0$\\
%$10$ & $2$  & $-1$  & $0$\\
%$11$ & $5$  & $-1$  & $0$\\
%$12$ & $9$  & $1$  & $2$\\
%\bottomrule
%\end{tabular}
%\end{table}
%\end{center}
Spolu s bodem v nekonečnu je $\# E(\mathbb{F}_{13})=13+1=14$ a jsme hotovi z důsledku \ref{super2}. \hfill $\square$\\

U speciálních případů křivek můžeme rafinovaně využít poznatky z elementární teorie čísel:

\begin{priklad}
Ukažme, že křivka:
 $$E/\mathbb{F}_p : y^2 = x^3 + kx$$
pro $ p \equiv -1 \pmod{4}$ je supersingulární.
\end{priklad}
\noindent \textit{Řešení.} Pro $p \equiv -1 \pmod{4}$ je $\genfrac{(}{)}{}{}{-1}{p} = -1$, takže pokud pro $a,b$ platí $p \mid a^2 + b^2$, jsou obě dělitelná $p$. V opačném případě totiž z $a^2 \equiv -b^2 \pmod{p}$ vyvodíme:
\begin{equation*}
\left(\frac{a}{b}\right)^2 \equiv -1 \pmod{p},
\end{equation*} 
spor. Nenulových čtverců v $\mathbb{F}_p$ je právě $\frac{p-1}{2}$, tudíž každý prvek $\mathbb{F}_p$ je buď čtverec, nebo mínus čtverec. Pro $x= 0$ máme pouze $y = 0$ a pro každé $x \in \mathbb{F}_p^{\times}$ je právě jedno z čísel $x^3+kx, (-x)^3-kx$ nenulovým čtvercem, protože je $x^2 \neq -1$. Pro každou dvojici $(x,-x)$ tak máme právě dvě řešení, dohromady $p-1$. Spolu s $(0,0)$ a bodem v nekonečnu je $\# E(\mathbb{F}_p) = p+1$, díky větě \ref{super2} je $E$ supersingulární. \hfill $\square$\\

\begin{priklad}
Ukažme, že křivka:
 $$E/\mathbb{F}_p : y^2 = x^3 + k$$
pro $ p \equiv -1 \pmod{3}$ je supersingulární.
\end{priklad}
\noindent \textit{Důkaz.} Ukážeme, že třetí mocnina je na $\mathbb{F}_p$ bijekcí. Pokud totiž pro $x \neq y$ platí $x^3 \equiv y^3 \pmod{p}$, tak:
\begin{equation*}
p \mid (x-y)(x^2+xy+y^2) \Rightarrow p \mid x^2+xy+y^2 
\end{equation*}
Ukážeme, že pak už $p \mid x,y$, v opačném případě $p$ nedělí ani jedno. Poslední rovnost pak vynásobíme čtyřmi a máme:
\begin{equation*}
p \mid (x+2y)^2 + 3 x^2 \Rightarrow \left(\frac{x+2y}{x} \right)^2 \equiv -3 \pmod{p}.
\end{equation*}

Pro $p \equiv -1 \pmod{3}$ je ale $-3$ kvadratický nezbytek, opět získáváme spor. Pro každé $y \in \mathbb{F}_p$ tedy existuje unikátní třetí odmocnina z $y^2 - k$ dávající bod $(x,y) \in E$. Dohromady máme na $E$ přesně $p$ afinních bodů a ten poslední samozřejmě leží v nekonečnu. \hfill $\square$\\

Protože supersingularita nezávisí na konkrétním rozšíření, křivky výše jsou supersingulární nad libovolným konečným tělesem s~charakteristikou po řadě $p \equiv -1 \pmod{4}$, resp. $p \equiv -1 \pmod{3}$.

Náš první postup počítání počtu bodů na křivce běží nejlépe v $O(p)$ čase, což je pro prvočísla  $\log_2(p) > 500$, tedy praktické kryptografické velikosti, jednoduše příliš pomalé. Jedním z~nejdřívějších velkých pokroků v oblasti počítání bodů byl \textit{Schoofův algoritmus}, zveřejněn roku 1985 v~\cite{Schoof}, který $\# E(\mathbb{F}_q)$ jako první dokáže spočíst deterministicky v čase polynomiálním v $\log(q)$. Poskytuje tedy exponenciální zrychlení oproti našemu předchozímu postupu.

Pojďme se podívat na samotnou strukturu bodů na supersingulární $E$ nad konečným tělesem. Ústřední při našem studiu isogenií je fakt, že supersingularita je pod působením isogenie zachována. 

\begin{veta}
Buď $E/\mathbb{F}_q$ eliptická křivka s $q = p^k$ a $\phi : E \longrightarrow E^\prime$ libovolná isogenie vycházející z $E$. Pak $E$ je supersingulární, právě pokud je $E^\prime$ supersingulární.
\end{veta}

\noindent \textit{Důkaz}. Mějme $\phi : E \longrightarrow E^\prime$ isogenii. Protože isogenie jsou homomorfismy grup bodů na křivkách nad $\mathbb{F}_q$, speciálně zachovají $p$-násobení:
\begin{equation*}
\phi  \circ[p]_E = [p]_{E^\prime}
\end{equation*}
a analogická rovnost platí pro duální isogenii. Pokud na $p$-torzi jedné z křivek existuje netriviální bod, tak nějaký leží v $p$-torzi i druhé křivky, tedy pokud jedna z křivek je obyčejná, obě jsou. Naopak pokud $p$ torze na $E$ triviální, díky $[p]_{E^\prime} = \phi \circ [p]_E$ je i~$E^\prime [p] \cong \lbrace \mathcal{O} \rbrace$ a~samozřejmě i naopak. \hfill $\square$\\

Speciálně toto tvrzení platí pro isomorfismy, každý $j$-invariant je proto exklusivní buď obyčejným, či supersingulárním křivkách, můžeme tedy $j$-invarianty rozřadit na obyčejné nebo supersingulární podle typu křivek jej sdílejících.

Pokud uvážíme graf všech $j$-invariantů nad $\overline{\mathbb{F}}_p$ (kterým přiřadíme jejich příslušnou třídu isomorfismů), kde dva vrcholy jsou propojené, právě pokud křivky jim náležící jsou isogenní pod isogenií prvočíselného stupně $\ell$, získáme neorientovaný(!) $\ell+1$-regulární (díky větě \ref{l+1}) graf rozdělený na obyčejné a supersingulární komponenty. Supersingulární křivky dokonce tvoří jednu jedinou souvislou komponentu, viz \cite[Cor. 78]{Kohel}. Ve čtvrté kapitole budeme tyto grafy studovat trochu podrobněji studovat a ukážeme, že pokud se zaměříme na isogenie definované pouze nad $\mathbb{F}_q$, grafy supersingulárních $j$-invariantů se zásadně liší od grafů těch obyčejných. Z každého vrcholu totiž vždy vede buď $0,1,2$ či $\ell+1$ hran, přičemž supersingulární komponenty jsou pořád $\ell+1$ regulární, zatímco komponenty obyčejné tvoří tzv. \textit{vulkány}, kde regulární graf stupně nejvýše $2$ slouží jako \uv{kráter} a každý jiný vrchol je buď listem, či má $\ell+1$ sousedů.

Na grafu isogenií nad $\mathbb{F}_{19}$ se nachází pouze $2$ supersingulární vrcholy, což potvrzuje fakt, že tyto křivky se nachází v menšině. Tento trend se drží i pro vyšší rozšíření, nad celým uzávěrem $\mathbb{F}_p$ se nachází relativně málo supersingulárních $j$-invariantů.

\begin{veta}
Označme $S$ množinu všech supersingulárních $j$-invariantů nad $\overline{\mathbb{F}}_{p}$. Pak platí:
\begin{equation*}
\# S = \left\lfloor \frac{p}{12} \right\rfloor + \begin{cases}
 0, \quad \text{pokud} \quad p \equiv 1 \pmod{12},\\
 1, \quad \text{pokud} \quad p \equiv 5,7 \pmod{12},\\
 2, \quad \text{pokud} \quad p \equiv 11 \pmod{12}.
 \end{cases}
\end{equation*}
\end{veta}

Důkaz se nachází na \cite[Cor. 4.40]{Washington}.

Předchozí věta nám říká, že když se budeme přesouvat do vyšších rozšíření tělesa $\mathbb{F}_p$, nenarazíme na další a další supersingulární $j$-invarianty. Dokonce se zastavíme už na $\mathbb{F}_{p^2}$:

\begin{veta}
Buď $E$ supersingulární eliptická křivka nad $\mathbb{F}_q$. Pak $j(E) \in \mathbb{F}_{p^2}$.
\end{veta}
\noindent \textit{Důkaz.} Isogenie $[p]$ na supersingulární křivce $E$ je neseparabilní s triviálním jádrem a~stupněm $p^2$. Podle věty \ref{separ2} je pak rovna složení dvou kopii Frobeniova endomorfismu s isomorfismem, $[p] = \iota \circ \pi ^2$. Isogenie $\pi^2$ zobrazuje:
\begin{equation*}
\pi^2 \quad : \quad E: y^2 = x^3+ax+b  \quad \longrightarrow \quad E^\prime : y^2 = x^3 + a^{p^2}x + b^{p^2},
\end{equation*}
tyto dvě křivky jsou proto isomorfní pod $\iota$. Díky vlastnostem charakteristiky:
\begin{align*}
 j(E) = j(E^\prime) = 1728 \frac{4 a^{3 p^2}}{4 a^{3 p^2} + 27 b^{2 p^2}} = \left( 1728 \frac{4 a^3}{4a^3+27b^2} \right)^{p^2} = j(E)^{p^2},
\end{align*}
$j$-invariant naší křivky je tedy fixovaný automorfismem $x^{p^2} = x$ na $\overline{\mathbb{F}}_q$ a leží tak v $\mathbb{F}_{p^2}$. \hfill $\square$\\

\begin{dusledek}\label{Fp2}
Buď $E/\mathbb{F}_q$ supersingulární křivka. Pak existuje supersingulární $E^\prime/\mathbb{F}_{p^2}$, která je s $E$ isomorfní.
\end{dusledek}
\noindent \textit{Důkaz.} Protože $j := j(E)$ leží v $\mathbb{F}_{p^2}$, příklad vyhovující křivky nad $\mathbb{F}_{p^2}$ pro $j \neq 0,1728$ dává křivka $E^\prime : y^2 = x^3 +3j(1728-j)x + 2j(1728-j)^2$ s $j(E^\prime) = j$, viz věta \ref{jjjj}, a~případy $j = 0,1728$ jsou zřejmé. \hfill $\square$\\

Při uvažování grafů supersingulárních $j$-invariantů se zajímáme vesměs pouze na třídy isomorfismů, postačí nám tedy uvažovat všechny křivky pouze nad $\mathbb{F}_{p^2}$.

\begin{znaceni}
Buďte $p, \ell$ prvočísla. Graf supersingulárních $j$-invariantů nad $\overline{\mathbb{F}}_p$ spojené isogeniemi stupně $\ell$ značme $G_{\ell} (\overline{\mathbb{F}}_p)$.
\end{znaceni}


\chapter{Uplatnění v kryptografii}

Přes Caesarovu šifru až po šifrování za pomocí Enigmy v období druhé světové války, po většinu lidské historie se využívaly kryptografické systémy založené na faktu, že obě komunikující partie si po domluvě vyberou způsob maskování zprávy a ten pro ostatní zůstává skrytý. Příkladem je právě o kolik písmen v Caesarově šifře transponujeme. Tento způsob nutně závisí na faktu, že se obě strany před výměnou mají možnost přes zabezpečný kanál na tomto způsobu domluvit. S přibývajícím počtem účastníků a~frekvencí komunikace, na příklad našeho každodenního interagování na internetu, kde musí konverzace mezi všemi účastníky být bezpečná, je bohužel na úkor ceny přenosu třeba vyšší počet a velikost klíčů, a přibývá riziko kompromitace.

Kvůli takovým obavám přišli Whitfield Diffie a Martin Hellman \cite{Diffie} roku 1976 s revolučním nápadem: asymetrickou kryptografií, kde každý z účastníků má svůj vlastní \textit{privátní klíč}, který s nikým nesdílí. Všechny strany, i potenciální útočník, znají několik informací, které jsou známé jako \textit{veřejné parametry}. Obě komunikující strany za pomocí veřejných informací tajně transformují svůj privátní klíč a výsledek, který budeme nazývat \textit{veřejným klíčem}, publikují. Oba účastníci vezmou veřejný klíč toho druhého a provedou s~ním ty samé tajné kroky závisící na jejich privátním klíči. Podstatou takové výměny je, že na jejím konci získají obě původní strany shodnou netriviální informaci, tedy informaci takovou, že žádná třetí strana ji nedokáže snadno uhodnout, za pomocí níž poté mohou společnou komunikaci šifrovat a nikdo jiný již jejich zprávy neuvidí. Předpokládá se, že pouze ze znalosti veřejného klíče je pro každou další partii těžké replikovat klíč privátní a že pole možných sdílených informací je obrovské. Vyhneme se tak přímočarým řešením hrubou silou.

Pojďme se podívat na protokol, který Diffie a Hellman navrhli. Budeme o něm dále mluvit jako o \textit{Diffie-Hellmanově výměně}. Ta je založena na problému \textit{diskrétního logaritmu} prvku $a \in \mathbb{Z}_p^{*}$, který po nás ze znalosti primitivního prvku $g$ modulo $p$ žádá najít $k$ splňující $g^k = a$ v $\mathbb{Z}_p$. Obecně můžeme $\mathbb{Z}_p$ nahradit cyklickou grupou $G$, kde $g$ je její libovolný generátor. Protokol požaduje, aby nebyl diskrétní logaritmus na $G$ spočitatelný efektivně, tj. v~polynomiálním čase vzhledem k velikosti grupy, jinak může útočník jednoduše privátní klíče obou stran spočíst, ale mocnění rychlé bylo. Umocnit číslo dokážeme v logaritmickém čase, a v konečné grupě nám stačí umocnit pouze na exponent modulo řádu grupy.

\begin{poznamka}
Polynomiální čas je zde pojem mírně zavádějící, nepožadujeme totiž algoritmus polynomiální v $\vert G \vert$, ale v $\log \vert G \vert$, totiž v jednotce místa, které $G$ zabere při zápisu. Jednoduché procházení všech mocnin $g$ proto běží v (očekávaném) exponenciálním čase. 
\end{poznamka}

%\footnote{Funkce, která je na každém vstupu efektivně spočitatelná, ale není efektivně invertibilní, se nazývá \textit{jednosměrná}. Existence takové funkce by byla velkým pokrokem pro kryptografie, bohužel žádná taková nebyla nalezena}
\begin{figure}[h]
\begin{center} 
\makebox[1cm]{\rule{15cm}{0.4pt}}\\
\hspace{-1.35cm} \textbf{Veřejné parametry:} Grupa $G$ řádu $p$, kde $p$ je prvočíslo, s generátorem $g$.\\

\vspace{-0.25cm}
\makebox[\linewidth]{\rule{15cm}{0.4pt}}\\
\vspace{0.2cm}
\begin{tabular}{l l c l}
\cline{2-2} \cline{4-4} 
& Alfréd & & Blažena \\ 
\cline{2-2} \cline{4-4} 
& \textbf{Vstup:} $a \in G^{*}$ & & \textbf{Vstup:} $b \in G^{*}$ \\
 & & $ \xlongrightarrow{\qquad \qquad  g^a \qquad \qquad}$ &  \\
&  & $\xlongleftarrow{\qquad \qquad  g^b \qquad \qquad} $ &  \\
& $G_{AB} := \left(g^b\right)^{a} = g^{ab}$ &  & $ G_{BA} := \big(g^a\big)^{b} = g^{ba}$ \\
& \textbf{Výstup:} $G_{AB}$ & & \textbf{Výstup:} $G_{BA}$
\end{tabular}
\caption*{Algoritmus 1: Diffie-Hellmanova výměna}
\vspace{-0.3cm}
\end{center}
\end{figure}

Díky předpokladu, že $G$ je cyklická, je i abelovská, tedy $G_{AB} = g^{ab} = g^{ba} = G_{BA}$.
\begin{figure}[h]
\begin{center} 
\begin{tikzcd}
 & g \arrow[dr, red, "a" red]\arrow[dl, blue, "b" blue] &\\
g^b \arrow[dr, blue, "g^a" red] & & g^a \arrow[dl, red, "g^b" blue]\\
& g^{ab} &
\end{tikzcd}
\end{center}
\end{figure}

Řád $G$ se prakticky bere prvočíslo $q = 2p+1$ takové, že $p$ je prvočíslo, pak $p$ nazveme tzv. \textit{Sophie-Germainovým prvočíslem} a $q$ zase \textit{bezpečným prvočíslem}. V takovém případě má $G$ podgrupu (velkého) prvočíselného řádu $p$, což je z kryptografického hlediska žádané, problém počítání diskrétního logaritmu totiž lze užitím Pohlig-Hellmanova algoritmu převést na počítání diskrétního logaritmu jeho podgrup. Navíc bezpečná prvočísla skýtají i výhody pro inicializování výměny, pro taková prvočísla dokážeme totiž snadno nalezneme primitivní kořen v $\mathbb{Z}_q$. Konkrétně, je-li $g$ primitivní kořen modulo $q$, má řád $q-1 = 2p$ modulo $q$, právě pokud $g^{p}$ i $g^2$ nedávají zbytek $1 \pmod{q}$. Najít $g^{p} \pmod{q}$ nám mohou usnadnit nástroje jako Eulerovo kritérium, díky kterému je postačující mít $g$ kvadratický nezbytek modulo $q$.

Veřejné klíče $g^a,g^b$, jsou nicméně, jak jejich název napovídá, veřejné, a má k nim přístup libovolná jiná osoba. Dejme tomu, že Eva, která má přístup pouze k veřejně dostupným informacím $G,g,g^a,g^b$, by chtěla též znát sdílené tajemství. Jeden ze způsobů, jak by mohla tajnou informací získat, spočívá ve výpočtu diskrétního logaritmu $\log_g(g^a) = a$, nicméně předpokládáme, že to je obtížné. Na klasických počítačích jsou nejlepší známé útoky na problémy, jako diskrétní logaritmus a faktorizace čísla, subexponenciální, nicméně na počítačích kvantových jsou už od poloviny $90$. let známé algoritmy polynomiální. V čem však takto podstatné zrychlení spočívá?


\section{Kvantové počítače}

\begin{center}
\begin{verse}
\setverselinenums{1}{3}
\qquad \textit{If computers that you build are quantum,}\\
\qquad \textit{Then spies of all factions will want 'em.}\\
\qquad \textit{Our codes will all fail,}\\
\qquad \textit{And they'll read our email,}\\
\qquad \textit{Till we've crypto that's quantum, and daunt 'em. }
\end{verse}
\hfill \textit{Jennifer a Peter Shorovi}
\end{center}

Klasický počítač operuje se základní jednotkou bit, což je diskrétní logický stav nabývající hodnot $0$ a $1$. Na $n$-bitovém počítači tak můžeme mít $2^n$ různých kombinací těchto stavů a v daný moment je počítač v právě jedné z těchto kombinací.

Ve světě kvantové mechaniky místo s klasickými bity pracujeme s \textit{qubity}. Qubit narozdíl od jeho tradičního protějšku nenabývá pouze hodnot $0$ a $1$, ale jejich jednotkové lineární kombinace. Přesněji řečeno, stavy $0$ a $1$ můžeme ztotožnit s vektory $\vert 0 \rangle = \begin{bmatrix}
1\\
0
\end{bmatrix}$, resp. $\vert 1 \rangle = \begin{bmatrix}
0\\
1
\end{bmatrix}$ ve vektorovém prostoru $\mathbb{C}^2$, qubit je pak libovolná lineární kombinace těchto vektorů s jednotkovou normu, tj. $\alpha \vert 0 \rangle + \beta \vert 1 \rangle$ s $\vert \alpha \vert ^2 + \vert \beta \vert ^2 =1$. Podobně můžeme uvažovat systém $n$ qubitů jako množinu všech vektorů normy jedna náležících součinu vektorových prostorů $(\mathbb{C} \times \mathbb{C})^n = \mathbb{C}^{2n}$. Systém dvou qubitů tedy můžeme uvážit jako množinu všech lineárních kombinací ortogonálních vektorů $\vert 00 \rangle = \begin{bmatrix}
1\\
0\\
0\\
0
\end{bmatrix}$, $\vert 01 \rangle = \begin{bmatrix}
0\\
1\\
0\\
0
\end{bmatrix}$, $\vert 10 \rangle = \begin{bmatrix}
0\\
0\\
1\\
0
\end{bmatrix}$, $\vert 11 \rangle = \begin{bmatrix}
0\\
0\\
0\\
1
\end{bmatrix}$ v~$\mathbb{C}^4$, kde součet druhých mocnin absolutních hodnot koeficientů bude roven jedné, takový stav nazveme tzv. \textit{kvantovou superpozici} vektorů $\vert 00 \rangle, \vert 01 \rangle, \vert 10 \rangle, \vert 11 \rangle$. V obecnosti stavy systému s $n$ qubity tvoří tzv. $2^n$-dimenzionální \textit{Hilbertův prostor}, jehož bázi tvoří klasické $n$-bity, viz \cite{Griffiths}.

Důležitou stránkou práce s kvantovým počítačem je, že nemůžeme jen tak kdykoliv zjistit stav našeho systému. Konkrétně uvážíme-li qubit ve stavu $ \alpha \vert 0 \rangle + \beta \vert 1 \rangle$, nedokážeme zjistit hodnoty $\alpha$ a $\beta$, pokud bychom je předtím neznali, jinak řečeno, náš systém nemůžeme v~libovolný čas přímo pozorovat. Místo toho musíme provést \textit{pozorování}, které nám podá informaci ohledně systému v podobě klasického bitu. Po pozorování však již nemůžeme dále pracovat s původním stavem, celý se při pozorování totiž kolabuje do pozorovaného klasického stavu. Jak? V případě pozorování stavu $\alpha \vert 0 \rangle + \beta \vert 1 \rangle$ získáme s pravděpodobností $\vert \alpha \vert ^2$ hodnotu $0$ a s pravděpodobností $\vert b \vert ^2$ hodnotu $1$. Obdobně koeficienty kvantového stavu určují pravděpodobnost, s jakou při pozorování získáme právě takový klasický stav. To je též důvod k normalizační podmínce $\vert \alpha \vert ^2 + \vert \beta \vert ^2 = 1$, součet všech pravděpodobností musí být roven $1$. Kvantové algoritmy proto nestudujeme pouze z pohledu časové složitosti, ale i se snahou získat správný výsledek s co nejvyšší pravděpodobností. 

Pojďme se nyní pobavit o tom, jak můžeme se systémem qubitů manipulovat. Systém našich qubitů můžeme samozřejmě vyjádřit ve vektorové podobě, přičemž tento vektor je díky normalizační podmínce jednotkový. V klasických obvodech jsou bity ovlivňovány \textit{branami}, které naše bity (ve vektorové podobě) transformují v jiné bity a zachovají normu příslušných vektorů, tj. jednotkové matice. Ku příkladu brána NOT bere pro vektory nad vektorovým prostorem dimenze $2$ nad $\mathbb{Z}^2$ formu $\begin{pmatrix}
0 & 1\\
1 & 0
\end{pmatrix}$ a brána OR je zase reprezentována maticí $\begin{pmatrix}
1 & 1\\
1 & 0
\end{pmatrix}$. Kvantové brány jsou přirozeným zobecněním těch klasických, tedy v případě dvojqubitového systému jsou to právě jednotkové matice ve vektorovém prostoru stupně $2$ nad $\mathbb{C}$. Kvantové obvody proto můžeme přirozeně studovat z pohledu lineární algebry.

Možná ale nejtěžší část práce s kvantovým počítačem je samotná jeho realizace. Běh kvantového počítače je velmi závislý na zdárné dualitě, ve které se každý qubit nachází, což klasická mechanika nepovoluje. Na konstrukci takového stroje se proto musíme ponořit do světa atomů a subatomárních částic, protože u takto malých částic lze pozorovat dualitu částic a vlnění. Například elektromagnetické záření lze vnímat jako vlnění elektromagnetického pole, i jako smršť fotonů, která přenáší energii. 

Jednou z největších překážek k překonání při praktické implementaci kvantového počítače je často velká nestabilita částic, se kterých pracujeme. V tomto ohledu se zdá vhodným kandidátem duální charakter polarizace fotonu, kterou lze vnímat jako superpozici jeho levotočivé a pravotočivé polarizace. Fotony lze stabilně udržovat i při pokojové teplotě, přičemž jako \uv{brány} se nabízí křemíkové hranoly reflektující světlo.

Klasické brány můžeme přirozeně převést do bran kvantových, a tedy libovolnou operaci, kterou provedeme na klasickém počítači, provedeme na tom kvantovém, zmíněný model kvantového počítače je tedy alespoň stejně silný jako klasický Turingův model. Již v $80$. letech minulého století, kdy užití kvantové mechaniky ve výpočetní technice bylo ve svých kojeneckých letech, byly objeveny procesy, které lze na kvantovém počítači vykonat mnohem rychleji, než na tom klasickém. Zmiňme, že jeden z nich, tzv. \textit{Gaussovo vzorkování bosonů}, byl užit jako důkaz kvantové nadvlády pod návrhem vědců z FJFI ČVUT \cite{Science}.

I když jedna libovolná instance nezaručí perfektně přesný výsledek, existuje-li netriviální pravděpodobnost správého výsledků, dostatečný počet opakování nám správný výsledek nalezne. Poskytují-li kvantové počítače při řešení daného problému exponenciální zrychlení, cena několika opakování celého procesu je více než uspokojivá.

%Jedním z důvodů, proč se věří, že s veřejně dostupnými kvantovými počítači přijde nová éra výpočetní techniky je, že existují procesy, o kterých se dodnes neví, zda jsou v~polynomiálním čase proveditelné na počítači klasickém, jejichž kvantové implementace byly již nalezeny.

Jeden problém, jenž je notoricky klasicky obtížný, je rozklad celého čísla na prvočinitele, který ve své nejčirejší podobě můžeme brát jako rozklad čísla $n$ na součin dvou (velkých) prvočísel. Tento problém lze snadno převést na hledání řádu čísla $a$ modulo $n$. V devadesátých letech minulého století přišel Peter Shor \cite{Shor} s řešením problému tzv. \textit{Hidden Subgroup Problem} na konečných abelovských grupách užívajícím polynomiálního počtu kvantových bran, což je třída problémů, do které patří jak diskrétního logaritmus, tak rozklad celého čísla. Mnoho v té době užívaných protokolů na šifrování a podpis dat bylo založeno na obtížnosti HSP, nejprominentější z~nich je známý jako RSA \cite{RSA}. Tento objev znamenal postupný pokles protokolů založených na těchto problémech a naopak se přesouvalo k protokolům založeným na tzv. \textit{akcích grup na množinu}.

Přestože pokrok lidstva v mnohých technologických objevech rostl povětšinou exponenciálně, u kvantových počítačů to tak jednoduché nebude. Problémy fyzické implementace architektury k času psaní této práce vyústily v největší kvantový počítač na světě, tzv. \textit{Jiuzhang}, mající přesně $76$ qubitů. I takto malé počítače běží řádově trilionkrát rychleji, nežli standardní počítač, a rekordy rozkládání čísel každých pár let padnou. Pro představu k roku 2012 bylo největší číslo rozložené kvantovým počítačem číslo $15$, do konce roku 2019 se povedlo výzkumníkům z IBM \cite{Karamlou} toto číslo zvýšit na $1099551473989 = 1048589 \times 1048601$.

Poznamenejme, že nejefektivnější známé klasické algoritmy rozkládající velká čísla (dejme tomu $\log_2(n)>200$) užívají poznatky z teorie eliptických křivek a algebraické teorie čísel, na kterou ještě přijde řeč. Tyto algoritmy nesou názvy \textit{Elliptic-curve factorization} a \textit{General number field sieve}. Oba běží v očekávaném subexponenciálním čase.

\section{Vzhůru k eliptickým křivkám}


Zjevnou adaptací Diffie-Hellmanova protokolu je výměna, která nese název ECDH (Elliptic Curve Diffie-Hellman):
\begin{figure}[h]
\begin{center} 
\makebox[1cm]{\rule{17.3cm}{0.4pt}}\\
\hspace{-1.35cm} \textbf{Veřejné parametry:} Prvočíslo $p$, eliptická křivka $E/\mathbb{F}_p$, bod $G \in E(\mathbb{F}_p)$ vysokého řádu\\

\vspace{-0.25cm}
\makebox[\linewidth]{\rule{17.3cm}{0.4pt}}\\
\vspace{0.2cm}
\begin{tabular}{l l c l}
\cline{2-2} \cline{4-4} 
& Alfréd & & Blažena \\ 
\cline{2-2} \cline{4-4} 
& \textbf{Vstup:} $a \in [1,p]$ & & \textbf{Vstup:} $b\in [1,p]$ \\
 & & $ \xlongrightarrow{\qquad \qquad  [a]G \qquad \qquad}$  &  \\
&  & $\xlongleftarrow{\qquad \qquad  [b]G \qquad \qquad}$   &  \\
& $G_{AB} := [a]([b]G) = [a][b]G$ &  & $ G_{BA} := [b]([a]G) = [b][a]G$ \\
& \textbf{Výstup:} $G_{AB}$ & & \textbf{Výstup:} $G_{BA}$
\end{tabular}
\caption*{Algoritmus 2: Protokol ECDH}

\end{center}
\end{figure}

Praktičnost tohoto protokolu je založena na předpokladu, že diskrétní logaritmus na eliptických křivkách, tedy ze znalosti $P$ a $[n]P$ spočíst $n$, je těžký problém. Tento předpoklad se obecně bere na klasickém počítači za platný, což je též důvod, proč protokol TLS chránící komunikaci na internetu užívá ECDH jako základ.

Protokol ECDH pracuje velmi podobně jako originální Diffie-Hellmanova výměna. Oproti ní nebo RSA má však mnohem menší klíče pro stejnou úroveň bezpečnosti, což je samozřejmě velmi žádoucí. Podobné vnitřní machinace obou protokolů přesto ECDH opět vystavují útoku via Schorův algoritmus. Ten totiž v polynomiálním čase nalezne periodu funkce $(a,b) \mapsto [a]P - [b]Q$ a tedy je schopen spočíst diskrétní logaritmus. O co víc, speciální případ diskrétního logaritmu na eliptických křivkách nabízí ještě rychlejší kvantové útoky \cite{Proos}. Supersingulární křivky jsou v~tomto ohledu dokonce o něco slabší, útok navržený v 90.letech \cite{MOV} redukuje tuto úlohu za pomocí Weilova párování na problém diskrétního logaritmu v~samotném konečném tělese.

Kvůli podobným útokům se kryptografie založena na supersingulárních křivkách v této éře (a skoro dvou desetiletích poté) nebrala v praktickém ohledu v potaz. Kdy se ale v této časové ose začalo uvažovat o isogeniích?

U originálního Diffie-Hellmanova protokolu bereme v předpoklad, že spočíst diskrétní logaritmus v konečném tělese je těžké. Obdobný těžký problém u isogenií se rovnou nabízí:

\begin{problem} (Isogeny Problem)
Ze znalosti eliptických křivek $E$ a $E^\prime$ a prvočísla $\ell$ spočtěte isogenii stupně $\ell^a$ mezi $E$ a $E^\prime$.
\end{problem}

První kryptografické schéma založené na problému výpočtu isogenií obyčejných eliptických křivek navrhl Couveignes \cite{Couveignes} již v roce 1997, nicméně svůj manuskript nepublikoval po dalších deset let. Grafy isogenií byly studovány přes přelom tisíciletí \cite{Galbraith}, \cite{Galbraith2}. Roku 2006 Rostovtsev a~Stolbunov \cite{Stolbunov} nezávisle na Couveignovi navrhli (prakticky totožný) protokol založen na cestách v grafu obyčejných isogenií. 

Bez přílišného noření se do detailů, tyto grafy jsou příliš řídké, a tedy existují subexponenciální algoritmy \cite{Galbraith3}, \cite{Childs} na hledání isogenií mezi nimi, supersingulární grafy jsou ale mnohem hustší. Výměna zmíněna o odstavec výše byla tedy z diskuze vyřazena, ne ale na dlouho, ke konci práce budeme diskutovat jeho spirituálního následníka, protokol CSIDH. Konečně, roku 2011 Luca De Feo, David Jao, navrhli praktickou výměnu užívající isogenie supersingulárních křivek, nesoucí název SIDH, - Supersingular Isogeny Diffie-Hellman \cite{DeFeo3}.

\section{SIDH}

Celá výměna SIDH je založena na faktu, že separabilní isogenie je, až na isomorfismus, jednoznačně určena svým jádrem. Dejme tomu, že $\langle A \rangle, \langle B \rangle$ jsou dvě (až na $\mathcal{O}$) disjunktní grupy bodů ležících na křivce $E$.

Složení separabilních isogenií $E \stackrel{\phi}{\longrightarrow} E/\langle A \rangle \stackrel{\psi}{\longrightarrow} (E / \langle A \rangle)/\langle B \rangle$, kde jádro $\phi$ je $A$ a jádro $\psi$ je $\phi (B)$, bude separabilní isogenie $\xi : E \longrightarrow (E / \langle A \rangle)/\langle B \rangle$ s jádrem $\langle A,B \rangle$. Protože separabilní isogenie je (až na isomorfismus) jednoznačně určena, máme následující komutující diagram:

\begin{figure}[h]
\begin{center} 
\begin{tikzcd}
 & E \arrow[dr, red, "\phi" red]\arrow[dl, blue, "\psi" blue] &\\
E/\langle B \rangle \arrow[dr, blue, "\phi ^\prime" red] & & E/\langle A \rangle \arrow[dl, red, "\psi ^\prime" blue]\\
&E/\langle A,B \rangle &
\end{tikzcd}
\end{center}
\end{figure}

Tento jednoduchý diagram je srdcem samotného protokolu. Samozřejmě isogenie $\phi$ je separabilní s jádrem $\langle A \rangle$ a $\psi^\prime$ je separabilní isogenie taková, že $\psi^\prime \circ \phi$ má jádro $\langle A,B \rangle$. V~případě, že bychom aplikovali obě posloupnosti isogenií, nezískáme nutně na výstupu tu samou křivku $E/\langle A,B \rangle$, křivky se budou lišit až na isomorfismus. Jako sdílené tajemství tak může posloužit příslušná třída isomorfismů, respektive $j$-invariant této křivky. S touto základní myšlenku na mysli, pusťme se do důležitých detailů, které samotnou výměnu usnadňují, či se vyhýbají známým útokům.

Uvažme výměnu, která ať proběhne mezi Alfrédem a Blaženou. Je dána supersingulární eliptická křivka $E$, kterou můžeme díky větě \ref{Fp2} bez ztráty bezpečnosti definovat nad $\mathbb{F}_{p^2}$ s nějakým dále specifikovaným prvočíslem $p$, protože nás u diagramu výše zajímají pouze třídy isomorfismů a ne nutně samotné křivky. Oba si vyberou tajné disjunktní podgrupy $A,B \subseteq E$ a poté spočtou isogenie zobrazující $E \longrightarrow E/A,E/B$. V případě obyčejných křivek isogenie ve formě cest v grafu isogenií spolu komutují (tedy nezáleží na pořadí jejich aplikace), což poskytuje poněkud přímočarý způsob spočíst isogenie $\phi^\prime,\psi^\prime$ založený na našem diagramu, viz právě \cite{Stolbunov}. Pro supersingulární křivky tvoří isogenie složitější strukturu, tento problém se ale dá obejít, pokud se nezaměříme na isogenie definované nad celým uzávěrem, ale pouze nad $\mathbb{F}_q$, o tom ale více později. 

Jeden z klíčových momentů v historii kryptografie založené na isogeniích nastal na přelomu tisíciletí, kdy Galbraith et al. kostruují isogenie mezi obyčejnými křivkami v subexponenciálním čase v $\log p$ i na klasickém počítači \cite{Galbraith},\cite{Galbraith2}. Důvod, proč pouze obyčejné křivky jsou zasaženy je kupodivu právě ona jednoduchá struktura isogenií na nich, vyhledávání v příslušném grafu je příliš jednoduché. V~případě supersingulárních křivek musí obě strany na úspěšný průběh výměny poskytnout té druhé ještě špetku informace o jejich podgrupě, nicméně na oplátku poskytuje struktura grafů supersingulárních isogenií větší bezpečnost.

Pokud Alfréd zveřejní obrazy generátorů $G_{1},G_{2}$ grupy $A$ v $\phi$, Blažena spočte libovolnou podgrupu $A$ generovanou bodem $P = [m]G_{1} + [n]G_2$ jednoduše jako:
\begin{equation*}\label{clubsuit}
\hspace*{-0.2 cm} \phi_B (\langle P \rangle) = \phi_A (\langle [m_A] G_{1A} + [n_A] G_{2A} \rangle) = \langle \phi_A ([m_A] G_{1A} + [n_A] G_{2A}) \rangle = \langle [m_A] \phi_A G_{1A} + [n_A] \phi_A G_{2A} \rangle. \tag{$\clubsuit$}
\end{equation*}
Zveřejnění obrazů generátorů $A$ v $\phi_A$ nám umožní spočíst obraz celé grupy $A$ v $\phi_A$, což může vést v jistých případech na efektivní útoky, to je ale oběť, kterou musíme podstoupit v uvažování supersingulárních křivek. Známé útoky na hledání isogenie mezi supersingulárními křivkami zatím nejsou příliš efektivní, to ale je též způsobeno faktem, že disciplína kryptografie pomocí isogenií supersingulárních křivek je na světě teprv dekádu a je stále v procesu rapidního vývoje.

Konečně, podívejme se ještě jednou na Diffie-Hellmanův protokol. Stejně důležitý jako předpoklad, že diskrétní logaritmus je obtížně spočitatelný, je i předpoklad, že mocnění naopak dokážeme provést v~ polynomiálním čase, tedy účastníci protokolu jej mohou v~rozumném čase sehrát.

Véluovy vzorce umožní účastníkům spočíst své separabilní isogenie v čase úměrném velikosti jejich jádra, což je čas exponenciální. Můžeme ale využít faktu, že se isogenie rozkládají na složení isogenií prvočíselných stupňů, přičemž isogenii malého prvočíselného stupně dokážeme spočíst v konstantním čase. Nabízí se proto vzít obě podgrupy hladké velikosti, což umožní účastníkům v logaritmickém čase spočíst přislušné isogenie.

Abychom zaručili dostatečně velké podgrupy pro obě partie, můžeme zvolit prvočíslo tvaru $p = f \ell_A ^{e_A} \ell_B ^ {e_A} - 1$, kde $\ell_A ^ {e_A} \approx \ell_B ^{e_A}$ jsou velké mocniny prvočísel a $f$ je malé. Zvolíme si křivku $E$ s $E(\mathbb{F}_{p^2}) \cong E[p+1] \cong E[f] \times E[\ell_A ^{e_A}] \times E[\ell_B ^{e_B}]$ obsahující dvě (přibližně stejně) velké podgrupy. To vynutíme například definováním $E$ nad $\mathbb{F}_p$, kde pak díky větě \ref{super2} $E$ nese $p+1$ bodů definovaných nad $\mathbb{F}_p$. Víme, že $E(\mathbb{F}_p)$ je podgrupou $E(\mathbb{F}_{p^2})$, tedy se příslušné počty bodů dělí. Hasseho věta značně omezuje počet bodů na $E(\mathbb{F}_{p^2})$ a pouze jeden z nich je dělitelný $p+1$, konkrétně $(p+1)^2$. Tento případ je tedy ekvivalentní s tím, že stopa Frobenia je nulová.

\begin{figure}[h]
\begin{center} 
\makebox[1cm]{\rule{17.3cm}{0.4pt}}\\
\hspace{-1.35cm} \textbf{Veřejné parametry:} Prvočíslo $p = f \ell_A ^{e_A} \ell_B ^{e_B} - 1$, supersingulární eliptická křivka $E/\mathbb{F}_{p^2}$ splňující $E(\mathbb{F}_{p^2}) = (p+1)^2$, generátory $G_{1A},G_{2A},G_{1B},G_{2B}$ po řadě $E[\ell_A ^{e_A}]$, resp. $E[\ell_B ^{e_B}]$\\

\vspace{-0.25cm}
\makebox[\linewidth]{\rule{17.3cm}{0.4pt}}\\
\vspace{0.2cm}
\hspace*{-1cm}\begin{tabular}{l l c l}
 \cline{2-2} \cline{4-4} 
& Alfréd & & Blažena \\ 
\cline{2-2} \cline{4-4} 
& \textbf{Vstup:} $m_A,n_A$ nedělitelná $p$ & & \textbf{Vstup:} $m_B,n_B$ nedělitelná $p$ \\
&spočte bod $A = [m_A]G_{1A}+[n_A]G_{2A}$ & & spočte bod $B = [m_B]G_{1B}+[n_B]G_{2B}$\\
&spočte separabilní isogenii& &spočte separabilní isogenii\\
&$\phi_A : E \longrightarrow E_A$ s jádrem $\langle A\rangle$ & &$\phi_B : E \longrightarrow E_B$ s jádrem $\langle B\rangle$\\
&spočte v ní obrazy $G_{1B},G_{2B}$& &spočte  v ní obrazy $G_{1A},G_{2A}$\\
 & & $\xlongrightarrow{E_A, \phi_A (G_{1B}), \phi_A (G_{2B})}$  &  \\
&  & $\xlongleftarrow{E_B, \phi_B (G_{1A}), \phi_B (G_{2A})} $ &  \\
& spočte křivku $E_{AB}$ := & & spočte křivku $E_{BA} :=$\\
& $:= E_A/\langle m_A \phi_B (G_{1A})+ n_A \phi_B (G_{2A}) \rangle$ &  & $ := E_B/\langle m_B \phi_A (G_{1B})+ n_B \phi_A (G_{2B}) \rangle$ \\
& \textbf{Výstup:} $j(E_{AB})$ & & \textbf{Výstup:} $j(E_{BA})$
\end{tabular}
\caption*{Algoritmus 3: Protokol SIDH}
\end{center}
\end{figure}

Vysvětleme, co se vlastně v každém kroku děje u Alfréda, Blažena postupuje obdobně. Alfréd si na začátku výměny zvolí tajný bod $A$ na $\ell_{A}^{e_A}$-torzi a spočte separabilní isogenii $\phi_A : E \longrightarrow E_A$ s jádrem rovným grupě $\langle A \rangle$ tak, že ji rozloží na složení $e_A$ isogenií stupně $\ell_A$. Křivka $E_A$ nese díky větě \ref{satotate} $(p+1)^2$ bodů. Aplikováním své isogenie na Blaženiny generátory $G_{1B},G_{2B} \subseteq E[\ell_B ^{e_B}]$ získá body $\phi_A (G_{1B}), \phi_A (G_{2B})$, které generují $\ell_B ^{e_B}$-torzi na křivce $E_A$. Tyto dva obrazy Alfréd publikuje spolu s křivkou $E_A$ a na oplátku obdrží Blaženinu křivku $E_B$ a obrazy $\phi_B (G_{1A}), \phi_B (G_{2A})$ generující $E_B [\ell_A ^{e_A}]$. Obraz bodu $A$ v Blaženině isogenii dokáže Alfréd spočíst jako:
\begin{equation*}
\phi_B (A) = \phi_B ([m_A] G_{1A} + [n_A] G_{2A}) = [m_A] \phi_B (G_{1A}) + [n_A] \phi_B (G_{2A}),
\end{equation*}
{\hypersetup{linkcolor=black}neboť $\phi_B$ je homomorfismus grup $E \longrightarrow E_B$. Bez znalosti Alfrédových tajných koeficientů $m_A,n_A$ není nikdo jiný schopen obraz $A$ vypočíst. Alfréd je poté schopen spočíst obraz celé grupy $\langle A \rangle$ v Blaženině isogenie, jak jsme si před chvíli ukázali u \eqref{clubsuit}.}

Nyní nastává čas, kdy Alfréd spočte isogenii $\phi_{AB}$ vycházející z křivky $E_B$ s jádrem $\langle \phi_B (A) \rangle = \phi_B (\langle A \rangle)$. Protože jsme grupy $A$ a $B$ vybrali až na bod $\mathcal{O}$ disjunktní, afinní body náležící do $\langle A \rangle$ budou po aplikaci $\phi_B$ afinní též a po dvou různé, neboli isogenie $\phi_{AB}$ je separabilní stupně $\ell^{e_A}$ a Alfréd ji dokáže efektivně (v logaritmickém čase) spočíst, čímž získá hledanou křivku $E_{AB}$. Obdobně Blažena získá křivku $E_{BA}$. Dle konstrukce existují separabilní isogenie vycházející z $E$, které mají jádro $\langle A,B \rangle$, přičemž jedna převádí $E$ na $E_{AB}$ a druhá na $E_{BA}$. Tyto křivky jsou proto isomorfní, čímž uzavíráme výměnu.
 
Pojďme nyní vyřešit pár důležitých detailů při seřizování takové výměny ze strany veřejně důvěrné třetí osoby, která nastavuje veřejné parametry, i několik maličkostí ze strany Alfréda a Blaženy.

Zamysleme se nejprve, jak často narazíme na prvočíslo $p$, které hledáme. Konkrétně, pokud si zvolíme fixní mocniny $\ell_A ^{e_A}, \ell_B ^{e_B}$, s jakou pravděpodobností nalezneme \uv{malé} prvočíslo $p \equiv -1 \pmod{\ell_{A}^{e_A} \ell_{B}^{e_B}}$. Dirichletova věta o aritmetických posloupnostech, klasický výsledek analytické teorie čísel, nám poví, že takových prvočísel existuje nekonečně mnoho, tu však potřebujeme ještě trochu zesílit. Na pomoc nám přijde Nikolai Chebotarev a jeho věta o hustotě, kterou lze adaptovat na postačující odhad hustoty prvočísel v aritmetické posloupnosti \cite{Lagarias}. Takové prvočíslo $p$ jsme pak schopni pro nějaké mocniny prvočísel $\ell_A ^{e_A},\ell_B ^{e_B}$ zaručeně najít.

Dále nastává problém najít počáteční křivku tak, aby potenciální útočník neměl při rozbíjení algoritmu přílišnou výhodu. Protokol založený na výměně SIDH, tzv. SIKE - Supersingular Isogeny Key Encapsulation, tento problém řeší jednoduchou volbou křivky $E : y^2 = x^3+x$ pro $p \equiv -1 \pmod{4}$, tedy například s $\ell_A = 2$ či $4 \mid f$. Volba křivky $E : y^2 = x^3+x$, či obecně křivek s $j$-invarianty $0$ a $1728$, umožní místo pouze $j$-invariantů rozlišovat i jednotlivé třídy twistů, viz \cite{Jao}.

Předem známá počáteční křivka ale může v některých případech značně ulehčit prolomení protokolu. Ke způsobu, jak se vyhnout možným trablím s fixní počáteční křivkou, se dostaneme o chvíli později u~protokolu SITH.

Konečně generátory příslušných torzí spočteme jednoduše, postačí vzít generátory $G_1,G_2$ naší křivky, jejich násobky $[\ell_B ^{e_B}] G_1, [\ell_B ^{e_B}] G_2$ generují grupu $E[\ell_A ^{e_A}]$.

Nyní již máme vyřešené aranžmá výměny ze strany nestranného prostředníka a můžeme se přesunout na naše účastníky, opět zaostříme na Alfréda. Po výběru bodu $A = [m_A] G_{1A} + [n_A] G_{2A}$ řádu $\ell_A ^{t_A}$ si Alfréd chce spočíst isogenii $E \longrightarrow E/\langle A \rangle$ stupně $\ell_A ^{t_A}$. Přímočará aplikace Véluových formulí je velmi pomalá a pro volbu $\ell_A ^{e_A} \approx \ell_A ^{e_B} \approx \sqrt{p}$ bychom průměrně očekávali $\ell_A ^{t_A} \approx \sqrt{\ell_A ^{e_A}}$, a tedy běžící čas $O(\sqrt[4]{p})$, což je exponenciální. Můžeme ale využít nápady spojené s rozkládáním isogenií na isogenie prvočíselných stupňů, viz věta \ref{prvoo}. Konkrétně můžeme naši isogenii rozložit na $t_A$ isogenií stupně $\ell_A$ a za pomocí Véluových formulí nyní celý výpočet končí po pouze $O(t_A \ell_A)$ operacích, speciálně se škáluje logaritmicky s rostoucí velikostí $p$.

Véluovy formule jako volba výpočtu všech isogenií též impaktují bezpečnost protokolu. Heuristicky můžeme ověřit, že křivky $E_{AB},E_{BA}$ získané na konci protokolu jsou nejenom isomorfní, ale dokonce tou samou křivkou. Jak bylo podotknuto v článku \cite{Leonardi}, toto pozorování je při užití Véluových formulí realitou. Tato skutečnost poskytuje protokolu větší bezpečnost, protože místo pouhého $j$-invariantu můžeme jako sdílené tajemství považovat celou křivku, či nějaký její parametr, který nabývá mezi všemi křivkami dostatečně mnoha různých hodnot a dá se co nejlépe komprimovat. 

V rámci projektu SOČ, kterého součást je tato práce, jsme v Sage 9.0 implementovali protokol SIDH, aby si čtenář mohl jeho běh vyzkoušet z první ruky. Na odkazu \url{https://github.com/zdenekpezlar/isogenie/tree/Implementation-Web} se nachází balíček obsahující implementované protokoly. Po rozbalení se postupuje podle README. Jakmile se čtenář otevře notebook, bude mít možnost si vybrat mezi protokoly SIDH a SITH, na ten druhý ještě přijde řeč. Zadá čísla odpovídající $p = \ell_A ^{e_A} \ell_B ^{e_B} - 1$ a program inicializuje výměnu. Samotný kód se nachází na \url{https://github.com/zdenekpezlar/isogenie/tree/SIDH-protocol}.

\section{Útoky na SIDH}

Pojďme se zběžně pobavit o možnostech, jak nad protokolem SIDH vyzrát. Budeme uvažovat prvočíslo $p$ dostatečně vysoké, aby řešení hrubou silou nebyla možností. Nejprve zmiňme, že i přes pojmenování tohoto protokolu má bezpečnost SIDH pramálo společného s tou Elliptic Curve Diffie-Hellman, oba jsou založené na velmi odlišných principech a proto není žádný důvod očekávat, že bezpečnost jedné ovlivňuje druhou. Dokonce, jak jsme zmínili, diskrétní logaritmus na supersingulárních křivkách je ještě jednodušší, než na obyčejných, u~isogenií tedy nastává (z naší momentální znalosti) přesně opačná situace jako u diskrétního logaritmu.

Ve své podstatě je problém prolomení SIDHu ekvivalentní s nalezením isogenie stupně $\ell_A ^{e_A}$ mezi známými křivkami $E,E_A$, spolu s trochu extra informacemi o našich isogeniích. Nemůžeme ale najít jen tak nějakou isogenii mezi křivkami, musí to být Alfrédova isogenie, jinak nejsme schopni užít zveřejněné body k výpočtu křivky $E/\langle A,B \rangle$.

Pozapomeňme na chvíli na obrazy generátorů a zabývejme se pouze problémem hledání isogenie. Pokud si, jak jsme zmínili v poslední sekci minulé kapitoly, vyznačíme supersingulární $j$-invarianty nad $\mathbb{F}_{p^2}$ a propojíme je via isogenie stupně $\ell \in \lbrace \ell_A, \ell_B \rbrace$  mezi nimi vedoucí, získáme spojitý neorientovaný $\ell+1$-regulární graf $G_{\ell} (\overline{\mathbb{F}}_p)$ čítající přibližně $p/12$ vrcholů. Každý z účastníků si vybere pseudo-náhodnou procházku a poté podle cesty svého protějšku provede další cestu a oba narazí na křivky se stejným $j$-invariantem. Pokusíme se ukázat, že je nepravděpodobné, že existuje více cest mezi $E,E_A$ délky $e_A = \log_{\ell_A} \ell_A ^{e_A} \approx \log_{\ell_A} \sqrt{p}$. K tomu nám pomůže malé lemma:

\begin{lemma*}
Graf $G_{\ell} (\overline{\mathbb{F}}_p)$ supersingulárních isogenií nad $\mathbb{F}_{p^2}$ má průměr alespoň $O(\log p)$.
\end{lemma*}
\noindent \textit{Důkaz.} Zvolme fixní vrchol $V$ a uvažme všechny z něj vedoucí cesty délky $n$. Podle věty \ref{bigl+1} vede v $G_{\ell} (\overline{\mathbb{F}}_p)$ z $V$ cesta do nejvýše $\ell^{n-1} (\ell+1)$ jiných vrcholů a tento odhad je dokonce ostrý, pokud $G_{\ell} (\overline{\mathbb{F}}_p)$ neobsahuje cykly. Průměr $d$ tohoto grafu pak musí splňovat nerovnost $\ell^{d-1} (\ell+1) \geqslant \lfloor \frac{p}{12} \rfloor + \varepsilon$, v opačném případě by existoval vrchol vzdálený od $V$ na vzdálenost více než $d$. To ale znamená $d \in O(\log p)$. \hfill $\square$\\

Pizer \cite[Thm. 1.]{Pizer} navíc ukazuje, že $G_{\ell} (\overline{\mathbb{F}}_p)$ má průměr i shora ohraničen $2 \log p + O(1)$. Graf $G_{\ell} (\overline{\mathbb{F}}_p)$ má tedy poměrně, ale ne příliš,  krátký průměr. Pokud mezi vrcholy $E,E_A$ nalezneme cestu délky $e_A < d/2$, s vysokou pravděpodobností je jediná, a tedy bude reprezentovat isogenii zvolenou Alfrédem. Jak tedy nalézt cestu mezi $E$ a $E_A$? Jistě můžeme jednoduše prohlédávat graf z $E$, případně $E_A$, do šířky. V nejhorším případě prohledáme všechny křivky $E/G$ s $G$ jádrem velikosti do $\ell^{e_A}$, kterých je přibližně $\ell^{e_A} \approx \sqrt{p}$. Tento postup jistě není optimální, pojďme se na něj podívat trochu chytřeji.

Místo toho, abychom prohledávali pouze z jednoho z vrcholů, můžeme prohledávat z~obou a iniciovat tzv. \textit{Meet in The Middle} attack. Při takovém útoku si z $E$ vypíšeme všechny cesty délky $\lfloor \frac{e_A}{2} \rfloor$ a z $E_A$ zase $\lceil \frac{e_A}{2} \rceil$, a budeme hledat v obou listech dvě isomorfní křivky. Oba seznamy pak budou mít přibližně stejnou velikost $\ell_A ^ {e_A/2} \approx \sqrt[4]{p}$. Problém hledání shody můžeme vyřešit v čase úměrném velikosti obou seznamů, pokud si sestavíme hashovací tabulky příslušných křivek (resp. jejich $j$-invariantů) a budeme hledat shodu, tedy skončíme v očekávaném čase $O(\sqrt[4]{p})$.

Obecně tento problém můžeme parafrázovat jako problém hledání shody dvou funkcí $f : A \longrightarrow C, g : B \longrightarrow C$, kde $A,B$ jsou množiny podgrup $E[\ell_A ^{e_A}]$,$E[\ell_B ^{e_B}]$ a $C$ sestává z $j$-invariantů supersingulárních křivek nad $\mathbb{F}_{p^2}$, tzv. \textit{claw finding}. Zmíněný postup s hashovací tabulkou problém řeší v $O(\vert A\vert + \vert B \vert)$, což je na klasickém počítači optimální \cite{Shengyu}. Kvantový počítač nepřekvapivě opět nad klasickým triumfuje a problém řeší v očekávaném čase $O( \sqrt[3]{\vert A \vert \cdot \vert B \vert})$, tedy v případě grafů isogenií $O(\sqrt[6]{p})$, užitím Taniho algoritmu \cite{Tani}. I tento postup je asymptoticky optimální bez uvažování dalších vlastností grafů isogenií. 

Pokud tedy chceme efektivně prolomit SIDH, s útokem v polynomiálním čase, který s~netriviální pravděpodobností vyustí ve společný klíč, musíme buď hlouběji studovat grafy isogenií, či uvažovat i obrazy generátorů torzí v jednotlivých isogeniích.

Analýza v \cite[Sec. 4.2]{Galbraith4} ukazuje, že jsme schopni spočíst hledanou isogenii mezi $E$ a~$E_A$, známe-li strukturu \textit{endomorfismů} na křivkách $E$ a $E_A$, tedy isogeniím $E \longrightarrow E$, resp. $E_A \longrightarrow E_A$, dokážeme spočíst hledanou isogenii. Výpočet této struktury je na obyčejné křivce a priori proveditelný v subexponenciálním čase \cite{Bisson}, což je z kryptografického hlediska nežádané, supersingulární křivky tuto strukturu mají velmi různou a útoku se vyhýbají. Přesto je důležité podotknout, že fixní počáteční křivka značně ulehčí práci útočníka, protože mu stačí spočíst množinu endomorfismů jednou.

Konečně, zvolme si na místě Evy trochu jiný plán útoku. Místo útoku zvenku na výměny mezi oběma partiemi, Eva infiltruje výměnu zevnitř. Dejme tomu, že Alfréd chce provést výměnu s Blaženou jako předtím, ale místo tentokrát bere Eva alias Blaženy, Alfréd netušíce, že byl ošizen a komunikuje s nečestnou partií. Oba celou výměnu sehrají jako normálně a Eva se dozví nějakou informaci o Alfrédově isogenii ve formě obrazu $\langle B \rangle$ v ní, přičemž tuto grupu si může volit, jak si zamane. Může se Eva opakováním výměny (Alfréd si ponechá svou tajnou isogenii) a šikovnými volbami svých grup dozvědět Alfrédovu isogenii? Právě takovou otázku si položili Galbraith, Petit, Shani a Ti v \cite{Galbraith4} a odpověď zní ano. V přibližně $\frac{1}{2} \log_2 (p)$ výměnách zaručují výpočet celé isogenie, resp. tajných koeficientů $m_A,n_A$ udávajících Alfrédův bod. Tento útok na SIDH \uv{zevnitř} je jedním z mála známých pracujících v polynomiálním čase, ať už v počtu výměn, a tvoří pro protokol reálnou hrozbu.

V obou právě zmíněných útocích hraje roli, že užívána nějaká křivka fixní, a je užívána znalost ohledně isogenií na nich. Chceme-li zaručit co nejlepší bezpečnost v duelu s těmito i se zatím nenalezenými útoky, je imperativní, že ze zveřejněné informace jsou nějakým způsobem zašifrovány, případně není dána fixní počáteční křivka. 



\section{Následníci výměny SIDH}

Subexponenciální řešení problému hledání isogenie mezi obyčejnými křivkami po nějakou dobu zcela zastavilo snahu nalézt efektivní a bezpečná kryptografická primitiva založená na isogeniích, SIDH tomuto hledání znovu vdechl život v podstatě supersingulárních křivek. Od zveřejnění jeho návrhu uplynula k psaní této práce celá dekáda a za tu dobu mnoho různých autorů s původním nápadem odeběhlo na všelijaká místa poskytující praktické vylepšení či (větší či menší) variaci na samém principu výměny. Zde dokumentujeme několik pár z těch nejperspektivnějších.

\textbf{SIKE.} \cite{SIKE} Náš první kandidát není tolik co následník SIDHu, jako jeho optimalizace. Oproti křivkám nad $\mathbb{F}_p$ je aritmetika křivek nad $\mathbb{F}_{p^2}$ velmi pomalá a SIKE se ji pokouší co nejvíce zrychlit. Místo křivek ve Weierstrassově tvaru pracuje s křivkami v tzv. \textit{Montgomeryho tvaru}, kde každou křivku vyjádří ve tvaru $b y^2 = x^3 + ax^2 + x$, kde $a,b \in \mathbb{F}_{p^2}$. Mimo jiné $j$-invariant této křivky závisí jen a pouze na $a$ a vzorce pro dvou a trojnásobek bodu jsou mnohem stravitelnější a~umožňují rychlejší počítání. Z tohoto důvodu jsou též fixována $\ell_A = 2, \ell_B = 3$, což opět neposkytuje útočníkům nějakou podstatnou výhodu. Konečně je implementována komprese posílaných bodů pro ještě menší klíče a tzv. KEM - \textit{Key Exchange Mechanism}, který poskytuje protokolu větší bezpečnost. Excelentní článek detailující tyto specifika je \cite{Costello}. 

\begin{poznamka}
Oproti kandidátům z řad post-kvantových výměn založených na mřížkách či kódech, SIKE je prakticky řádově pomalejší \cite[Ch. 6]{SIKE}. Původní Couveignův/Rostovtsev a Stolbunovův protokol bez optimalizací je navíc ještě o~několik řádů pomalejší. Naopak velikost klíčů má SIKE z mnoha těchto protokolů nejmenší, jak bylo podotknuto na standardizační soutěži NIST \cite{NIST}, na které SIKE roku 2017 soutěžil a probojoval se do třetího a posledního kola jako alternativní kandidát.
\end{poznamka}

\textbf{eSIDH.} \cite{eSIDH} Pravděpodobně nejvíce přímočará adaptace výměny SIDH spočívá ve změně prvočísla na tvar $p = f 2^{e_A} \ell_B ^{e_B} \ell_C ^{e_C} - 1$, kde $\ell_B,\ell_C$ jsou malá prvočísla a $2^{e_A} \approx \ell_B ^{e_B} \ell_C ^{e_C}$. Alfréd si tentokrát vybírá isogenie stupně dělícího $2^{e_A}$ a Blažena isogenie stupně dělícího $\ell_B ^{e_B} \ell_C ^{e_C}$. Tato varianta v mnoha případ paradoxně má rychlejší aritmetiku, než samotný SIDH.

\textbf{BSIDH.} \cite{BSIDH} Tento protokol užívá faktu, že supersingulární křivky s $\#E(\mathbb{F}_p^2) = (p+1)^2$ mají kvadratický twist s počtem bodů $(p-1)^2$, viz \ref{twister}. Jeden z účastníků pak pracuje na $(p+1)$-torzi křivky $E$ a druhý na $(p-1)$-torzi jejího kvadratického twistu $\tilde{E}$, přičemž $E$ a $\tilde{E}$ jsou isomorfní nad $\mathbb{F}_{p^4}$.

\textbf{SITH.} \cite{Dark} Zde navržená výměna řeší problém známé počáteční křivky velmi jednoduše, Alfréd provede z křivky $E$ náhodnou procházku v grafu isogenií stupně $\ell_A$ na $E^\prime$, ze které je poté výměna inicializována podobně jako v SIDHu.

\textbf{CSIDH.} \cite{CSIDH} Náš poslední protokol nese název \uv{Commutative Supersingular Isogeny Diffie-Hellman}, přesto není přímou adaptací ani vylepšením SIDHu. Jeho běh užívá hlubších znalostí ohledně eliptických křivek a jejich spojitosti s algebraickou teorií čísel, konkrétně pracuje na principu akce \textit{grupy tříd ideálů} na grafu supersingulárních eliptických křivek. Pro tentokrát bereme pouze křivky definované nad $\mathbb{F}_p$ a zajímavě isogenie též, speciálně jako vrcholy bere třídy křivek isomorfní \uv{nad $\mathbb{F}_p$}. Každá třída isomorfismů je proto reprezentovaná dvěma vrcholy příslušícími křivkám a jejich kvadratických twistům. Díky této volbě potřebujeme užívat pouze aritmetiku čísel nad $\mathbb{F}_p$, která je velmi rychlá. V dalších kapitolách budeme studovat oblasti matematiky potřebné k pochopení tohoto protokolu.

\section{SITH}

Při statické počáteční křivce $E_0$ se protokol vystavuje většímu riziku prolomení, proto bychom chtěli při každé výměně zvolit novou křivku. Nestranný prostředník může ale jen zveřejnit příslušnou křivku a její parametry, tato úloha proto padá na samotné účastníky výměny. Protokol SITH - Supersingular Isogeny Two-Party Handshake diktuje, že na začátku výměny například Alfréd zvolí podgrupu $G \subseteq E_0 [\ell_A]$ a spočte novou křivku $E = E_0/G$, zbytek výměny postupuje analogicky jako v SIDHu, pouze na křivce $E$.

\begin{figure}[h]
\begin{center} 
\begin{tikzcd}
E_0 \arrow[rr, bend left, red] & & E \arrow[dr, red, "\phi" red]\arrow[dl, blue, "\psi" blue] &\\
& E/\langle B \rangle \arrow[dr, blue, "\phi ^\prime" red] & & E/\langle A \rangle \arrow[dl, red, "\psi ^\prime" blue]\\
& &E/\langle A,B \rangle &
\end{tikzcd}
\end{center}
\end{figure}

Tato jednoduchá volba řeší mnoho útoků, které závisí na známé počáteční křivce. I když protokol nyní není symetrický pro oba účastníky (například Alfréd musí poslat dohromady dvakrát tolik informací co Blažena), tento problem je snadno řešen během dvou protokolů proti sobě, jeden iniciovaný Alfrédem, druhý Blaženou. 

Nápad tohoto protokolu byl převzat ze článku \cite{Dark}, kde autoři výměnu pouze nastiňují jako prospektivního následníka SIDHu. Kolektiv autorů ani nikdo jiný doposud tuto konkrétní výměnu neimplementoval, autoři pouze navrhují provést podrobné porovnání velikostí klíču a rychlosti protokolů SIDH a SITH.

My můžeme bohužel zmínit pouze povrchová pozorování, jakožto není v našich možnostech simulovat SIDH a SITH ve větším měřítku a tyto simulace porovnávat. Na Blaženině straně protokolu se pramálo změní, Alfréd ale musí provést jednu náhodnou procházku navíc (což prodlouží délku jeho procedury na přibližně $150$ $\%$) a zveřejnit křivku a její generátory, což velikost jeho publikovaných klíčů přibližně zdvojnásobí. Alfréd též může jednoduše zveřejnit pouze křivku $E$ a nechat na Blaženě, ať si zjistí příslušné generátory, což zase přidá na délce procedury na její straně.

V rámci projektu SOČ jsme tedy obdobně jako v případě SIDHu protokol SITH implementovali v Sage 9.0. Adresa \url{https://github.com/zdenekpezlar/isogenie/tree/Implementation-Web} obsahuje aplet umožňující vyzkoušet si náhodnou instanci protoku SIDH, nicméně ten tvoří pouze polovinu stránky. \uv{Temnou stranu} obývá protokol SITH, kde opět po zadání prvočísel určujících $p = \ell_A ^{e_A} \ell_B ^{e_B} - 1$ můžeme vidět všechny kalkulace v náhodné instanci i tohoto protokolu. Opět je zdrojový kód k dispozici na \url{https://github.com/zdenekpezlar/isogenie/tree/SITH-protocol}, který si čtenář může dle své vůle upravovat.



\begin{figure}[h]
\begin{center} 
\makebox[1cm]{\rule{17.3cm}{0.4pt}}\\
\hspace{-1.35cm} \textbf{Veřejné parametry:} Prvočíslo $p = f \ell_A ^{e_A} \ell_B ^{e_B} - 1$, supersingulární eliptická křivka $E_0/\mathbb{F}_{p^2}$ splňující $E_0(\mathbb{F}_{p^2}) = (p+1)^2$, generátory $G_{1A},G_{2A},G_{1B},G_{2B}$ po řadě $E_0[\ell_A ^{e_A}]$, resp. $E_0[\ell_B ^{e_B}]$\\

\vspace{-0.25cm}
\makebox[\linewidth]{\rule{17.3cm}{0.4pt}}\\
\vspace{0.2cm}
\hspace*{-1cm}\begin{tabular}{l l c l}
 \cline{2-2} \cline{4-4} 
& Alfréd & & Blažena \\ 
\cline{2-2} \cline{4-4} 
& \textbf{Vstup:} $p \nmid m_A,n_A$, grupa $G \subseteq E_0[\ell_A]$ & & \textbf{Vstup:} $p \nmid m_B,n_B$ \\
&spočte křivku $E = E_0/G$ & &\\
&a její generátory $G_1,G_2$ & &\\
&spočte generátory $G_{1A} = [\ell_A ^{e_A}] G_1,\dots$ & &\\
 & & $\xlongrightarrow{E,G_{1A},G_{2A},G_{1B},G_{2B}}$  &  \\
&spočte bod $A = [m_A]G_{1A}+[n_A]G_{2A}$ & & spočte bod $B = [m_B]G_{1B}+[n_B]G_{2B}$\\
&spočte separabilní isogenii& &spočte separabilní isogenii\\
&$\phi_A : E \longrightarrow E_A ^\prime$ s jádrem $\langle A\rangle$ & &$\phi_B : E \longrightarrow E_B$ s jádrem $\langle B\rangle$\\
&spočte v ní obrazy $G_{1B},G_{2B}$& &spočte  v ní obrazy $G_{1A},G_{2A}$\\
 & & $\xlongrightarrow{E_A, \phi_A (G_{1B}), \phi_A (G_{2B})}$  &  \\
&  & $\xlongleftarrow{E_B, \phi_B (G_{1A}), \phi_B (G_{2A})} $ &  \\
& spočte křivku $E_{AB}$ := & & spočte křivku $E_{BA} :=$\\
& $:= E_A/\langle m_A \phi_B (G_{1A})+ n_A \phi_B (G_{2A}) \rangle$ &  & $ := E_B/\langle m_B \phi_A (G_{1B})+ n_B \phi_A (G_{2B}) \rangle$ \\
& \textbf{Výstup:} $j(E_{AB})$ & & \textbf{Výstup:} $j(E_{BA})$
\end{tabular}
\caption*{Algoritmus 3: Protokol SITH}
\end{center}
\end{figure}


\chapter{Algebraická teorie čísel}

Ve snaze vybudovat teorii k hlubšímu studiu eliptických křivek a isogenií, natož diskuzi prakticky užívaných protokolů, se musíme na tyto objekty podívat v naprosto odlišném světle. Opustíme proto na okamžik eliptické křivky a ponoříme se do říše algebraické teorie čísel.

Na světě se nachází myriáda kvalitních a podrobných materiálů ke studiu této krásné oblasti matematiky, já osobně vřele doporučuji texty \cite[Ch. XIII]{Chen}, \cite{Ireland}, \cite{Neukirch} či \cite{Pupik}. Jako velmi stručný úvod motivovaný poznatky z elementární teorie čísel může též posloužit má SOČ, \cite{Pezlar}. 

\section{Moduly nad okruhem}

Při definici vektorového prostoru požadujeme, aby byl sestrojen nad tělesem. Objekt mající obdobné vlastnosti můžeme však obecněji sestrojit nad libovolným okruhem. Pro jednoduchost se omezíme pouze na okruhy komutativní.

\begin{definice}
Mějme abelovskou grupu $G$ a množinu $X$. Pod \textit{akcí} $G$ \textit{na} $X$ rozumíme zobrazení $\cdot : G \times X \longrightarrow X$, které splňuje $1 \cdot x = x$ a $g \cdot (h \cdot x) = (g \cdot h) \cdot x$ pro libovolná $g,h \in G, x \in X$.
\end{definice}
\begin{definice}
Akci $\cdot : G \times X \longrightarrow X$ nazveme \textit{volnou}, pokud pro libovolná $x \in X$ a~$g \in G$ rovnost $g \cdot x = x$ znamená $g = 1$. Akci $\cdot$ též nazveme \textit{tranzitivní}, pokud pro každou dvojici $(x,y) \in X^2$ existuje $g \in G$ splňující $g \cdot x = y$.
\end{definice}

\begin{definice}
Abelovskou grupu $M$ s operací $+$ pro okruh $R$ nazveme $R$-\textit{modulem} s~akcí $\cdot : R \times M \longrightarrow M$, pokud $\cdot$ je asociativní a na $+$ oboustranně distributivní.
\end{definice}

Vzpomeňme na definici volné abelovské grupy $G$, jakožto $G \cong \mathbb{Z}^r$ pro nějaké nezáporné $r$, obdobně definujeme i volný modul.
\begin{definice}
Modul $M$ okruhu $R$ nazveme \textit{volným}, pokud obsahuje $R-$bázi, tj. pro nějaká $m_i \in M$ lineárně nezávislá nad $R$ je $M = \left\lbrace r_1 m_1 + r_2 m_2 + \cdots  \vert r_i \in R \right\rbrace$. Říkáme, že množina $\lbrace m_1,m_2,\dots \rbrace$ \textit{generuje} $M$.
\end{definice}

\begin{definice}
Buď $M$ volný $R$-modul. Pokud je $k$ nejmenší přirozené číslo takové, že existuje $k$ prvků $M$ generujících $M$ nad $R$, řekneme, že $R$-\textit{rank} $M$ je $k$.
\end{definice}

Pro $R$ těleso je $M$ volným modulem, tedy vektorovým prostorem nad $R$, protože každý vektorový prostor vyžaduje existenci báze. $R$-rank $M$ je pak roven dimenzi $M$ nad $R$.

Nejprve si ukážeme jednoduchý způsob, jak poznat, zda je grupa $\mathbb{Z}$-modulem.

\begin{priklad}\label{modulgrupa}
Ukažme, že grupa je abelovská, právě pokud je $\mathbb{Z}$-modulem.
\end{priklad} 
\noindent \textit{Důkaz.} Každá abelovská grupa $G$ s operací $+$ je $\mathbb{Z}$-modulem s akcí $n \cdot a$, jakožto součet $n$ prvků $a \in G$, pro záporná čísla $(-n) \cdot a = - (n \cdot a)$. Navíc pro $\mathbb{Z}$-modul s~operací $+$ platí:
\begin{equation*}
x+y+x+y = 1\cdot (x+y) + 1 \cdot (x+y) = (1+1)\cdot (x+y) = (1+1)\cdot x + (1+1)\cdot y = x+x+y+y,
\end{equation*}
tedy $y+x = x+y$.\hfill $\square$\\

Každý komutativní okruh $R$ je volným $R$-modulem, jehož $R$-rank je $1$. Mezi volné $\mathbb{Z}$-moduly patří například okruh Gaussových celých čísel $\mathbb{Z}[i]$, jenž má $\mathbb{Z}$-rank $2$, ale okruh zbytkových tříd $\mathbb{Z}_{101}$ volným $\mathbb{Z}$-modulem není, ale volným $\mathbb{Z}_{101}$ modulem již je. Naopak tělesa $\mathbb{Q}, \mathbb{C}$ jsou po řadě $\mathbb{Z}$-modul, resp. $\mathbb{Q}$-modul bez konečné báze, volné přesto jsou.

Poněkud zajímavějším příkladem modulu je grupa nejvýše kvadratických polynomů nad reálnými čísly $\mathbb{R}[x]/x^3\mathbb{R}$, což je volný $\mathbb{R}$-modul ranku $3$ s bází $\lbrace 1, x, x^2 \rbrace$, či grupa $E[n]$ pro křivku nad $K$ s $\char K \nmid n$, což je volný $\mathbb{Z}_n$-modul, který má díky větě \ref{nesoudtorze} rank $2$. 

\begin{poznamka}
Roku 1922 Luis Mordell v \cite{Mordell} dokázal, že pro libovolnou eliptickou křivku $E$ je grupa $E(\mathbb{Q})$ konečně generovaná. Tento výsledek rozšířil André Weil v roce 1928 pro libovolnou projektivní křivku nad číselným tělesem \cite{Weil}, což je pojem, který si za chvíli objasníme. Obecně charakterizovat tuto grupu, či efektivně spočíst její rank, jsou dnes problémy stále velmi obtížné. Clayův institut tuto oblast matematiky považoval za tak důležitou, že roku 2000 mezi problémy tisíciletí (Millenium Prize Problems) zařadil tzv. \textit{Birch-Swinnerton-Dyerovu domněnku}, která se zabývá asymptotickým chováním $E(\mathbb{F}_p)/p$ vzhledem k ranku naší křivky. 
\end{poznamka}

Podmnožiny $R$-modulu uzavřené na sčítání a násobení prvky $R$ jsou též $R$-moduly. Takový modul pak nazveme podmodulem.

\begin{definice}
Nechť $M$ a $N$ jsou $R$-moduly, přičemž $N$ je podgrupa $M$. Pak $N$ nazveme \textit{podmodulem} $M$. \textit{Index} podmodulu $N$ v $M$ definujeme jako počet prvků faktorgrupy $M / N$, pokud je tato grupa konečná.
\end{definice}

\begin{veta}\label{podmodul}
Nechť $M$ je volný $\mathbb{Z}$-modul a $N$ jeho podmodul. Pak rank $N$ je nejvýše tak velký, jako rank $M$. Speciálně je $N$ volný.
\end{veta}

Hezký důkaz indukcí je podán v \cite[Věta~1.3.8]{Pupik}. Pokud bychom však místo $\mathbb{Z}$ uvážili libovolný komutativní okruh $R$, tvrzení již ne nutně platí!
\begin{priklad}
Ukážme, že $\mathbb{Z}_6$-modul $2\mathbb{Z}_6 = \lbrace 0,2,4 \rbrace$ není volný. Pokud by totiž modul $2 \mathbb{Z}_6$ měl nad $\mathbb{Z}_6$ bází, musely by její prvky nad $\mathbb{Z}_6$ být lineárně nezávislé. Nicméně platí $3 \cdot 0 = 3 \cdot 2  = 3 \cdot 4 = 0$, přičemž $3 \neq 0$ v $\mathbb{Z}_6$. Žádná podmnožina $S \subseteq 2 \mathbb{Z}_6$ tedy není nad naším okruhem lineárně nezávislá, protože $3 S = \lbrace 0 \rbrace$.
\end{priklad}

Obdobně vidíme, že pokud $R$ je okruh, který není oborem integrity, obsahující nenulové prvky $x, y$ se součinem $0$, a $M$ je jeho volný podmodul, pak $x M$ je podmodul $M$, který není volný.

Čtenář se mohl setkat s pojmem \textit{tenzorový součin} vektorových prostorů $V$ a $W$, neboli vektorový prostor $U$ disponující univerzálním bilineárním zobrazením $V \times W \longrightarrow U$. My tuto definici rozvineme na moduly nad komutativním okruhem, tedy akci $R \times G \longrightarrow G$ rozšíříme na akci $M \times G \longrightarrow G$, kde $M$ je $R$-modul.

\begin{definice}
Buďte $R$ okruh a $M$ a $N$ volné $R$-moduly. Uvažme prvky $m \in M,n \in N$ jednotlivých modulů. \textit{Tenzorový součin} $m \otimes n$ definujeme jako výraz, který je na sčítání oboustranně distributivní a pro každé $r \in R$ splňuje:
\begin{equation*}
(rm) \otimes n = r (m \otimes n) = m \otimes (rn).
\end{equation*}
Pak \textit{tenzorový součin} volných modulů $M$ a $N$ definujeme jako volný $R$-modul generovaný prvky $m \otimes n$ pro $m \in M$, $n \in N$. Jeho prvky nazveme \textit{tenzory}.
\end{definice}

Uveďme si jednoduchý příklad tenzorového součinu. 

\begin{priklad}
Ukažme, že $\mathbb{Z}_m \otimes_{\mathbb{Z}} \mathbb{Z}_n = \lbrace 0 \rbrace$ pro nesoudělná celá $m,n$. Máme:
\begin{align*}
m (1 \otimes 1) = m \otimes 1  &= 0 \otimes 1 = 0,\\
n (1 \otimes 1) = 1 \otimes n &= 1 \otimes 0 = 0.
\end{align*}
Dle Bezoutovy věty existují $x,y \in \mathbb{Z}$, že $xm + yn = 1$. Pak:
\begin{align*}
1 \otimes 1 = (xm+yn)(1 \otimes 1) = xm (1 \otimes 1) + yn (1 \otimes 1) = 0.
\end{align*}
Pro každá $x \in \mathbb{Z}_m, y \in \mathbb{Z}_n$ pak platí $x \otimes y = x(1 \otimes y) = xy (1 \otimes 1)  = 0$.
\end{priklad}

Případ $N = \mathbb{Q}$ a $R = \mathbb{Z}$ je zajímavější:

\begin{veta}\label{qtensor}
Pokud je $M$ $\mathbb{Z}$-modul, každý prvek $\mathbb{Q} \otimes_{\mathbb{Z}} M$ se dá zapsat ve tvaru $r \otimes m$ pro $r \in \mathbb{Q}, m \in M$.
\end{veta}

\textit{Důkaz.} Je postačující ukázat, že pro $x,y \in \mathbb{Q}, m,n \in M$ se $x \otimes m + y \otimes n$ dá vyjádřit v~takovém tvaru. Zvolme celá $a,b,c$ splňující $x = \frac{a}{c}, y = \frac{b}{c}$. Pak:
\begin{align*}
\frac{a}{c} \otimes m + \frac{b}{c} \otimes n = \frac{1}{c} \otimes am + \frac{1}{c} \otimes bn = \frac{1}{c} \otimes (am+bn),
\end{align*}
kde $am+bn \in M$, je hledaného tvaru. \hfill $\square$



\section{Číselná tělesa}

Za pomocí vlastností modulů můžeme začít studovat konečná rozšíření racionálních čísel, tzv. číselná tělesa.

\begin{definice}
Komplexní číslo $\alpha$, které je kořenem polynomu $P \in \mathbb{Z}[x]$, nazveme \textit{algebraické}. Pokud je navíc $\alpha$ kořenem monického (normovaného) polynomu nad $\mathbb{Z}$, nazveme jej \textit{celým algebraickým} číslem.
\end{definice}

\begin{definice}
Konečná rozšíření racionálních čísel obsahují pouze čísla algebraická, tato tělesa proto nazveme \textit{algebraická číselná tělesa}, pro jednoduchost je budeme nazývat pouze \textit{číselná tělesa}.
\end{definice}


\begin{definice}
Pod stupněm číselného tělesa rozumíme stupeň jeho rozšíření nad $\mathbb{Q}$ jakožto vektorového prostoru. Číselná tělesa stupně $2$ nazveme \textit{kvadratická}.
\end{definice}

Jistě obor komplexních čísel s racionální reálnou i imaginární složkou je kvadratickým tělesem, jako je též těleso $\mathbb{Q}(\sqrt{2})$. Obecně každé těleso dáno rozšířením $\mathbb{Q}$ o jednu jedinou odmocninu je kvadratické. Opačná inkluze je též nasnadě:

\begin{veta}
Buď $K$ kvadratické těleso. Pak $K = \mathbb{Q}(\sqrt{m})$ pro nějaké celé bezčtvercové $m$.
\end{veta}
\noindent \textit{Důkaz.} $K$ je vektorový prostor nad $\mathbb{Q}$ stupně dvě, má tedy nad racionálními čísly bázi $\lbrace 1, \theta \rbrace$ a $K$ je rozšířením $\mathbb{Q}(\theta)$ pro algebraické $\theta$. Číslo $\theta ^2$ náleží do $K$, musí proto existovat vyjádření $a+b\theta = \theta ^2$ pro $a,b$ racionální čísla, tedy $\theta = \frac{s+t\sqrt{m}}{2}$ pro vhodná racionální $s,t$. Pak $K = \mathbb{Q}\left(\frac{s+t\sqrt{m}}{2} \right) = \mathbb{Q}(\sqrt{m})$.\hfill $\square$\\

Pro $m > 0$ nazveme $K$ \textit{reálným} kvadratickým tělesem, v opačném případě ($m < 0$) jej nazveme \textit{imaginárním} kvadratickým tělesem. Pokud $m$ je čtvercem celého čísla, je $K$ rovno $\mathbb{Q}$, není tedy kvadratickým tělesem.

Toto tvrzení můžeme zobecnit na všechna číselná tělesa. Konkrétně každé konečné rozšíření racionálních čísel je jednoduché, jak je ukázáno v \cite[Věta 11.12]{Rosicky}. Dokonce si takové $\theta$ můžeme zvolit celé algebraické, viz \cite[Lemma 4.3.8]{Perutka}. Báze $K$ jakožto vektorového prostoru je poté $\lbrace 1,\theta,\dots,\theta ^{n-1} \rbrace$, kde $n = [K : \mathbb{Q}]$ je stupeň minimálního polynomu prvku $\theta$ nad racionálními čísly. 

\begin{poznamka} 
Je zajímavé uvážit případ rozšíření $\mathbb{Q}(\theta)$, kde $\theta$ není kořenem žádného polynomu s racionálními koeficienty, takové $\theta$ se nazývá \textit{transcendentní}. Pak zobrazení dané $P \mapsto P(\theta)$ pro racionální lomenou funkci $P$ je prosté a dává isomorfismus mezi tělesem racionálních lomených funkcí $\mathbb{Q}(x)$ a $\mathbb{Q}(\theta)$.
\end{poznamka}

Pojďme si trochu charakterizovat celá algebraická čísla v číselném tělese.

\begin{veta}\label{alg}
Komplexní číslo $\alpha$ je celé algebraické, právě pokud je $\mathbb{Z}[\alpha]$ volným $\mathbb{Z}$-modulem.
\end{veta}
\noindent \textit{Důkaz.} Je-li $\alpha$ celé algebraické číslo s minimálním polynomem $f \in \mathbb{Z}[x]$ stupně $n$, pak $\mathbb{Z}[\alpha]$ je volný $\mathbb{Z}$-modul s bází $\lbrace 1,\alpha,\dots,\alpha^{n-1} \rbrace$, číslo $\alpha^k$ pro $k\geqslant n$ totiž dokážeme z $\alpha^{k-n} P(\alpha) = 0$ vyjádřit jako $\mathbb{Z}$-lineárních kombinace mocnin $\alpha$ ostře nižších $k$, protože je $P$ monický.

Naopak pokud je $\mathbb{Z}[\alpha]$ volný $\mathbb{Z}$-modul, je generovaný prvky $f_i(\alpha) \in \mathbb{Z}[\alpha]$ pro polynomy $f_1,\dots,f_k \in \mathbb{Z}[x]$. Pro číslo $t$ ostře větší $\max{(\deg f_i)}$, leží $\alpha^t$ v $\mathbb{Z}[\alpha]$, je proto vyjádřitelné jako $\mathbb{Z}$-lineární kombinace $f_i(\alpha)$. Pro nějaká $a_i \in \mathbb{Z}$:
\begin{equation*}
\alpha^t = \sum a_i f_i(\alpha),
\end{equation*}
tedy $\alpha$ je kořenem monického polynomu $x^t - \sum a_i f_i(x)$, dle definice je celé algebraické. \hfill $\square$\\

Pro všechna algebraická čísla $\theta$, která nejsou celá algebraická, tedy okruh $\mathbb{Z}[\theta]$ není konečně generovaný jako $\mathbb{Z}$-modul, stejně jako v případě $\theta$, které není kořenem žádného polynomu nad racionálními čísly.  Díky tomuto tvrzení můžeme jednoduše odůvodnit, proč necelá racionální čísla nejsou celá algebraická.
\begin{priklad}\label{racalg}
Ukažme, že pro $p,q$ nesoudělná celá s $\vert q\vert > 1$ je racionální číslo $\frac{p}{q}$ algebraické číslo, ale již není celé algebraické.
\end{priklad}
\noindent \textit{Důkaz.} Číslo $\frac{p}{q}$ je kořenem polynomu $qx-p \in \mathbb{Z}[x]$, tedy je algebraické. Dále uvažme pro spor polynom $P = x^n + a_{n-1} x^{n-1} + \cdots  + a_0$ s kořenem $\frac{p}{q}$. Rovnost $P\left(\frac{p}{q} \right)=0$ přenásobíme číslem $q^n$ a získáme:
\begin{equation*}
p^n + a_{n-1} p^{n-1} q + \cdots + a_1 p q^{n-1} + a_0 q^n = 0.
\end{equation*}
Dejme tomu, že $\vert q \vert > 1$, a uvažme prvočíslo $r$ dělící $q$. Pak $r$ dělí číslo $-(a_{n-1} p^{n-1} q + \cdots + a_1 p q^{n-1} + a_0 q^n) = p^n$, což je spor s faktem, že $p$ a $q$ jsou nesoudělná. Žádné takové prvočíslo proto neexistuje a $q = \pm 1$. \hfill $\square$ 

\begin{poznamka}
Na toto tvrzení můžeme nahlížet i jako na problém ukázat, že okruh $\mathbb{Z}\left[\frac{p}{q}\right]$ není volným $\mathbb{Z}$-modulem. Pokud by totiž $\lbrace a_1,\dots,a_k \rbrace$ byla jeho báze, posloupnost mocnin $\frac{p}{q}, \left(\frac{p}{q}\right)^2, \dots, \left(\frac{p}{q}\right)^i, \dots \in \mathbb{Z}\left[\frac{p}{q}\right]$ má pro prvočíslo $r \mid q$ klesající celočíselné hodnoty $r$-adických valuací. To je nicméně spor, protože množina $\lbrace \nu_r(a_1),\dots,\nu_r(a_k) \rbrace$ je zdola omezená a~platí $\nu_r(a+b) \geqslant \min\left\lbrace\nu_r(a),\nu_r(b)\right\rbrace$.
\end{poznamka}

Důležitým faktem o celých algebraických číslech je, že v číselném tělese tvoří okruh, jak si dále ukážeme.

\begin{veta}
Celá algebraická čísla číselného tělesa $K$ tvoří okruh $\mathcal{O}_K$.
\end{veta}
\noindent \textit{Důkaz.} Ukážeme, že součet a součin dvou algebraických čísel $\alpha$ a $\beta$ je opět algebraické číslo. Mějme $\mathbb{Z}[\alpha]$ a $\mathbb{Z}[\beta]$ volné moduly a uvažme okruh $\mathbb{Z}[\alpha,\beta]$, jenž je množinou všech polynomů ve dvou proměnných nad celými čísly evaluovaných v bodě $(\alpha,\beta)$. Ten je abelovskou grupou a díky příkladu \ref{modulgrupa} i $\mathbb{Z}$-modulem.

V důkazu věty \ref{alg} jsme si ukázali, že pokud minimální polynom $P_\alpha$ má stupeň $n$, číslo $\alpha^k$ pro $k \geqslant n$ se dá vyjádřit jako $\mathbb{Z}$-lineární kombinace prvků $\alpha$ s mocninami ostře nižšími $n$. Víme, že $\mathbb{Z}[\alpha,\beta]$ je množinou $\mathbb{Z}$-lineárních kombinací čísel $\alpha^i \cdot \beta^j$, z čehož plyne, že $\mathbb{Z}[\alpha,\beta]$ je generovaný množinou $S = \left\lbrace \alpha^i \beta^j \vert i \in \left\lbrace 0,1,\dots,n-1 \right\rbrace, j \in \left\lbrace 0,1,\dots,m-1 \right\rbrace \right\rbrace$, kde minimální polynom $\beta$, $P_\beta$, má stupeň $m$. Protože $K$ je oborem integrity, nějaká podmnožina $S$ lineárně nezávislá nad $\mathbb{Q}$ tvoří bází $\mathbb{Z}[\alpha,\beta]$. Okruh $\mathbb{Z}[\alpha,\beta]$ je proto volným $\mathbb{Z}$-modulem ranku nejvýše $mn$.

Okruhy $\mathbb{Z}[\alpha+\beta]$ a $\mathbb{Z}[\alpha \beta]$ jsou díky jejich komutativitě $\mathbb{Z}$-moduly a navíc jsou oba zjevně podmoduly $\mathbb{Z}[\alpha,\beta]$. Díky větě \ref{podmodul} jsou oba volné (ranku nejvýše $mn$), tedy $\alpha+\beta$ a $\alpha \beta$ jsou celá algebraická čísla. Speciálně pro libovolné $\alpha$ celé algebraické je $-\alpha$ celé algebraické. Množina $\mathcal{O}_K$ celých algebraických čísel tělesa $K$ proto tvoří okruh. \hfill $\square$\\

\begin{poznamka}
Okruhy celých algebraických čísel značíme $\mathcal{O}_K$ a později uvedeme \textit{pořádky}, které budeme povětšinou značit $\mathcal{O}$. Shodně jsme značili bod v nekonečnu na křivce, mějme proto na paměti kdy diskutujeme který pojem!
\end{poznamka}

\begin{lemma}\label{asob}
Buď $\theta$ libovolné nenulové algebraické číslo. Pak existuje nenulové celé $m$ takové, že $m \theta$ je celé algebraické.
\end{lemma}
\noindent \textit{Důkaz.} Pokud minimální polynom $\theta$ nad celými čísly je:
\begin{equation*}
P : a_n x^n + a_{n-1} x^{n-1} + \cdots + a_0,
\end{equation*}
kde $a_n \neq 0$, pak $a_n \theta$ je kořenem monického polynomu:
\begin{equation*}
P^* : x^n +a_n  a_{n-1} x^{n-1} + a_n ^2 a_{n-2} x^{n-2} + \cdots + a_n ^n a_0,
\end{equation*}
toto číslo je tedy celé algebraické. \hfill $\square$\\

Toto tvrzení je vše, co nám stačí k určení podílového tělesa $\mathcal{O}_K$. Pokud si představíme situaci nad $\mathbb{Z}$ či $\mathbb{Z}[i]$, jistě se dovtípíme, které těleso to bude.

\begin{dusledek}
Číselné těleso $K$ je podílovým tělesem okruhu $\mathcal{O}_K$. 
\end{dusledek}
\noindent \textit{Důkaz.} Víme, že podílové těleso okruhu $\mathcal{O}_K$, které označíme $L$, je nejmenší těleso obsahující $\mathcal{O}_K$, tedy je podtělesem $K$. Navíc, pro libovolné $\alpha \in K$ existuje celé $m$ s $m \alpha \in \mathcal{O}_K$, tedy $\alpha = \frac{m \alpha}{m}$ je podílem dvou prvků $\mathcal{O}_K$, tudíž $K \subseteq L$. \hfill $\square$\\

Známe okruh celých algebraických čísel těles $\mathbb{Q}$ a $\mathbb{Q}(i)$. V~libovolném kvadratickém tělese však dokážeme $\mathcal{O}_K$ za pomocí znalosti řešení kvadratické rovnice jednoduše popsat též.

 \begin{veta}\label{cela}
Nechť $m \neq 0,1$ je bezčtvercové celé číslo a $K = \mathbb{Q}(\sqrt{m})$ je algebraické číselné těleso. Pak platí:
\begin{equation*}
\mathcal{O}_K = \begin{cases}
      \mathbb{Z}[\sqrt{m}], & \textit{pokud} \quad  m \equiv 2,3 \pmod{4},\\
      \mathbb{Z}\left[\frac{1+\sqrt{m}}{2}\right], & \textit{pokud} \quad m \equiv 1 \pmod{4}.
    \end{cases}
\end{equation*}
\end{veta}

\noindent \textit{Důkaz.} Jistě $\mathbb{Z}[\sqrt{m}]$, resp. $\mathbb{Z}\left[\frac{1+\sqrt{m}}{2}\right]$, je podmnožinou $\mathcal{O}_K$, neboť minimální polynomy prvků $a+b\sqrt{m}$, resp. $a+b\frac{1+\sqrt{m}}{2}$, jsou po řadě $(x-a)^2 - b m^2$, resp. $(x-a)^2 - bx + ab + b^2 \frac{1-m}{4}$.

Ze tvaru řešení kvadratických rovnic plyne, že prvky $\mathcal{O}_K$ jsou ve tvaru $\frac{a+b\sqrt{m}}{2}$ pro $a,b \in \mathbb{Z}$. Zjevně pro $b \neq 0$ sdílí $\frac{a+b\sqrt{m}}{2}$ a $\frac{a-b\sqrt{m}}{2}$ minimální polynom, ten je proto roven:
\begin{equation*}
\left(x - \frac{a+b\sqrt{m}}{2} \right)\left( x - \frac{a-b\sqrt{m}}{2}\right) = x^2 - ax + \frac{a^2 - b^2 m}{4}.
\end{equation*} 
Pokud $\frac{a+b\sqrt{m}}{2} \in \mathcal{O}_K$, je tento monický polynom definovaný nad celými čísly. Proto $a^2 - b^2 m$ je dělitelné čtyřmi. Je-li $m$ je sudé, je $a$ též, tedy $a^2 $ je dělitelné čtyřmi. Za předpokladu, že $m$ je bezčtvercové, je $m \equiv 2 \pmod{4}$, tedy i $b$ je sudé.

Nyní již předpokládejme, že $m$ je liché. Pokud je $m \equiv 3 \pmod{4}$, platí $4 \mid a^2 + b^2$, což nutně znamená $2 \mid a,b$, protože kvadráty dávají zbytky $0,1$ po dělení čtyřmi. Pak $\frac{a+b\sqrt{m}}{2} \in \mathbb{Z}[\sqrt{m}]$. Konečně uvažme $m \equiv 1 \pmod{4}$. Máme $a^2 \equiv b^2 \pmod{4}$, tedy $a \equiv b \pmod{2}$. To ale znamená, že $\frac{a+b\sqrt{m}}{2} \in \mathbb{Z}\left[\frac{1+\sqrt{m}}{2}\right]$. \hfill $\square$\\

\begin{poznamka}
Okruh $\mathcal{O}_K$ v kvadratickém tělese $K = \mathbb{Q}(\sqrt{m})$ s $m$ bezčtvercovým můžeme kompatněji vyjádřit jako $\mathbb{Z}\left[\frac{d+\sqrt{d}}{2}\right]$, kde $d = m$, pokud $m \equiv 1 \pmod{4}$, a $4m$ jinak. Toto $d$ se nazývá \textit{diskriminant} číselného tělesa $K$.
\end{poznamka}

Každé číselné těleso je jednoduchým rozšířením racionálních čísel, platí však obdobná vlastnost pro okruhy celých algebraických čísel a celá čísla? U kvadratických těles jsme si to právě potvrdili, tělesa vyšších řádů tentokrát tuto vlastnost ne nutně sdílí. Minimální příklad se dokonce nachází již mezi kubickými tělesy, konkrétně $\mathbb{Q}(\sqrt[3]{19})$, viz \cite[Ex. 2.3.]{Conrad3}.

Ke konci této sekce ještě zběžně definujme \textit{pořádky}, tj. podokruhy číselného tělesa, které mají rank shodný se stupněm tělesa.

\begin{definice}
Okruh $\mathcal{O}$ obsažen v číselném tělese $K$ nazveme \textit{pořádkem}, pokud je volným $\mathbb{Z}$-modulem ranku $[K:\mathbb{Q}]$.
\end{definice}

Nejprve si všimněme, že věta \ref{cela} říká, že okruh celých algebraických čísel kvadratického tělesa je pořádkem. Tuto vlastnost dokonce sdílí všechny okruhy $\mathcal{O}_K$ v číselném tělese $K$.

\begin{veta}
Okruh $\mathcal{O}_K$ je pořádkem $K$.
\end{veta}
\noindent \textit{Důkaz.} Největší problém, který při dokazování tohoto tvrzení musíme překonat, je fakt, že $\mathcal{O}_K$ je konečně generovaný $\mathbb{Z}$-modul. Ten se klasicky dokazuje s pomocí \uv{stopy}, kterou si představíme za chvíli. Pro tuto část se proto odkazujeme na \cite[Lemma (2.9)]{Neukirch}, kde je ukázáno, že $\mathcal{O}_K$ je konečně generovaný a počet jeho generátorů je shora ohraničen $n$. 

Ať nyní $\lbrace a_1,\dots,a_n \rbrace$ je báze vektorového prostoru $K/\mathbb{Q}$. Podle lemmatu \ref{asob} existují nenulová celá $m_i$ taková, že každé z čísel $m_i a_i$ je celé algebraické. Protože $a_i$ musela být navzájem lineárně nezávislá nad $\mathbb{Q}$, čísla $m_i a_i$ jsou lineárně nezávislá nad $\mathbb{Z}$. Volný $\mathbb{Z}$-modul $\mathcal{O}_K$ proto obsahuje volný $\mathbb{Z}$-modul $\mathbb{Z}[m_1 a_1,\dots, m_n a_n]$ ranku $n$ a~díky větě \ref{podmodul} sám má tedy rank roven $n$. \hfill $\square$\\

Mezi pořádky má $\mathcal{O}_K$ speciální postavení, je totiž vzhledem k inkluzi největší.

\begin{veta}\label{podporadek}
Nechť $K$ je číselné těleso stupně $n$ a $\mathcal{O}$ jeho pořádek. Pak $\mathcal{O}$ je podmodulem $\mathcal{O}_K$.
\end{veta}

\noindent \textit{Důkaz.} Buď $\lbrace a_1, \dots,a_n \rbrace$ báze $\mathcal{O}$ jakožto $\mathbb{Z}$-modulu. Protože $\mathcal{O} = \mathbb{Z}[a_1,\dots,a_n]$ je volný modul a $\mathbb{Z}[a_i]$ jsou jeho podmoduly, podle věty \ref{podmodul} jsou všechny volné. Díky větě \ref{alg} jsou $a_i$ celá algebraická čísla, tedy $a_i \in \mathcal{O}_K$. Protože $\mathbb{Z} \subseteq \mathcal{O}_K$, leží každá $\mathbb{Z}$-lineární kombinace $a_i$ v~$\mathcal{O}_K$, jinak řečeno $\mathcal{O} \subseteq \mathcal{O}_K$. \hfill $\square$\\

O $\mathcal{O}_K$ tak můžeme hovořit jako o \uv{maximálním} pořádku.

\begin{definice}
Buď $\mathcal{O}$ pořádek číselného tělesa $K$. Pak \textit{vodič} $\mathcal{O}$ \textit{v} $\mathcal{O}_K$ definujeme jako index $\vert \mathcal{O}_K / \mathcal{O}\vert$.
\end{definice}

Kromě faktu, že pořádky jsou $\mathbb{Z}$-moduly ranku $n$ a obsaženy v okruhu $\mathcal{O}_K$, můžeme je přesně vzhledem k maximálnímu pořádku charakterizovat.

\begin{veta}
Nechť $\mathcal{O}$ je pořádek číselného tělesa $K$ stupně $n+1$. Pak existují čísla $a_1,\dots,a_n \in K$ a celá $k_1,\dots,k_n$ splňující $k_i \mid k_{i+1}$ a:
\begin{equation*}
\mathcal{O}_K = \mathbb{Z}[a_1,a_2,\dots,a_n], \qquad \qquad \mathcal{O} = \mathbb{Z}[k_1 a_1, k_2 a_2, \dots, k_n a_n].
\end{equation*}
\end{veta}
\noindent \textit{Důkaz.} Buďte $\lbrace 1,\alpha_1,\dots,\alpha_{n-1} \rbrace$, resp. $\lbrace 1,\beta_1,\dots,\beta_{n-1} \rbrace$ báze $\mathcal{O}_K$ a $\mathcal{O}$ jakožto $\mathbb{Z}$-modulů. Zobrazení $\xi:\mathcal{O}_K \longrightarrow \mathcal{O}$ dané $\alpha_i \longmapsto \beta_i$ pro každé $i$ můžeme reprezentovat $n \times n$ maticí $M$ nad celými čísly. Je známé (viz \textit{Smithova normální forma}), že existují $n \times n$ celočíselné matice $L,N$ takové, že $LMN$ je diagonální matice, jejíž hlavní diagonála obsahuje celá $k_i$ splňující $k_i \mid k_{i+1}$. Tato čísla jsou ne všechna nulová, protože index $\mathcal{O}$ v $\mathcal{O}_K$ je konečný. Násobení maticemi pouze mění bázi $\mathcal{O}$, tedy můžeme položit $LMN$ matici udávající $\xi^\prime : \mathcal{O}_K \longrightarrow \mathcal{O}$ a ta definuje $k_i$ ze zadání. \hfill $\square$\\

Jelikož víme, že $\mathcal{O}_K$ obsahuje bázi vektorového prostoru $K/\mathbb{Q}$, každý jeho pořádek ji obsahuje též. O co víc, předchozí věta aplikovaná na pořádky kvadratických těles tvrdí, že můžeme zvolit $\lbrace 1,d \rbrace$ a $\lbrace 1,fd \rbrace$ báze $\mathcal{O}_K$, resp. $\mathcal{O}$ s $f$ vodičem $\mathcal{O}$ v $\mathcal{O}_K$, a proto symbolicky říci $\mathcal{O} = \mathbb{Z}+ f\mathcal{O_K}$. Z toho navíc plyne, že platí inkluze pořádků $\mathcal{O} \subseteq \mathcal{O}^\prime$ pouze a jenom, když vodič $\mathcal{O}^\prime$ dělí vodič $\mathcal{O}$.

Konečně, protože každý pořádek obsahuje racionální bázi pro $K$, můžeme si uvést ještě  jednu ekvivalentní definici pořádku pomocí tenzorového součinu.
\begin{dusledek}\label{poradektensor}
Buď $K$ číselné těleso. Pak podokruh $\mathcal{O} \subseteq K$ je pořádkem, právě pokud je volným $\mathbb{Z}$-modulem splňujícím $\mathbb{Q} \otimes_{\mathbb{Z}} \mathcal{O} \cong K$.
\end{dusledek}

Pozor, pořádky se od okruhu $\mathcal{O}_K$ obecně liší v několika zásadních oblastech, ke kterým se budeme vracet. Pro jedno, pořádky nejsou nikdy celouzavřené nad jejich podílovým tělesem, každý prvek $\mathcal{O}_K \setminus \mathcal{O}$ je totiž celý nad $\mathbb{Z}$ a tedy i nad $\mathcal{O}$. Ku příkladu číslo $\frac{1+\sqrt{5}}{2}$ je celé nad pořádkem $\mathbb{Z}[\sqrt{5}] \subseteq \mathbb{Q}(\sqrt{5})$ a neleží v něm.

\section{Norma, stopa a zkoumání dělitelnosti v okruzích}

V této části užijeme pár základních poznatků ze studia lineární algebry ke studiu vlastnosti minimálních polynomů prvků číselného tělesa. Po čtenáři tedy požadujeme, aby se alespoň \uv{stopově} orientoval v této teorii, pro velmi podrobný úvod do této oblasti matematiky může posloužit \cite{Johnson}.

Mějme $K$ číselné těleso a $L$ jeho konečné rozšíření s $[L:K] = n$. Zobrazení na $L$ dané předpisem $a(x) : x \mapsto ax$, tedy násobení prvkem $a \in L$, definuje $K$-lineární endomorfismus vektorového prostoru $L$ nad $K$. Pokud si vybereme bázi $\lbrace \alpha_1,\dots,\alpha_n \rbrace$ prostoru $L$ nad $K$, zobrazení $a(x)$ působí na tuto bázi jako matice:
\begin{equation*}
\begin{bmatrix}
a(\alpha_1) \\
a(\alpha_2) \\
\vdots\\
a(\alpha_n)
\end{bmatrix} = \begin{pmatrix}
a_{11} & a_{12} & \cdots & a_{1n}\\
a_{21} & a_{22} & \cdots & a_{2n}\\
\vdots & \vdots & \ddots &\vdots\\
a_{n1} & a_{n2} & \cdots& a_{nn}
\end{pmatrix} \begin{bmatrix}
\alpha_1 \\
\alpha_2 \\
\vdots\\
\alpha_n
\end{bmatrix},
\end{equation*}
kde $a_{ij} \in K$, a rozšiřuje se $K$-lineárně na celém $L$. Pokud vyjádříme $a = t_1 \alpha_1 + \cdots + t_n \alpha_n$ jako lineární kombinaci prvků báze, můžeme díky vyjádření $a(\alpha_i) = \sum_j a_{ij} \alpha_{j}$ jednoznačně určit celou matici.

My se zaměříme na případ $K=\mathbb{Q}$ a $L$ číselné těleso stupně $n$, který popíšeme jednodušeji. Víme, že $L$ je jednoduché rozšíření $\mathbb{Q}(\theta)$ s bází $\lbrace 1,\theta,\dots,\theta^{n-1} \rbrace$, vzhledem ke které budeme psát $a(x)$. Ukážeme, že toto zobrazení ve své podstatě souvisí s minimálním polynomem prvku $a$ nad racionálními čísly.

\begin{definice}
Ať $K = \mathbb{Q}(\theta)$ je číselné těleso a $\tau$ je jeho prvek. Pak pod pojmem \textit{charakteristický polynom} $\tau$ rozumíme charakteristický polynom lineárního zobrazení $\tau(x)$. 
\end{definice}

Nejprve se podívejme na zobrazení $\theta(x)$, kde $\theta$ má minimální polynom nad $\mathbb{Q}$ roven $x^n + b_{n-1} x^{n-1} + \cdots + b_0$. Máme dáno $\theta \cdot \theta^{i-1} = \sum_j a_{ij} \theta^{j-1}$, tedy protože prvky množiny $\lbrace 1,\theta,\dots,\theta^{n-1} \rbrace$ jsou lineárně nezávislé nad $\mathbb{Q}$, můžeme psát $\theta(x)$ jako akci matice $M_{\theta}$ udávající $\theta(x)$:
\begin{equation*}
 \begin{pmatrix}
0 & 1 & 0 & \cdots & 0\\
0 & 0 & 1 & \cdots & 0\\
\vdots & \vdots & \vdots & \ddots &\vdots\\
0 & 0 & \cdots & 0& 1\\
-b_0 & -b_1 & \cdots & -b_{n-2} & -b_{n-1}
\end{pmatrix} 
\end{equation*}
na $\mathbb{Q}(\theta)$. Charakteristický polynom $\theta$ je charakteristický polynom matice $M_\theta$, který je daný $\det (x I - M_{\theta})$,  tedy:
\begin{equation*}
 \det \begin{pmatrix}
x & -1 & 0 & \cdots & 0\\
0 & x & -1 & \cdots & 0\\
\vdots & \vdots & \ddots & \ddots &\vdots\\
0 & 0 & \cdots & x& -1\\
b_0 & b_1 & \cdots & b_{n-2} & x +b_{n-1}
\end{pmatrix} ,
\end{equation*}
což je $b_0 + b_1 x + \cdots + b_{n-1} x^{n-1} + x^n$, minimální polynom prvku $\theta$.

Nyní již uvažme libovolné $ \tau \in \mathbb{Q}(\theta)$. Připomeňme známý fakt ze studia tělesových rozšíření: $[\mathbb{Q}(\theta) : \mathbb{Q}(\tau)] \cdot [\mathbb{Q}(\tau) : \mathbb{Q}] = [\mathbb{Q}(\theta) : \mathbb{Q}]$, speciálně stupeň tělesa $\mathbb{Q}(\tau)$ dělí stupeň $\mathbb{Q}(\theta)$, a totéž proto platí pro stupně minimálních polynomů příslušných $\tau$ a $\theta$.

\begin{lemma}
Nechť $K = \mathbb{Q}(\theta)$ je číselné těleso stupně $mn$ a $\tau \in K$ má minimální polynom stupně $m$. Pak charakteristický polynom $\tau(x)$ je $n$-tou mocninou minimálního polynomu $\tau$ nad $\mathbb{Q}$. 
\end{lemma}
\noindent \textit{Důkaz.} Uvažme $\lbrace 1,\theta,\dots,\theta^{mn-1} \rbrace$ bázi $K$ nad $\mathbb{Q}$ a $\lbrace b_1,\dots,b_m \rbrace$ bázi $K$ nad $\mathbb{Q}(\tau)$. Množina všech prvků $\theta^i b_j$ je zřejmě nad $\mathbb{Q}$ lineárně nezávislá a tvoří proto bázi prostoru $K/\mathbb{Q}$. Vzhledem k této bázi snadným porovnáním koeficientů zjistíme, že $\tau(x)$ působí na $\mathbb{Q}(\theta)$ jako blokově diagonální matice obsahující $n$ matic:
\begin{equation*}
 \begin{pmatrix}
0 & 1 & 0 & \cdots & 0\\
0 & 0 & 1 & \cdots & 0\\
\vdots & \vdots & \vdots & \ddots &\vdots\\
0 & 0 & \cdots & 0& 1\\
-c_0 & -c_1 & \cdots & -c_{m-2} & -c_{m-1},
\end{pmatrix},
\end{equation*}
kde $x^m + c_{m-1} x^{m-1} + \cdots  + c_0$ je minimální polynom $\tau$ nad racionálními čísly. Charakteristický polynom $\tau$ je pak jeho minimální polynom umocněn na $n$-tou mocninu, což je v~souladu s větou Cayley-Hamiltona zaobírající se charakteristickými polynomy matic. \hfill $\square$\\

Kvůli této korespondenci zobrazení $\tau(x)$ a minimálního polynomu $\tau$ definujeme pojmy stopa a norma, které nám pomohou s prací v okruzích, například při zkoumání dělitelnosti.

\begin{definice}
Buď $K$ číselné těleso a $\tau$ jeho prvek. Pak definujeme jeho \textit{stopu} $\Tr(\tau)$ a~\textit{normu} $N(\tau)$ jako stopu, resp. determinant matice udávající $\tau(x)$:
\begin{align*}
\Tr_K(\tau) &:= \Tr M_\tau,\\
N_K(\tau) &:= \det M_{\tau}.
\end{align*} 
\end{definice}

Norma i stopa prvků číselného tělesa jsou tedy racionální čísla. Podotkněme, že stopa i determinant matice nezávisí na konkrétní volbě báze, definice výše je proto korektní. Abychom se nezadusili notací, pokud bude jasné těleso nad kterým pracujeme, budeme psát jednoduše $Tr(\tau),N(\tau)$.

Pojďme se si spočíst normu a stopu pár prvků v číselným tělesech, abychom získáli intuici, s čím to pracujeme.

\begin{priklad}
V tělese $\mathbb{Q}(\sqrt{-2})$ mějme číslo $a+b\sqrt{-2}$. Báze tohoto tělesa jakožto vektorového prostoru nad $\mathbb{Q}$ je $\lbrace 1,\sqrt{-2}\rbrace$, pojďme spočíst akci $(a+b\sqrt{-2})(x)$ na tomto tělese, k čemuž nám stačí určit akci na bázi:
\begin{align*}
(a+b\sqrt{-2}) \cdot 1 &= a + b \sqrt{-2},\\
(a+b\sqrt{-2}) \cdot \sqrt{-2} &= -2b + a\sqrt{-2},
\end{align*}
tedy $(1+2\sqrt{2})(x)$ působí na $\mathbb{Q}(\sqrt{-2})$ jako matice:
\begin{equation*}
\begin{pmatrix}
a & b\\
-2b & a 
\end{pmatrix}.
\end{equation*}    
Její stopa je $2a$ a determinant $a^2+2b^2$, což souhlasí s tím, že minimální polynom $a+b\sqrt{-2}$ je pro $b \neq 0$ roven $x^2 - 2ax + a^2+2b^2$ a pro $b=0$ jednoduše $x-a$.

Uvažme dále těleso $\mathbb{Q}(\theta)$, kde $\theta$ je kořenem polynom $x^3 - x + 3$, který je zjevně iracionální. Libovolný jeho prvek $\tau$ vyjádřený podle báze $\lbrace 1,\theta,\theta^2 \rbrace$ jako $a + b\theta + c \theta^2$ působí na bázi jako:
\begin{align*}
(a + b\theta + c \theta^2) \cdot 1 &= a + b\theta + c \theta^2,\\
(a + b\theta + c \theta^2) \cdot \theta &=  -3c + (a+c)\theta + b \theta^2,\\
(a + b\theta + c \theta^2) \cdot \theta &=  -3b + (b-3c)\theta + (a+c) \theta^2,
\end{align*}
tedy udává matici:
\begin{equation*}
\begin{pmatrix}
a & b & c\\
-3c & a+c & b\\
-3b & b-3c & a+c 
\end{pmatrix}
\end{equation*} 
se stopou $3a+2c$ a determinantem $ a^3 - 3 b^3 + 2 a^2 c + 3 b c^2 + 9 c^3 - ab^2 - 9 ab c - ac^2$. Buď je $\tau$ racionální číslo, či je jeho minimální polynom roven třemi. V prvním případě je jeho stopa $3a$ a norma $a^3$, v druhém případě stopa $a$ a norma rovna determinantu $M_\tau$.
\end{priklad}

Když máme dobrou představu o normě a stopě, pojďme se o těchto funkcích ukázat několik málo důležitých faktů. K tomu nám pomohou klasické výsledky ohledně stop a~determinantů matic.

\begin{veta}
Norma je mutiplikativní a stopa je $\mathbb{Q}(\theta)$-lineární funkce.
\end{veta}
\noindent \textit{Důkaz.} Důkaz plyne z faktů, že $\det(A \cdot B) = \det(A) \cdot \det(B)$ a $\Tr(kA+\ell B) = \Tr(kA) + \Tr(\ell B) = k\Tr(A)+\ell \Tr(B)$ pro libovolné čtvercové matice $A,B$ a $k,\ell \in \mathbb{Q}(\theta)$. \hfill $\square$\\

Normu a stopu $\tau$ můžeme díky vlastnostem mapy $\tau(x)$ pevněji ukotvit k minimálnímu polynomu $\tau$:

\begin{veta}
Buď $K$ číselné těleso stupně $n$ a $\tau$ jeho prvek s minimálním polynomem $x^k + c_{k-1} x^{k-1} + \cdots  + c_0$ nad $\mathbb{Q}$. Pak:
\begin{align*}
\Tr(\tau) &= - n/k \cdot c_{k-1},\\
N(\tau) &= (-1)^n {c_0}^{n/k}.
\end{align*} 
\end{veta}

Ekvivalentně věta říká, že pokud $\tau, \tau_2,\dots,\tau_n$ jsou kořeny charakteristického polynomu $\tau$, včetně multiplicity, platí $\Tr_K(\tau) = \tau + \tau_2+\cdots+\tau_n$ a $N(\tau) = \tau \cdot \tau_2 \cdots \tau_k$. Pokud je tedy $\tau$ celé algebraické číslo, jeho norma i stopa jsou celá čísla.\\

\noindent\textit{Důkaz.} Tvar stopy plyne ihned z faktu, že stopa matice je součtem prvků po hlavní diagonále.  Determinant blokové matice je součin determinantů bloků na diagonále, tedy $n/k$-tá mocnina determinantu matice:
\begin{equation*}
 \begin{pmatrix}
0 & 1 & 0 & \cdots & 0\\
0 & 0 & 1 & \cdots & 0\\
\vdots & \vdots & \vdots & \ddots &\vdots\\
0 & 0 & \cdots & 0& 1\\
-c_0 & -c_1 & \cdots & -c_{k-2} & -c_{k-1},
\end{pmatrix}.
\end{equation*}
což je $(-1)^k c_{0}$. \hfill $\square$\\

Minimální polynomy prvků $\alpha$ a $\beta$ nám toho říkají pouze pramálo o minimálních polynomech čísel $\alpha+\beta$ či $\alpha \cdot \beta$, nicméně za pomocí spojení minimálních polynomů s normami a~stopami můžeme za pomocí vět výše přesně popsat některé jejich koeficienty.

Protože je norma multiplikativní a na celých algebraických číslech celočíselná, můžeme ji propojit s dělitelností v okruzích. Pokud $b = ac$ pro $a,b,c \in R$ nenulová, máme $N(b) = N(ac) = N(a)N(c)$.
\begin{veta}\label{normalni}
Mějme $a,b \in R \subseteq \mathcal{O}_K$ nenulová pro $K$ číselné těleso. Pokud $a$ dělí $b$, ve smyslu $b = a \cdot c$ pro $c \in R$, tak platí:
\begin{equation*}
N(a) \mid N(b).
\end{equation*}
\end{veta}

Pokud $a$ je v okruhu $R$ invertibilní, tedy $a\cdot b = 1$ pro nějaké $b \in R$, nutně platí $N(a)N(b) = N(ab) = N(1) = 1$ a díky celočíselnosti norem je $N(a)$ rovno $\pm 1$. 
\begin{definice}
Prvek $a \in R$, který je v $R$ invertibilní, nazveme \textit{jednotkou}.
\end{definice}

\begin{definice}
Pokud je podílem dvou prvků $a,b \in R$ jednotka, nazveme je \textit{asociované}.
\end{definice}

Jednotky v okruzích tvoří multiplikativní grupu, přičemž v okruhu celých algebraických čísel kvadratických tělesech jsou určena řešeními kvadratických forem. V okruhu celých čísel tělesa $\mathbb{Q}$ jsou jednotky zjevně pouze $\pm 1$, Gaussova celá čísla připouští multiplikativní inverze prvků $\pm 1$ i $\pm i$. Případ reálných kvadratických těles $\mathbb{Q}(\sqrt{d})$, tedy $0 < d \not\equiv 1 \pmod{4}$, je obzvlášťe zajímavý, jednotky $a+b\sqrt{-d} \in \mathbb{Z}[\sqrt{-d}]$ totiž splňují:
\begin{equation*}
a^2 - d b^2 = \pm 1,
\end{equation*}
tedy rozšířenou Pellovu rovnici.

Studium Pellových rovnic \cite[2. díl]{Prase} poté poukazuje na fakt, že tato grupa je vesměs cyklická, tedy že všechny jednotky vygenerujeme jako $\pm \omega^n$ pro $n \in \mathbb{Z}$ a $\omega$ tzv. \textit{fundamentální jednotku} tohoto okruhu.

Od jednotek se přesuňme na zobecnění prvočísel, tzv. \textit{ireducibilních} prvků, v okruzích. 
\begin{definice}
Prvek $a \in R$ nazveme \textit{ireducibilním}, nelze-li jej zapsat jako součin dvou prvků $R$, z nichž ani jeden není jednotka.
\end{definice}

Multiplikativita normy tvrdí, že prvky s prvočíselnou normou jsou nad $R$ ireducibilní. Zajímalo by nás tedy, zda dokážeme s ireducibilními prvky operovat podobně jako s~prvočísly, tedy rozkládat čísla na ireducibilní prvky. Takový rozklad v obecném okruhu je $a \neq 1$ zjevně existuje, a díky multiplikativitě normy je pro nenulové prvky konečný, bohužel však ne vždy je jednoznačně určený. Koncept dělení v okruzích přivádí na mysl dělení se zbytkem.

Ze školních lavic víme, že v dokážeme v celých číslech dělit se zbytkem. Tuto vlastnost ale sdílí některé další okruhy, neprominentněji $\mathbb{Z}[i]$. Ukážeme si tedy, jak na to.

Vskutku, ukážeme, že pro libovolná nenulová $a,b \in \mathbb{Z}[i]$ můžeme zvolit Gaussova celá čísla $q,r$ taková, že $a = bq+r$ a $N(r) < N(b)$, kde normu bereme normu komplexního čísla.  Norma je multiplikativní, tedy ekvivalentně píšeme $N\left(\frac{r}{b}\right) < 1$ a $\frac{a}{b} = q+\frac{r}{b}$. Existence takových $r$ a $b$ je nicméně zřejmá, pokud se na problém podíváme geometricky. Rovnice $N(z) \leqslant 1$ definuje v~komplexní rovině jednotkový kruh se středem v počátku, tedy požadujeme, aby šlo zvolit $q \in \mathbb{Z}[i]$, které je na méně než jednotkovou vzdálenost od libovolného komplexního čísla $z$. To jistě dokážeme, protože Gaussova celá čísla tvoří v~komplexní rovině jednotkovou mřížku.

\begin{figure}
\begin{center}
\begin{tikzpicture}[scale=2]
	\draw[style=help lines] (0,0) grid (3,3);
	\draw[style=help lines] (3,1) -- (4,1);
	\draw[style=help lines] (3,2) -- (4,2);
	\draw[style=help lines] (3,3) -- (4,3);
	\draw[style=help lines] (1,3) -- (1,4); 
	\draw[style=help lines] (2,3) -- (2,4);
	\draw[style=help lines] (3,3) -- (3,4); 
    \fill[red] (2.33,1.67) circle (0.5 pt);
  \draw (2.33,1.67) circle (28.4 pt);
  \draw[->] (-0.2,0) -- (4,0);
  \draw[->] (0,-0.2) -- (0,4);
  \draw[->] (2.275,1.725) -- (2.055,1.945);
  \draw[densely dotted] (2.33,1.67) -- (3.33,1.67);
  \node[above left, outer sep=2pt] (0,0) {0};
  \node[left, outer sep=2pt] at (0,1) {$i$};  
  \node[left, outer sep=2pt] at (0,2) {$2i$};  
  \node[left, outer sep=2pt] at (0,3) {$3i$};  
  \node[below, outer sep=2pt] at (1,0) {$1$}; 
  \node[below, outer sep=2pt] at (2,0) {$2$}; 
  \node[below, outer sep=2pt] at (3,0) {$3$};
  \node[below right, outer sep=2pt] at (2.33,1.67) {$\frac{7+5i}{3}$};
\end{tikzpicture}
\end{center}
\caption{Nejbližší mřížový bod je blíže, než $1$.}
\end{figure}

Díky tomuto poznatku můžeme v $\mathbb{Z}[i]$ dělit se zbytkem, tudíž existuje pro libovolná $a,b  \in \mathbb{Z}[i]$ (až na násobení jednotkou) jednoznačný nejvyšší společný dělitel, a výše uvedená vlastnost efektivně dává vzniku Euklidovu algoritmu v $\mathbb{Z}[i]$. Bezoutova identita proto platí pro dva členy a tedy i pro libovolný počet Gaussových celých čísel.
\begin{definice}
Obor integrity $R$ takový, že existuje funkce $N : R \longrightarrow \mathbb{N}_0$ splňující pro každá $a,b \in R$ a $b$ nenulové:
\begin{enumerate}
\item $N(a) = 0$, právě pokud $a = 0$,
\item $N(a) \leqslant N(ab)$,
\item Existují $q,r \in R$ splňující $a = bq + r$ a $N(r) < N(b)$,
\end{enumerate}
nazveme \textit{euklidovým}.
\end{definice}

\begin{poznamka}
Pouze konečně mnoho okruhů celých algebraických čísel imaginárních kvadratických těles $\mathbb{Q}(\sqrt{d})$ pro $d<0$ je euklidových, vyhovující $d$ se nazývají \textit{Heegnerova}. Z~nich v~absolutní hodnotě nejvyšší je $-163$.
\end{poznamka}

V euklidově okruhu můžeme díky vlastnostem pokládaným na $N$ dělit se zbytkem obdobně jako v případě $\mathbb{Z}[i]$. V těchto oborech díky platnosti Bezoutovy rovnosti vykazují ireducibilní prvky podobné vlastnosti jako prvočísla v~celých číslech.

\begin{veta}
Buď $R$ euklidův okruh a $p \in R$ ireducibilní. Pokud pro $a,b \in R$ platí $p \mid ab$, pak buď $p \mid a$, či $p \mid b$.
\end{veta}
\noindent \textit{Důkaz.} Nechť naopak platí $p \mid ab$ a $p$ nedělí ani jeden z činitelů. Existuje (až na násobení jednotkou) jednoznačný společný dělitel $d$ prvků $p$ a $a$, který díky ireducibilitě $p$ je buď s $p$ asociovaný, či je jednotka. Pokud by nastal první případ, pak $p \mid a$, spor. Je tak $d$ jednotkou, vhodným přenásobením $a$ jednotkou uvažujme $d = 1$. Z Bezoutovy rovnost existují $x,y \in R$ splňující $xa+yp = 1$. Analogicky dojdeme k existenci $z,t \in R$ s $zb+tp = 1$. Vynásobením těchto rovnosti získáme:
\begin{equation*}
1 = (xa+yp)(zb+tp) = xyab + p(xta+yzb+ytp),
\end{equation*}
díky předpokladu úlohy $p$ dělí pravou stranu a tedy i levou, což je hledaný spor. \hfill $\square$\\

S předchozí větou na mysli se není příliš obtížné dovtípit, že euklidovy okruhy připouští jednoznačný rozklad, protože se ireducibilní prvky opravdu chovají podobně jako prvočísla.

\begin{dusledek}
Buď $R$ euklidův okruh. Pak se každý jeho prvek jednoznačně (až na pořadí prvků a násobení jednotkou) rozkládá na součin ireducibilních prvků a jednotky.
\end{dusledek}
\noindent \textit{Důkaz}. Kvůli multiplikativitě normy je každý nenulový prvek $n \in R \setminus \lbrace 1 \rbrace$ rozložitelný na konečně mnoho činitelů. Dejme tomu, že existují dvě posloupnosti $p_1, \dots, p_k$ a~$q_1,\dots, q_{\ell}$ ireducibilních prvků takových, že platí:
\begin{equation*}
u_1 p_1 \cdots p_k = n = u_2 q_1 \cdots q_{\ell}
\end{equation*}
pro nějaké jednotky $u_i \in R$. Platí, že $p_1$ dělí druhý rozklad, tedy dělí jedno z $q_i$, bez újmy na obecnosti ať to je $q_1$, tedy $q_1 = v_1 p_1$. Tento proces opakujeme s číslem $\frac{n}{p_1} =\frac{u_1}{v_1} p_2 \cdots p_k = u_2 q_2 \cdots q_{\ell}$, čímž docházíme k tomu, že množiny $\lbrace p_1, \dots, p_k \rbrace$ a~$\lbrace q_1,\dots,q_{\ell} \rbrace$ jsou až na asociaci shodné, včetně násobnosti, což jsme chtěli. \hfill $\square$\\
 
V obecném okruhu, dokonce ani $\mathcal{O}_K$, jednoznačnost rozkladu však neplatí. Klasický protipříklad dává okruh $\mathbb{Z}[\sqrt{-5}]$ a dva rozklady čísla $6 = 2 \cdot 3 = (1+\sqrt{-5})(1-\sqrt{-5})$. Ukážeme, že všichni čtyři činitelé jsou v $\mathbb{Z}[\sqrt{-5}]$ ireducibilní.

Norma obecného prvku $a+b\sqrt{-5}$ našeho okruhu je daná $a^2+5b^2$, tedy normy našich dělitelů jsou po řadě rovny $4,9,6$ a $6$. Pokud by nějaký z nich šel rozložit jako součin dvou ireducibilních prvků, s ohledem na multiplikativitu a nezápornost normy v $\mathbb{Z}[\sqrt{-5}]$ by oba měly normu buď $2$ či $3$. Nicméně $2$ a $3$ nejsou kvadratické zbytky modulo $5$, tedy rovnost $2,3 = N(a+b\sqrt{-5}) = a^2+5b^2$ nemá řešení, taková čísla proto neexistují a všichni čtyři dělitelé jsou ireducibilní. Tyto rozklady jsou pak díky uvedeným normám různé. V příštích sekcích se k jednoznačnosti rozkladu ještě vrátíme, nicméně prozatím mějme na paměti, že ne vždy nutně platí.

Některé prvky číselných okruhu můžeme tedy vyjádřit jako součin ireducibilních prvků vícero různými způsoby, přinejmenším bychom alespoň očekávali, že počet ireducibilních faktorů je vždy konzistentní. Opět bychom se však mýlili, okruh $\mathbb{Z}[\sqrt{-14}]$ poskytuje následující dva rozklady čísla $81$:
\begin{equation*}
3 \cdot 3 \cdot 3 \cdot 3 = 81 = (5+2\sqrt{-14})(5-2\sqrt{-14}),
\end{equation*}
rozborem norem a dělitelností opět můžeme dospět k závěru, že všichni tři přítomní dělitelé jsou ireducibilní a dělitelé z jednotlivých rozkladů nejsou asociovaní.

Pojďme se ještě na chvíli pozastavit u okruhu $\mathbb{Z}[i]$, ve kterém jednoznačnost rozkladu platí, a ukázat jednu roztomilou aplikaci předchozí věty. Norma Gaussova celého čísla je $N(a+bi) = a^2+b^2$, tedy druhá mocnina klasické komplexní absolutní hodnoty. Její vlastnosti pomohou odhalit, přesně která přirozená čísla jsme schopni vyjádřit jako součet dvou čtverců.


\begin{veta} 
Přirozené číslo $n$ lze vyjádřit jako součet dvou čtverců, právě pokud $n$ není dělitelné prvočíslem $p \equiv -1 \pmod{4}$ v liché mocnině.
\end{veta}

\noindent \textit{Důkaz.} Odůvodníme, proč v okruhu $\mathbb{Z}[i]$ jsou prvočísla $p \equiv -1 \pmod{4}$ a $\pm 1 \pm i$ ireducibilní prvky a naopak prvočísla $p \equiv 1 \pmod{4}$ již rozložitelná jsou. Dejme tomu, že jsme schopni zapsat $p \equiv -1 \pmod{4}$ jako součin dvou prvků, obou ne jednotek. Rovnost $p = ab$ v $\mathbb{Z}[i]$ díky multiplikativitě normy znamená $p^2 = N(p)=N(ab)=N(a)N(b)$. Protože norma komplexního čísla je nezáporná a $a,b$ nejsou jednotky, platí $N(a)=N(b)=p$, tedy pro $a=x+yi$ platí $x^2+y^2 = p$. To ale porušuje pravidlo, že čtverce dávají pouze zbytky $0$ a~$1$ modulo čtyřmi. Tato $p$ jsou proto ireducibilní. Dále Gaussova celá čísla $\pm 1 \pm i$ s normou $2$ jsou jistě ireducibilní, a jsou jediná s normou $2$.

Pokud je naopak $p \equiv 1 \pmod{4}$ prvočíslo, je $-1$ kvadratický zbytek modulo $p$. Pro nějaké $x$ proto platí $p \mid x^2+1$, což můžeme v rámci $\mathbb{Z}[i]$ zapsat jako $p \mid (x+i)(x-i)$. Pokud by $p$ nešlo rozložit, muselo by dělit právě jednu ze závorek a tak $p \mid i$, což je nemožné. Existuje proto netriviální rozklad $ab = p$ s normou $N(a)N(b) = N(ab) = N(p)=p^2$, tedy $N(a)=N(b)=p$ pro nějaká Gaussova celá $a,b$. Pro $a = x+yi$ pak platí $p = N(a) = x^2+y^2$.

Jestliže $n$ je dělitelné čtyřmi či prvočíslem $p \equiv -1 \pmod{4}$ v liché mocnině, nelze zjevně zapsat jako součet dvou čtverců, protože kvadratické zbytky modulo $4$ jsou pouze $0$ a~$1$ a~$p \mid a^2+b^2$ znamená, že buď $-1$ je kvadratický zbytek modulo $p$, neboli $p \equiv 1 \pmod{4}$, či $p \mid a$ a~$p \mid b$. Naopak pokud nenastává ani jeden z těchto případů, můžeme každé prvočíslo $p \equiv 1 \pmod{4}$ dělící $n$ zapsat jako součet dvou čtverců, tedy díky rovnostem $(a^2+b^2)(c^2+d^2) = (ac-bd)^2+(ad+bc)^2$ a $q^2 a^2 = (qa)^2$ lze $n$ vyjádřit jako součet dvou čtverců. \hfill $\square$\\

Obdobné charakterizace můžeme provést rozkladem v ostatních euklidovských kvadratických okruzích, což staví základy charakterizace vyjadřování celých čísel kvadratickými formami. Podrobněji je toto téma studováno v \cite{Tomas}, či v předloze oné práce \cite{Cox}, na které je též založena notná část této kapitoly.

 Chtěli bychom tedy hledat strukturu, která poslouží tam, kde nás prvky $\mathcal{O}_K$ selhaly, u~jednoznačného rozkladu na ireducibilní prvky. Eduard Kummer v 19. století tento problém vyřešil vložením množiny algebraických celých čísel tělesa $K$ do množiny tzv. \textit{ideálních čísel}, která se jednoznačně rozkládají na součin \textit{ideálních prvočísel}. Tento koncept Richard Dedekind, další z~titánů teorie čísel, později nazval \textit{ideály}.

\section{Ideály}

\begin{definice}
Neprázdnou aditivní podgrupu $\mathfrak{a}$ komutativního okruhu $R$ takovou, že $a \cdot r \in \mathfrak{a}$ platí pro $a \in \mathfrak{a}, r \in R$ označíme jako ideál. 
\end{definice}
O ideálech můžeme proto přemýšlet jako o (jediných neprázdných) podmodulech $R$-modulu $R$.

Pokud $\mathfrak{a}$ je podgrupa $R$, tak faktorgrupa $R/\mathfrak{a}$ se stane okruhem, právě pokud $\mathfrak{a}$ je ideálem. Ideály tedy konstruujeme v podobném duchu jako normální podgrupy, kde podgrupa $H$ grupy $G$ je normální, právě když $G/H$ je grupa. Zobrazení $G \longrightarrow G/H$ přiřazující každému prvku $G$ jeho příslušnou třídu v $G/H$ je poté homomorfismus grup.

Každý ideál $\mathfrak{a} \subseteq R$ tedy definuje faktorokruh $R/\mathfrak{a}$, kde projekce $R \longrightarrow R/\mathfrak{a}$ redukující každé $r \in R$ na jeho příslušnou třídu v $R/\mathfrak{a}$ dává kanonický homomorfismus mezi těmito dvěma okruhy. Navíc homomorfismus jemu inverzní udává bijektivní zobrazení mezi třídami $R/\mathfrak{a}$ a ideály $R$ obsahující $\mathfrak{a}$.

\begin{definice}
Pokud $\theta_1,\dots,\theta_n \in R$ je konečná množina generátorů ideálu (ve smyslu $R$-modulu) $\mathfrak{a} \subseteq R$, značíme:
\begin{equation*}
\mathfrak{a} = (\theta_1,\dots,\theta_n).
\end{equation*}
\end{definice}

Ne každý ideál libovolného okruhu je konečně generovaný, například ideál $(x_1,x_2,\dots)$ v~okruhu $\mathbb{R}[x_1,x_2,\dots]$ s nekonečně mnoha proměnnými jistě konečně generovaný není, my si však dále odůvodníme, proč v pořádcích tomu tak je. Nejprve se však pozastavíme u okruhů zbytků.

Ideál generovaný prvkem $x^2+1$ v $\mathbb{Z}[x]$ má příslušný okruh zbytků $\mathbb{Z}[x]/(x^2+1)$ isomorfní okruhu $\mathbb{Z}[i]$, není tedy konečný, jako není v mnoha dalších \uv{divokých} okruzích. V pořádku $\mathcal{O}$ přesto každý ideál konečný okruh zbytků má.

\begin{veta}
Buď $\mathcal{O}$ pořádek číselného tělesa $K$ stupně $n$ a $\mathfrak{a} \subseteq \mathcal{O}$ ideál. Pak $\mathfrak{a}$ má v $\mathcal{O}$ konečný index.
\end{veta}
\noindent \textit{Důkaz.} Nulový ideál tvrzení zjevně splňuje, dále uvažme opak a nenulový prvek $x \in \mathfrak{a}$. Pokud $x_2,\dots,x_k$ jsou kořeny minimálního polynomu $x$, platí $N(x) = x \cdot x_2 \cdots x_k \in \mathfrak{a}$ je díky větě \ref{podporadek} celé číslo. Pak $\mathcal{O}/\mathfrak{a}$ je podokruhem $\mathcal{O}/(N(x))$. Pokud $\lbrace a_1,\dots,a_n \rbrace$ je báze pořádku $\mathcal{O}$ jako $\mathbb{Z}$-modulu, libovolné číslo $z \in \mathcal{O}$ můžeme vyjádřit jako $t_1 a_1 + \cdots + t_n a_n$ pro celá $t_i$ a leží ve stejné třídě jako $z^\prime = t_1 ^\prime a_1 + \cdots + t_n ^\prime a_n$, kde $t_i ^\prime$ je zbytek, který $t_i$ dává po dělení $N(x)$. Každé $t_i ^\prime$ nabývá jednoho z $N(x)$ těchto zbytků, tedy v $\mathcal{O}/(N(x))$ leží nejvýše $N(x)^n$ prvků a tento okruh je konečný. \hfill $\square$\\

Norma čísla v $\mathbb{Z}[i]$ nám dává představu o jeho vzdálenosti od počátku souřadné soustavy, normu ideálu proto definujeme s podobným účelem.
\begin{definice}
Buď $\mathcal{O}$ pořádek číselného tělesa $K$ a $\mathfrak{a} \subseteq \mathcal{O}$ ideál. Pod normou $N_{\mathcal{O}}(\mathfrak{a})$ ideálu $\mathfrak{a}$ rozumíme index $\mathfrak{a}$ v $\mathcal{O}$.
\end{definice}

\begin{veta}\label{noether}
Každý stoupající řetězec inkluzí ideálů pořádku $\mathcal{O}$ je shora omezený.
\end{veta}
\textit{Důkaz.} Pokud pro ideály $\mathfrak{b},\mathfrak{c}$ platí $\mathfrak{b} \subset \mathfrak{c}$, tak jistě i $\mathcal{O}/\mathfrak{c} \subset \mathcal{O}/\mathfrak{b}$, tedy $N(\mathfrak{b}) > N(\mathfrak{c})$. Nekonečný řetězec (ostrých) inkluzí ideálů by znamenal posloupnost norem těchto ideálů klesající pod všechny meze, speciálně by existoval ideál se zápornou normou, zjevný spor. \hfill $\square$\\

% Uvažme pro spor nekonečný řetězec ideálů $\mathfrak{a}_1  \subseteq \mathfrak{a}_2 \subseteq \cdots$ a buď $\mathfrak{b}$ jejich sjednocení, které je pak též konečně generovaným ideálem $\mathcal{O}$. Mějme proto $\mathfrak{b} = (a_1,\dots,a_n)$, kde $a_i$ náleží do ideálu $\mathfrak{a}_{ik}$. Pokud $m$ je mezi těmito $ik$ nejvyšší, platí z definice $\mathfrak{b} \subseteq \mathfrak{a}_{m} \subseteq \mathfrak{b}$ a~každý ideál s indexem alespoň $m$ je tak roven $\mathfrak{b}$. \hfill $\square$\\

Se znalostí předchozí věty můžeme pak definitivně obhájit definici ideálů $\mathcal{O}$ jako konečně generovaných:
\begin{veta}
Buď $\mathfrak{a} \subseteq \mathcal{O}$ ideál. Pak je konečně generovaný jako $\mathcal{O}$-modul.
\end{veta}
\noindent \textit{Důkaz.} Ideál obsahující pouze $0$ je konečně generovaný, dále uvažme nenulový prvek $a_1 \in \mathfrak{a}$. Pokud $\mathfrak{a}$ není generovaný $a_1$, obsahuje prvek $a_2$ takový, že $(a_1) \subset (a_1,a_2)$. Pokud $\mathfrak{a}$ není generovaný těmito dvěma prvky, existuje $a_3 \in \mathfrak{a}$ takový, že $(a_1,a_2) \subset (a_1,a_2,a_3)$. V případě, že bychom takovéto prvky mohli hledat do neurčita, získali bychom nekonečný ostře rostoucí řetězec ideálů $(a_1) \subset (a_1,a_2) \subset (a_1,a_2,a_3) \subset \cdots$, spor s předchozí větou. Řetězec se proto musí na nějakém místě rozlomit a zůstane nám konečná množina generátorů. \hfill $\square$\\


V pořádků číselného tělesa $K$ stupně $n$ platí $\mathcal{O} \cong \mathbb{Z}^{n}$, tedy fundamentální věta konečně generovaných abelovských grup tvrdí, že konečná podgrupa $\mathcal{O}$ je buď nulová, či isomorfní direktnímu součinu několika, nejvýše však $n$, kopií $\mathbb{Z}$. Speciálně každý ideál má nejvýše $n$ generátorů. V další sekci si počet generátorů ideálů $\mathcal{O}_K$ omezíme dokonce číslem $2$.

V euklidově okruhu existuje jednoznačně (až na násobení jednotkou) určený největší společný dělitel čísel $\theta_i$, nějaké $d$. Jistě pak libovolný prvek $(\theta_1,\dots,\theta_n)$ náleží do $(d)$. Navíc dle Bezoutovy identity platí opačná inkluze, tedy $(\theta_1,\dots,\theta_n)$ je ideál generovaný největším společným dělitelem čísel $\theta_i$.  

\begin{definice}
Ideály generované jediným prvkem označíme jako \textit{hlavní}.
\end{definice}

Zajímavé propojení s námi již známou normou prvků $\mathcal{O} \subseteq K$ lze pozorovat právě u~ideálů hlavních. Norma hlavního ideálu $(\alpha)$ je dána $[\mathcal{O}:\alpha \mathcal{O}]$, je tedy rovna stupni zobrazení $\alpha(x)$ na $K$, což je definice čísla $N_K(\alpha)$. Navíc norma $\alpha$ patří do ideálu $(\alpha)$, protože je součinem jeho sdružených čísel. Každý hlavní ideál pořádku obsahuje svou normu, tedy i každý jiný obsahuje celé číslo, konkrétně normu libovolného jeho generátoru. Dokonce všechny ideály obsahují svou vlastní normu, známá věta připisovaná Lagrangemu říká, že každý řád prvku konečného okruhu $\mathcal{O}/\mathfrak{a}$, dělí jeho velikost, tedy normu $\mathfrak{a}$. Speciálně $N(\mathfrak{a}) \cdot 1 \in \mathfrak{a}$.

Pojďme si dále definovat na ideálech pár základních operací.
\begin{definice}\label{soucindef}
Buďte $\mathfrak{a},\mathfrak{b}$ ideály okruhu $R$. Pak jejich součet a součin definujeme následovně:
\begin{itemize}
\item $\mathfrak{a}+\mathfrak{b}  =\left\lbrace  a+b \vert a \in \mathfrak{a}, b \in \mathfrak{b} \right\rbrace$,
\item $\mathfrak{a} \mathfrak{b} =\left\lbrace \left. \sum_{i=1}^{n} a_i b_i \right \vert a_i \in \mathfrak{a}, b_i \in \mathfrak{b}, n \in \mathbb{N} \right\rbrace$.
\end{itemize}
\end{definice}


Vidíme, že jak součet, tak součin dvou ideálů je též ideálem, první generovaný sjednocením množin generátorů obou ideálů, druhý součiny po jednom generátoru $\mathfrak{a}$ a druhém generátoru $\mathfrak{b}$. Sčítání je jistě asociativní a jeho neutrální prvek je nulový ideál $(0) = \lbrace 0 \rbrace$. Násobení ideálů je taktéž asociativní, neboť: $$(\mathfrak{a} \mathfrak{b}) \mathfrak{c}= \left\lbrace \left. \sum_{i=1}^{n} a_i b_i c_i \right\vert a_i \in \mathfrak{a}, b_i \in \mathfrak{b}, c_i \in \mathfrak{c}, n \in \mathbb{N}  \right\rbrace = \mathfrak{a}(\mathfrak{b}\mathfrak{c}),$$ a neutrální prvek je vždy celý okruh $R$. Ideály okruhu $R$ proto tvoří se sčítáním grupu a~násobením monoid, při komutativitě $R$ je monoid komutativní též.

Prostřednictvím násobení si můžeme definovat dělitelnost ideálů:
\begin{definice}
Buďte $\mathfrak{a},\mathfrak{b}$ ideály komutativního okruhu $R$. Pokud pro nějaký ideál $\mathfrak{c} \subseteq R$ platí $\mathfrak{b} = \mathfrak{a} \mathfrak{c}$, píšeme $\mathfrak{a} \mid \mathfrak{b}$ a říkáme, že $\mathfrak{a}$ \textit{dělí} $\mathfrak{b}$.
\end{definice}

\begin{definice}
Buďte $\mathfrak{a},\mathfrak{b}$ ideály okruhu $R$. Tyto ideály nazveme \textit{nesoudělné}, pokud platí rovnost ideálů:
\begin{equation*}
\mathfrak{a}+\mathfrak{b} = (1).
\end{equation*}
\end{definice}

Dva ideály jsou tedy nesoudělné, právě pokud součet nějakých dvou jejich prvků je jednotkou $R$. Nedefinujeme největší společný dělitel, neboť ten ne vždy existuje, alespoň ne v~obecném okruhu $R$. Nesoudělné ideály mají další zajímavé vlastnosti, jejich součin je totiž shodný s jejich průnikem.

\begin{lemma}
Ať $\mathfrak{a},\mathfrak{b}$ jsou nesoudělné ideály komutativního okruhu $R$. Pak platí rovnost $\mathfrak{ab} = \mathfrak{a} \cap \mathfrak{b}$.
\end{lemma}
\noindent \textit{Důkaz.} Jistě platí $\mathfrak{ab} \subseteq \mathfrak{a} \cap \mathfrak{b}$. Naopak ale násobení ideálů je na sčítání zjevně distributivní, máme tedy:
\begin{equation*}
(1)(\mathfrak{a} \cap \mathfrak{b}) = (\mathfrak{a}+\mathfrak{b}) (\mathfrak{a} \cap \mathfrak{b}) = \mathfrak{a}(\mathfrak{a} \cap \mathfrak{b}) + \mathfrak{b} (\mathfrak{a} \cap \mathfrak{b}) = \mathfrak{a b}+ \mathfrak{ba} \subseteq \mathfrak{ab} 
\end{equation*}
díky komutativitě $R$. \hfill $\square$\\

Nesoudělné ideály v celých číslech jsou generované nesoudělnými celými čísly $m,n$ a~podle Čínské zbytkové věty platí $\mathbb{Z}/(m) \times \mathbb{Z}/(n) \cong \mathbb{Z}/(mn)$. Toto tvrzení můžeme pak samozřejmě zobecnit do libovolných okruhu.

\begin{veta}(Čínská zbytková věta)\label{CRT}
Ať $\mathfrak{a},\mathfrak{b}$ jsou nesoudělné ideály komutativního okruhu~$R$. Pak platí:
\begin{equation*}
R/\mathfrak{a} \times R/\mathfrak{b} \cong R/\mathfrak{ab}.
\end{equation*}
\end{veta} 
\noindent \textit{Důkaz.} Označme $f : R \longrightarrow R/\mathfrak{a} \times R/\mathfrak{b}$ homomorfismus okruhů redukující každý prvek $R$ na příslušné zbytkové třídy v $R/\mathfrak{a},R/\mathfrak{b}$. Do jádra $f$ spadají právě prvky $\mathfrak{a} \cap \mathfrak{b} = \mathfrak{ab}$, tedy $f$ dává vzniku injektivnímu homomorfismu $g : R/\mathfrak{ab} \longrightarrow R/\mathfrak{a} \times R/\mathfrak{b}$. Navíc pokud $a \in \mathfrak{a}, b\in \mathfrak{b}$ jsou prvky se součtem $1$, libovolná jejich lineární kombinace $ax+by$ patří vždy do třídy $(x,y)$ v $R/\mathfrak{a} \times R/\mathfrak{b}$ pro všechna $x,y \in R$, což dokazuje surjektivitu $g$ a tedy hledaný isomorfismus. \hfill $\square$\\

Tato věta mimo jiné znamená, že i norma ideálu je (ne nutně kompletně) multiplikativní funkcí. V příští kapitole ukážeme, že pro okruh $\mathcal{O}_K$ je dokonce kompletně multiplikativní, tj. platí $N(\mathfrak{a}) N(\mathfrak{b}) = N(\mathfrak{ab})$ pro libovolné ideály $\mathfrak{a},\mathfrak{b}$.

V následující podkapitole dokážeme slíbené tvrzení, že ideály $\mathcal{O}_K$ se rozkládají jednoznačně na součin prvoideálů, a odůvodníme, proč toto tvrzení nedosahuje na zbylé pořádky tělesa $K$.

\section{Rozklad na prvoideály}


V celých číslech jsou krom samotného okruhu $\mathbb{Z}$ ideály generované prvočísly $(p)$ jediné nenulové, které pro libovolná $a,b \in R$ splňující $ab \in (p)$ vynucují alespoň jedno z $a$ či $b$ náležící do $(p)$. Tento koncept si zobecníme do obecných okruhů.

\begin{definice}
Ideál $\mathfrak{p} \subset R$ takový, že pro každá $a,b \in R$ splňující $ab \in \mathfrak{p}$ platí buď $a \in \mathfrak{p}$, či $b \in \mathfrak{p}$, nazveme \textit{prvoideálem}.
\end{definice}

Prvoideály v pořádcích můžeme ve zkratce charakterizovat v následující větě:
\begin{veta}\label{prvoid}
Buď $\mathfrak{p} \subseteq \mathcal{O}$ ideál. Pak následující skutečnosti jsou ekvivalentní:
\begin{enumerate}
\item $\mathfrak{p}$ je prvoideál,
\item Faktorový okruh $\mathcal{O}/\mathfrak{p}$ je konečné těleso,
\item $\mathfrak{p}$ je maximální, neboli neexistuje ideál $\mathfrak{a}$ splňující $\mathfrak{p} \subset \mathfrak{a} \subset \mathcal{O}$,
\item Rovnost $\mathfrak{p} = \mathfrak{a} \mathfrak{b}$ znamená buď $\mathfrak{a} = \mathfrak{p}$, či $\mathfrak{b} = \mathfrak{p}$.
\end{enumerate} 
\end{veta}
\textit{Důkaz.} Případ, kdy v okruhu zbytků $\mathcal{O}/\mathfrak{p}$ rovnost tříd $(a+\mathfrak{p})(b+\mathfrak{p}) = \mathfrak{p}$ platí jenom pokud jedno z~$a,b$ náleží do $\mathfrak{p}$, nastane právě když $\mathfrak{p}$ je prvoideál. Faktorokruh $\mathcal{O}/\mathfrak{p}$ je proto oborem integrity pouze a jenom když $\mathfrak{p}$ je prvoideál. Klasický výsledek abstraktní algebry ale tvrdí, že konečný obor integrity je těleso, což stvrzuje ekvivalenci bodů $(i)$ a $(ii)$.

Dále mějme $\mathfrak{p}$ prvoideál a $\mathfrak{a}$ ideál $\mathcal{O}$ splňující $\mathfrak{p} \subset \mathfrak{a}$. Ukážeme, že $\mathfrak{a}$ je roven samotnému $\mathcal{O}$. Buď $a \in \mathfrak{a} \setminus \mathfrak{p}$, pak $a$ leží v nenulové třídě $\mathcal{O}/\mathfrak{p}$. Tento prvek má v $\mathcal{O}/\mathfrak{p}$ pak multiplikativní inverze $b$, tedy $ab=1+c$ pro nějaké $c \in \mathfrak{p} \subset \mathfrak{a}$, což znamená $1 = ab-c$. Všechna tři čísla $a,b,c$ leží v $\mathfrak{a}$, tedy $1 \in \mathfrak{a}$ a $\mathfrak{a} = \mathcal{O}$. Naopak pokud $\mathfrak{p}$ je maximální, uvažme libovolné $a \in \mathcal{O}\setminus\mathfrak{p}$. Nenulová třída $a+\mathfrak{p} \in \mathcal{O}/\mathfrak{p}$ dává vzniku ideálu $(a)+\mathfrak{p} \subseteq \mathcal{O}$, který obsahuje jak $a$, tak ideál $\mathfrak{p}$, tedy díky maximalitě $\mathfrak{p}$ i~okruh $\mathcal{O}$ samotný. Jednotka náleží do $(a)+\mathfrak{p}$, platí tedy $ra+p = 1$ pro nějaká $r \in \mathcal{O}, p \in \mathfrak{p}$. Platí pak rovnost $(r+\mathfrak{p})(a+\mathfrak{p}) = ra+\mathfrak{p} = 1+\mathfrak{p}$, čili každá nenulová třída $a+\mathfrak{p}$ má v $\mathcal{O}/\mathfrak{p}$ multiplikativní inverzi.

Konečně ať $\mathfrak{p}$ je roven součinu dvou ideálů $\mathfrak{a}$ a $\mathfrak{b}$, speciálně jej oba obsahují. Pokud je $\mathfrak{p}$ prvoideálem, tak je maximální, tedy je jeden z $\mathfrak{a}$ roven $\mathfrak{p}$ a ten druhý okruhu $\mathcal{O}$. Naopak pokud platí bod $(iv)$, tak $\mathcal{O}/\mathfrak{p}$ je oborem integrity, tedy konečným tělesem. \hfill $\square$\\


Tyto výsledky nejsou exlusivní pro pořádky číselných těles, mimo ně však musíme být na pozoru, podmínka $(ii)$ totiž není splněna například pro prvoideál $(x)$ v okruhu $\mathbb{Z}[x]$.

Důsledek předchozí věty, Bezoutovy věty a faktu, že každý ideál obsahuje svoji normu, mluví o normě prvoideálů:
\begin{dusledek}
Pokud ideál $\mathfrak{p} \subset \mathcal{O}$ je prvoideál, pak obsahuje unikátní prvočíslo, jehož některá mocnina je norma $\mathfrak{p}$.
\end{dusledek}

Nyní jsme konečně připraveni diskutovat jednoznačnost rozklad na prvoideálů.

Vzpomeňme si na náš postup, když jsme dokazovali jednoznačnost rozkladu na ireducibilní prvky v euklidovských doménách. Ten se skládal ze tří kroků, i)  ireducibilní prvek dělící součin dvou prvků dělí jeden z nich, ii) každý prvek je součinem několika ireducibilních prvků a iii) rozklad na ireducibilní prvky je (až na násobení jednotkou) jednoznačný.

Tuto proceduru se pokusíme zopakovat a poté odůvodníme, proč v pořádcích zcela zreplikovat nelze, hlavní problém bude činit bod ii). První část přichází bezbolestně:

\begin{veta}\label{prvobsah}
Buďte $\mathfrak{p},\mathfrak{a},\mathfrak{b} \subseteq \mathcal{O}$ nenulové ideály. Pak $\mathfrak{p}$ je prvoideál, právě pokud inkluze $\mathfrak{p} \supseteq \mathfrak{a} \mathfrak{b}$ znamená $\mathfrak{p} \supseteq \mathfrak{a}$, či $\mathfrak{p} \supseteq \mathfrak{b}$.
\end{veta}

\noindent \textit{Důkaz}. Nejprve uvažme $\mathfrak{p}$ prvoideál. Pokud platí $p \supseteq \mathfrak{a} \mathfrak{b}$ a $\mathfrak{p} \not\supseteq \mathfrak{a}$, uvažme číslo $x \in \mathfrak{a} \setminus \mathfrak{p}$. Pro každé $y \in \mathfrak{b}$ je $xy \in \mathfrak{ab} \subseteq \mathfrak{p}$, tedy $y \in \mathfrak{p}$, neboli platí $\mathfrak{p} \supseteq \mathfrak{b}$. Nyní mějme implikaci ze zadání platnou. Pro libovolná $x,y \in \mathfrak{p}$ platí $ (x)(y) = (xy) \subseteq \mathfrak{p}$, tedy jeden ze dvou ideálů generovaných $x,y$ náleží do $\mathfrak{p}$. Jedno z těchto čísel proto leží uvnitř $\mathfrak{p}$ a $\mathfrak{p}$ je prvoideál. \hfill $\square$\\

Pokračujme s naším seznamem, tentokrát ukážeme, že každý ideál $\mathcal{O}$ obsahuje součin prvoideálů.

\begin{veta}\label{obsahprvo}
Každý nenulový ideál $\mathfrak{a} \subseteq \mathcal{O}$ splňuje:
\begin{equation*}
\mathfrak{a} \supseteq \mathfrak{p}_1 \mathfrak{p}_2 \cdots \mathfrak{p}_r.
\end{equation*}
pro nějaké nenulové prvoideály $\mathfrak{p}_i$.
\end{veta}

\noindent \textit{Důkaz}. Dejme tomu, že existují ideály, které toto tvrzení nesplňují, a uvažme mezi nimi exemplář $\mathfrak{a}$ s nejnižší normou. Ten jistě není prvoideálem, existují proto $x,y \not\in \mathfrak{a}$, jejichž součinem v $\mathfrak{a}$ leží. Pak ideály $\mathfrak{a}+(x)$ a $\mathfrak{a}+(y)$ oba ostře obsahují samotný ideál $\mathfrak{a}$ a mají tedy nižší normu, díky našim předpokladům oba obsahují součin nějakých prvoideálů. Díky platné inkluzi $(\mathfrak{a}+(x) ) (\mathfrak{a}+(y)) \subseteq \mathfrak{a}$ tak získáváme toužený spor. \hfill $\square$\\


Konečně, na dokončení důkazu budeme do boje muset povolat novou definici:
\begin{definice}
Buď $\mathfrak{p} \subset \mathcal{O}_K$ prvoideál a definujme jeho inverzi jako $\mathcal{O}_K$-modul:
\begin{equation*}
\mathfrak{p}^{-1} := \left\lbrace x \in K \left\vert \right. x \mathfrak{p} \subseteq \mathcal{O}_K \right\rbrace
\end{equation*}
a definujme násobení $\mathfrak{a} \mathfrak{p}^{-1} := \lbrace \sum a_i p_i \vert a_i \in \mathfrak{a}, p_i \in \mathfrak{p}^{-1} \rbrace := \mathfrak{p}^{-1} \mathfrak{a}$.
\end{definice}

Poznamenejme, že inverze libovolného prvoideálu je jistě $\mathcal{O}_K$-modulem a navíc platí inkluze $\mathcal{O}_K \subseteq \mathfrak{p}^{-1}$. Poslední část definice je díky komutativitě $\mathcal{O}_K$ dobře definovaná a koresponduje s komutativitou násobení ideálů $\mathcal{O}_K$.

Důvod zavedení tohoto pojmu závisí na jeho schopnosti \textit{krátit} prvoideály, to je ale (v~plné obecnosti) unikátní pro maximální pořádek, proč si osvětlíme brzy.

\begin{veta}
Buď $\mathfrak{p} \subset \mathcal{O}_K$ prvoideál maximálního pořádku. Pak platí $\mathcal{O}_K \subset \mathfrak{p}^{-1}$ a rovnost $\mathfrak{p} \mathfrak{p}^{-1} = \mathcal{O}_K$.
\end{veta}
\noindent \textit{Důkaz.} Protože $\mathfrak{p}$ náleží do $\mathcal{O}_K$, jeho inverze jistě obsahuje celý $\mathcal{O}_K$. Vyberme nyní nenulové $x \in \mathfrak{p}$. Ideál $(x) \subseteq \mathfrak{p}$ podle věty \ref{obsahprvo} obsahuje součin prvoideálů $\mathfrak{p}_1 \cdots \mathfrak{p}_k$, kde $k$ je mezi všemi množinami $\lbrace \mathfrak{p}_1,\dots,\mathfrak{p}_k \rbrace$ nejnižší možné. Podle věty \ref{prvobsah} $\mathfrak{p}$ je roven jednomu z $\mathfrak{p}_i$, bez újmy na obecnosti ať $\mathfrak{p} = \mathfrak{p}_1$. Díky výběru $k$ ideál $(x)$ neobsahuje $\mathfrak{p}_2 \cdots \mathfrak{p}_k$, uvažme $y \in \mathfrak{p}_2 \cdots \mathfrak{p}_k \setminus (x)$, pak $y/x \not\in \mathcal{O}_K$. Platí ale inkluze $y \mathfrak{p} \subseteq \mathfrak{p} \mathfrak{p}_2 \cdots \mathfrak{p}_k \subseteq (x)$, tedy $(y/x) \mathcal{O} \subseteq \mathfrak{p}$ a $y/x$ je proto prvkem $\mathfrak{p}^{-1} \setminus \mathcal{O}_K$.

Nyní již můžeme předpokládat existenci $a \in \mathfrak{p}^{-1} \setminus \mathcal{O}_K$ splňujícího $a \mathfrak{p} \subseteq \mathcal{O}_K$. Platí inkluze $\mathfrak{p} \subseteq \mathfrak{p}+a \mathfrak{p} \subseteq \mathcal{O}_K$, tedy z maximality prvoideálů nastane v jedné z inkluzí rovnost. Dejme tomu, že $\mathfrak{p} = \mathfrak{p} + a \mathfrak{p}$, pak musí platit $a \mathfrak{p} \subseteq \mathfrak{p}$. Protože $\mathfrak{p}$ je konečně generovaný $\mathbb{Z}$-modul, tato inkluze nutně znamená, že $a$ je celý nad $\mathbb{Z}$, spor s $a \not\in \mathcal{O}_K$. Platí proto $\mathfrak{p} + a \mathfrak{p} = \mathcal{O}_K$ pro každé $a \in \mathfrak{p}^{-1} \setminus \mathcal{O}_K$, neboli $\mathfrak{p} \mathfrak{p}^{-1} = \mathcal{O}_K$ díky maximalitě prvoideálů. \hfill $\square$\\

\begin{dusledek}\label{cancel}
Buďte $\mathfrak{a},\mathfrak{b},\mathfrak{p}$ ideály $\mathcal{O}_K$, $\mathfrak{p}$ prvoideál. Pokud platí $\mathfrak{p} \mathfrak{a} = \mathfrak{p} \mathfrak{b}$, tak $\mathfrak{a} = \mathfrak{b}$.
\end{dusledek}

S důkazem předchozího tvrzení je dovršen kopec teorie, kterou potřebujeme k důkazu jednoznačnosti rozkladu ideálů $\mathcal{O}_K$ na prvoideály.

\begin{veta}
Každý nenulový ideál $\mathfrak{a} \subset \mathcal{O}_K$ lze jednoznačně rozložit na součin prvoideálů.
\end{veta}
\noindent \textit{Důkaz.} Nejprve ukážeme, že každý ideál rozložit lze. Postupujme indukcí vzhledem k $t$, počtu prvoideálů, jejichž součin $\mathfrak{a}$ obsahuje. Případ $t=1$ dává $\mathfrak{a}$ prvoideál, dále ať věta platí pro nějaké $t$ a uvažme (ne prvoideál) $\mathfrak{a}$ obsahující $\mathfrak{p}_1 \cdots \mathfrak{p}_{t+1}$. Jistě $\mathfrak{a}$ je obsažen v~nějakém maximálním $\mathfrak{p}$, tedy podle věty \ref{prvobsah} je jeden z $\mathfrak{p}_i$ roven $\mathfrak{p}$, ať to je $\mathfrak{p}_1$. Násobením řetězce $\mathfrak{p}_1 \supseteq \mathfrak{a} \supseteq \mathfrak{p}_1 \cdots \mathfrak{p}_{t+1}$ modulem $\mathfrak{p}_1^{-1}$ dává $\mathcal{O}_K \supseteq \mathfrak{a} \mathfrak{p}_1^{-1} \supseteq \mathfrak{p}_2 \cdots \mathfrak{p}_{t+1}$, modul $\mathfrak{a} \mathfrak{p}_1^{-1}$ je tedy ideál $\mathcal{O}_K$ a je rozložitelný, tedy $\mathfrak{a} = \mathfrak{a} \mathfrak{p}_1^{-1} \mathfrak{p}_1$ je též.

Nyní se pusťme na jednoznačnost. Dejme tomu, že existují dva rozklady $\mathfrak{p}_1 \cdots \mathfrak{p}_k = \mathfrak{a} = \mathfrak{q}_1 \cdots \mathfrak{q}_{\ell}$. Ideál $\mathfrak{p}_k$ dělí součin $\mathfrak{q}_i$, je proto roven jednomu z nich. Podle důsledku \ref{cancel} a~komutativity $\mathcal{O}_K$ je můžeme oba pokrátit (vynásobit $\mathfrak{p}_k^{-1}$) a pokračovat s ideálem $\mathfrak{a} \mathfrak{p}_k^{-1}$, čímž ostře snížíme $\max (k,\ell)$, sestupem dojdeme k závěru $k = \ell$ a že množiny prvoideálů na obou stranách musely být, včetně násobnosti, shodné. \hfill $\square$\\

Pokud definujeme mocninu ideálu $\mathfrak{a}^k := \underbrace{\mathfrak{a} \cdots \mathfrak{a}}_{k}$, můžeme každý prvoideál jednoznačně rozložit na součin mocnin prvoideálů.

Jednoznačnost rozkladu pospolu s existencí inverzních prvoideálů (a tedy všech ideálů) v~$\mathcal{O}_K$ nám umožňuje pozorovat mnoho paralel s celými čísly. Mimo jiné můžeme dělitelnost přeformulovat pomocí inkluze, pro ideály $\mathcal{O}_K$ platí ekvivalence $\mathfrak{a} \mid \mathfrak{b} \Leftrightarrow \mathfrak{b} \subseteq \mathfrak{a}$. Tato vlastnost je dokonce jednou z ekvivalentních definicí tzv. \textit{Dedekindových oborů}, další z nich je jednoznačný rozklad ideálů na prvoideály či invertibilita každého ideálu (což platí díky invertibilitě prvoideálů). Invertibilita ideálů nám též umožňuje ideály krátit, ve smyslu implikace $\mathfrak{a b} =\mathfrak{ac} \Rightarrow \mathfrak{b} = \mathfrak{c}$ pro nenulový ideál $\mathfrak{a}$.

Čínská zbytková věta říká, že norma ideálů je multiplikativní, přičemž každý ideál $\mathcal{O}_K$ se jednoznačně rozkládá na součin prvoideálů. Dá se ukázat \cite[Věta 4.3.18.]{Pupik}, že norma mocniny prvoideálu $\mathfrak{p}$ je rovna příslušné mocnině normy $\mathfrak{p}$, tedy norma ideálů maximálního pořádku je kompletně multiplikativní.

Navíc jednoznačnost rozkladu nám pomůže omezit počet generátorů libovolného ideálu okruhu $\mathcal{O}_K$ číslem $2$:
\begin{veta}\label{dav}
Každý ideál $\mathcal{O}_K$ je generovaný nejvýše dvěma prvky.
\end{veta}
\noindent \textit{Důkaz.} Dejme tomu, že $\mathfrak{a}$ není hlavní ideál a uvažme nenulové $x \in \mathfrak{a}$. Pak $(x) \subset \mathfrak{a}$, neboli $\mathfrak{a} \mid (x)$ podle předchozí diskuze. Uvažme pak rozklady ideálů $\mathfrak{a} = \mathfrak{p}_1 ^{a_1} \cdots \mathfrak{p}_k ^ {a_k},(x) = \mathfrak{p}_1 ^{b_1} \cdots \mathfrak{p}_k ^{b_k}$ splňující $a_i \leqslant b_i$ pro každé $i$. Čínská zbytková věta \ref{CRT} nám pak umožňuje najít $y$ splňující $y \in \mathfrak{p}_k^{a_k} \setminus \mathfrak{p}_k^{a_k+1}$ pro každé $k$, což znamená $\mathfrak{a} = \mathfrak{p}_1 ^{a_1} \cdots \mathfrak{p}_k ^ {a_k} = (x)+(y) = (x,y)$. \hfill $\square$\\ 

Nyní nastává vhodná chvíle se zamyslet nad naší volbou zúžit se pouze na maximální pořádky. Ve všech pořádcích opravdu platí, že každý ideál obsahuje součin nějakých prvoideálů a~prvoideály se chovají podobně jako prvočísla, ztrácíme ale nutnou existenci inverzního prvoideálu a s ní i vyjádření všech ideálů jako součin prvoideálů i možnost krátit. Než si zmíníme ucelenou větu o faktorizaci ideálů v pořádcích, ukažme si příklad selhání faktorizace.

\begin{priklad}
Uvažme pořádek $\mathbb{Z}[2i] \subset \mathbb{Q}[i]$ a jeho ideál $(2,2i)$. Ukážeme, že nemůže být rozložitelný na prvoideály. Platí totiž $(2,2i)^2 = (4,4i) = (2)(2,2i)$, tedy pokud by byl tento ideál rozložitelný na prvoideály, musela by platit rovnost ideálů $(2,2i) = (2)$, prvek $4+2i$ ale leží pouze v prvním ideálu. Problém zde nastává, protože $2$ je sudé číslo, ideál $(2,2i)$ tedy náleží do ideálu $\mathfrak{c} = \lbrace x \in \mathbb{Q}(i) \vert x \mathbb{Z}[i] \subseteq \mathbb{Z}[2i] \rbrace$, kde $\mathcal{O} = \mathbb{Z}[2i]$, a není invertibilní jako $\mathcal{O}$-modul. Tento ideál $\mathfrak{c}$ je (vzhledem k inkluzi) největší ideál $\mathcal{O}_K$ obsažen v $\mathcal{O}$ a nese název \textit{conductor ideal} (vodící ideál), více informací o něm se nachází na \cite{Conrad3}.
\end{priklad}

\begin{veta}
Buď $\mathcal{O}$ pořádek číselného tělesa $K$ a označme $\mathfrak{c} = \lbrace x \in K \vert x \mathcal{O}_K \subseteq \mathcal{O} \rbrace$. Každý ideál $\mathcal{O}$ nesoudělný s $\mathfrak{c}$ je součinem invertibilních prvoideálů a je sám jako $\mathcal{O}$-modul invertibilní, speciálně je též jednoznačně rozložitelný na součin invertibilních prvoideálů. Navíc neinvertibilních prvoideálů je pouze konečně mnoho.
\end{veta}

Důkaz se nachází na \cite[Sec. 3.]{Conrad3}. Speciálně ideály nesoudělné s $\mathfrak{c}$ se chovají prakticky identicky jako ideály maximálního pořádku, jsou všechny generované nejvýše dvěma prvky, můžeme je krátit, většina hezkých vlastností, které jsme zde zmínili. Ideály s~$\mathfrak{c}$ soudělné u~většiny těchto vlastností takovým či onakým způsobem selžou.

 

\section{Grupa tříd ideálů a jednoznačnost rozkladu}

Pojďme si nyní ideály pořádku $\mathcal{O}$ rozšířit na jeho moduly. Každý nenulový konečně generovaný $\mathcal{O}$-modul je roven $\mathfrak{a} =  a_1 \mathcal{O} + a_2 \mathcal{O} + \cdots + a_k \mathcal{O}$ pro nenulová $a_i \in K$, přičemž víme, že existuje (nenulový) celý násobek každého z nich ležící v $\mathcal{O}_K$ a tedy i v $\mathcal{O}$. Pokud $d$ je nejmenším společným násobkem všech těchto skalárů, $d \mathfrak{a}$ je ideálem $\mathcal{O}$ a naopak násobek $\mathcal{O}$-ideálu prvkem tělesa $K$ je jistě konečně generovaný $\mathcal{O}$-modul.

\begin{definice}
Buď $K$ podílové těleso okruhu $R$. Pokud $\mathcal{O}$-modul $\mathfrak{a} =  m \mathfrak{b}$ je ideál $R$ pro $m \in R$, nazveme $\mathfrak{b}$ \textit{lomeným ideálem} $K$. Budeme značit $\mathfrak{b} = \frac{\mathfrak{a}}{m}$.
\end{definice}

\begin{definice}
Buď $K$ podílové těleso $R$. Pro $\alpha \in K$ nazveme $(\alpha) = \alpha \mathcal{O}$ \textit{hlavním lomeným ideálem} $R$.
\end{definice}

Mezi zástupce lomených ideálů v $\mathcal{O}_K$ patří například $\mathfrak{p}^{-1}$, inverze libovolného prvoideálu v $\mathcal{O}_K$. Hlavní lomené ideály nedávají příliš překvapivé příklady, v okruhu celých čísel tělesa $\mathbb{Q}$ je typickým příkladem $\frac{(3)}{2} = \frac{3}{2} \mathbb{Z}$.

Součet i součin ideálů pořádku $\mathcal{O}$ přirozeně generalizuje i na lomené ideály, z nich součin nás bude zajímat více.

Uvažme nyní $\mathfrak{a},\mathfrak{b}$ lomené ideály pořádku $\mathcal{O}$ a definujme relaci \textit{ekvivalence} $\sim$ s tím, že $\mathfrak{a},\mathfrak{b}$ jsou ekvivalentní, pokud existují $x,y \in \mathcal{O}$ taková, že $\mathfrak{a} \cdot (x) = \mathfrak{b} \cdot (y)$, tedy pokud \uv{podíl} dvou takových ideálů je hlavní lomený ideál $\mathcal{O}$ ($K$ je podílovým tělesem $\mathcal{O}$). Relace $\sim$ pak rozkládá množinu lomených ideálů $\mathcal{O}$ na třídy ekvivalence $[\mathfrak{a}]$, kde násobení hlavním (lomeným) ideálem ponechá třídu.

\begin{veta}
Buďte $\mathfrak{a},\mathfrak{b} \subseteq \mathcal{O}$ lomené ideály. Pak platí:
\begin{equation*}
[\mathfrak{a}] \cdot [\mathfrak{b}] = [\mathfrak{ab}].
\end{equation*}
\end{veta}
\noindent \textit{Důkaz.} Buďte $a,a^\prime \in [\mathfrak{a}], b, b^\prime \in [\mathfrak{b}]$ lomené ideály. Existují pak nenulové prvky $\alpha,\beta \in K$ takové, že $a^\prime = \alpha \cdot a$ a $b^\prime = \beta \cdot b$. Součin $a^\prime,b^\prime$ je roven $\alpha \beta a b$ a vždy tedy leží ve třídě $[ab]$. \hfill $\square$\\

Součin dvou ideálů z dvou tříd spadá vždy do té samé třídy, můžeme pak přirozeně definovat na třídách násobení. To je jistě komutativní i asociativní a třída $[\mathcal{O}]$ hlavních lomených ideálů $\mathcal{O}$ skrz něj působí jako identita.

\begin{veta}
Třídy invertibilních ideálů $\mathcal{O}$ tvoří grupu.
\end{veta}
\noindent \textit{Důkaz.} Uvažme třídu obsahující nenulový lomený ideál $\mathfrak{a}$ pořádku $\mathcal{O}$. Pokud $\mathfrak{a}$ je invertibilní lomený ideál (existuje $\mathfrak{b}$ s $\mathfrak{ab} = \mathcal{O}$), třída $[\mathfrak{a}]$ je invertibilní též, s inverzí $[\mathfrak{b}]$. Naopak pokud je třída $[\mathfrak{a}]$ invertibilní, ve smyslu $[\mathfrak{a}][\mathfrak{b}] = [(1)]$, součin $\mathfrak{a}\mathfrak{b}$ je hlavní lomený ideál $x\mathcal{O}$, tedy platí $\mathfrak{a} \frac{\mathfrak{b}}{x} = \mathcal{O}$. Násobení tříd je zřejmě asociativní, invertibilní třídy, neboli třídy invertibilních ideálů, pak tvoří s násobením tříd grupu. \hfill $\square$\\

\begin{definice}
Buď $\mathcal{O}$ pořádek číselného tělesa $K$. Definujeme pak \textit{grupu tříd ideálů} $Cl(\mathcal{O})$ jako grupu všech invertibilních tříd $[\mathfrak{a}]$ rozkladu podle relace $\sim$ definované výše spolu s operací násobení tříd ideálů.
\end{definice}

Existuje ještě jeden způsob jak definovat grupu tříd ideálů, pro některé čtenáře možná přirozenější. Označíme-li množiny $\mathsf{G}$, $\mathsf{H}$ invertibilních lomených, případně invertibilních hlavních lomených ideálů, spolu s operací násobení ideálů se obě stavají grupami, přičemž $\mathsf{H}$ je podgrupou $\mathsf{G}$. V případě maximálního pořádku jsou $\mathsf{G},\mathsf{H}$ prostě množiny lomených, resp. hlavních lomených ideálů, protože díky jednoznačnosti rozkladu je každý ideál invertibilní. Grupu tříd ideálů pak můžeme zapsat jako faktorgrupu $\mathsf{G}/\mathsf{H}$.

\begin{veta}
Každá třída ideálů $Cl (\mathcal{O})$ má reprezentanta z ideálů $\mathcal{O}$.
\end{veta}
\noindent \textit{Důkaz.} Buď $\mathfrak{a}/m$ nějaký lomený ideál. Pak $\mathfrak{a}/m \cdot (m) = \mathfrak{a}$ je ideál $\mathcal{O}$ a leží ve stejné třídě jako $\mathfrak{a}/m$. \hfill $\square$\\

Mluvíme-li o pořádcích v číselném tělese, nalezneme u nich pouze konečně mnoho takových tříd ideálů, i když okruhy obecně mohou mít grupu tříd ideálů nekonečnou. Tento fakt není na první pohled zjevný a nebudeme se jím nějak zvlášť zabývat. Klasické důkazy v každé třídě naleznou ideál normy nižší než počet tříd, z čehož konečnost po uvedení pár dalších tvrzení plyne, zaujatý čtenář ocení \cite[Kap. 5]{Pupik}.

\begin{veta}
Grupa tříd ideálů pořádku $\mathcal{O}$ je konečná.
\end{veta}

 S konečností grupy tříd ideálu se pak můžeme bavit o počtu jejích prvků.

\begin{definice}
\textit{Třídové číslo} $h_{\mathcal{O}}$ pořádku $\mathcal{O}$ definujeme jako počet prvků grupy $Cl(\mathcal{O})$.
\end{definice}

Důkaz konečnosti třídového čísla též navádí na jeho nalezení, není to však jednoduchý proces. Často je redukován na rozkládání ideálů generovaných prvočísly pod danou hranici, viz například \cite[Kap. 5.]{Pupik}. Obecný její výpočet, i s pomocí počítače, je obtížný, jak může dosvědčit fakt, že nejsou ani známa všechna reálná kvadratická tělesa, jejichž maximální pořádek má třídové číslo 1. Brzy si totiž ukážeme, že tyto okruhy jsou právě ty mající jednoznačný rozklad na ireducibilní prvky.

Každý prvek konečné grupy umocněn na její řád se stává neutrálním. Tento fakt v případě grupy tříd ideálů zní:
\begin{veta}
Buď $\mathcal{O}$ pořádek a $\mathfrak{a}$ jeho ideál. Pak ideál $\mathfrak{a}^{h_{\mathcal{O}}}$ je hlavním ideálem $\mathcal{O}$.
\end{veta}

Jen takto banální poznatek o grupě tříd ideálů přirozeně spojuje grupu tříd ideálů s~jednoznačností rozkladu na ireducibilní prvky. Pokud je totiž grupa tříd ideálů pořádku triviální, každý jeho (lomený) ideál je hlavní.

\begin{veta}
Buď $\mathcal{O}_K$ maximální pořádek číselného tělesa. Pak každý prvek $\mathcal{O}_K$ se, až na permutaci a~násobení jednotkou, jednoznačně rozkladá na ireducibilní prvky $\mathcal{O}_K$ právě pokud platí $h_{\mathcal{O}_K} = 1$.
\end{veta}

\noindent \textit{Důkaz.} Nejprve ať je třídové číslo $\mathcal{O}_K$ rovno jedné. Každý jeho ideál je pak hlavní. Pokud pak $p_1 \cdots p_k = n = q_1 \cdots q_{\ell}$ jsou dva rozklady čísla $n$ na (ne nutně různé) ireducibilní prvky, ideály generované příslušnými výrazy jsou:
\begin{equation*}
(p_1) \cdots (p_k) = (n) = (q_1) \cdots (q_\ell).
\end{equation*}
Pokud by ideál generovaný například $p_1$ nebyl prvoideálem, byl by vyjádřitelný jako součin dvou (hlavních) ideálů $(p_1) = (a)(b) = (ab)$, tedy platí $p_1 \mid ab \mid p_1$ a existují $x,y \in \mathcal{O}_K$ splňující $p_1 = abx = p_1 xy$, $x,y$ jsou pak jednotkami. Prvky $p_1$ a $ab$ jsou asociované a~díky ireducibilitě $p$ je jedno z $a,b$ jednotkou též, ideál $(p_1)$ prvoideálem. Rovnosti výše jsou proto vyjádřeními prvoideálů a musí se tak příslušné množiny ideálů rovnat. To znamená, že množiny generátorů musí být, včetně násobnosti a bez ohledu na násobení jednotkou, shodné. Naopak ať $\mathcal{O}_K$ připouští jednoznačnost rozkladu na ireducibilní prvky a buď $\mathfrak{p}$ jeho prvoideál. Můžeme pak nenulový prvek $n \in \mathfrak{p}$ rozložit $n = p_1 \cdots p_k$  na (ne nutně různé) ireducibilní $p_i$. Jeden z těchto ireducibilních prvků, ať to je $p_1,$ leží v prvoideálu $\mathfrak{p}$, tedy platí $(p_1) \subseteq \mathfrak{p}$. Díky ireducibilitě $p_1$ je $(p_1)$ je prvoideál a maximalita prvoideálů říká $(p_1) = \mathfrak{p}$. Každý prvoideál je hlavní a tedy díky jednoznačnému rozkladu každého ideálu na prvoideály je i každý jiný ideál. \hfill $\square$\\

Předchozí trvzení se samozřejmě přirozeně zobecňuje na ideály pořádků nesoudělné s~vodícím ideálem. Soudělné ideály mnoho podobných hezkých vlastností ztrácí, několik pár z nich je k nalezení v \cite[Ch. 3.]{Conrad3}.

\begin{poznamka}
Lze ukázat, že okruh celých algebraických čísel tělesa $\mathbb{Q}(\sqrt{d})$ s $d < 0$ má třídové číslo $1$, neboli připouští jednoznačnost rozkladu, pokud je euklidovým okruhem. Opačná implikace neplatí, příkladem toho je okruh $\mathbb{Z}\left[\frac{1+\sqrt{-19}}{2} \right]$.
\end{poznamka}

Další zajímavé vlastnosti platí pro okruhy s třídovým číslem $2,3,4$ a více, například třídové číslo nejvýše dva znamená, že byť se některé prvky rozkládají do více různých množin prvočinitelů, jejich počet (včetně násobnosti) zůstane vždy konzistentní. Například dříve zmíněný okruh $\mathbb{Z}[\sqrt{-14}]$ s dvěma rozklady $81$ na různé počty faktorů má příslušnou grupu tříd ideálů čtyřprvkovou.

\begin{priklad}
Každý pořádek, který je euklidovým okruhem, má třídové číslo $1$. Naopak okruh celých algebraických čísel $\mathbb{Z}[\sqrt{-5}] \subseteq \mathbb{Q}(\sqrt{-5})$ má třídové číslo $2$. Ne každý jeho ideál je hlavní, například $(2,1+\sqrt{-5})$ je ideál s normou $2$. Pokud by byl generovaný prvkem $a+b\sqrt{-5} \in \mathbb{Z}[\sqrt{-5}]$, bylo by $2 = N((a+b\sqrt{-5})) = N(a+b\sqrt{-5}) = a^2+5b^2$, což nemá řešení modulo $5$.
\end{priklad}
 
Dále si zkonstruujeme přirozený injektivní homomorfismus vedoucí z grupy tříd ideálů pořádku do grupy tříd ideálů okruhu celých algebraických čísel, který nám poví o vztahů příslušných třídových čísel. 

\begin{veta}
Buď $\mathcal{O}$ pořádek číselného tělesa $K$. Pak $h_{\mathcal{O}_K} \mid h_{\mathcal{O}}$.
\end{veta}
\noindent \textit{Důkaz.} Uvažme třídu $[\mathfrak{a}] \in Cl(\mathcal{O}_K)$, kde $\mathfrak{a} \subseteq \mathcal{O}_K$ je ideál nesoudělný s vodícím ideálem $\mathfrak{c}$. Ideál $\mathfrak{a} \cap \mathcal{O}$ je ideál nesoudělný s $\mathfrak{c}$ a jeho $\mathcal{O}_K$ násobek $\mathcal{O}_K (\mathfrak{a} \cap \mathcal{O})$ je zjevně celý $\mathfrak{a}$, $\mathfrak{a} \cap \mathcal{O}$ je tedy invertibilní $\mathcal{O}$-modul a zobrazení $[\mathfrak{a} \cap \mathcal{O}] \mapsto [\mathcal{O}_K (\mathfrak{a} \cap \mathcal{O})]= [\mathfrak{a}]$ udává surjektivní homomorfismus $Cl(\mathcal{O}_K) \longrightarrow Cl(\mathcal{O})$, speciálně se příslušná třídová čísla dělí. \hfill $\square$\\
 
O co víc, čistě pro zajímavost uveďme, že dokážeme  s pomocí vodícího ideálu $\mathfrak{c}$ přesně určit vztah svazující $h_{\mathcal{O}}$ a $h_{\mathcal{O}_K}$. Důkaz následujícího tvrzení není jednoduchý, uvedeme ho proto bez důkazu, ten je k nalezení na \cite[Thm. 5.2.]{Conrad3}.
\begin{veta} 
Buď $\mathcal{O}$ pořádek číselného tělesa s vodícím ideálem $\mathfrak{c}$. Pak platí:
\begin{equation*}
\frac{h_\mathcal{O}}{h_{\mathcal{O}_K}} = \frac{[(\mathcal{O}_K/\mathfrak{c})^{\times} : (\mathcal{O}/\mathfrak{c})^{\times}]}{[\mathcal{O}_K ^\times : \mathcal{O}^\times]}.
\end{equation*}
\end{veta}

Třídová čísla propojují rozklad v okruhu s jeho ideály, není to však ani zdaleka jediné, kde toto číslo působí. V analytické teorii čísel má své místo ve tvrzení známém jako \uv{class number formula} dokázané Peterem Dirichletem, spojující kvadratické formy, L-funkce, diskriminant číselného tělesa i číslo $\pi$ (tentokrát opravdu ono číslo splňující $\pi \approx 3$) v~jedné elegantní formuli. Opusťme ale nyní svět algebraické teorie čísel a pojďme zužitkovat nabyté znalosti na teorii eliptických křivek.
 
 %k End(E) : Neutrálním prvkem našeho okruhu pro sčítání je $[0]$ a pro kompozici zase $[1]$. Nyní pojďme zkoumat spojitost isogenie $\phi $ stupně $n$ s isogenií $[n]$. Při definici isogenie jsme formualovali výrok \uv{$E$ je isogenní s $E^\prime$} jako ekvivalentní relaci. Každé isogenií totiž lze jednoznačně přiřadit její \textit{duál}, jehož vlastnosti nám pomohou studovat jak samotnou isogenii, tak i $[n]$.
 
\chapter{Okruhy Endomorfismů} 
 
Jak napovídá název této sekce, endomorfismy na eliptické křivce tvoří okruh. Tento okruh se budeme snažit s pomocí teorie představené v předchozích kapitolách charakterizovat. Omezíme se pro tentokrát na křivky (a tedy i endomorfismy) nad $\mathbb{F}_p$, kde $p$ je prvočíslo, což nám mnohé věci podstatně usnadní.
\begin{definice}
Mějme $E/\mathbb{F}_p$ eliptickou křivku. Označme $\End(E)$ množinu isogenií $\phi : E \longrightarrow E$, které jsou definované nad $\mathbb{F}_p$, pospolu s $[0]$. Prvky $\End(E)$ nazvěme \textit{endomorfismy} na $E$.
\end{definice}

Endomorfismy definované nad $\mathbb{F}_p$ jsou právě endomorfismy, které převádí množinu $E(\mathbb{F}_p)$ samu na sebe. 
 
\begin{veta}
Množina $\End(E)$ tvoří spolu s operacemi $+$ a $\circ$ okruh.
\end{veta}
\noindent \textit{Důkaz.} Sčítání i skládání endomorfismů je jistě opět definované nad $\mathbb{F}_p$ a je isogenií, $\End(E)$ je proto uzavřený na sčítání i skládání. Sčítání endomorfismů na $E$ je komutativní i asociativní, přičemž $[0]$ je neutrálním prvkem pro sčítání, a ke každé isogenii $\phi$ je isogenie $[-1] \circ \phi$ opačnou k $\phi$. Dále skládání isogenií je asociativní a $[1]$ je jeho neutrálním prvkem. Konečně, skládání je na sčítání oboustranně distributivní, protože endomorfismy na $E$ jsou homomorfismy grup $E(\overline{\mathbb{F}}_p) \longrightarrow E(\overline{\mathbb{F}}_p)$. \hfill $\square$\\

V této kapitole se pokusíme přijít na kloub samotné struktuře okruhu endomorfismů a~grafům isogenií, které nám pospolu s teorií, kterou jsme si představili v předchozí kapitole, pomohou osvětlit funkčnost dalšího kryptografického schématu založeného na isogeniích.

Než začneme tento okruh studovat, všimněme si všudypřítomného injektivního homomorfismu okruhů $\mathbb{Z} \longrightarrow \End(E)$ daného $m \mapsto [m]$. Protože množina složená ze skalárních násobků na $E$ je isomorfní okruhu $\mathbb{Z}$, můžeme v $\End(E)$ isogenie $[m]$ ztotožnit s jejich základem $m$ a považovat inkluzi $\mathbb{Z} \subseteq \End(E)$ za platnou. Kvůli tomuto rozhodnutí bude též přirozenější skládání isogenií zapisovat ve stylu násobení.

\begin{umluva}
V okruhu endomorfismů $\End(E)$ budeme složení isogenií $\phi \circ \psi$ psát jako $\phi \psi$ a~isogenii $[m]$ ztotožníme s číslem $m$.
\end{umluva}


Kvůli multiplikativitě stupnů isogenií (a $0$), můžeme o okruhu endomorfismů říci, že je oborem integrity a díky inkluzi $\mathbb{Z} \subseteq \End(E)$ má nulovou charakteristiku. Surjektivita isogenií nám též umožňuje nenulové endomorfismy oboustranně \uv{krátit}, jak bychom očekávali u okruhu s nenulovou charakteristikou.

\section{Stopa endomorfismu}

Vraťme se nyní na chvíli k první kapitole a duální isogenii. Ta má několik vlastností, které by po seznámení s normou a stopou prvku kvadratického tělesa měly znít povědomě.

Konkrétně, vzpoměňme si na důkaz věty \ref{super}, speciálně že součet $\pi+\widehat{\pi}$ je v $\End(E)$ celé číslo. Naprosto stejně můžeme postupovat u libovolného jiného endomorfismu.
\begin{veta}
Každý endomorfismus $\phi$ na $E$ splňuje $\phi+\widehat{\phi} \in \mathbb{Z}$.
\end{veta}
\noindent \textit{Důkaz}.  Nulová isogenie tvrzení jistě splňuje. Pro ostatní endomorfismy na $E$ si roznásobme výraz $(1 - \phi)(1 - \widehat{\phi})$:
\begin{align*}
\deg (1-\phi) &= (1-\phi)\widehat{(1-\phi)} = (1-\phi)(1-\widehat{\phi}) = 1-(\phi+\widehat{\phi})+\phi \widehat{\phi},\\
\phi+\widehat{\phi} &= 1 + \phi \widehat{\phi} - \deg (1-\phi) = 1 + \deg \phi - \deg (1-\phi) \in \mathbb{Z},
\end{align*}
což jsme chtěli. \hfill $\square$\\


\begin{definice}
Buď $\phi \in \End(E)$ endomorfismus. Pak definujeme jeho \textit{stopu} jako:
\begin{equation*}
\Tr \phi := \phi+\widehat{\phi} \in \mathbb{Z}.
\end{equation*}
\end{definice} 

Můžeme pak ukázat, že každý endomorfismus splňuje v $\End(E)$ kvadratickou rovnici. 
\begin{veta}
Každý endomorfismus $\phi$ na $E$ je v $\End(E)$ kořenem \textit{charakteristického polynomu} $\phi$:
\begin{equation*}
x^2 - \Tr \phi + \deg \phi \in \mathbb{Z}[x].
\end{equation*}
\end{veta}
\noindent \textit{Důkaz.} Víme, že $\Tr \phi = \phi+ \widehat{\phi} = \Tr \widehat{\phi}$ a $\deg \phi = \phi \widehat{\phi} = \deg \widehat{\phi}$ jsou v $\End(E)$ celá čísla. Viétovy vztahy pak tvrdí, že $\phi$ a $\widehat{\phi}$ jsou kořeny polynomu výše. \hfill $\square$\\

Stopa v číselném tělese je aditivní funkcí. Jelikož stopa endomorfismu sdílí s touto funkcí název, nepřekvapí nás její následující vlastnost.

\begin{lemma}
Buďte $\phi, \psi \in \End(E)$. Pak platí $\Tr (\phi+\psi) = \Tr \phi + \Tr \psi$.
\end{lemma}
\noindent \textit{Důkaz.} Díky větě \ref{dual} platí:
\begin{equation*}
\Tr  (\phi+\psi) = \phi+\psi+\widehat{\phi+\psi} = \phi+\psi+\widehat{\phi}+\widehat{\psi} = \Tr \phi + \Tr \psi.
\end{equation*}
\hfill $\square$\\


Zúžení $\phi \vert_n := \phi \vert_{E[n]}$ je endomorfismem na volném $\mathbb{Z}_n$-modulu $E[n]$, který má rank nejvýše $2$, v~případě $n$ nesoudělného s $p$ právě $2$. Matice $M_n$ udávající akci $\phi \vert_n$ je určena volbou báze $E[n]$ jako $\mathbb{Z}_n$-modulu, vždy je však zachován determinant i stopa, můžeme proto takovému zúžení přiřadit determinant i stopu.
\begin{veta}
Buď $E/\mathbb{F}_p$ eliptická křivka a $\phi$ endomorfismus na ní. Pokud $p \nmid n$ je nesoudělné s $\deg \phi$ a $M$ matice $2 \times 2$ působící jako $\phi$ na $E[n]$, platí:
\begin{equation*}
\Tr \phi \equiv \Tr M \pmod{n}, \qquad \qquad \deg \phi \equiv \det M \pmod{n}.
\end{equation*}
\end{veta}
\noindent \textit{Důkaz}. Ať $\phi$ není nulový endomorfismus a buď $x^2 - sx + t$ charakteristický polynom $\phi \vert_n$ jakožto lineárního zobrazení, $M = \begin{pmatrix} a & b\\ c & d \end{pmatrix}$ matice udávající jeho akci, matice $N$ reprezentující $\widehat{\phi} \vert_n$. Protože složení $\phi \vert_n$ a duálu zúžení působí na $E[n]$ jako matice $t I$ s $t$ nesoudělným s $n$, matice $M$ je invertibilní a navíc:
\begin{equation*}
N = t M^{-1} = \frac{t}{\det M} \begin{pmatrix} d & -b\\ -c & a \end{pmatrix}.
\end{equation*} 
Díky $\phi \vert_n + \widehat{\phi} \vert_n = s$ platí dále:
\begin{equation*}
\begin{pmatrix} a & b\\ c & d \end{pmatrix} + \frac{t}{\det M} \begin{pmatrix} d & -b\\ -c & a \end{pmatrix} = sI = \begin{pmatrix} s & 0\\ 0 & s \end{pmatrix}.
\end{equation*}
Platí poté $b - \frac{b t}{\det M} = 0$, $c - \frac{c t}{\det M} = 0$, buď tedy platí $b=c=0$ a~následně $a=d$, nebo $b=c=0$ a~následně $a=d$. První případ dává $t = \det M$ a $s = a+d = \Tr M$, jsme pak hotovi. Druhý případ říká, že $\phi \vert _n$ působí jako skalární násobení na $E[n]$. Tento případ je těžší a pro dostatečně vysoké $n$ je jej též možné dotáhnout do konce, viz \cite[Thm. 7.17]{Sutherland}, plný důkaz tvrzení užívá Weilových párování. \hfill $\square$\\ 

Důkaz lze též vést cestou jiných, tzv. \textit{Tateových}, párování, viz \cite[Prop. III.8.6., Prop. V.2.3.]{Silverman}, to je ale opět nad rozsahem práce.

Připomeňme standardní výsledek, že každá $2 \times 2$ matice $M$ splňuje charakteristickou rovnici $M^2 - \Tr M M + \det M I = 0$. Isogenie působící na $E[\ell]$ splňuje tedy stejnou charakteristickou rovnici modulo $\ell$, jako matice udávající její akci na tuto torzi.

Endomorfismus, o kterém jsme v 1. kapitole často mluvili, je ten pojmenovaný po Frobeniovi, pojďme jej studovat trochu hlouběji. Jeho charakteristický polynom je:
\begin{equation*}
x^2 - tx + p = 0,
\end{equation*}
kde $t = \pi + \widehat{\pi}$ je stopa Frobenia. Tu jsme v důkazu věty \ref{super} určili roznásobením výrazu $(1-\pi)(1-\widehat{\pi}) = \deg (1 - \pi)$:
\begin{align*}
\pi + \widehat{\pi} = p+1-\# E(\mathbb{F}_p).
\end{align*}
V případě supersingulární křivky je stopa Frobenia nulová a $\pi =  \pm \sqrt{-p} \not\in \mathbb{Z}$. Platí tak inkluze $\mathbb{Z}[\sqrt{-p}] \subseteq \End(E)$. V další sekci okruh endomorfismů umístíme do kvadratického tělesa, čímž pak zásadně omezíme jeho možné tvary.

Pozastavme se ještě nad kvadratickým vztahem udávajícím Frobeniův endomorfismus, ne nutně již nad supersingulární křivkou. Ten nám pomůže poodhalit tajemství struktury grafů isogenií prvočíselného stupně $\ell$, konkrétně kolik hran vychází z $j$-invariantu reprezentujícího příslušnou třídu isomorfismu. K tomu se na rovnice udávající isogenie musíme podívat ne jako celé, ale pouze modulo $\ell$. Tato sekce postupuje volně podle \cite[Sec. 6]{Schoof2}.

Nejprve si propojíme zúžení Frobeniova morfismu na $\ell$-torze s rovnicí udávající Frobeniova morfismu modulo $\ell$.

\begin{lemma}
Buď $\phi : E \longrightarrow E^\prime$ isogenie mezi křivkami nad $\mathbb{F}_p$ prvočíselného stupně $\ell$. Pak $\phi$ je definovaná nad $\mathbb{F}_p$, právě pokud platí $\pi (\ker \phi) = \ker \phi$.
\end{lemma}
\noindent \textit{Důkaz.} Endomorfismus $\pi$ je na bodech obou křivek $E, E^\prime$ prostý. Druhou podmínku zadání můžeme proto relaxovat na $\pi \ker \phi \subseteq \ker \phi$. Pokud je $\phi$ definovaná nad $\mathbb{F}_p$, tak je invariantní pod kompozicí s $\pi$ a každý prvek jádra je zobrazen sám na sebe. Druhý směr lze ukázat studiem akce Galoisovy grupy $\mathrm{Gal}(\overline{\mathbb{F}}_p /\mathbb{F}_p)$ na jádro $\phi$, pro více informací viz \cite[Lemma 24.]{Suchanek}. \hfill $\square$\\

\begin{dusledek}
Buďte $\ell$ prvočíslo a $P \in E[\ell]$ afinní bod. Pak je separabilní isogenie $\phi : E \longrightarrow E/\langle P \rangle$ definovaná nad $\mathbb{F}_p$, právě pokud $\pi (P) = \lambda P$ pro nějaké $\lambda$.
\end{dusledek}
\noindent \textit{Důkaz.} Pokud je $\phi$ definovaná nad $\mathbb{F}_p$, tak dle předchozího lemmatu je $\pi (P) \in \ker \phi = \langle P \rangle$, tj. $\pi (P) = \lambda P$ pro $\lambda$ celé. Naopak je-li $\pi(P) = \lambda P (\neq \mathcal{O})$ pro celé $\lambda$, tak platí $\pi \ker \phi = \pi \langle P \rangle = \langle \pi( P) \rangle = \langle \lambda P \rangle$, což je rovno $\langle P \rangle = \ker \phi$. \hfill $\square$\\

Pojďme si ještě jednou přeformulovat podmínku na skutečnost, že isogenie generovaná bodem $P$ řádu $\ell$ je definovaná nad $\mathbb{F}_p$. Následující tvrzení ukotví jeho souvislost s charakteristickým polynomem $\pi$.

\begin{lemma}
Nechť $P \in E[\ell]$ je afinní. Pokud existuje celé $\lambda$ splňující $\pi (P) = \lambda P$, tak je kořenem rovnice $x^2 - tx + p$ modulo $\ell$. 
\end{lemma}
\noindent \textit{Důkaz.} Ať pro celé $\lambda$ platí $\pi(P) = \lambda P$. Navíc též $\pi^2 (P) = \pi(\pi (P)) = \pi(\lambda P) = \lambda \pi(P) = \lambda^2 P$ a tedy:
\begin{equation*}
\mathcal{O} = (\pi^2 - t \pi + p)P = (\lambda ^ 2 - t \lambda + p)P.
\end{equation*}
Protože řád $P$ je $\ell$, je $\lambda$ kořenem polynomu $x^2 - tx + p$ modulo $\ell$. \hfill $\square$\\


Problém zjišťování akce $\pi$ na $\ell$-torzi můžeme tedy převést na řešení charakteristického polynomu $\pi$ modulo $\ell$. 

Díky předchozí větě konečně můžeme charakterizovat isogenie stupně $\ell$ vycházející z $j(E)$ definované nad $\mathbb{F}_p$ v grafu $j$-invariantů, či ekvivalentně podgrupy $E[\ell]$ řádu $\ell$, na kterých $\pi$ působí jako skalár.

\begin{veta}
Buď $E/\mathbb{F}_p$ eliptická křivka a $\ell \neq p$ prvočíslo. Pak počet isogenií definovaných nad $\mathbb{F}_p$ stupně $\ell$ vycházejících z $E$ je, až na isomorfismus, roven buď $0,1,2$ či $\ell+1$. 
\end{veta}
\noindent \textit{Důkaz.} Isogenie stupně $\ell \neq p$ zjevně jsou separabilní a podle věty \ref{isomor} jsou v korespondenci s podgrupami $E[\ell]$, přičemž takových grup je na celé $E[\ell]$ díky větě \ref{l+1} $\ell+1$. Abychom získali isogenii definovanou nad $\mathbb{F}_p$, musí $\pi$ na příslušnou grupu působit jako skalár. 

Ukážeme nyní, že pokud existují alespoň $3$ isogenie definované nad $\mathbb{F}_p$, pak už jich musí existovat $\ell+1$. Ať jsou $P,Q,R \in E[\ell]$ tři různé body, které definují isogenie definované nad $\mathbb{F}_p$ stupně $\ell$ s jádry po řadě $\langle P\rangle, \langle Q \rangle, \langle R \rangle$. Podle předchozích dvou vět musí $\pi\vert_\ell$ na těchto třech bodech působit jako skalár, přičemž možné hodnoty jsou (nejvýše dvě) vlastní čísla matice udávající $\pi \vert_{\ell}$. Na některých dvou bodech, ať to jsou $P$ a $Q$, působí $\pi \vert_{\ell}$ jako ten samý skalár, $\lambda$. Krom $\mathcal{O}$ se $\langle P \rangle$ a $\langle Q \rangle$ neprotínají a obě mají $\ell$ prvků, tj. $(P,Q)$ tvoří bázi $E[\ell]$. Pro každý bod $L \in E[\ell]$ s $L = aP+bQ$ pak platí $\pi L = \pi(aP+bQ) = a \pi P + b \pi Q = a \lambda P + b \lambda Q = \lambda(aP + bQ) = \lambda L$, $\pi$ tedy působí na celou $\ell$-torzi jako násobení $\lambda$. Speciálně tak působí na každou z $\ell+1$ podgrup $E[\ell]$ a každá z~nich tak definuje jednu separabilní isogenii. Pokud tedy právě $0,1$ či $2$ podgrup $E[\ell]$ řádu $\ell$ není jádrem isogenie nad $\mathbb{F}_p$, tak jsou jádrem isogenie nad $\mathbb{F}_p$ všechny. \hfill $\square$\\

Podíváme-li se pozorněji na vlastní čísla udávající $\pi \vert_{\ell}$, dokážeme dokonce přesně určit, kdy který z případů nastane.

\begin{veta}
Buď $E/\mathbb{F}_p$ eliptická křivka a $\ell \neq p$ prvočíslo. Pak počet isogenií definovaných nad $\mathbb{F}_p$ stupně $\ell$ vycházejících z $E$ je, až na isomorfismus, roven:
\begin{enumerate}
\item $0$, pokud charakteristická rovnice $\pi$ nemá řešení modulo $\ell$,
\item $1$, pokud charakteristická rovnice $\pi$ má jeden kořen $\lambda$ a působí jako násobení $\lambda$ na právě jednu podgrupu $E[\ell]$ řádu $\ell$,
\item $2$, pokud charakteristická rovnice $\pi$ má dva různé kořeny modulo $\ell$,
\item $\ell+1$, pokud charakteristická rovnice $\pi$ má jeden kořen $\lambda$ a působí jako násobení $\lambda$ na všech podgrupách $E[\ell]$ řádu $\ell$.
\end{enumerate}

\end{veta}

\noindent \textit{Důkaz.} Dejme nejprve tomu, že se $x^2 - tx+p$ rozkládá modulo $\ell$ jako:
\begin{equation*}
x^2 - tx + p \equiv (x-\lambda) (x- \mu) \pmod{\ell},
\end{equation*} 
pro $\lambda,\mu \in \mathbb{Z}_{\ell}$ ne nutně různá vlastní čísla matice udávající $\pi \vert_{\ell}$. Pro některé prvky $E[\ell]$ se pak $\pi_{\ell}$ chová jako skalární násobek $\lambda$, resp. $\mu$ a navíc podle předchozí věty tyto prvky generují separabilní isogenii definovanou nad $\mathbb{F}_p$. Pokud se tedy charakteristický polynom nerozkládá, $\pi$ nepůsobí na žádném prvku $E[\ell]$ jako skalár a hledaná isogenie neexistuje.

Dále ať se polynom rozkládá a rozdělme si práci podle násobnosti kořenů. Nejprve uvažme, že $x^2 - tx + p$ má dva různé kořeny $\lambda, \mu \in \mathbb{Z}_{\ell}$. To znamená, že $\pi \vert_{\ell}$ se na nějakých dvou bodech $P,Q$ chová jako násobení $\lambda$, resp. $\mu$. Speciálně se chová jako příslušný skalární násobek na $\langle P \rangle$, resp. $\langle Q \rangle$, přičemž tyto grupy jsou až na bod v nekonečnu disjunktní a obě mají řád $\ell$. Nutně pak $\langle P,Q \rangle$ tvoří bázi $E[\ell]$. Ukážeme nyní, že krom grup generovaných $P,Q$ se $\pi$ na žádných jiných bodech $E[\ell]$ nechová jako skalární násobek. Ať naopak $R = [a]P + [b]Q$ je bod splňující $\pi R = \eta R$ pro $\eta \in \mathbb{Z}$ a $a,b$ jsou v $\mathbb{Z}_{\ell}$ nenulová. Pak:
\begin{equation*} 
[a \eta] P + [b \eta] Q = \eta ([a]P + [b]Q) = \eta R = \pi R = \pi ([a]P + [b]Q) = [a]\pi P + [b] \pi Q = [a \lambda] P + [b \mu] Q,
\end{equation*}
tedy protože $P,Q$ tvoří bázi $E[\ell]$, musí platit $a \eta \equiv a \lambda \pmod{\ell}$ a $b \eta \equiv b \mu \pmod{\ell}$. Protože $\ell \nmid a,b$, musí nastat $\lambda \equiv \eta \equiv \mu \pmod{\ell}$, což je spor. $\pi$ se proto chová jako skalární násobek pouze na $\langle P \rangle$ a $\langle Q \rangle$ a obě grupy generují (jediné) isogenie stupně $\ell$.

Konečně, ať $x^2 - tx + p$ má kořen  násobnosti $2$, tj. $x^2 - tx + p \equiv (x-\lambda)^2 \pmod{\ell}$.  To znamená, že $\pi \vert_{\ell}$ se na nějakém bodě $P \in E[\ell]$ chová jako násobení $\lambda$. Případ, kdy $\pi$ se chová jako násobení $\lambda$ pouze a jenom na $\langle P \rangle$, dává pouze jednu isogenii stupně $\ell$ mající jádro $\langle P \rangle$.

Pokud naopak $\pi P = \lambda P$ a $\pi Q = \lambda Q$ platí na dvou bodech, které neleží na torzích toho druhého, tak tato dvojice musí tvořit bázi $E[\ell]$ a tedy $\pi$ působí jako násobení $\lambda$ na celé $\ell$-torzi. Ta má díky větě \ref{l+1} $\ell+1$ podgrup řádu $\ell$, každá definuje jednu separabilní isogenii. \hfill $\square$\\

Zmiňme na konec, že byť kvůli dalším sekcím této kapitoly pracujeme pouze s~endomorfismy na $E$ definovanými nad $\mathbb{F}_p$, naprosto obdobně bychom mohli postupovat i v případě eliptické křivky nad $\mathbb{F}_q$. Počet separabilních isogenií vedoucích z křivky $E/\mathbb{F}_q$ stupně $\ell \neq p$ definovaných nad $\mathbb{F}_q$ je (až na isomorfismus) roven $0,1,2$ či $\ell+1$.

\section{Algebra endomorfismů}

V této sekci okruh endomorfismů rozšíříme do vektorového prostoru nad $\mathbb{Q}$ za pomoci tenzorového součinu. Ten skrývá známou strukturu imaginárního kvadratického tělesa a~to ne jen tak libovolného, dokonce $\mathbb{Q}(\pi)$. Představme si proto tento modul:

\begin{definice}
Buď $E/\mathbb{F}_p$ eliptická křivka. Pak $\mathbb{Z}$-modul definovaný jako:
\begin{equation*}
\End^0 (E) := \mathbb{Q} \otimes_{\mathbb{Z}} \End(E)
\end{equation*}
nazveme \textit{algebrou endomorfismů} $E$.
\end{definice}

Algebra endomorfismů je generovaná formálními výrazy $r \otimes \phi$, kde $r \in \mathbb{Q}, \phi \in \End(E)$, podle věty \ref{qtensor} je každý její prvek právě takového tvaru. Zásadní problém s propozicí, že algebra endomorfismů je těleso, je součin tenzorů, který jsme si nedefinovali. Protože ale $\End^0 (E)$ má \uv{jednoduché} prvky, součin dvou tenzorů přichází přímočaře.
\begin{definice}
Buďte $r \otimes \phi, s \otimes \psi \in \End^0(E)$. Pak definujeme:
\begin{equation*}
(r \otimes \phi) (s \otimes \psi) := rs \otimes \phi \psi.
\end{equation*} 
\end{definice}

Mnoho vlastností okruhu endomorfismů sáha i do $\End^0 (E)$, speciálně zřejmě je oborem integrity a má nulovou charakteristiku.

Jistě opět panují injektivní homomorfismy $\mathbb{Q} \longrightarrow \End^0 (E)$ a $\End(E) \longrightarrow \End^0 (E)$ dané zobrazeními $r \mapsto r \otimes 1$, resp. $\phi \mapsto 1 \otimes \phi$. Ty bychom znovu chtěli uvažovat raději jako inkluze. Jednou z vlastností tenzorového součinu je $\frac{a}{b} \otimes \phi = \frac{1}{b} \otimes a \phi$ pro $a,b \in \mathbb{Z}$ a tedy ztotožnění $r \otimes \phi$ s $r \phi$ je dobře definované. 

\begin{umluva}
Prvky algebry endomorfismů budeme místo $r \otimes \phi$ značit $r \phi$ a považovat inkluze $\mathbb{Q} \subseteq \End^0 (E)$ a $\End(E) \subseteq \End^0 (E)$ za platné.
\end{umluva}

Zastavme se nyní, poohlédněme se na předchozí kapitoly, a naplánujme další postup útoku. Prostředek, který připomíná vlastnosti kvadratických těles nejvíce, je duální isogenie jako sdružené číslo prvku. Tento koncept si proto rozšíříme i na algebru endomorfismů:
\begin{definice}
Buďte $r \in \mathbb{Q}$ a $\phi \in \End(E)$ endomorfismus. Pak pro $r \phi \in \End^0 (E)$ definujeme \textit{Rosatiho involuci} $\widehat{r \phi} := r \widehat{\phi}$.
\end{definice}
Jistě položením $\phi = 1$ v definici výše dává $r = \widehat{r}$ a opět snadno dojdeme k tomu, že $\widehat{r \phi}$ je opravdu involuce. I ostatní vlastností involuce $\widehat{\phi}$, tedy že je aditivní a antihomomorfismem, samozřejmě platí v $\End^0 (E)$. No a kde se vyskytuje konjugát, tam se podívají i stopa a~norma.
\begin{definice}
Normu a stopu prvku $\alpha \in \End^0(E)$ definujeme jako:
\begin{align*}
\Tr \alpha &= \alpha+\widehat{\alpha},\\
\N \alpha &= \alpha \widehat{\alpha}. 
\end{align*}
\end{definice}

Pojďme tedy začít v rychlosti budovat korespondence mezi stopou a normou endomorfismu a těmi příslušící prvku (imaginárního) kvadratického tělesa, kterých je opravdu velká spousta.

\begin{veta}
Norma i stopa $\alpha \in \End^0 (E)$ jsou racionální čísla, přičemž norma je nezáporná. Navíc norma je nulová, jen pokud $\alpha = 0$.
\end{veta}
\noindent \textit{Důkaz}. Norma prvku $r \phi$ je rovna $r \phi \widehat{r \phi} = r^2 \phi \widehat{\phi} = r^2 \deg \phi$, což je nezáporné racionální číslo, a jeho stopa je $r \phi + \widehat{r \phi} = r (\phi+\widehat{\phi}) = r \Tr \phi$, racionální číslo též. Pokud je norma $\alpha$ nulová, je buď $r=0$, nebo $\deg \phi = 0$, každopádně $\alpha = 0$. \hfill $\square$\\

Norma i stopa jsou úzce spojeny s konjugáty prvku racionálního čísla, konkrétně ji všechny sdílí. Vlastností, které jsme o normě a stopě odvozovali ve třetí kapitole, přichází prakticky zadarmo.

\begin{veta}
Pro libovolná $\alpha,\beta \in \End^0 (E)$ a $k, \ell \in \mathbb{Q}^{+}$ platí $\Tr \left( k\alpha + \ell \beta \right) = k\Tr \alpha + \ell\Tr \beta$, $\Tr \alpha = \Tr \widehat{\alpha}$, $N \alpha = N \widehat{\alpha}$ a~$\N \alpha \beta = \N \alpha \N \beta$.
\end{veta}
\noindent \textit{Důkaz.} Aditivita a stopy plyne z aditivity involuce $\widehat{\phi}$ a $\mathbb{Q}$-linearita pak plyne z definice. Protože je Rosatiho involuce involucí, platí:
\begin{equation*}
\Tr \widehat{\alpha} = \widehat{\alpha} + \widehat{\widehat{\alpha}} = \widehat{\alpha} + \alpha = \Tr \alpha.
\end{equation*}
Dále:
\begin{equation*}
\alpha \N \widehat{\alpha} = \alpha \widehat{\alpha} \alpha = \N \alpha \alpha = \alpha \N \alpha,
\end{equation*} tedy, protože algebra endomorfismu je oborem integrity, platí $\N \alpha = \N \widehat{\alpha}$. Konečně, podle věty \ref{dual} platí:
\begin{equation*}
\N \alpha \beta = \alpha \beta \widehat{\alpha \beta} = \alpha \beta \widehat{\beta} \widehat{\alpha} = \alpha (\N \beta) \widehat{\alpha} = \alpha \widehat{\alpha} \N \beta = \N \alpha \N \beta.
\end{equation*}
\hfill $\square$\\


\begin{dusledek}
Buď $\alpha \in \End^0 (E)$ nenulové. Pak má v $\End^0 (E)$ multiplikativní inverzi.
\end{dusledek}
\noindent \textit{Důkaz.} Položme $1/\alpha = \widehat{\alpha}/\N \alpha \in \End^0 (E)$. Ukážeme, že toto $1/\alpha$ je hledanou inverzí, platí totiž $\alpha \cdot 1/ \alpha = \alpha \widehat{\alpha} /\N \alpha = 1$, je tedy levou inverzí. Analogicky $(\widehat{\alpha}/\deg \alpha) \alpha =  1$, tedy $1/\alpha$ je opravdu hledaným prvkem. \hfill $\square$\\

Předchozí věta nám opodstatní fakt, že algebra endomorfismů tvoří \textit{division ring}, tedy splňuje všechny podmínky na těleso až na nutnost komutativity násobení. Tento objekt je proto tělesem, právě pokud je násobení komutativní. Tímto způsobem algebru endomorfismů klasifikovat nebudeme, zvolíme trochu mazanější přístup. Nejprve vidíme, že každý prvek této algebry je nad racionálními čísly nejvýše kvadratický.\\

\begin{veta}\label{kv}
Každé $\alpha \in \End^0 (E)$ je kořenem polynomu:
\begin{equation*}
x^2 - \Tr \alpha x + \N \alpha \in \mathbb{Q}[x].
\end{equation*}
\end{veta}
\noindent \textit{Důkaz.} Viétovy vztahy říkají, že kořeny tohoto polynomu jsou $\alpha$ a $\widehat{\alpha}$. \hfill $\square$\\

Povšimněme si, že pokud bychom endomorfismy i křivky místo nad $\mathbb{F}_p$ doteď definovali nad nějakým jeho rozšířením, pramálo by se změnilo, násobení se ale už lišit bude. Významným výsledkem připisovaným Maxu Deuringovi \cite{Deuring} je klasifikace plných algeber endomorfismů nad libovolným konečným tělesem. Ukáže se, že buď jsou isomorfní tělesu racionálních čísel (pouze křivky nad racionálními čísly), imaginárnímu kvadratickému tělesu, či tzv. \textit{kvaternionové algebře}, tedy rozšíření $\mathbb{Q}(\alpha,\beta)$ s $\alpha \beta = - \beta \alpha$, kde na pořadí zápisu násobení jistě záleží. Obecně všechny supersingulární křivky mají algebru endomorfismů kvaternionovou algebru a obyčejné algebry endomorfismů berou formu kvadratického tělesa. Hezký, poměrně elementární důkaz je k nalezení na \cite[Thm. 13.17]{Sutherland}. V případě naší \uv{zjednodušené} algebry endomorfismů, kde endomorfismy bereme definované pouze nad $\mathbb{F}_p$, však pro supersingulární křivky nastává opačný případ. Nejprve ukažme, že endomorfismy komutují s racionálními čísly.

\begin{lemma}
Buďte $r \in \mathbb{Q}$ a $\alpha \in \End^0 (E)$. Pak platí $r \alpha = \alpha r$.
\end{lemma}
\noindent \textit{Důkaz.} Položme $\alpha = s \phi$ pro $s \in \mathbb{Q}$ a $\phi$ endomorfismus. Podle definice násobení:
\begin{equation*}
r \alpha = (r) (s \phi) = (rs) \phi = (s \phi) r = \alpha r,
\end{equation*}
což jsme chtěli. \hfill $\square$\\

\begin{veta}
Buďte $E/\mathbb{F}_p$ supersingulární křivka a libovolné $\alpha \in \End^0 (E)$. Pak $\alpha \in \mathbb{Q}(\pi)$.
\end{veta}
\noindent \textit{Důkaz.} Nejprve vidíme, že protože definujeme křivku i endomorfismy definované nad $\mathbb{F}_p$, $\pi = \pm \sqrt{-p}$ není racionální číslo, a navíc platí $\alpha \pi =  \pi \alpha$.
Dále si všimněme, že komutují-li dva prvky $x,y \in \End^0 (E)$, tak aplikace lineární transformace $x \mapsto ax+b$ pro $a,b \in \mathbb{Q}$ komutativitu zachová:
\begin{equation*}
(ax+b)y  = axy + by = ayx + by  = yax + yb = y(ax+b).
\end{equation*}
Speciálně transformace $\alpha \mapsto \alpha - \frac{\Tr \alpha}{2}$ dává:
\begin{equation*}
\left( \alpha - \frac{\Tr \alpha}{2} \right) \pi = \pi \left( \alpha - \frac{\Tr \alpha}{2} \right),
\end{equation*}
tedy roznásobením opět platí $\alpha \pi = \pi \alpha$. Prvek $\alpha$ má ale nulovou stopu, jelikož: $$\Tr \left(\alpha - \frac{\Tr \alpha}{2} \right) = \Tr \alpha - \Tr \left( \frac{\Tr \alpha}{2} \right) = \Tr \alpha - \left(\frac{\Tr \alpha}{2} + \frac{\widehat{\Tr \alpha}}{2} \right) = 0,$$
kde užíváme základní vlastnosti stopy. Obdobně můžeme zvolit $\tilde{\pi} = \pi - \frac{\Tr \pi}{2}$, který má stopu nulovou, sám však nulový není, a $\alpha, \tilde{\pi}$ spolu komutují. Dále, zvolme $\tilde{\alpha} = \alpha - \frac{\Tr \alpha \tilde{\pi}}{2 \tilde{\pi}}$, tento prvek zjevně komutuje s $\tilde{\pi}$. Jeho stopa je rovna:
\begin{equation*}
\Tr \tilde{\alpha} = \Tr \alpha - \Tr \frac{\Tr \alpha \tilde{\pi}}{2 \tilde{\pi}} = \Tr \alpha - \left( \frac{\Tr \alpha \tilde{\pi}}{2}\right) \Tr \frac{1}{\tilde{\pi}} = -\left( \frac{\Tr \alpha \tilde{\pi}}{2}\right) \Tr \frac{1}{\tilde{\pi}}.
\end{equation*}
Platí ale:
\begin{equation*}
\Tr \frac{1}{\tilde{\pi}} = \frac{1}{\tilde{\pi}} + \frac{1}{\widehat{\tilde{\pi}}} = \frac{\tilde{\pi} + \widehat{\tilde{\pi}}}{\tilde{\pi} \widehat{\tilde{\pi}}} = 0
\end{equation*}
díky supersingularitě $E$, tedy $\Tr \tilde{\alpha} = 0$. Konečně, součin $\tilde{\alpha}$ a $\tilde{\pi}$ má stopu nulovou též:
\begin{equation*}
\Tr \tilde{\pi} \tilde{\alpha} = \Tr \left( \alpha \tilde{\pi} - \frac{\Tr \alpha \tilde{\pi}}{2} \right) = \Tr \alpha \tilde{\pi} - \Tr \frac{\Tr \alpha \tilde{\pi}}{2} = 0,
\end{equation*}
což jsme chtěli. Prvky $\tilde{\alpha}, \tilde{\pi}$ proto splňují $\Tr \alpha = \Tr \tilde{\pi} = \Tr \tilde{\alpha} \tilde{\pi} = 0$. Pak díky komutativitě $\tilde{\alpha}$ a $\tilde{\pi}$:
\begin{equation*}
\tilde{\alpha} \tilde{\pi} = -\widehat{\tilde{\alpha} \tilde{\pi}} = - \widehat{\tilde{\pi}} \widehat{\tilde{\alpha}} = - (- \tilde{\pi}) (-\tilde{\alpha}) = - \tilde{\pi} \tilde{\alpha}= - \tilde{\alpha} \tilde{\pi}, 
\end{equation*}
neboli $2 \tilde{\alpha} \tilde{\pi} = 0$. Algebra $\End^0 (E)$ je oborem integrity, tedy musí být jeden z $\tilde{\alpha},\tilde{\pi}$ nulový. Frobeniův endomorfismus není racionálním číslem, musí proto $\tilde{\alpha}$ být nulové, z čehož plyne $\alpha = \frac{\Tr \alpha \tilde{\pi}}{2 \tilde{\pi}} \in \mathbb{Q}(\pi)$. \hfill $\square$\\

\begin{dusledek}
Ať $E/\mathbb{F}_p$ je supersingulární křivka. Pak $\End^0 (E) = \mathbb{Q}(\pi)$.
\end{dusledek}
\noindent \textit{Důkaz.} Předchozí věta naznačuje inkluzi $\End^0 \subseteq \mathbb{Q}(\pi)$. Víme ale, že $\pi$ má stupeň $2$ nad racionálními čísly a každý jeho racionální násobek v algebře endomorfismů leží, čímž svíráme algebru endomorfismů z obou stran: $\mathbb{Q}(\pi) \subseteq \End ^0 (E) \subseteq \mathbb{Q}(\pi)$, nutně musí nastat rovnost $\End^0 (E) = \mathbb{Q}(\pi)$. \hfill $\square$\\

Speciálně, protože racionální čísla i $\pi$ komutují s libovolným prvkem algebry endomorfismů, tento obor je komutativní. 

Tak a nyní si můžeme užívat mnoho vlastností algebry endomorfismů (a tedy i okruhu endomorfismů) jako kvadratického tělesa, které dokazovat přímo by bylo bolestivé. Začněme s okruhem endomorfismů.

\begin{veta}
Buď $E/\mathbb{F}_p$ supersingulární křivka. Pak $\End(E)$ je pořádkem v $\mathbb{Q}(\pi)$. 
\end{veta}
\noindent \textit{Důkaz.} Protože každý prvek $\End(E)$ je kořenem kvadratického monického polynomu nad celými čísly (viz věta \ref{kv}), je celý algebraický a okruh endomorfismů je obsažen v maximálním pořádku $\mathcal{O}_{\mathbb{Q}(\pi)}$, což je $\mathbb{Z}$-modul ranku $2$. Navíc díky $\pi \not\in \mathbb{Q}$ a~inkluzi $\mathbb{Z}[\pi] \subseteq \End(E)$ svíráme okruh endomorfismů mezi dvěma volnými $\mathbb{Z}$-moduly ranku $2$. Díky větě \ref{podmodul} je $\End(E)$ sám volným $\mathbb{Z}$-modulem ranku $2$. \hfill $\square$\\

Maximální pořádek je určen zbytkem, který $p$ dává po dělení čtyřmi, pojďme si oba případy rozebrat. Pokud $p \equiv 1 \pmod{4}$, maximální pořádek v $\mathbb{Q}(\pi) \cong \mathbb{Q}(\sqrt{-p})$ je $\mathbb{Z}[\pi]$ a~tak musí platit $\End(E) = \mathbb{Z}[\pi]$. V případě $p \equiv -1 \pmod{4}$ máme zase řetězec inkluzí:
\begin{equation*}
\mathbb{Z}[\pi] \subseteq \End (E) \subseteq \mathbb{Z} \left[ \frac{1+\pi}{2} \right].
\end{equation*}

\begin{poznamka}
Výraz $\frac{1+\pi}{2}$ jako takový nedává v okruhu endomorfismů smysl, protože je formálně roven $\frac{1}{2} \otimes (1+\pi)$. Isogenie $[2]$ na $E$ je ale surjektivní, tedy můžeme tento výraz považovat jako prvek splňující $2 \frac{1+\pi}{2} = 1+\pi$ a šťastně s ním pracovat jako s endomorfismem.
\end{poznamka}

Při hledání okruhu endomorfismů nás už pouze volba prvočísla nezachrání.
\begin{priklad}
Podívejme se na supersingulární křivky $E_1/\mathbb{F}_{19} : y^2 = x^3 + x$ a $E_2/\mathbb{F}_{19} : y^2 = x^3-x$. Obě křivky mají shodný Frobeniův endomorfismus $\sqrt{-19}$ a tedy i algebru endomorfismů $\mathbb{Q}(\sqrt{-19})$. Křivka $E_1$ má okruh endomorfismů pouze pořádek $\mathbb{Z}[\pi]$, zato okruh endomorfismů křivky $E_2$ je maximální pořádek $\mathbb{Z}\left[ \frac{1+\pi}{2} \right]$.
\end{priklad}

Krom komutativity samotného okruhu endomorfismů můžeme ukázat i komutativitu isogenií s endomorfismy.

\begin{lemma}\label{komut}
Buďte $E_1,E_2$ dvě křivky nad $\mathbb{F}_p$ a $\phi : E_1 \longrightarrow E_2$ isogenie. Pokud jsou $\mathcal{O}_1,\mathcal{O}_2$ pořádky $\mathbb{Q}(\pi)$ příslušící po řadě $\End(E_1)$, resp. $\End(E_2)$, tak libovolný endomorfismus $\alpha \in \mathcal{O}_1 \cap \mathcal{O}_2$ s $\phi$ komutuje, tedy $\phi \alpha = \alpha \phi$.
\end{lemma}
\noindent \textit{Důkaz.} Endomorfismus $\alpha \in \mathcal{O}_1 \cap \mathcal{O}_2$ můžeme vyjádřit jako $\frac{a+b \pi}{2}$, kde $a,b$ jsou celá čísla, protože $\End(E_i) \subseteq \mathbb{Z}\left[ \frac{1+\pi}{2} \right]$. Případ $\alpha = 0$ je zřejmý, jinak jsou endomorfismy surjektivní, pro každý bod $P \in E$ tedy existuje $Q$ splňující $2Q = P$. Pak pro každý $P \in E$ platí:
\begin{equation*}
\phi \alpha P = \phi \frac{a+b \pi}{2} P = \phi (a+b \pi) Q = (a+b \pi) \phi Q = \frac{a+b \pi}{2} \phi 2Q = \frac{a+ b \pi}{2} \phi P = \alpha \phi P.
\end{equation*}
\hfill $\square$\\

V případě obyčejných křivek jsou okruhy endomorfismů velmi rozmanité, nebudeme se jimi ale dále zabývat. U supersingulárních křivek můžeme však jednoduchost okruhů endomorfismů využít při charakterizaci isogenií mezi křivkami.

\begin{veta}\label{cond}
Buďte $E,E^\prime$ křivky nad $\mathbb{F}_p$, mezi kterými existuje isogenie $\phi$ stupně $\ell$, a~$\mathcal{O},\mathcal{O}^{\prime}$ pořádky příslušící jejich okruhu endomorfismů. Pak nastává jeden z následujících případů:
\begin{itemize}
\item $\mathcal{O} = \mathcal{O}^{\prime}$, pak $\phi$ nazveme \textit{horizontální.} V opačném případě ji nazveme vertikální a~platí jedno z následjících:
\item $[\mathcal{O} : \mathcal{O}^{\prime}] = \ell$,
\item $[\mathcal{O}^{\prime} : \mathcal{O}] = \ell$.
\end{itemize}
\end{veta}
\noindent \textit{Důkaz.} Existují $\vartheta,\upsilon \in \End^0 (E)$ taková, že $\mathcal{O} = \mathbb{Z}[\vartheta]$ a $\mathcal{O}^{\prime} = \mathbb{Z}[\upsilon]$ jsou v~$\End ^0 (E)$ pořádky. Endomorfismus $\phi \vartheta \widehat{\phi} \in \mathbb{Z}[\upsilon]$ má duál $\phi \widehat{\vartheta} \widehat{\phi}$ díky komutativitě endomorfismů a~isogenií. Spočtěme si jeho normu a stopu:
\begin{align*}
\N \phi \vartheta \widehat{\phi} &= \phi \vartheta \widehat{\phi} \phi \widehat{\vartheta} \widehat{\phi} = \phi \vartheta \ell \widehat{\vartheta} \widehat{\phi} = \phi \ell \vartheta \widehat{\vartheta} \ell \widehat{\phi} = \phi \ell (\N \vartheta) \widehat{\phi} = \ell (\N \vartheta) \phi \widehat{\phi} = \ell ^2 \N \vartheta = \N \ell \vartheta,\\
\Tr \phi \vartheta \widehat{\phi} &= \phi \vartheta \widehat{\phi} + \phi \widehat{\vartheta} \widehat{\phi} = \phi (\vartheta + \widehat{\vartheta}) \widehat{\phi} = \phi (\Tr  \vartheta) \widehat{\phi} = \phi \widehat{\phi} \Tr \vartheta = \ell \Tr \vartheta = \Tr \ell \vartheta.
 \end{align*}
 Endomorfismus $\ell \vartheta$ tedy splňuje stejnou kvadratickou rovnici jako $\phi \vartheta \widehat{\phi}$ a je buď roven jemu, nebo jeho duálu. Tak či tak $\ell \vartheta  \in \mathbb{Z} [ \upsilon ]$ a analogicky $\ell \upsilon \in \mathbb{Z} [ \vartheta]$. To znamená, že existují $a,b,x,y \in \mathbb{Z}$, která splňují $\ell \vartheta = a \upsilon + b$ a $\ell \upsilon = x \vartheta + y$. Pak:
\begin{equation*}
\ell^2 \vartheta = a \ell \upsilon + b \ell = a (x \vartheta + y) + b \ell,
\end{equation*}
neboli porovnáním koeficientů $\ell^2 = ax$ a $0 = ay + b \ell$. Musí nastat jedna z rovností $(a,x) = (1,\ell^2), (\ell,\ell), (\ell^2,1)$. Druhý případ dává $ \ell \mid b$ a $\vartheta = \upsilon + b/\ell$, tedy $\mathbb{Z}[\vartheta] = \mathbb{Z}[\upsilon]$. Naopak první případ dává $\ell \vartheta = \upsilon + b$ a $[\mathbb{Z}[\upsilon] : \mathbb{Z}[\vartheta]] = \ell$, třetí případ je analogický. \hfill $\square$\\

V případě vertikálních isogenií mezi supersingulárními křivkami víme, že možné stupně rozšíření jsou pouze $1$ a $2$, tedy mezi takovými křivkami existují vertikální isogenie stupně nejvýše $2$ a všechny ostatní nutně zachovají okruh endomorfismů. Toto rozdělení isogenií a speciálně pak horizontální isogenie a jejich vlastnosti ve spojení s grafy isogenií jsou mnohem podrobněji studovány v \cite[Ch. 4.]{Suchanek}, čtenáře s pocitem nedostatku informací obsažených v následující sekci proto vřele odkazujeme. 

\section{Isogenie generované ideály}

Mějme v této sekci $E/\mathbb{F}_p$ supersingulární eliptickou křivku a $\mathcal{O}$ pořádek v kvadratickém tělese $\mathbb{Q}(\pi)$ s vodičem dělícím $2$ příslušící $\End(E)$. Libovolný invertibilní ideál $\mathfrak{a} \subseteq \mathcal{O}$ se jednoznačně rozkládá na součin prvoideálů. Z~každého takového ideálu zkonstruujeme isogenii vycházející z $E$, který má mnoho společného s jeho jádrem. Abychom si nepletli pořádky a bod v nekonečnu, ten pozdější budeme ve zbytku této sekce značit $O$.

Nejprve si zobecníme $n$-torzi na ideály pořádku příslušícího okruhu endomorfismů.

\begin{definice}
Buď $E$ supersingulární křivka a invertibilní ideál $\mathfrak{a} \subset \mathcal{O}$.  Definujme pak $\mathfrak{a}$-\textit{torsor} jako podgrupu $E$:
\begin{equation*}
E[\mathfrak{a}] := \bigcap_{\alpha \in \mathfrak{a}} \ker \alpha.
\end{equation*}
\end{definice}

Torsor příslušící hlavnímu ideálu $(m)$ s $m \in \mathbb{Z}$ je roven pouze $\ker m = E[m]$. Tato definice je proto opravdu pouze přirozeným zobecněním torzních podgrup. 

\begin{definice}
Buď $\mathfrak{a} \subset \mathcal{O}$ invertibilní ideál. Pak isogenii $\phi_{\mathfrak{a}}$ definujeme jako separabilní isogenii vycházející z $E$ s jádrem $E[\mathfrak{a}]$, přičemž značme $E/\mathfrak{a}$ cílovou křivku této isogenie.
\end{definice}

Křivka $E/\mathfrak{a}$ i isogenie $\phi_{\mathfrak{a}}$ jsou definované nad $\mathbb{F}_p$ a křivka je až na isomorfismus jednoznačně určena, opodstatňující notaci $E/\mathfrak{a}$. Jistě pro každou separabilní isogenii s~jádrem $G$ můžeme vybrat ideál $\mathfrak{a}$, pro který platí $E/\mathfrak{a} \cong E/G$, postačí vzít $G$ generovanou prvky, průnik jejichž jader je $\mathfrak{a}$. 

Pojďme tyto isogenie vycházející z ideálů zkoumat. Započněme u endomorfismů.
\begin{lemma}\label{endo}
Buď $\alpha \in \End(E)$ endomorfismus. Pak $E/(\alpha) \cong E$.
\end{lemma}
\noindent \textit{Důkaz.} Separabilní isogenie $\phi_{(\alpha)} : E \longrightarrow E/(\alpha)$ určuje křivku $E/(\alpha)$ jednoznačně až na isomorfimus. Tato isogenie má jádro $E[(\alpha)] = \cap_{a \in (\alpha)} \ker a = \ker \alpha$, které sdílí s~endomorfismem $\alpha$. Věta \ref{isomor} říká, že $E/(\alpha) \cong E/\ker \alpha \cong E$. \hfill $\square$\\


Dále se podívejme, jak se chovají isogenie generované prvoideály. 

\begin{veta}\label{normm}
Buď $\mathfrak{a} \subset \mathcal{O}$ invertibilní ideál, který neobsahuje $2$, a $\ell \neq p$ prvočíslo. Pak isogenie $\phi_{\mathfrak{a}} : E \longrightarrow E/\mathfrak{a}$ má stupeň $\ell$, právě pokud $\mathfrak{a}$ je prvoideál, jehož norma je $\ell$.
\end{veta}
\noindent \textit{Důkaz.} Nejprve ať $\mathfrak{a}$ generuje isogenii stupně $\ell$. Jistě $(\ell) \subseteq \mathfrak{a}$, tedy protože $(\ell)$ a $\mathfrak{a}$ neobsahují dvojku, jednoznačně se oba rozkládají na prvoideály, což znamená, že $\mathfrak{a} \mid (\ell)$ a tedy $N(\mathfrak{a}) \mid N(\ell) = \ell ^2$. Rozeberme hodnoty $N(\mathfrak{a})$. Případ, kdy norma $\mathfrak{a}$ je $1$, nastává jenom pokud $\mathfrak{a} = (1)$, což je spor, protože jádro $\phi_{\mathfrak{a}}$ je netriviální. Dále pokud $ \ell^2 = N(\mathfrak{a})  \mid N(\ell) = \ell ^2$, tak musí nastat i rovnost $\mathfrak{a} = (\ell)$. Pokud $\ell \neq 2$, tak se charakteristická rovnice Frobeniova endomorfismu $x^2 + p$ rozkládá na dvě různá čísla modulo $\ell$. Jinak řečeno, $\pi \vert_{\ell}$ má dvě vlastní čísla, buď $\lambda$ takové, že $\pi P = \lambda P$ pro každý $P \in E[\mathfrak{a}]$. Platí pak rovnost $(\pi - \lambda) P = O$, tedy $\pi - \lambda \in \mathfrak{a} = (\ell)$, neboli $\pi - \lambda$ působí jako $\ell \xi$ na celou $E[\ell]$. To je ale spor, protože pak $\pi P = \lambda P$ pro každé $P \in E[\ell]$, ale $\pi$ má dvě různá vlastní čísla. Musí proto nastat $N(\mathfrak{a}) = \ell$, což jistě znamená, že $\mathfrak{a}$ je prvoideálem.

Naopak ať $\mathfrak{a}$ je prvoideál normy $\ell$. Ukážeme, že $E[\mathfrak{a}]$ čítá $\ell$ prvků. Platí $\mathfrak{a} \subseteq (\ell)$, tedy $E[\mathfrak{a}] \subseteq E[\ell]$, neboli počet prvků $E[\ell]$ musí dělit počet prvků $E[\ell] = \ell^2$. Pokud by bylo $\vert E[\mathfrak{a}] \vert = \ell^2$, tak musí nastat $E[\mathfrak{a}] = E[\ell]$ a tedy $\ell \mid \alpha$ pro každé $\alpha \in \mathfrak{a}$, tj. $\mathfrak{a} = (\ell)$, což má normu $\ell^2$ a ne $\ell$. Nyní uvažme případ $E[\mathfrak{a}] = \lbrace O \rbrace$. Díky analogu věty \ref{dav} pro pořádky existují $\alpha,\beta \in \mathcal{O}$ s $\mathfrak{a} = (\alpha,\beta)$, pak $\ker \alpha \cap \ker \beta = \lbrace O \rbrace$. Platí $\ell \mid N(\alpha),N(\beta) \in \mathfrak{a}$, tedy existují afinní body $P \in \ker \alpha \cap E[\ell]$ a $Q \in \ker \beta \cap E[\ell]$, které nepatří do po řadě $\ker \beta$, resp. $\ker \alpha$. Umíme vyjádřit $\ell \in \mathfrak{a}$ jako lineární kombinaci generátorů $x \alpha + y \beta$, tedy $O = \ell P = (x \alpha + y \beta) P  = y \beta P$, což znamená $\ell \mid y$, protože $P \not\in \ker \beta$, a obdobně $\ell \mid x$. Pak ale $1 = x\alpha /\ell + y \beta/\ell $ leží v $\mathfrak{a}$, je to celý okruh endomorfismů, spor. Platí proto $\vert E[\mathfrak{a}] \vert = \ell$ a~isogenie $\phi_\mathfrak{a}$ má stupeň $\ell$ též. \hfill $\square$\\

Navíc prvoideály generující isogenii stupně $\ell$ můžeme dokonce přesně charakterizovat.

\begin{veta}
Ať $\mathfrak{p} \subseteq \End(E)$ je prvoideál a $\phi_{\mathfrak{p}} : E \longrightarrow E/\mathfrak{p}$ separabilní isogenie stupně $\ell \neq p$. Pak $\mathfrak{p} = (\ell, \pi - \lambda)$, kde $\lambda$ je vlastní číslo $\pi \vert_{\ell}$ .
\end{veta}
\noindent \textit{Důkaz.} Uvažme $P \in E[\mathfrak{p}]$ afinní. Pak platí $\pi P = \lambda P$ pro nějaké celé $\lambda$, neboli $(\pi - \lambda) P = O$. Protože je $\langle P \rangle = \ker \phi_{\mathfrak{p}} \subseteq E[\ell]$, platí i $\ell P = O$, neboli $E[(\ell,\pi-\lambda)] \subseteq E[\mathfrak{p}]$. Oba prvoideály mají normu $\ell$, musí proto nastat v inkluzi rovnost.  \hfill $\square$\\ 

\begin{dusledek}\label{duslodok}
Buď $\ell \neq p$ liché prvočíslo takové, že charakteristická rovnice $\pi$ má pod modulem $\ell$ dva různé kořeny $\lambda$ a $\mu$. Pak se ideál $(\ell)$ rozkládá jako $(\ell) = (\ell,\pi - \lambda)(\ell, \pi - \mu)$.
\end{dusledek}
\noindent \textit{Důkaz.} Tato prvočísla jsou právě taková, že $x^2 - tx + p \equiv 0 \pmod{\ell}$ má dvě různá řešení, tedy $\genfrac{(}{)}{}{}{t^2-4p}{\ell} = 1$ a $(\ell)$ s normou $\ell^2$ se tedy rozkládá na dva různé prvoideály stupně $\ell$. Podle předchozí věty tato faktorizace musí vypadat právě jako $(\ell) = (\ell,\pi-\lambda)(\ell,\pi-\mu)$. \hfill $\square$\\

Jednoznačnost rozkladu invertibilních ideálů pořádku příslušícího okruhu endomorfismů nám bude vhod vzápětí. Pak totiž můžeme isogenii generovanou ideálem na isogenie prvočíselných stupňů, které jsou generované prvoideály.

\begin{veta}\label{multivitamin}
Buďte $\mathfrak{a},\mathfrak{b} \subset \mathcal{O}$ invertibilní ideály neobsahující $2$. Pak platí $\ker \phi_\mathfrak{a} \phi_\mathfrak{b} = \ker \phi_\mathfrak{ab}$.
\end{veta}
\noindent \textit{Důkaz.} Uvažme $P$ libovolný bod v jádře $\phi_{\mathfrak{a}} \phi_{\mathfrak{b}}$. Ekvivalentně tento bod splňuje $\phi_\mathfrak{b} (P) \in \ker \phi_{\mathfrak{a}} = E[\mathfrak{a}]$. Každý endomorfismus $\alpha \in \mathfrak{a}$ pak splňuje $\alpha \phi_{\mathfrak{b}} (P) = O$. Předpokládáme, že $\mathfrak{b}$ neobsahuje $2$ a tak díky větě \ref{cond} je $\phi_{\mathfrak{b}}$ horizontální isogenie, vodič $\mathcal{O}$ totiž dělí $2$. Nyní užijeme větu \ref{komut}, dle které můžeme prohodit pořadí aplikace našich isogenií:
\begin{equation*}
\phi_{\mathfrak{b}} \alpha (P) =O
\end{equation*}
pro každý $\alpha \in \mathfrak{a}$, neboli $\alpha (P) \in \ker \phi_{\mathfrak{b}} = E[\mathfrak{b}]$.  Tato skutečnost je ekvivalentní s faktem, že pro každé $\beta \in \mathfrak{b}$ platí $\beta \alpha (P) = O$, tj. pro každý endomorfimus $\gamma \in \mathfrak{ba} = \mathfrak{ab}$ je pak $\gamma (P) = 0$, protože pro libovolné dva endomorfismy, které nulují $P$, tuto vlastnost sdílí i~jejich součet. Poslední skutečnost je ekvivalentní s $P \in E[\mathfrak{ab}]$ a~tedy každý bod $P$ leží v~$\mathfrak{ab}$-torsoru, právě pokud leží v jádře $\phi_{\mathfrak{a}} \phi_{\mathfrak{b}}$. \hfill $\square$\\

Protože isogenie generované ideály jsou separabilní, nutně se isogenie $\phi_\mathfrak{a} \phi_\mathfrak{b}$ a $\phi_{\mathfrak{ab}}$ liší až na isomorfismus. Můžeme pak určit normu libovolné isogenie definované ideálem.
\begin{dusledek}
Buď $\mathfrak{a} \subset \mathcal{O}$ invertibilní ideál. Pak $\deg \phi_{\mathfrak{a}} = N(\mathfrak{a})$.
\end{dusledek}
\noindent \textit{Důkaz.} Pokud se $\mathfrak{a}$ rozkládá na prvoideály $\mathfrak{p}_{1} ^{a_1} \mathfrak{p}_2 ^{a_2} \cdots \mathfrak{p}_{k} ^{a_k}$, podle předchozích dvou tvrzení platí: 
\begin{equation*}
\deg \phi_{\mathfrak{a}} = (\deg \phi_{\mathfrak{p}_1} )^{a_1} \cdots (\deg \phi_{\mathfrak{p}_k} )^{a_k} = N(\mathfrak{p}_1)^{a_1} \cdots N(\mathfrak{p}_k)^{a_k} = N(\mathfrak{p}_1 ^{a_1} \cdots \mathfrak{p}_k ^{a_k}) = N(\mathfrak{a}),
\end{equation*}
díky multiplikativitě normy. \hfill $\square$\\

Věta \ref{multivitamin} má krom výše uvedeného důsledku užití při studiu působení grupy tříd ideálů na třídy isomorfismů. Dále totiž ukážeme, že na třídách isomorfismů definuje akci. Okruh endomorfismů je komutativní a proto násobení jeho ideálů je též, platí tedy:
\begin{equation*}
\ker \phi_{\mathfrak{a}} \phi_{\mathfrak{b}} = \ker \phi_{\mathfrak{ab}} =\ker \phi_{\mathfrak{ba}} = \ker \phi_{\mathfrak{b}} \phi_{\mathfrak{a}}.
\end{equation*}


\begin{veta}\label{idealiso}
Buď $\mathcal{O}$ pořádek příslušící okruhu endomorfismů $\End(E)$. Pak každá třída $[\mathfrak{a}] \in Cl(\mathcal{O})$ definuje až na isomorfismus unikátní separabilní isogenii $\phi_{[\mathfrak{a}]} : E \longrightarrow E/[\mathfrak{a}]$.
\end{veta}
\noindent \textit{Důkaz.} Uvažme $\mathfrak{a},\mathfrak{b} \in [\mathfrak{a}]$ dva ideály příslušící do stejné třídy ideálů. Existují poté endomorfismy $\alpha,\beta$ takové, že $\mathfrak{a}(\alpha) =  \mathfrak{b}(\beta)$. Lemma \ref{endo} a věta \ref{isomor} pak dávají:
\begin{equation*}
E/\mathfrak{a} \cong E/\mathfrak{a}(\alpha)\cong E/\mathfrak{b}(\beta) \cong E/\mathfrak{b}.
\end{equation*}
Pokud naopak dva ideály $\mathfrak{a},\mathfrak{b}$ jsou takové, že $E/\mathfrak{a} \cong E/\mathfrak{b}$, pro nějaká $\alpha,\beta \in \End(E)$, která reprezentují isomorfismy, platí:
\begin{equation*}
E[\mathfrak{a}(\alpha)] = \ker \phi_{\mathfrak{a}(\alpha)} =  \ker \phi_{\mathfrak{a}} \phi_{(\alpha)} = \ker  \phi_{\mathfrak{b}} \phi_{(\beta)} = \ker \phi_{\mathfrak{b} (\beta)} = E[\mathfrak{b}(\beta)],
\end{equation*}
neboli $\mathfrak{a}(\alpha)$, $\mathfrak{b}(\beta)$ leží ve stejné třídě $Cl(\mathcal{O})$ a $\mathfrak{a}, \mathfrak{b}$ proto též.
\hfill $\square$\\

Vidíme, že třídy ideálů jednoznačně určují separabilní isogenie, tedy můžeme každé třídě isogenií isomorfních nad $\mathbb{F}_p$, jenž patří křivkám se shodným okruhem endomorfismů, přiřadit akci grupy tříd ideálů. Již nemůžeme uvažovat $j$-invarianty, protože křivka a její kvadratický twist neleží ve stejných tříd isomorfismů nad $\mathbb{F}_p$.

\begin{definice}
Buď $\mathcal{O}$ pořádek v $\mathbb{Q}(\pi)$. Označme pak $\mathrm{Ell}_{\mathcal{O}}$ množinu všech tříd supersingulárních eliptických křivek isomorfních nad $\mathbb{F}_p$ sdílích okruh endomorfismů $\End(E) \cong \mathcal{O}$.  
\end{definice}

\begin{veta}
Buď $\mathcal{O}$ pořádek v $\mathbb{Q}(\pi)$ takový, že množina $\mathrm{Ell}_{\mathcal{O}}$ je neprázdná. Pak zobrazení $Cl(\mathcal{O}) \times \mathrm{Ell}_{\mathcal{O}} \longrightarrow \mathrm{Ell}_{\mathcal{O}}$ dané $([\mathfrak{a}],E) \longmapsto E/\mathfrak{a}$, kde $\mathfrak{a} \subseteq \mathcal{O}$ je reprezentant třídy $[\mathfrak{a}]$, je volná akce $Cl(\mathcal{O})$ na $\mathrm{Ell}_{\mathcal{O}}$.
\end{veta}

\noindent \textit{Důkaz.} Zjevně platí $E/(1) \cong E$ a navíc větě \ref{multivitamin} platí $(E/\mathfrak{a})/\mathfrak{b} \cong E/\mathfrak{ab}$, tj. zobrazení $([\mathfrak{a}],E) \longmapsto E/\mathfrak{a}$ je akce. Navíc pokud platí $E/\mathfrak{a} \cong E$, tak podle věty \ref{idealiso} musí být $[\mathfrak{a}] = [1]$ a $\mathfrak{a}$ je hlavní ideál, tedy tato akce je volná. \hfill $\square$\\ 

Následující vlastnost, tranzitivita, je bohužel daleko nad rozsahem této práce.
\begin{veta}
Buď $\mathcal{O}$ pořádek v $\mathbb{Q}(\pi)$ takový, že množina $\mathrm{Ell}_{\mathcal{O}}$ je neprázdná. Pak akce $Cl(\mathcal{O}) \times \mathrm{Ell}_{\mathcal{O}} \longrightarrow \mathrm{Ell}_{\mathcal{O}}$ daná $([\mathfrak{a}],E) \longmapsto E/\mathfrak{a}$, kde $\mathfrak{a} \subseteq \mathcal{O}$ je reprezentant, je tranzitivní.
\end{veta}
Důkaz se nachází na \cite[Thm. 4.5]{Waterhouse}.

Díky tranzitivitě můžeme přiřadit třídám $\mathrm{Ell}_\mathcal{O}$ křivek isomorfních nad $\mathbb{F}_p$ třídy $Cl(\mathcal{O})$, tedy počet křivek $\mathrm{Ell}_\mathcal{O}$ až na isomorfismus nad $\mathbb{F}_p$ je roven $h_\mathcal{O}$. Tato korespondence nám krom eliptických křivek pomáhá i studovat samotnou grupu tříd ideálů, v závislosti na zbytkové třídě $p$ modulo $12$ totiž můžeme určit, které křivky mají twist vyšší než kvadratický. Mimo případné křivky s~$j$-invarianty $0,1728$ má každá křivka pouze kvadratický twist a tedy jsme naopak schopni charakterizovat paritu třídového čísla $h_\mathcal{O}$. Konkrétně v~případě $p \equiv -1 \pmod{4}$ jsou křivky s $j$-invariantem $1728$ supersingulární, třídová čísla pořádků $\mathbb{Z}[\sqrt{-p}]$ a $\mathbb{Z}\left[\frac{1+\sqrt{-p}}{2} \right]$ tělesa $\mathbb{Q}(\sqrt{-p})$ jsou proto lichá. 

Množina všech tříd isomorfismů supersingulárních křivek nad $\mathbb{F}_p$ s grupou tříd ideálů tvoří tzv. \textit{těžký homogenní prostor} (hard homogenous space), viz \cite{Couveignes}, což je ekvivalentní s faktem, že grupa tříd ideálů na tuto množinu definuje volnou a~tranzitivní akci. Tento případ je velmi zajímavou instancí těžkého homogenního prostoru, je totiž jediným známým případem takového prostoru, který by nebyl založen na mocnění v grupě. Neboť problém diskrétního logaritmu je efektivně řešitelný Shorovým algoritmem, tato akce je jedinou známou post-kvantovou instancí těžkého homogenního prostoru.

Se znalostí, že akce grupy tříd ideálů na supersingulární eliptické křivky s daným okruhem endomorfismů je volná a tranzitivní lze ukázat následující charakterizaci okruhu endomorfismů:

\begin{veta}\label{YO}
Buď $p \equiv 3 \pmod{8}$ prvočíslo vyšší než $3$ a $E/\mathbb{F}_p$ supersingulární křivka. Pak $\End(E) = \mathbb{Z}[\pi]$, právě pokud existuje $A \in \mathbb{F}_p$ takové, že $E$ je nad $\mathbb{F}_p$ isomorfní s~křivkou $y^2 = x^3 + Ax^2 + x$. Navíc existuji-li takové $A$, pak je unikátní.
\end{veta}
Důkaz je k nalezení na \cite[Prop. 8.]{CSIDH}. 
Takové vyjádření křivky patří do třídy křivek tvaru \textit{Montgomeryho}, které jsou tvaru $B y^2 = x^3 + Ax^2 + x$. Výhodou této reprezentace je, že třídy isomorfismů jsou závislé jenom a pouze na $A$, viz \cite[Lemma 29.]{Karaskova}. Tyto křivky se často užívají, protože mají poměrně jednoduché tvary dvoj a trojnásobku bodů. To je též důvod, proč se tyto křivky používají v protokolu SIKE.

Pro úplnost dodejme, že isomorfismy Montgomeryho křivek jsme si explicitně nedefinovali. Isomorfismus takových křivek opět bereme jako invertibilní zobrazení mezi nimi, tj. dané lineární záměnou koeficientů, na uvedených odkazech jsou k nalezení detaily.

\section{CSIDH}

Okruh endomorfismů definovaných nad $\mathbb{F}_p$ pro supersingulární eliptickou křivku má strukturu pořádku v imaginárním kvadratickém tělese a jak jsme zmínili, podobnou strukturu tvoří plný okruh endomorfismů příslušící křivce obyčejné. Toto pozorování nás nabádá vzkřísit nápady Couveigna, Rostovstseva a Stolbunova \cite{Couveignes}, \cite{Stolbunov} týkající se obyčejných křivkách a adaptovat je do supersingulárního prostředí. Původní protokoly byly založeny na následujícím komutativním diagramu, kde $\mathfrak{a},\mathfrak{b}$ jsou ideály pořádku příslušící plnému okruhu endomorfismů obyčejné křivky:

\begin{figure}[h]
\begin{center} 
\begin{tikzcd}
 & E \arrow[dr, red] \arrow[dl, blue] &\\
E/[\mathfrak{b}] \arrow[dr, blue] & & E/[\mathfrak{a}] \arrow[dl, red]\\
&E/[\mathfrak{ab}] &
\end{tikzcd}
\end{center}
\end{figure}

Supersingulární křivky a jejich plný okruh endomorfismů jsou podstatně odlišné od těch obyčejných, nekomutativita okruhu endomorfismů neumožňuje přímou adaptaci. Z tohoto důvodu jsou v SIDHu přenášeny obrazy generátorů torzí, ty ale vedou na polynomiální aktivní útoky. Supersingulární křivky definované nad $\mathbb{F}_p$ mají komutativní okruh endomorfismů, který s plným okruhem endomorfismů obyčejných křivek sdílí strukturu imaginárního kvadratického tělesa, a tedy poskytuje adaptaci velmi jednoduše. 

Rozdíly se SIDHem začínají hned při fázi volení parametrů. Volíme totiž prvočíslo $p = 4 \ell_1 \cdots \ell_n - 1$, kde $\ell_i$ jsou různá malá lichá prvočísla, a $E_0 : y^2 = x^3 + x$ supersingulární křivku. Pak platí $p \equiv 3 \pmod{8}$. Navíc věta \ref{YO} tvrdí, že okruh endomorfismů takové křivky je $\mathbb{Z}[\pi] \cong \mathbb{Z}[\sqrt{-p}]$. Frobeniův endomorfismus na supersingulární křivce má charakteristickou rovnici $x^2 + p = 0$, přičemž modulo $\ell_i$ se levá strana rozkláda jako $(x-1)(x+1)$. Důsledek \ref{duslodok} pak tvrdí, že ideál $(\ell_i)$ se rozkládá jako $(\ell_i,\pi - 1)(\ell_i, \pi + 1)$. Akce takových prvoideálů lze spočíst jednoduše \cite{CSIDH}.

\begin{poznamka}
V tomto kroku mimo jiné spočívá výhoda CSIDHu oproti původnímu protokolu na obyčejných křivkách. Je totiž obtížným problémem najít obyčejnou křivku s~počtem prvků dělitelným mnoha malými prvočísly \cite{DeFeo5} a obecný výpočet akce grupy tříd ideálů je výpočetně náročný. V supersingulárním případě stačí vhodná volba prvočísla.
\end{poznamka}
Pojďme si načrtnout myšlenku CSIDHu. Jak jsme zmiňovali ve 2. kapitole, chceme postkvantovou výměnu založit na struktuře, kterou snadno spočteme, ale obtížně invertujeme. V takovém případě se nabízí počítat akci třídy ideálů $[\mathfrak{l}_1 ^{e_1} \cdots \mathfrak{l}_n ^{e_n}]$, kde $\mathfrak{l}_i = (\ell_i,\pi - 1)$. Tyto ideály mají lichou normu a tedy isogenie $\phi_{\mathfrak{l}_i}$ jsou díky větě \ref{cond} horizontální. Akce této třídy na Montgomeryho křivku $E_0 : y^2 = x^3 + x$ definuje novou křivku $E_{\mathfrak{a}} =E_0 /\mathfrak{a} : y^2 = x^3 + Ax^2 + x$ v Montgomeryho tvaru se shodným okruhem endomorfisimů. Protože akce grupy tříd ideálů na tyto Montgomeryho koeficienty (které přísluší třídám isomorfismů) je tranzitivní a volná, pro dvě třídy ideálů $[\mathfrak{a}], [\mathfrak{b}]$ máme diagram shodný jako u původního protokolu na obyčejných křivkách.

Po spočtení příslušných akcí oba účastníci získají isomorfní Montgomeryho křivky, tj. tyto křivky sdílí koeficient u $x^2$, který může posloužit jako sdílené tajemství. Dokonce získají křivku stejnou. 

\begin{figure}[h]
\begin{center} 
\makebox[1cm]{\rule{17.3cm}{0.4pt}}\\
\hspace{-1.35cm} \textbf{Veřejné parametry:} Prvočíslo $p = 4 \cdot \ell_1 \cdots \ell_n - 1$, kde $\ell_i$ jsou různá malá lichá prvočísla. Supersingulární eliptická křivka $E_0/\mathbb{F}_p : y^2 = x^3+x$ s okruhem endomorfismů $\mathbb{Z}[\pi]$. Prvoideály $\mathfrak{l}_i = (\ell_i, \pi - 1)$. Číslo $m \in \mathbb{N}$.\\

\vspace{-0.25cm}
\makebox[\linewidth]{\rule{17.3cm}{0.4pt}}\\
\vspace{0.2cm}
\hspace*{-1cm}\begin{tabular}{l l c l}
 \cline{2-2} \cline{4-4} 
& Alfréd & & Blažena \\ 
\cline{2-2} \cline{4-4} 
& \textbf{Vstup:} vektor $(a_1,a_2,\dots,a_n)$, kde $\vert a_i \vert < m$  & &  \textbf{Vstup:} vektor $(b_1,b_2,\dots,b_n)$, kde $\vert b_i \vert < m$\\
&spočte akci $[\mathfrak{l}_1 ^{a_1} \cdots \mathfrak{l}_n ^{a_n}]$ na $E_0$ & &spočte akci $[\mathfrak{l}_1 ^{b_1} \cdots \mathfrak{l}_n ^{b_n}]$ na $E_0$\\
&získá křivku $E_A : y^2 = x^3 + Ax + x$& &získá křivku $E_B: y^2 = x^3 + Bx + x$\\
 & & $\xlongrightarrow{A}$  &  \\
&  & $\xlongleftarrow{B} $ &  \\
& spočte akci $[\mathfrak{l}_1 ^{a_1} \cdots \mathfrak{l}_n ^{a_n}]$ na $E_B$ & & spočte akci $[\mathfrak{l}_1 ^{b_1} \cdots \mathfrak{l}_n ^{b_n}]$ na $E_A$\\
& získá křivku $E_{AB} = y^2 + Sx^2 + x$ &  & získá křivku $E_{AB} = y^2 + Sx^2 + x$ \\
& \textbf{Výstup:} $S$ & & \textbf{Výstup:} $S$
\end{tabular}
\caption*{Algoritmus 4: Protokol CSIDH}
\end{center}
\end{figure}

Těžkým problémem v CSIDHu je invertování akce grupy tříd ideálů. Chtěli bychom proto tento úkon co nejvíce znesnadnit a supersingulární křivky nad $\mathbb{F}_p$ jsou pro tento účel perfektní. Plná algebra endomorfismů obyčejné křivky nad $\overline{\mathbb{F}}_{p}$ a supersingulární nad $\mathbb{F}_p$ je totiž isomorfní tělesu $ K = \mathbb{Q}(\sqrt{t^2-4p})$, kde $t$ je stopa Frobenia. Velikost grupy tříd ideálů se asymptoticky blíží odmocnině $d$ diskriminantu $K$ \cite{Siegel}, který je roven buď $\vert t^2-4p \vert$ či $4\vert t^2 - 4p \vert$. Pro $p  \equiv -1 \pmod{4}$ je tato hodnota nejvyšší při $t = 0$, kdy $d \approx \sqrt{p}$.

Aby nebyl přístupný útok hrubou silou, musí být $m$ dostatečně velké. V \cite[Sec. 7.1]{CSIDH} je odvozeno omezení na $m$, které je dostatečně velkorysé na to, aby praktičnost protokolu nebyla ovlivněna. Dále útok prohledáváním v grafu isogenií (kde postupujeme podobně jako v SIDHu) a jeho varianta Meet In The Middle mají očekávaný čas nejvýše $O(\sqrt[4]{p})$ viz \cite{Delfs}. Kvantový počítač prezentuje útoky skrz vyhledávání užitím \textit{Groverova algoritmu} v~čase $O(\sqrt[6]{p})$ popsané v \cite{DeFeo3}, volba dostatečně velkého prvočísla, například $p > 2^{400}$, těmto útokům však obstojí.

Samotný protokol takový, jak ho prezentujeme výše může opět být využit jednou zlomyslnou stranou k napadení svého protivníka. Podobně jako u zmíněných polynomiálních útoků na SIDH, Eva pod zástěrou Blaženy by mohla do světa vysílat nevyhovující křivky. Ověření, zda křivka nad celým uzávěrem je supersingulární není obecně příliš rychlé, optimalizovaný Schoofův algoritmus běží v očekávaném čase $O(\log ^ 5 p )$ \cite{Sutherland3}, užití křivek nad $\mathbb{F}_p$ tento problém podstatně zjednodušuje. Díky volbě prvočísla $p$ známé přesně řád $E(\mathbb{F}_p)$ a~za pomocí Hasseho věty lze efektivně supersingularitu ověrit, viz \cite[Alg. 1. a Sec. 8]{CSIDH} pro více detailů.

Důvodů, proč se upustilo od originálního protokolu navrženého Couveignem/Rostovtsevem a Stolbunovem, je několik a jedním z nejdůležitějších je, že Childs, Jao a Soukharev na protokol zveřejnili subexponenciální útok \cite{Childs}, který byl založen na faktu, že okruh endomorfismů a jeho grupa tříd ideálů jsou komutativní. Tento útok je též důvodem, proč SIDH užívá křivky supersingulární. Jelikož okruh endomorfismů supersingulárních křivek nad $\mathbb{F}_p$ tvoří shodnou strukturu jako plný okruh endomorfismů křivky obyčejné, subexponenciální útok je stále hrozbou. I když subexponenciální útok nezastavil užívání například schémat založených na rozkládání celých čísel, toto omezení musíme brát v~potaz. SIDH takto efektivně prolomitelný není, to ale ne nutně znamená superioritu nad CSIDHem.

Klíče v CSIDHu sestávají v případě obou účastníků z jediného prvku $\mathbb{F}_p$, oproti SIDHu s několika prvky nad $\mathbb{F}_{p^2}$, i v případě optimalizovaného SIKE jsou klíče podobné velikosti se CSIDHem. Aritmetika čísel nad $\mathbb{F}_p$ je též mnohem rychlejší, než ta nad $\mathbb{F}_{p^2}$, CSIDH bez optimalizací může se SIKE na tomto poli soupeřit taky \cite{DeFeo4}. 

CSIDH ale navíc poskytuje tzv. \textit{neinteraktivní výměnu}, tj. jakmile Alfréd zveřejní svůj klíč a~obdrží ten Blaženin, žádná další interakce není třeba. To kontrastujme se SIDHem (a všemi ostatními účastníky soutěže NIST), kde kvůli polynomiálnímu útoku Galbraitha et al. je nějaká další interakce třeba, ku příkladu zveřejnění Alfrédovy procházky v SITH. Tato vlastnost navádí na využití ne ve směru výměn klíčů, ale například jejich autentizace či efektivní podpisy. Zmiňme zde na konec pár nedávno (vzhledem ke psaní této práce) zveřejněných návrhů na podepisovací schémata založených na nápadech CSIDHu. První návrh v~tomto ohledu přišel ve formě protokolu SeaSign \cite{SeaSign}. Od doby publikace toho článku byly zveřejněny protokoly CSi-FiSh \cite{CSIFISH} a SQISign \cite{SQISign}, každý poskytující menší klíče a rychlejší běh, než ten předchozí. 


\chapter*{Závěr}
\addcontentsline{toc}{chapter}{Závěr}
\markboth{Závěr}{}

Na světě se nachází několik různých obtížných matematických problémů, na kterých jsou založené prospektivní kryptografické protokoly domněle rezistentní kvantovým počítačům. Z těch nejstudovanějších jsou však výměny založené na isogeniích zdaleka nejvíce teoreticky náročné, jejich plné pochopení vyžaduje expertní znalosti z velmi rozsáhlých oborů algebraické geometrie a algebraické teorie čísel. V naší práci jsme čtenáře neznalého těchto částí matematiky provedli k poměrně hlubokému chápání struktur, které isogenie provází. Krom toho poskytujeme k protokolu SIDH a jeho následníku SITH implementace, díky kterým si čtenář může vyzkoušet jejich běh z první ruky.

Konstrukce založené na isogeniích poskytují schémata jak na efektivní výměnu klíčů, tak na jejich autentizaci i podpisy, která konkurují s vedoucími post-kvantovými schématy v mnoha ohledech. Zejména ve velikosti klíčů dokonce stojí v popředí momentálně populárních protokolů. Fanfáry ale nemohou zaznít příliš brzy, užití isogenií se v~kryptografii uvažuje pouze 20 let a isogenie supersingulárních křivkek pouze dekádu. Na místě je proto naprosto jistě hlubší studium struktur isogenií a speciálně grafů supersingulárních isogenií definovaných pouze nad $\mathbb{F}_p$, které se před zveřejněním CSIDHu v roce 2018 zanedbávaly. 

Jako přirozené pokračování práce lze považovat detailní porovnání průběhu, velikosti klíčů a obecně bezpečnost protokolu SITH s těmi SIDHu. 




%\begin{algorithm}
%\caption{A}
%\begin{algorithmic}
%\REQUIRE $n \geq 0 \vee x \neq 0$
%\ENSURE $y = x^n$
%\STATE $y \leftarrow 1$
%\IF{$n < 0$}
%\STATE $X \leftarrow 1 / x$
%\STATE $N \leftarrow -n$
%\ELSE
%\STATE $X \leftarrow x$
%\STATE $N \leftarrow n$
%\ENDIF
%\WHILE{$N \neq 0$}
%\IF{$N$ is even}
%\STATE $X \leftarrow X \times X$
%\STATE $N \leftarrow N / 2$
%\ELSE[$N$ is odd]
%\STATE $y \leftarrow y \times X$
%\STATE $N \leftarrow N - 1$
%\ENDIF
%\ENDWHILE
%\end{algorithmic}
%\end{algorithm}



\begin{thebibliography}{97}

\bibitem{SIKE}
\textsc{Azarderakhsh}, Reza, Matthew \textsc{Campagna}, Craig \textsc{Costello}, Luca \textsc{De Feo}, Basil \textsc{Hess}, Amir \textsc{Jalali}, Brian \textsc{Koziel}, Brian \textsc{LaMacchia}, Patrick \textsc{Longa}, Michael \textsc{Naehrig}, Joost \textsc{Renes}, Vladimir \textsc{Soukharev} a David \textsc{Urbanik}: \textit{SIKE: Supersingular Isogeny Key Encapsulation}. 2017.

\bibitem{CSIFISH}
\textsc{Beullens}, Ward, Thorsten \textsc{Kleinjung} a Frederik \textsc{Vercauteren}: \textit{CSI-FiSh: Efficient Isogeny based Signatures through Class Group Computations}. 2019. Dostupné z: \url{https://eprint.iacr.org/2019/498}.

\bibitem{Dark}
\textsc{Bottinelli}, Paul, Victoria \textsc{de Quehen}, Christopher \textsc{Leonardi}, Anton \textsc{Mosunov}, Filip \textsc{Pawlega} a Milap \textsc{Sheth}: \textit{The Dark SIDH of Isogenies}. ISARA Corporation, Waterloo, Canada. 2019. Dostupné z: \url{https://eprint.iacr.org/2019/1333}.


\bibitem{Bisson}
\textsc{Bisson}, Gaetan a Andrew V. \textsc{Sutherland}: \textit{Computing the Endomorphism Ring of an Ordinary Elliptic Curve Over a Finite Field}. 2009. Dostupné z: \url{https://arxiv.org/abs/0902.4670}.

\bibitem{CSIDH}
\textsc{Castirik}, Wouter, Tanja \textsc{Lange}, Chloe \textsc{Martindale}, Lorenz \textsc{Panny} a Joost \textsc{Renes}: \textit{CSIDH: An Efficient Post-Quantum Commutative Group Action.} 2018.

\bibitem{Prase}
\textsc{Čermák}, Filip a Matěj \textsc{Doležálek}: \textit{Teorie nejen čísel}. Seriál korespondenčního matematického semináře.

\bibitem{eSIDH}
\textsc{Cervantes-Vázquez}, Daniel, Eduaro \textsc{Ochoa-Jiménez} a Francisco \textsc{Rodríguez-Henríquez}: \textit{eSIDH: the revenge of the SIDH}. 2020.

\bibitem{Chen}
\textsc{Chen}, Evan: \textit{An Infinitely Large Napkin}. Dostupné z: \url{https://venhance.github.io/napkin/Napkin.pdf}.

\bibitem{Childs}
\textsc{Childs}, Andrew, David \textsc{Jao} a Vladimir \textsc{Soukharev}: \textit{Constructing elliptic curve isogenies in quantum subexponential time}. Journal of Mathematical Cryptology,8(1), 2014. Dostupné z: \url{https://arxiv.org/abs/1012.4019}

\bibitem{Chuang}
\textsc{Chuang}, Isaac L. a Michael A. \textsc{Nielsen}: \textit{Quantum Computation and Quantum Information}. Cambridge University Press, Cambridge, 2000. 

\bibitem{Conrad1}
\textsc{Conrad}, Keith: \textit{Trace and Norm}. University of Connecticut, Connecticut. Dostupné z: \url{https://kconrad.math.uconn.edu/blurbs/galoistheory/tracenorm.pdf}.

\bibitem{Conrad2}
\textsc{Conrad}, Keith: \textit{Ideal Factorization}. University of Connecticut, Connecticut. Dostupné z: \url{https://kconrad.math.uconn.edu/blurbs/gradnumthy/idealfactor.pdf}.

\bibitem{Conrad3}
\textsc{Conrad}, Keith: \textit{The Conductor Ideal}. University of Connecticut, Connecticut. Dostupné z: \url{https://kconrad.math.uconn.edu/blurbs/gradnumthy/idealfactor.pdf}.

\bibitem{BSIDH} 
\textsc{Costello}, Craig: \textit{B-SIDH: supersingular isogeny Diffie-Hellman using twisted torsion}. Microsoft Research, USA, 2019. Dostupné z: \url{https://eprint.iacr.org/2019/1145}.

\bibitem{Costello}
\textsc{Costello}, Craig: \textit{Supersingular isogeny key exchange for beginners}. Microsoft Research, USA, 2019. Dostupné z: \url{https://eprint.iacr.org/2019/1321}.


\bibitem{Couveignes}
\textsc{Couveignes}, Jean-Marc: \textit{Hard Homogenous Spaces}. 2006. Dostupné z: \url{https://eprint.iacr.org/2006/291.pdf}.

\bibitem{Cox}
\textsc{Cox}, David: \textit{Primes of the form $x^2+n y^2$ : Fermat, Class Field Theory and Complex Multiplication}. New York, 1989.

\bibitem{DeFeo}
\textsc{De Feo}, Luca: \textit{Fast Algorithms for Towers of Finite Fields and Isogenies}. EcolePolytechnique X, 2010.

\bibitem{DeFeo3}
\textsc{De Feo}, Luca, David \textsc{Jao} a Jérôme \textsc{Plût}: \textit{Towards quantum-resistant cryptosystems from supersingular elliptic curve isogenies}. Math. Cryptol. 8(3): 209-247, 2014. Dostupné z: \url{https://eprint.iacr.org/2011/506.pdf}.

\bibitem{DeFeo2}
\textsc{De Feo}, Luca: \textit{Mathematics of Isogeny Based Cryptography}. Université de Versailles \& Inria Saclay, 2017. Dostupné z: \url{https://arxiv.org/abs/1711.04062}.

\bibitem{DeFeo4}
\textsc{De Feo}, Luca: \textit{Isogeny based Cryptography: what’s under the hood?} École des Mines de Saint-Étienne, Gardanne, 2018. Dostupné z: \url{http://defeo.lu/docet/talk/2018/11/15/gardanne/}.

\bibitem{DeFeo5}
\textsc{De Feo}, Luca, Jean \textsc{Kieffer} a Benjamin \textsc{Smith}: \textit{Towards practical key exchange from ordinary isogeny graphs}. 2018. Dstupné z: \url{https://eprint.iacr.org/2018/485}.

\bibitem{SeaSign}
\textsc{De Feo}, Luca a Steven \textsc{Galbraith}: \textit{SeaSign: Compact isogeny signatures from class group actions}. EUROCRYPT 2019. Dostupné z: \url{https://eprint.iacr.org/2018/824}.

\bibitem{SQISign}
\textsc{De Feo}, Luca, David \textsc{Kohel}, Antonin \textsc{Leroux}, Christopher \textsc{Petit} a Benjamin \textsc{Wesolowski}: \textit{SQISign: compact post-quantum signatures from quaternions and isogenies}. 2020. Dostupné z: \url{https://eprint.iacr.org/2020/1240}.

\bibitem{Science}
\textsc{Deng}, Yu-Hao, Xing \textsc{Ding}, Lin \textsc{Gan}, Peng \textsc{Hu}, Yi \textsc{Hu}, Ming-Cheng \textsc{Chen}, Xiao \textsc{Jiang}, Hao \textsc{Li}, Li \textsc{Li}, Yuxuan \textsc{Li}, Nai-Le \textsc{Liu}, Chao-Yang \textsc{Lu}, Yi-Han \textsc{Luo}, Jian-Wei \textsc{Pan}, Li-Chao \textsc{Peng}, Jian \textsc{Qin}, Hui \textsc{Wang}, Zhen \textsc{Wang}, Zhen \textsc{Wang}, Guangwen \textsc{Yang}, Lixing \textsc{You}, Han-Sen \textsc{Zhong}:\textit{Quantum computational advantage using photons.} Science Magazine. 2020. Dostupné z: \url{https://science.sciencemag.org/content/370/6523/1460.full}

\bibitem{Delfs}
\textsc{Delfs}, Christina a Steven D. \textsc{Galbraith}: \textit{Computing isogenies between super-singular elliptic curves over} $\mathbb{F}_p$. Des. Codes Cryptography, 78(2), 2016. Dostupné z: \url{https://arxiv.org/abs/1310.7789}.

\bibitem{Deuring}
\textsc{Deuring}, Max: \textit{Die typen der multiplikatorenringe elliptischer funktionenkörper}. Abhandlungen aus dem mathematischen Seminar der Universität Hamburg 14, 1941. 

\bibitem{Diffie}
\textsc{Diffie}, Whitfield a Martin \textsc{Hellman}: \textit{New Directions in Cryptography}. IEEE Transactions on Information Theory 22, 1976.

\bibitem{Petit}
\textsc{Eisentr{\"a}ger}, Sean H., Kristin \textsc{Lauter}, Travis \textsc{Morrison} a Christopher \textsc{Petit}: \textit{Supersingular Isogeny Graphs and Endomorphism Rings: Reductions and Solutions.}
Advances in Cryptology – EUROCRYPT 2018, Lecture Notes in Computer Science, pages 329–368. Springer International Publishing, 2018.

\bibitem{Feynman}
\textsc{Feynman}, Richard P.: \textit{Simulating physics with computers}. Int J Theor Phys 21, 467–488, 1982. Dostupné z: \url{https://doi.org/10.1007/BF02650179}.

\bibitem{Galbraith}
\textsc{Galbraith}, Steven D.: \textit{Constructing Isogenies Between Elliptic Curves Over Finite Fields}. LMS J. Comput. Math., 199, 118-138, 1999. Dostupné z: \url{https://www.math.auckland.ac.nz/~sgal018/iso.pdf}.

\bibitem{Galbraith2}
\textsc{Galbraith}, Steven D., Florian \textsc{Hess} a Nigel P. \textsc{Smart}: \textit{Extending the GHS Weil descent attack.} EUROCRYPT 2002,  Springer LNCS 2332 29-44, 2002.

\bibitem{Galbraith3}
\textsc{Galbraith}, Steven D. a Anton \textsc{Stolbunov}: \textit{Improved Algorithm for the Isogeny Problem for Ordinary Elliptic Curves}. Applicable Algebra in Engineering, Communication and Computing, Vol. 24, No. 2, 2013. Dostupné z: \url{https://arxiv.org/abs/1105.6331}.

\bibitem{Galbraith4}
\textsc{Galbraith}, Steven D., Christopher \textsc{Petit}, Barak \textsc{Shani} a Yan \textsc{Bo Ti}: \textit{On the security of supersingular isogeny cryptosystems}. International Conference on the Theory and Application of Cryptology and Information Security. Springer, 2016.

\bibitem{Griffiths}
\textsc{Griffiths}, Robert B.: \textit{Hilbert Space Quantum Mechanics}. 2014.

\bibitem{Grover}
\textsc{Grover}, Lov K.: \textit{A fast quantum mechanical algorithm for database search}.28th Annual ACM Symposium on the Theory of Computing, 1996. Dostupné z: \url{https://arxiv.org/abs/quant-ph/9605043}.

\bibitem{Hartshorne}
\textsc{Hartshorne}, Robin: \textit{Algebraic  Geometry}. Berkley: Springer-Verlag, 1977.

\bibitem{Ireland}
\textsc{Ireland}, Kenneth a Michael \textsc{Rosen}: \textit{A Classical Introduction to Modern Number Theory}. New York, Berlin a Heidelberg: Springer-Verlag, 1982.

\bibitem{Jao}
\textsc{Jao}, David a David \textsc{Urbanik}: \textit{Extra Secrets from Automorphisms and SIDH-based NIKE}, 2018.

\bibitem{ECDSA}
\textsc{Johnson}, Don, Alfred \textsc{Menenzes} a Scott \textsc{Vanstone}: \textit{The Elliptic Curve Digital Signature Algorithm (ECDSA)}. Certicom a Department of Combinatorics \& Optimization, University of Waterloo,  Ontario, Canada. 2001.

\bibitem{Johnson}
\textsc{Johnson}, Lee W., Ronald Dean \textsc{Riess} a Jimmy Thomas \textsc{Arnold}: \textit{Introduction to Linear Algebra}. Fifth edition. Virginia Polytechnic Institute and State University: Addison-Wesley, 2002.

\bibitem{Karamlou}
\textsc{Karamlou}, Amir H, Willieam A. \textsc{Simon}, Amara \textsc{Katabarwa}, Travis L. \textsc{Scholten}, Borja \textsc{Peropandre} a Yudong \textsc{Cao}: \textit{Analyzing the Performance of Variational Quantum Factoring on a Superconducting Quantum Processor}. Zapata Computing, Boston; Research Laboratory of Electronics, Massachusetts Institute of Technology, Cambridge a IBM Quantum, IBM T. J. Watson Research Center, New York, 2020. Dostupné z: \url{https://www.zapatacomputing.com/publications/analyzing-the-performance-of-variational-quantum-factoring-on-a-superconducting-quantum-processor/}.

\bibitem{Karaskova}
\textsc{Karásková}, Zdislava: \textit{Supersingulární isogenie a jejich využití v kryptografii}. Diplomová práce. Brno: Masarykova univerzita, 2019. Dostupné z: \url{https://is.muni.cz/th/mt87i/}.

\bibitem{Koblitz}
\textsc{Koblitz}, Neal: \textit{Elliptic curve cryptosystems}. Mathematics of Computation. 48 (177): 203–209, 1987.

\bibitem{Kohel}
\textsc{Kohel}, David R.: \textit{Endomorphism rings of elliptic curves over finite fields}. University of California, Berkley, 1996.

\bibitem{Lagarias}
\textsc{Lagarias}, Jeffrey C. a Andrew M. \textsc{Odlyzko}: \textit{Effective Versions of the Chebotarev Density Theorem}. Algebraic Number Fields,L-Functions and Galois Properties (A. Fröhlich, ed.), pp. 409–464. New York, London: Academic Press, 1977.

\bibitem{Leonardi}
\textsc{Leonardi}, Christopher: \textit{A Note on the Ending Elliptic Curve in SIDH}. 2020. Dostupné z: \url{https://eprint.iacr.org/2020/262}.

\bibitem{Marcus}
\textsc{Marcus}, Daniel A.: \textit{Number fields}. New York: Springer-Verlag, 1977.

\bibitem{Matushak}
\textsc{Matushak}, Andy a Michael A. \textsc{Nielsen}: \textit{Quantum computing for the very curious}. San Francisco, 2019. Dostupné z: \url{https://quantum.country/qcvc}.

\bibitem{MOV}
\textsc{Menezes}, Afred, Tatsuki \textsc{Okamoto} a Scott \textsc{Vanstone}: \textit{Reducing Elliptic Curve Logarithms to Logarithms in a Finite Field}. IEEE Transactions on Information Theory 39, 1993.

\bibitem{Miller}
\textsc{Miller}, Victor: \textit{Use of elliptic curves in cryptography}. Advances in Cryptology—CRYPTO ’85, Lecture Notes in Computer Science, vol 218. Springer, pp 417–426, 1986.

\bibitem{Mordell}
\textsc{Mordell}, Luis J.: \textit{On the rational solutions of the indeterminate equations of the third and fourth degrees}. Cambridge, 1922.

\bibitem{Neukirch}
\textsc{Neukirch}, J{\"u}rgen: \textit{Algebraic Number Theory}. New York: Springer-Verlag, 1999.

\bibitem{NIST}
\textsc{NIST}. Post-Quantum Cryptography. Dostupné z: \url{https://csrc.nist.gov/Projects/Post-Quantum-Cryptography/}.

\bibitem{Tomas}
\textsc{Perutka}, Tomáš: \textit{Vyjadřování prvočísel kvadratickými formami.} Středoškolská odborná činnost. Brno: Masarykova univerzita, 2017. Dostupné z: \url{https://socv2.nidv.cz/archiv39/getWork/hash/ff6e75d5-f922-11e6-848a-005056bd6e49}.

\bibitem{Perutka}
\textsc{Perutka}, Tomáš: \textit{Užití dekompoziční grupy k důkazu zákona kvadratické reciprocity.} Středoškolská odborná činnost. Brno: Masarykova univerzita, 2018. Dostupné z: \url{https://socv2.nidv.cz/archiv40/getWork/hash/1984482c-1298-11e8-90e4-005056bd6e49}.


\bibitem{Pezlar}
\textsc{Pezlar}, Zdeněk: \textit{Zajímavá využití algebraické teorie čísel}. Středoškolská odborná činnost. Brno: Masarykova univerzita, 2020. Dostupné z: \url{https://socv2.nidv.cz/archiv42/getWork/hash/921aa7aa-568d-11ea-9fea-005056bd6e49}.

\bibitem{Pizer}
\textsc{Pizer}, Arnold K.: \textit{Ramanujan graphs and Hecke operators.} Bulletin of the American Math Society, 23, 1990.

\bibitem{Proos}
\textsc{Proos}, John a Christof \textsc{Zalka}: \textit{Shor’s discrete logarithm quantum algorithm for elliptic curves}. Department of Combinatorics \& Optimization, University of Waterloo,  Ontario, Canada, 2008. Dostupné z: \url{https://arxiv.org/abs/quant-ph/0301141}.

\bibitem{Pupik}
\textsc{Pupík}, Petr: \textit{Užití grupy tříd ideálů při řešení některých diofantických rovnic}. Diplomová práce. Brno: Masarykova univerzita, 2009. Dostupné z: \url{https://is.muni.cz/th/v8xsj/}.

\bibitem{Raclavsky}
\textsc{Raclavský}, Marek: \textit{Racionální body na eliptických křivkách}. Bakalářská práce. Praha: Univerzita Karlova, 2014. Dostupné z: \url{https://is.cuni.cz/webapps/zzp/detail/143352/}.

\bibitem{RSA}
\textsc{Rivest}, Ronald L., Adi \textsc{Shamir} a Leonard M. \textsc{Adleman}: \textit{A Method for Obtaining Digital Signatures and Public-Key Cryptosystems}. 1977. Dostupné z: \url{https://people.csail.mit.edu/rivest/Rsapaper.pdf}. 

\bibitem{Rosicky}
\textsc{Rosický}, Jiří: \textit{Algebra}. Brno: Masarykova univerzita, 2002.

\bibitem{Shengyu}
\textsc{Shengyu}, Zhang: \textit{Promised and Distributed Quantum Search Computing and Combinatorics}. Proceedings of the Eleventh  Annual  International Conference on Computing  and Combinatorics, Berlin, Heidelberg, 2005.

\bibitem{Shor}
\textsc{Shor}, Peter W.: \textit{Polynomial-Time Algorithms for Prime Factorization and Discrete Logarithms on a Quantum Computer}. New York: Springer-Verlag, 1994. Dostupné z: \url{https://arxiv.org/abs/quant-ph/9508027}.

\bibitem{Schoof}
\textsc{Schoof}, René: \textit{Elliptic Curves Over Finite Fields and the Computation of Square Roots $\! \operatorname{mod} \, p$.} Journal de Théorie des Nombres de Bordeaux 7, 1985. Dostupné z: \url{https://www.ams.org/journals/mcom/1985-44-170/S0025-5718-1985-0777280-6/S0025-5718-1985-0777280-6.pdf}.

\bibitem{Schoof2}
\textsc{Schoof}, René: \textit{Counting points on elliptic curves over finite fields.} Journal de Théorie des Nombres de Bordeaux 7, 1995. Dostupné z: \url{https://www.mat.uniroma2.it/~schoof/ctg.pdf}.

\bibitem{Siegel}
\textsc{Siegel}, Carl: \textit{Über die Classenzahl quadratischer Zahlkörp}. Acta Arithmetica, 1(1), 1935.

\bibitem{Silverman}
\textsc{Silverman}, Joseph H.: \textit{The Arithmetic of Elliptic Curves}. New York: Springer-Verlag, 1992. 

\bibitem{Silverman2}
\textsc{Silverman}, Joseph H.: \textit{Advanced Topics in the Arithmetic of Elliptic Curves}. New York: Springer-Verlag, 1994. 

\bibitem{Stolbunov}
\textsc{Rostovtsev}, Alexander a Anton \textsc{Stolbnov}:\textit{ Public-key cryptosystem based on isogenies}. 2006. Dostupné z: \url{http://eprint.iacr.org/2006/145/}. 


\bibitem{Suchanek}
\textsc{Suchánek}, Vojtěch: \textit{Vulkány isogenií v kryptografii}. Diplomová práce. Brno: Masarykova univerzita, 2020. Dostupné z: \url{https://is.muni.cz/th/pxawb/}.

\bibitem{Sutherland2}
\textsc{Sutherland}, Andrew V.: \textit{Isogeny Volcanoes}. 2012. Dostupné z: \url{https://arxiv.org/abs/1208.5370}.

\bibitem{Sutherland3}
\textsc{Sutherland}, Andrew V.: \textit{Identifying supersingular elliptic curves}. 2012. Dostupné z: \url{https://arxiv.org/abs/1107.1140}

\bibitem{Sutherland}
\textsc{Sutherland}, Andrew V.: \textit{Elliptic Curves}. Massachusetts Institute of Technology, 2017. Dostupné z: \url{https://math.mit.edu/classes/18.783/2017/lectures.html}. 

\bibitem{Tani}
\textsc{Tani}, Seiichiro: \textit{Claw Finding Algorithms Using Quantum Walk}. Theoretical Computer Science, 410(50):5285-5297, 2009.

\bibitem{Tate}
\textsc{Tate}, John: \textit{Endomorphisms of Abelian Varieties over Finite Fields}. Inventiones Mathematicae, 2 (2): 134–144, Cambridge, 1966.

\bibitem{Velu}
\textsc{Vélu}, Jacques: \textit{Isogénies entre courbes elliptiques}. Comptes Rendus de l’Académie des Sci-ences de Paris, 1971. 

\bibitem{Washington}
\textsc{Washington}, Lawrence C.: \textit{Elliptic Curves: Number theory and cryptography}. Maryland, 2008. 

\bibitem{Waterhouse}
\textsc{Waterhouse}, William C.: \textit{Abelian varieties over finite fields}. Annales scientifiques de l’École Normale Supérieure, 1969.

\bibitem{Weil}
\textsc{Weil}, André: \textit{L'arithmétique sur les courbes algébriques}.  Acta Mathematica 52, 1929. 


\end{thebibliography}
\end{document}



%\begin{veta} Nechť $p,q$ jsou různá lichá prvočísla. Potom 
%$$\left( \frac{p}{q} \right) = \left( \frac{q}{p} \right) \cdot (-1)^{\frac{(p-1)(q-1)}{4}}.$$

%Dále navíc Pro libovolná celá čísla $a,b$ a liché prvočíslo $p$ platí:
%\begin{enumerate}
%\item $\bigl( \frac{a}{p} \bigr)\cdot\bigl( \frac{b}{p} \bigr)=\bigl( %\frac{ab}{p} \bigr),$
%\item $\bigl( \frac{-1}{p} \bigr) = (-1)^{\frac{p-1}{2}},$
%\item $\bigl( \frac{2}{p} \bigr) = (-1)^{\frac{p^2-1}{8}}.$ 
%\end{enumerate}
%\end{veta}

%\ Vzhledem k důležitosti těchto tvrzení uvedeme ještě ekvivalentní formu, jíž je možné některé z nich vyjádřit -- a to pomocí kongruencí:

%\begin{veta} Nechť $p,q$ jsou různá lichá prvočísla. Potom 
%$$\left( \frac{p}{q} \right) = \begin{cases}
%\left( \frac{q}{p} \right) \qquad \text{pokud} \; p\;\text{nebo}\; q\equiv 1\pmod4;\\ 
%-\left( \frac{q}{p} \right) \qquad \mbox{pokud} \; p\equiv q\equiv 3 \pmod{4}. \end{cases} $$

%Dále navíc pro libovolná celá čísla $a,b$ a liché prvočíslo $p$ platí:
%\begin{enumerate}
%\item $\bigl( \frac{a}{p} \bigr)\cdot\bigl( \frac{b}{p} \bigr)=\bigl( \frac{ab}{p} \bigr),$
%\item $\bigl( \frac{-1}{p} \bigr) =
%\begin{cases}
%1 \quad \text{pokud} \; p\equiv1\pmod4\\
%-1\quad\text{pokud}\; p\equiv3\pmod4
%\end{cases}$
%\item $\bigl( \frac{2}{p} \bigr) = 
%\begin{cases}
%1 \quad \text{pokud} \; p\equiv\pm1\pmod8\\
%-1\quad\text{pokud}\; p\equiv\pm3\pmod8.
%\end{cases}$
%\end{enumerate}
%\end{veta


% Cvičení: falešný násobení PO_L


% \Zdroje: Dumit Foote, Zakony reciprocity, Rosicky, Cox, Marcus, clanek o Mihaelescau?, Ireland Rosen?, Pupik?

%\begin{definice} Množinu $G$ spolu s binární operací $\odot$ na ní definovanou nazveme grupou, pokud splňuje tyto podmínky:
%\begin{enumerate}
%\item operace je asociativní, tzn.\ pro každé $x,y,z \in G$ platí $(x\odot y)\odot z=x\odot(y\odot z)$,
%\item existuje tzv.\ neutrální prvek, tedy nějaké $e \in G$ takové, že pro každé $x \in G$ platí $e\odot x=x=x\odot e$,
%\item ke každému prvku můžeme nalézt prvek k němu inverzní, tedy pro každé $x\in G$ existuje $y\in G$ tak, že $x\odot y=e=y\odot x$. 
%\end{enumerate}
%\ Pokud je operace navíc komutativní, hovoříme o abelovské nebo komutativní grupě.
%\end{definice}

%\begin{definice} Nechť R je množina, $+$, $\cdot$ binární operace na ní definované. Pak $(R,+,\cdot)$ je okruh, pokud:
%\begin{enumerate}
%\item $(R,+)$ je komutativní grupa,
%\item operace $\cdot$ je asociativní a existuje vzhledem k ní neutrální prvek,
%\item platí oboustranná distributivita, tedy pro libovolné $a,b,c\in R$ platí $a\cdot(b+c)=a\cdot b+a\cdot c,$ $(b+c)\cdot a=b\cdot a+ c\cdot a.$
%\end{enumerate}
%\ Pokud je i operace $\cdot$ komutativní, hovoříme o komutativním okruhu.
%\end{definice}

%\ Operaci + běžně nazýváme sčítání a neutrální prvek vůči ní značíme symbolem 0, operaci $\cdot$ nazýváme násobení a neutrální prvek vůči ní značíme jako $1$.



%\begin{definice} Komutativní okruh $R$ nazýváme obor integrity, pokud pro libovolná $a,b\in R$ platí, že pokud $a\cdot b=0$, tak $a=0$ nebo $b=0$. \end{definice}


%\begin{definice} Nechť T je množina, $+$, $\cdot$ binární operace na ní definované. Pak $(T,+,\cdot)$ je těleso, pokud:
%\begin{enumerate}
%\item $(T,+)$ je komutativní grupa,
%\item $(T\smallsetminus\{0\},\cdot)$ je komutativní grupa,
%\item platí oboustranná distributivita, tedy pro libovolné $a,b,c\in T$ platí $a\cdot(b+c)=a\cdot b+a\cdot c,$ $(b+c)\cdot a=b\cdot a+ c\cdot a.$
%\end{enumerate}
%\end{definice}

%\ Jinak řečeno, těleso je takový obor integrity, jehož každý nenulový prvek je jednotkou, neboli je \textit{invertibilní} -- tedy má vůči operaci $\cdot$ inverzní prvek.






%V případě $p=2$ se nám situace ztíží tím, že pokud $m\equiv1\pmod4$, nemůžeme aplikovat větu \ref{polynomy}, jelikož 2 dělí $|\z[\frac{1+\sqrt m}2]/\z[\sqrt m]|$. Přesto ale dokážeme následující větu:

%\begin{veta} Nechť $K=\q(\sqrt m)$. Potom: $$2\o_K=
%\begin{cases}
%(2,\sqrt m)^2 \quad \text{pokud}\; m\equiv2\pmod4, \\
%(2,1+\sqrt m)^2 \quad \text{pokud} \;m\equiv3\pmod4, \\
%(2,\frac{1-\sqrt m}2)(2,\frac{1+\sqrt m}2) \quad \text{pokud}\; m\equiv1\pmod 8, \\
%2\o_K,\; \text{tj. je prvoideál} \quad \text{pokud}\; m\equiv 5\pmod 8.
%\end{cases}$$
%\end{veta}

%\begin{proof} Zamysleme se nejprve nad diskriminantem okruhu $\o_K$. Z věty \ref{tabulka} víme, že v případě $m\equiv 2,3\pmod4$ platí $d(\o_K)=4m$ a v případě $m\equiv1\pmod4$ platí $d(\o_K)=m$.

%\ Uvažujme nejprve $m\equiv2,3\pmod4$. V tomto případě můžeme aplikovat větu \ref{polynomy}. Jelikož $d(\o_K)=4m$, tak se 2 bude vždy větvit.

%\ Pokud $m\equiv2\pmod4$, tak $2|m$ a tedy $x^2-m\equiv(x)^2\pmod2$. Tudíž $2\o_K=(2,\sqrt m)^2$ (analogicky jsme postupovali v důkazu předchozí věty).

%\ Pokud $m\equiv3\pmod4$, tak jelikož se 2 větví a zároveň nedělí $m$, platí $x^2-m\equiv x^2+1\equiv x^2+2x+1\equiv (x+1)^2\pmod2$ a tedy $2\o_K=(2,1+\sqrt m)^2$.

%\ Nyní uvažujme $m\equiv1\pmod4$, tedy $\o_K=\z[\frac{1+\sqrt m}2]$. Víme již, že nemůžeme použít větu \ref{polynomy}, musíme tedy použít jiné argumenty.

%\ Pokud $m\equiv1\pmod 8$, tak $2\in (2,\frac{1-\sqrt m}2)(2,\frac{1+\sqrt m}2)=(4,1+\sqrt m,1-\sqrt m,\frac{1- m}4)$ protože $\nsd(4,\frac{1-m}4)=2$ (a díky Bezoutově rovnosti s každými dvěma celými čísly ležícími v daném ideálu v něm leží i jejich největší společný dělitel). Tudíž $2\o_K\s(2,\frac{1-\sqrt m}2)(2,\frac{1+\sqrt m}2)$ a proto $(2,\frac{1-\sqrt m}2)(2,\frac{1+\sqrt m}2)|2\o_K$. Aby platila věta \ref{eifi}, musí už platit přímo rovnost.

%\ Zbývá případ $m\equiv 5\pmod 8$. Nechť $\P$ je prvoideál $o\_K$, $\P|2\o_K$. Ukážeme $f(\P|2)=2$. To uděláme sporem: pokud $f(\P|2)=1$, tak $\o_K/\P\cong\z/2\z$. Uvažujme polynom $x^2-x+\frac{1-m}4$. Ten má v $\o_K$ kořen $\frac{1+\sqrt m}4$; má tedy kořen i v $\o_K/\P$. Na druhou stranu tento polynom v $\z/2\z$ žádný kořen nemá, jelikož $x^2-x+\frac{1-m}4\equiv x^2-x+1\pmod2$. To je spor s tím, že jsou tělesa $\o_K/\P$ a $\z/2\z$ izomorfní a $2\o_K$ je tedy opravdu prvoideál.

%\
%\end{proof}

%\ Případ $p=2$ nás zajímá spíše pro úplnost, případ $p$ je liché bude hrát ústřední roli v důkazu kvadratické recprocity a dalších tvrzení. 





\begin{poznamka} Často nastává situace, kdy $K$ je \uv{skoro} podtěleso $L$. Uvažujme například těleso $\R$ s klasickým sčítáním a násobením a těleso $\R^2=\{(a,b)\mid (a,b)\in\R\}$ s operacemi sčítání po složkách (tj. $(a,b)+(c,d)=(a+c,b+d)$) a s násobením definovaným jako $(a,b)\cdot(c,d)=(ac-bd,ad+bc)$ pro všechna $a,b,c,d\in\R$.Sice $\R$ není podtělesem tělesa $\R^2$, ale existuje vnoření (tj. injektivní homomorfismus) $f: \R\rightarrow\R^2$ (např. definované jako $f(a)=(a,0)$), tzn. $\R$ je izomorfní s nějakým podtělesem $f(\R)=\{f(a)\mid a\in\R\}$ tělesa $\R^2$. Sice tedy přísně vzato nemůžeme hovořit o rozšíření $\R\s\R^2$, ale jelikož $\R$ a $f(\R)$ jsou izomorfní, tudíž mají s algebraického hlediska identické vlastnosti, tak někdy nebudeme zcela korektní a např. v této situaci budeme mluvit o rozšíření $\R\s\R^2$ místo o $f(\R)\s\R^2$. \label{nejsmekorektni} \end{poznamka}

\ Zohledníme-li tuto poznámku, můžeme psát $[\R^2:\R]=2$.

\section{Základní poznatky}

\ V této části stručně připomeneme pojmy z algebry, které budeme v práci nejčastěji používat -- především hlavní větu o faktorgrupách, ideál okruhu a vlastnosti okruhu polynomů jedné proměnné.

\ Uveďme tedy nejprve hlavní větu o faktorgrupách:

\begin{veta} Nechť $f:G\rightarrow K$ je homomorfismus grup, $H$ normální podgrupa grupy $G$ splňující $H\s\ker f$. Nechť $\pi:G\rightarrow G/H$ je projekce grupy $G$ na faktorgrupu $G/H$. Pak existuje, a to jediné, zobrazení $\fii:G/H\rightarrow K$ splňující $\fii\odot\pi=f$. Navíc platí: \begin{enumerate}
\item $\fii$ je homomorfismus grup,
\item $\fii$ je injekce, právě když $H=\ker f$,
\item $\fii$ je surjekce, právě když $f$ je surjekce. \end{enumerate} \end{veta}
$$
\xymatrix{
G\ar[rr]^f\ar[dr]^\pi&&K \\
&G/H\ar@{-->}[ur]_\fii&
}
$$

\ Věta má podstatné důsledky:

\begin{dusledek} Nechť $f:G\rightarrow K$ je homomorfismus grup, $f(G)=\{f(g)\mid g\in G\}$. Pak $G/\ker f\cong f(G)$. \end{dusledek}

\begin{dusledek} Homomorfismus grup $f:G\rightarrow K$ je injektivní, právě když $\ker f$ je triviální grupa. \end{dusledek}

Nyní přejděme k pojmu ideál.
 

\begin{poznamka} V dalším textu budeme používat následující značení: pro těleso $K$ symbolem $K(a_1,...,a_n)$ míníme těleso generované množinou $K\cup\{a_1,...,a_n\}$. V případě okruhů používáme obdobné značení -- nejmenší okruh obsahující nějaký okruh $R$ a množinu $\{a_1,...,a_n\}$ značíme $R[a_1,...,a_n]$. Tedy např. $\mathbb{Q}(\sqrt 2)$ nejmenší těleso obsahující racionální čísla a odmocninu ze dvou a $\z[i]$ je nejmenší okruh obsahující celá čísla a imaginární jednotku~$i$. \end{poznamka}


\begin{definice} Nechť R je okruh. Neprázdnou množinu $\I\subseteq R$ nazveme ideálem okruhu R, pokud:
\begin{enumerate}
\item pro libovolné $a,b\in \I$ platí $a+b\in \I$,
\item pro libovolné $r\in R, a\in \I$ platí $ar\in \I, ra\in \I$.
\end{enumerate}
\end{definice}

\ Každý okruh má alespoň dva ideály, a to celý okruh a triviální ideál $\{0\}$ -- říkáme jim nevlastní ideály a ostatní ideály nazýváme vlastní. 

\begin{definice} Ideál $\I$ okruhu $R$ nazýváme hlavní, pokud je ve tvaru $aR=\{ar|r\in R\}$ pro nějaké $a\in R$. Danému oboru integrity říkáme okruh hlavních ideálů, pokud je každý jeho ideál hlavní. \end{definice}

\ Typickým okruhu hlavních ideálů jsou celá čísla: jediné ideály jsou zde tvaru $n\mathbb{Z}$, kde $n$ je libovolné nezáporné celé číslo.

\begin{poznamka} Hlavní ideál $aR$ někdy značíme $(a)$. Obecně je možné definovat ideál generovaný množinou a ideál generovaný konečnou množinou $\{a_1,a_2,...,a_n\}$ značíme $(a_1,a_2,...,a_n)$.\end{poznamka}

\ Všimněme si, že z definice ideálu přímo plyne důležitý poznatek:

\begin{veta} Nechť $\I$ je ideál okruhu R. Pak $(\I,+)$ je normální podgrupa grupy (R,+). \end{veta}

\ Ideály jsou úzce spjaté s homomorfismy okruhů. Platí totiž následující věta:

\begin{veta} Nechť $f:R\rightarrow S$ je homomorfismus okruhů. Pak platí:
\begin{enumerate}
\item je-li $\J$ ideál okruhu $S$, pak $f^{-1}(\J)=\{x\in R|f(x)\in \J\}$ je ideál okruhu $R$,
\item je-li $f$ surjekce a $\I$ ideál okruhu $R$, pak $f(\I)=\{f(x)|x\in \I\}$ je ideál okruhu $S$.
\end{enumerate}
\end{veta}

\ To mimo jiné znamená, že jádro libovolného homomorfismu $f:R\rightarrow S$ je ideál okruhu $R$, jelikož $\ker f=f^{-1}(0)$.

\ Podle ideálů můžeme faktorizovat. Jelikože je pro každý ideál $\I$ je $(\I,+)$ normální podgrupa $(R,+)$, můžeme sestrojit faktorgrupu $(R/\I,+)$, kde + je nyní sčítání tříd pomocí reprezentantů. Lze ukázat, že na této faktorgrupě je možné definovat i násobení pomocí reprezentantů tak, že $R/\I$ je s těmito operacemi okruh. Takto vzniklý okruh nazýváme faktorokruh. Existuje hlavní věta o faktoroktuzích analogická hlavní větě o faktorgrupách.

\ Existují dvě významné skupiny ideálů, které má smysl definovat:

\begin{definice} Nechť R je okruh, $\I$ jeho vlastní ideál. O ideálu $\I$ říkáme, že je:
\begin{enumerate}
\item prvoideál, pokud pro libovolné $a,b\in R$ platí implikace $ab\in \I\Rightarrow a\in \I nebo b\in \I$,
\item maximální ideál, pokud neexistuje žádný ideál $\J$ okruhu $R$ splňující $\I\subsetneq \J\subsetneq R$.
\end{enumerate}
\end{definice}

\ Pokud je okruh $R$ komutativní, můžeme tyto skupiny ideálů poznat podle toho, jak vypadá jimi určený faktorokruh:

\begin{veta} Nechť $R$ je komutativní okruh, $\I$ jeho vlastní ideál. Pak je $\I$ prvoideál, právě když faktorokruh $R/\I$ je obor integrity, a $\I$ je maximální ideál, právě když je $R/\I$ těleso. \label{prvmax} \end{veta}

\ Dále stručně připomeňme vlastnosti polynomů jedné proměnné.

\begin{veta} Nechť $R$ je okruh. Pak množina $R[x]$ polynomů jedné proměnné s koeficienty z $R$ tvoří rovněž okruh (s obvykle definovaným sčítáním a násobením polynomů). Navíc je-li $R$ komutativní (resp. obor integrity), tak i $R[x]$ je komutativní (resp. obor integrity). \end{veta}

\begin{veta} Nechť $R$ je obor integrity. Pak polynom $f\in R[x]$ má v $R$ nejvýše tolik kořenů, kolik je jeho stupeň (který značíme $\deg f$). \end{veta}

\begin{veta} Nechť $R$ je obor integrity. Pak je $R[x]$ euklidovský okruh, přesněji pro každé dva polynomy $f,g\in R[x]$ existují právě jedna dvojice polynomů $k,r\in R[x]$ taková, že $f=kg+r$ a $\deg r<\deg g$. \end{veta} 

\ Na závěr definujme podílové těleso a uveďme některé jeho příklady.

\begin{definice} Nechť $R$ je obor integrity. Nejmenší těleso obsahující $R$ nazveme podílové těleso okruhu R. \end{definice}

\begin{veta} Nechť $R$ je obor integrity. Pak jeho podílové těleso $F$ můžeme psát ve tvaru $$F=\left\{\frac ab\mid a,b\in R, b\ne0\right\},$$ kde $\frac ab=\frac cd$, právě když $ad=bc$ a operace sčítání a násobení provádíme následovně: $$\frac ab+\frac cd=\frac{ad+bc}{bd},$$ $$\frac ab\cdot \frac cd= \frac{ac}{bd}.$$ \end{veta}

\ Podílové těleso okruhu $\z$ je těleso racionálních čísel. Podílové těleso okruhu $R[x]$ nazýváme \textit{těleso racionálních funkcí} a značíme ho $R(x)$. Tyto pojmy můžeme zobecnit: množina polynomů více proměnných nad oborem integrity $R$ tvoří rovněž obor integrity a jeho podílové těleso také nazýváme těleso racionálních funkcí.
