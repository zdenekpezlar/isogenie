\documentclass [12pt]{report}
\usepackage{multicol}
\usepackage[czech]{babel}
\usepackage{multicol}
\usepackage{a4wide}
\usepackage{algorithm}
\usepackage{algorithmic}
\usepackage{amsmath, amssymb, amsthm}
\usepackage[utf8]{inputenc}
\usepackage[all]{xypic}
\usepackage{graphicx}
\usepackage{fancyhdr}
\usepackage{caption}
\usepackage{verse}
\usepackage{enumerate}
\renewcommand{\labelenumi}{(\roman{enumi})} 
\usepackage{tabularx}
\usepackage{wrapfig}
\usepackage{color}
\usepackage{array}
\usepackage{textcomp}
\usepackage{siunitx}
\usepackage{epsfig}
\usepackage{hyperref}
\usepackage{url}
\usepackage{tabularx}
\usepackage[metapost, mplabels,truebbox,clip]{mfpic}


\makeatletter
\newcommand\figcaption{\def\@captype{figure}\caption}
\makeatother

\renewcommand{\sectionmark}[1]{ \markright{-- \thesection.\ #1}{}}

\fancypagestyle{plain}{%
\fancyhf{}
\lhead[L]{}
\chead{}
\rhead[R]{\nouppercase\leftmark}
\lfoot{}
\cfoot{}
\rfoot{\thepage}
\renewcommand{\headrulewidth}{0.4pt}}

\pagestyle{fancy}
\fancyhf{}
\lhead[L]{}
\chead{}
\rhead[R]{\nouppercase\leftmark}
\lfoot{}
\cfoot{}
\rfoot{\thepage}
\begin{document}
\newcommand{\ZZ}{{\mathbb{Z}}}
\newcommand{\cyc}[1]{{\langle #1 \rangle}}
%\newtheorem{veta}{Věta}[section]
%\newtheorem{definice}{Definice}[section]
%\newtheorem{dusledek}{Důsledek}[section]
%\newtheorem{lemma}{Lemma}[section]

\newtheorem{veta}{Věta}[section]
\newtheorem{definice}[veta]{Definice}
\newtheorem{dusledek}[veta]{Důsledek}
\newtheorem{lemma}[veta]{Lemma}
\newtheorem{poznamka}[veta]{Poznámka}
\newtheorem{priklad}[veta]{Příklad}
\theoremstyle{definition}



\floatname{algorithm}{Algoritmus}


\def\nsd{\operatorname{nsd}}
\def\id{\operatorname{id}}
\def\char{\operatorname{char}}
\def\ker{\operatorname{ker}}
\def\Aut{\operatorname{Aut}}
\def\Gal{\operatorname{Gal}}
\def\Fix{\operatorname{Fix}}
\def\c{\operatorname{\mathbb{C}}}
\def\R{\operatorname{\mathbb{R}}}
\def\q{\operatorname{\mathbb{Q}}}
\def\e{\operatorname{\mathcal{V}}}
\def\z{\operatorname{\mathbb{Z}}}
\def\n{\operatorname{\mathbb{N}}}
\def\s{\operatorname{\subseteq}}
\def\w{\operatorname{\zeta}}
\def\fii{\operatorname{\varphi}}
\def\o{\operatorname{\mathcal{O}}}
\def\I{\operatorname{\mathcal{I}}}
\def\J{\operatorname{\mathcal{J}}}
\def\P{\operatorname{\mathcal{P}}}
\def\pn{\operatorname{\mathfrak{P}}}
\def\pn{\operatorname{\mathfrak{p}}}
\def\res{\operatorname{res}}

\begin{titlepage}
{
\centering
\LARGE \textbf{STŘEDOŠKOLSKÁ ODBORNÁ ČINNOST}\\
\Large\textbf{Obor č. 1: Matematika a statistika}\\
\vspace{6cm}
\LARGE\textbf{Isogenie v kryptografii}\\
}
\vspace{10cm}
{\noindent\large\bfseries Zdeněk Pezlar\\ 
	\large\bfseries Jihomoravský kraj\\ }
\center\large Brno 2020
	
\end{titlepage}

\begin{titlepage}
{
\centering
\LARGE \textbf{STŘEDOŠKOLSKÁ ODBORNÁ ČINNOST}\\
\Large\textbf{Obor č. 1: Matematika a statistika}\\
\vspace{6cm}
\LARGE\textbf{Isogenie v kryptografii}\\
\vspace{1cm}
\LARGE\textbf{Isogeny Based Cryptography}\\
}
\vspace{6cm}
{\noindent\large\bfseries Autor: Zdeněk Pezlar\\ 
	\large\bfseries Škola: Gymnázium Brno, třída Kapitána Jaroše, p. o.\\
    \large\bfseries Kraj: Jihomoravský \\
	\large\bfseries Konzultant: Bc. Vojtěch Suchánek\\}

\end{titlepage}

\newpage
\thispagestyle{empty}
\vspace*{14cm}
\subsubsection*{Prohlášení}

Prohlašuji, že jsem svou práci SOČ vypracoval samostatně a použil jsem pouze prameny a literaturu uvedené v seznamu bibliografických záznamů.
Prohlašuji, že tištěná verze a elektronická verze soutěžní práce SOČ jsou shodné. 
Nemám závažný důvod proti zpřístupňování této práce v souladu se zákonem č. 121/2000 Sb., o právu autorském, o právech souvisejících s právem autorským a o změně některých zákonů (autorský zákon) v platném znění. \\[1cm]
V Brně dne: \dotfill \ \ \ \ \ \  Podpis: \dotfill

\newpage
\thispagestyle{empty}
\begin{center}
\includegraphics[width=0.35\textwidth]{podpora_soc-horizontalni.png}
\end{center}
\begin{center}
\includegraphics[width=0.45\textwidth]{logo_JMK_pruhledne.png}
\end{center}
\vspace*{0.7cm}
\begin{center}
\includegraphics[width=0.35\textwidth]{jcmm-logotype-positive1.png}
\end{center}
\vspace*{9.5cm}
\subsection*{Poděkování}
++Tato práce byla vypracována za finanční podpory JMK.


\newpage
\thispagestyle{empty}
\subsection*{Abstrakt}
abstrakt


\subsection*{Klíčová slova}
isogenie;klíčové slovo.


\vspace*{4cm}

\subsection*{Abstract}
abstrakt

\subsection*{Key words}
isogenie;klíčové slovo.






\tableofcontents
\thispagestyle{empty}

\chapter*{Úvod}
\addcontentsline{toc}{chapter}{Úvod}
\markboth{Úvod}{}

celkem úvod




\chapter*{Použitá značení}

\begin{flalign*}
&a \mid b  &&a \text{ dělí } b\\
&\mathcal{D}(a,b) &&\text{největší společný dělitel } a,b\\
&a \sim b  &&a \text{ je asociované s } b\\
&\overline{a+b\sqrt{m}} &&\text{konjugát } a+b\sqrt{m} \text{, neboli } a-b\sqrt{m}\\
&\mathbb{N},\mathbb{Z},\mathbb{Q},\mathbb{R},\mathbb{C} &&\text{množina přirozených, celých, racionálních, reálných, komplexních čísel} \\
&\mathbb{Z}_d &&\text{okruh zbytků modulo } d \\
&R[x] &&\text{okruh polynomů s koeficienty nad okruhem } R\\
&K(a_1,\dots, a_n) &&\text{nejmenší podtěleso } L, \text{ které obsahuje těleso } K \text{ i prvky } a_1, \dots, a_n \in L\\ 
&[K:L] &&\text{stupeň rozšíření tělesa } K \text{ nad } L, \text{t.j. dimenze vektorového prostoru } K:L\\ 
&\mathcal{O}_K &&\text{okruh celých algebraických čísel tělesa } K\\
&Cl(\mathcal{O}_K) &&\text{grupa tříd ideálů tělesa } K\\
&h_K &&\text{řád grupy tříd ideálů tělesa } K\\
&\mathcal{U}(\mathcal{O}_K) &&\text{grupa jednotek tělesa } K\\
&(a) &&\text{hlavní ideál generovaný prvkem } a\\
&\frac{\mathcal{I}}{m} &&\text{lomený ideál } \frac{\mathcal{I}}{m}\\
&\Big(\frac{a}{m}\Big) &&\text{hlavní lomený ideál } \frac{(a)}{m}\\
&N(a) &&\text{norma prvku } a\\
&N((a)) &&\text{norma ideálu generovaného } a\\
&\mathcal{I} \vert \mathcal{J} &&\text{ideál } \mathcal{I} \text{ dělí ideál } \mathcal{J}\\
&P_\alpha &&\text{minimální polynom } \alpha \text{ nad } K\\
&\mathsf{G} / \mathsf{H} &&\text{faktorgrupa } \mathsf{G} \text{ podle } \mathsf{H}
\end{flalign*}

\chapter{Eliptické křivky}

\section{Základy}

V naší první kapitole se budeme věnovat isogeniím eliptických křivek a práci s nimi. Budeme budovat teorii a intuici potřebnou k smysluplné diskuzi protokolu SIDH. Pro porozumění textu je třeba ovládat základy [čeho zjistím, až to napíšu]. Budeme postupovat vesměs dle \cite{DeFeo2}, nicméně další vhodný úvodní materiál se nachází na \cite{Sutherland}. Ne vždy budeme uvádět důkazy tvrzení, neboť jsou mnohdy příliš pokročilé či technické, v takových případech se odkážeme na relevantní literaturu. \\

Po celou dobu budeme pracovat nad projektivním prostorem nad uzávěrem tělesa $K$, což je množina bodů v $\overline{K}^n$, kde dva body považujeme za ekvivalentní, pokud leží v přímce s počátkem, můžeme proto místo jednotlivých bodů pracovat s přímkami skrz počátek. Chtěli bychom, aby se každé dvě $n-1$ rozměrné roviny protínaly, a s tím máme problém pouze pokud protínáme dvě rovnoběžné. V každém směru si tak můžeme definovat projektivní prostor stupně $n-1$ v nekonečnu, kde se protínají rovnoběžné roviny.

\begin{definice}
Projektivní prostor $\mathbb{P}^n (\overline{K})$ definujeme jako množinu tříd nenulových vektorů $(a_0, \dots, a_n) \in \overline{K}^{n+1}$ s ekvivalentní relací $(a_0, \dots, a_n) \sim (b_0, \dots, b_n)$, pokud existuje nenulové $\lambda$, že $(a_0, \dots, a_n) = \lambda (b_0, \dots, b_n)$. Tyto třídy ekvivalence budeme značit $(a_0 : \dots : a_n)$.\\

Pokud je jedno z $a_i$ nulové, získáme $n-1$ rozměrný prostor v nekonečnu.
\end{definice}

Projektivní prostor $\mathbb{P}^2 (\mathbb{R})$ je známý jako projektivní rovina. Každé dvě přímky se protínají v jednom bodě, přičemž rovnoběžné přímky se protínají v bodě v nekonečnu v daném směru.

\begin{poznamka}
Je zajímavé uvážit spojitost projektivních prostorů s barycentrickými souřadnicemi, kde je každý bod vyjádřen jako vážený průměr vrcholů referenčního simplexu. Tyto souřadnice jsou též homogenní a každé dvě přímky se protínají, byť některé v~nekonečnu, takové body mají součet vah roven $0$. Můžeme tedy o barycentrických souřadnicích přemýšlet jako o projektivním prostoru s jiným základem.
\end{poznamka}

Přípomeňme si pak definici eliptické křivky. Často se definuje jako nesingulární projektivní křivka genu $1$, pro naše účely si definici zúžíme.
\begin{definice}
Mějme $K$ těleso charakteristiky různé od $2$ a $3$. Pro $a,b \in K$, že\\ $4a^2+27b^3 \neq 0$, definujeme v $\mathbb{P}^2 (\overline{K})$ eliptickou křivku jako množinu bodů $(X:Y:Z)~\in~\overline{K}^3$ splňující:
\begin{equation*}
Y^2 Z = X^3 + a X Z^2 + b Z^3. 
\end{equation*}
\end{definice}

Definice vylučující tělesa s charakteristikou $2$ a $3$ nám umožňuje zapsat křivku ve výše uvedené jednoduché formě. Avšak čtenář, jenž je již obeznámen s eliptickými křivkami, může protestovat, že eliptická křivka je množina bodů $x,y \in K$ splňující:
\begin{equation*}
y^2 = x^3 + a x + b.
\end{equation*}
Pokud máme v rovnici eliptické křivky $Z=0$, pak i $X = 0$ a máme jediný bod $(0:1:0)$. Jinak můžeme celou rovnici podělit $Z^3$ a přejít na proměnné $x := \frac{X}{Z}, y := \frac{Y}{Z}$ a získat nám známou formu, kterou budeme dále označovat jako \textit{afinní}, často se v literatuře uvádí jako \textit{Weierstrassova}. Pak naše křivka je množina bodů $(x,y) \in \overline{K}^2$ splňujících $y^2 = x^3 + a x + b$ spolu s bodem v nekonečnu $\mathcal{O} = (0:1:0)$.
\begin{definice}
 Množinu všech bodů $E$ nad $K$ (společně s $\mathcal{O}$) budeme značit $E(K)$ a počet jejích prvků budeme značit $\# E(K)$.
\end{definice}


Podívejme se nyní na eliptickou křivku $E$ geometricky. Je zjevné, že má graf symetrický podle osy $x$, definujme proto k $P \in E$ bod $-P \in E$ jako obraz $P$ podle osy $x$. Pokud bychom na bodech naší křivky definovali součet, chtěli bychom, aby součet $P$ a $-P$ byl $\mathcal{O}$.\\

OBrázky\\

Pokud řekneme, že tečna k $E$ ji protíná ve dvou stejných bodech, pak každá přímka protíná $E$ v právě třech bodech včetně multiplicity. Speciálně tečna v bodě s $y=0$ tento bod protíná dvakrát a ten třetí je bod v nekonečnu $E$. Pak si sčítání $+$ na $E$ můžeme definovat tak, že součet každých tří bodů v přímce je $\mathcal{O}$. Pokud tak přímka procházející $P,Q \in E$ protíná $E$ potřetí v $R$, pak definujeme $P+ Q = -R$. Pro součet bodů $P,Q \in E$ můžeme pak odvodit několik důležitých vlastností:
\begin{enumerate}
\item $P + Q = Q + P$,
\item $(P + Q) + R =P + ( Q + R)$,
\item $P + \mathcal{O} = P$,
\item $P + (-P) = \mathcal{O}$.
\end{enumerate} 

Při takto definovaném součtu můžeme s body na $E$ pracovat jako s abelovskou grupou se sčítáním $+$ a~neutrálním prvkem $\mathcal{O}$. Samozřejmě součet dvou bodů dokážeme za pomocí analytické geometrie přímo spočíst:
\begin{veta}\label{sum}
Mějme afinní body $P = (x_1,y_1), Q = (x_2,y_2)$ s $P \neq -P$. Pak $P+Q = (x_3,y_3)$ je daný:
\begin{align*}
x_3 &= \lambda^2 - x_1 - x_2,\\
y_3 &= - \lambda x_3 - y_1 + \lambda x_1,
\end{align*}
kde:
\begin{equation*}
\lambda = \begin{cases}
\frac{y_2 - y_1}{x_2-x_1}, \text{ pokud } x_1 \neq x_2,\\
\frac{3x_1 ^2 + a}{2y_1}, \text{ pokud } x_1 = x_2.
\end{cases}
\end{equation*}

\end{veta}
Důkaz s prominutím neuvádím. Pro zkrácení zápisu si budeme definovat skálární násobky bodů následovně:
\begin{definice}
Mějme bod $P \in E$. Pak pro $n$ přirozené definujeme jeho $n$-násobek:
\begin{equation*}
[n]_E P = \underbrace{P+ \cdots + P}_{n},
\end{equation*}
příčemž pro $n < 0$ definujeme $[n]_E P = [-n]_E (-P)$ a $[0]_E P = \mathcal{O}$.
\end{definice}

Pokud bude z kontextu jasné, nad kterou eliptickou křivkou pracujeme, budeme značit násobení skalárem pouze $[n]P$.

\begin{priklad}\label{priklad1}
tu příklad sčítání, jak v Q tak v Fq, hezky graficky\\
\end{priklad}



\begin{priklad}\label{priklad2}
Určeme dvojnásobek a trojnásobek bodu $P = (x,y)$ na $E : y^2 = x^3 + ax + b$. 
\end{priklad}
\textit{Řešení: } V duchu značení \eqref{sum} máme pro $[2] P = (x_1,y_1)$:
\begin{align*}
\lambda = \frac{3x^2 + a}{2y},
\end{align*}
tedy:
\begin{align*}
x_1 &= \lambda^2 - 2x = \frac{(3x^2 + a)^2}{4 y ^2} - 2x = \frac{(3x ^2 + a)^2 - 8 y^2 x }{4 y ^2}\\
y_1 &=  - \lambda x_1 - y + \lambda x = - \frac{(3x^2 + a)[(3x ^2 + a)^2 - 8 y^2 x ]}{8y^3} - y - \frac{x(3x^2 + a)}{2y}\\
&= \frac{(3x^2 + a)[-(3x ^2 + a)^2 + 8 y^2 x - 4 y^2 x ] - 8y^4}{8y^3}
\end{align*}

Všimneme si, že pro $P$ s $y=0$ je $[2] P = \mathcal{O}$. Na příkladu \eqref{priklad1} též ale vidíme, že trojnásobek bodu ??? dává též $\mathcal{O}$. Obecně by nás mohlo zajímat, které body pošle násobení $n$ do nekonečna.

\section{Torzní body}

\begin{definice}
Buď $n$ celé číslo. O množině všech $P \in E$, že $[n] P = \mathcal{O}$, řekneme, že tvoří $n$-\textit{torzi} $E$ a tuto množinu budeme značit $E[n]$.
\end{definice}

Torzní podgrupy nám pomáhají hlouběji studovat eliptické křivky v mnohých směrech. Například pro eliptickou křivku $E$ nad konečným tělesem $\mathbb{F}_q$, je konečná grupa $E(\mathbb{F}_q)$ ????????????

\begin{veta}
Nechť je $E$ eliptická křivka nad $K$ a $m$ nenulové číslo. Pak:
\begin{itemize}
\item Pokud $\char K \nmid m$, tak $E[m] \cong \mathbb{Z}_m \times \mathbb{Z}_m $,
\item Pokud označíme $p = \char K > 0$:
\begin{equation*}
E[p^i] \cong  \begin{cases}
      \lbrace \mathcal{O} \rbrace, & \text{pro každé nezáporné } i, \\
      \mathbb{Z}_{p^i}, & \text{pro každé nezáporné } i.
    \end{cases}
\end{equation*}
\end{itemize}
\end{veta}

Navíc pokud $m \neq \char K $ je prvočíslo, tak  . Nicméně vidíme, že existuje rodina křivek, která má pouze triviální $c$-torzi.

\begin{definice}
Pokud máme $E[c] \cong \lbrace \mathcal{O} \rbrace $, nazveme $E$ \textit{supersingulární}. Jinak $E$ budeme říkat \textit{obyčejná}.
\end{definice}

Rozdělení křivek na obyčejné a supersingulární bude vhodné v mnoha ohledech, jak při diskuzi vlastností křivek, tak z kryptografického hlediska. 

\begin{veta}\label{super}
Nechť $E$ je křivka nad $\mathbb{F}_q$, kde $q = p^r$ je mocnina prvočísla $p > 3$. Pak: $$\# E(\mathbb{F}_q) \equiv 1 \pmod{p}$$  nastane právě pokud $E$ je supersingulární.
\end{veta}

\begin{veta}
Ať $E$ je křivka nad $\mathbb{F}_p$ s $p > 3$. Pak: $$\# E(\mathbb{F}_p) = p+1$$  nastane právě pokud $E$ je supersingulární.
\end{veta}

Pro určení supersingularity $E$ nás tak bude do jisté míry zajímat číslo $t = p+1 - \# E(\mathbb{F}_p)$.

Pojďme se pak podívat na strukturu bodů na supersingulární $E$ nad $\mathbb{F}_q$.
 
 \begin{veta}
Nechť $E$ je křivka nad $\mathbb{F}_q$, kde $q = p^r$ je mocnina prvočísla $p > 3$. 
\end{veta}

\begin{priklad}
Ukažme, že křivka:
 $$E : y^2 = x^3 + x$$
nad $\mathbb{F}_{p^2}$ pro $p \equiv -1 \pmod{4}$ je supersingulární.
\end{priklad}



\section{Zobrazení mezi eliptickými křivkami}

Násobení bodů v $E$ skalárem nám dává homomorfismus $E \longrightarrow E$. Tvoří proto endomorfismus z $E$ daný lomenou funkcí nad $K$. My se nyní podíváme na zobrazení mezi jednotlivými eliptickými křivkami, konkrétně homomorfismy grup $E_1(K) \longrightarrow E_2(K)$.\\

Uvažme zobrazení $(x,y) \mapsto (u^2 x, u^3 y)$, které převádí křivky:
\begin{equation*}
E_1 : y^2 = x^3 + u^4 a x + u^6 b \mapsto E_2 :  y^2 = x^3 + ax + b 
\end{equation*}
pro libovolné $u \in \overline{K}$. To je lineární zobrazení mezi $E_1(K)$ a $E_2(K)$, které zachovává přímky a tedy i součet bodů na našich křivkách, definuje proto homomorfismus z $E_1(K)$ do $E_2(K)$. Navíc je zobrazení zjevně invertibilní, tudíž dokonce mezi $E_1(K)$ a $E_2(K)$ dává isomorfismus nad $\overline{K}$.

\begin{veta} (Sato-Tate)
Dvě křivky $E_1,E_2$ nad $K$ jsou nad $K$ isomorfní právě pokud $\# E_1(K) = \# E_2(K)$.
\end{veta}

Speciálně dvě křivky, které mají nad $K$ pouze body v nekonečnu, jsou nad $K$ isomorfní. Ne vždy máme nutně isomorfismus nad $K$, ale nad jeho rozšířením. Aby byl náš isomorfismus nad $\overline{K}$ definovaný, musí být díky předpisu $(x,y) \mapsto (u^2 x, u^3 y)$ nutně nad rozšířením $K$ stupně dělícího $6$.
\begin{definice}
Buďte $E,E^\prime$ křivky isomorfní nad rozšířením $K$, ale ne nad $K$. Pak řekneme, že $E^\prime$ je \textit{twistem} $E$ nad $K$.
\end{definice}
Zobrazení $(x,y) \mapsto \big(\frac{x}{d}, \frac{y}{\sqrt{d^3}}\big) $ pro $\sqrt{d} \not \in K, d \in K$ nám dává isomorfismus z  $E:  y^2 = x^3 + ax + b$ na:
\begin{equation*}
E:  y^2 = x^3 + ax + b \simeq E_d : y^2 = x^3 + d^2 a x + d^3 b \Leftrightarrow d y^2 = x^3 + ax + b,
\end{equation*}
avšak ne nad $K$, ale nad jeho kvadratickým rozšířením $K(\sqrt{d})$. $E_d$ nazveme \textit{kvadratickým twistem} $E$.\\

Pro křivky s $a=0$, resp. $b=0$, můžeme analogicky najít \textit{kubický} a \textit{sextický}, resp. \textit{kvartický} twist:
\begin{align*}
y^2 = x^3 + b &\mapsto y^2 = x^3 + d^2 b ,\\
y^2 = x^3 + b &\mapsto y^2 = x^3 + d b, \\
y^2 = x^3 +ax &\mapsto y^2 = x^3 + d ax ,
\end{align*}
dané po řadě $(x,y)  \mapsto \big(\frac{x}{\sqrt[3]{d^2}}, \frac{y}{d}\big)$ a $(x,y)  \mapsto \big(\frac{x}{\sqrt[3]{d}}, \frac{y}{\sqrt{d}}\big)$, resp. $(x,y)  \mapsto \big(\frac{x}{\sqrt{d}}, \frac{y}{\sqrt[4]{d^3}}\big)$. Vidíme, že poslední dvě zmíněné křivky jsou navíc kvadratickými twisty po řadě kubického a kvadratického twistu $E$.\\


Chtěli bychom říci, kdy mezi dvěma eliptickými křivkami existuje isomorfismus, tedy najít nějaký invariant, který isomorfní křivky sdílí. Takovou funkci splňuje právě $j$-invariant.

\begin{definice}
Pro eliptickou křivku $E: y^2 = x^3 + ax + b$ definujeme její $j$-invariant jako:
\begin{equation*}
j(E) = 1728 \frac{4a^3}{4a^3+27b^2}.
\end{equation*}
\end{definice}
Poznamenejme, že ten je vždy nad $K$ definovaný, neboť eliptické křivky mají nenulový diskriminant.
\begin{veta}
Dvě křivky definované nad $K$ jsou isomorfní nad $\overline{K}$ právě pokud mají stejný $j$-invariant.
\end{veta}

Mějme následujících pět křivek nad $\mathbb{Z}_{101}$:
\begin{align*}
E_1 : y^2 &= x^3+x+1,\\
E_2 : y^2 &= x^3+5x+23,\\
E_3 : y^2 &= x^3+x-1,\\
E_4 : y^2 &= x^3+2,\\
E_5 : y^2 &= x^3+2x.
\end{align*}
Spočtěme si pak jejich $j$-invarianty nad $\mathbb{Z}_{101}$:
\begin{align*}
j(E_1) &= 1728 \frac{4}{31},\\
j(E_2) &= 1728 \frac{4 \cdot 5^3}{4 \cdot 5^3+27 \cdot 23^2} = 1728 \frac{4 \cdot 24}{4 \cdot 24 + 27 \cdot 24} = 1728 \frac{4}{31},\\
j(E_3) &= 1728 \frac{4}{31},\\
j(E_4) &= 1728,\\
j(E_5) &= 0.
\end{align*}
Vidíme, že $j$-invarianty $E_1$ a $E_2$ se rovnají, přičemž v $\mathbb{Z}_{101}$ se oba rovnají $1728 \cdot 4 \cdot 88$, nutně mezi nimi existuje isomorfismus. Snadno ověříme, že zobrazení:
\begin{equation*}
(x,y) \mapsto (3^2 x, 3^3 y) = (9x,27y) 
\end{equation*}
převádí:
\begin{align*}
y^2 &= x^3+x+1 \mapsto 27^2 y^2 = 9^3 x^3 + 9x + 1,\\
&\Leftrightarrow 22 y^2 = 22 x^3 + 9x + 1,\\
&\Leftrightarrow 22 y^2 = 22 x^3 + 110x + 506,\\
&\Leftrightarrow y^2 = x^3 + 5x + 23.
\end{align*}
Inverzní isomorfismus $E_2 \longrightarrow E_1$ je pak daný $(x,y) \mapsto (34^2 x, 34^3 y) = (45x,15y)$, neboť multiplikativní inverz $3$ v $\mathbb{Z}_{101}$ je $34$.\\

Křivka $E_3$ má ale též stejný $j$-invariant jako $E_1$ a $E_2$, nad $\mathbb{Z}_{101}$ mezi nimi a $E_3$ přesto isomorfismus neexistuje. $E_3$ je kvadratickým twistem $E_1$ nad $\mathbb{Z}_{101^2} = \mathbb{Z}_{101}[i]$, jakožto zobrazení $(x,y) \mapsto \Big(\frac{x}{i^2}, \frac{y}{i^3}\Big) = (-x,iy)$ převádí:
\begin{align*}
y^2 &= x^3+x+1 \mapsto -y^2 = -x^3-x+1,\\
\Leftrightarrow y^2 &=  x^3 + x - 1.
\end{align*}
Dvě speciální hodnoty $j$-invariantu jsou $0$ a $1728$, kterých nabývají křivky, které mají po řadě lineární, resp. konstantní člen roven $0$. Právě křivky s $j$-invariantem $0$ mají kubický (a~sextický) twist, ty s $j$-invariantem $1728$ zase kvartický.\\

Mohli bychom se nicméně zajímat, proč se v~$j$-invariantu násobí číslem $1728$. Důvodem jsou tělesa charakteristik $2$ a $3$, $j$-invariant je totiž definován pro libovolnou nesingulární projektivní křivku genu $1$, tj.:
\begin{equation*}
y^2 + a_1 xy + a_3 y =  x^3 + a_2 x^2 + a_4 x + a_6,
\end{equation*}
konkrétně jako:
\begin{equation*}
\frac{(b_2 ^2 - 24 b_4)^2}{\Delta},
\end{equation*}
kde $\Delta$ je diskriminant naší křivky a $b_2 = a_1^2 + 4a_2, b_4 = 2a_4 + a_1 a_3$.
Pro tělesa s $\char K \neq 2,3$ můžeme definovat zobrazení, která převádí naši křivku na nám známý afinní tvar, detaily důkazu jsou k nalezení v \cite[kap.~3]{SilvermanArithm}. Obraz rovnice $j$-invariantu je právě takový, jak ho zde definujeme, násoben konstantou $1728$.\\

Jak násobení bodů $E$ skalárem, tak twistování, jsou homomorfismy bodů křivek nad tělesem $K$, resp. jeho rozšířením. Spadají tak pod rodinu zobrazení eliptických křivek zvaných \textit{isogenie}, o kterých se budeme dále bavit.

\section{Isogenie}



\begin{definice}
Ať $E_1,E_2 \in \overline{K}$ jsou eliptické křivky. Pak surjektivní morfismus $E_1 \longrightarrow E_2$  daný racionální funkcí nad $K$,  který posílá bod v nekonečnu $E_1$ na bod v nekonečnu $E_2$, nazveme isogenií. Pokud mezi $E_1,E_2$ existuje isogenie, nazveme je \textit{isogenní}.
\end{definice}

\begin{definice}
Pod stupňem isogenie $\phi$ budeme rozumět jejímu stupni jako racionální funkci v $x$, budeme značit $\deg \phi$. Obraz křivky $E$ v $\phi$ budeme značit $\phi(E)$. 
\end{definice}

Stejně jako jsme se zabývali torsní podgrupou našich křivek, nebude překvapením, že bude pro studium isogenií důležité, které body zobrazí do nekonečna. 


\begin{definice}
Pod jádrem $\phi$ rozumíme jádru $\phi$ jako homomorfismu grup.
\end{definice}

S isogeniemi jsme se již na naší (prozatím) krátké cestě několikrát setkali, jak násobení skalárem, tak naše isomorfismy zmíněné na konci předchozí kapitoly, jsou isogeniemi. Násobení $[n]$ má z Véluových formulí stupeň $n^2$, isomorfismy jsou isogenie lineární a jejich jádry jsou po řadě $E[n]$ a $\mathcal{O}$. Zobrazení:
\begin{equation*}
\phi : y^2 = x^3+x \longrightarrow y^2 =  x^3 + 11x + 62
\end{equation*}
mezi křivkami nad $\mathbb{Z}_{101}$ dané $(x,y) \mapsto \Big(\frac{x^2 + 10x - 2}{x+10},\frac{x^2 y + 20xy + y}{x^2 + 20x - 1}\Big)$ je též isogenií, tentokrát stupně dvě. Jádrem $\phi$ je množina $\lbrace \mathcal{O},10 \rbrace$, protože $x^2 + 20x - 1 = (x+10)^2$ v~$\mathbb{Z}_{101}$.\\

Pojďme se nyní pobavit o několika základních vlastností isogenií.\\



teext

\begin{veta}
Buď $\phi: E \longrightarrow E^\prime$ isogenie stupně $n$. Pak existuje jediná isogenie $\hat{\phi}: E^\prime \longrightarrow E$, která pro každou jinou isogenii $\psi: E^\prime \mapsto E$ splňuje:
\begin{enumerate}
\item $\phi \circ \hat{\phi} = [n]_{E^\prime}$,
\item $\hat{\phi} \circ \phi = [n]_E$,
\item $\widehat{\phi \circ \psi} = \hat{\psi} \circ \hat{\phi}$,
\item $\widehat{\phi + \psi} = \hat{\phi} + \hat{\psi}$,
\item $\hat{\hat{\phi} } = \phi $.
\end{enumerate} 
Isogenii $\hat{\phi}$ budeme označnovat jako isogenii duální k $\phi$.
\end{veta}

Díky $\deg \phi \circ \psi = \deg \phi \cdot \deg \psi$ pro libovolné racionální funkce $\deg \phi, \psi $, je $\hat{\phi}$ též isogenií stupně $n$.

\begin{definice}
Mějme $E,E_1 \in \overline{K}$ a $\phi: E \longrightarrow E_1$ isogenii stupně $k$. Pokud je $\# \ker \phi = k$, pak o $\phi$ řekneme, že je separabilní. Jinak řekneme, že $\phi$ je neseparabilní. V případě, že je $\deg \phi $ roven mocnině $\char K$, mluvíme o $\phi$ jako o čistě neseparabilní.
\end{definice}

Jak tomu bylo v případě násobení skalárem, $\ker \phi$ tvoří podgrupu $E(K)$. Ukáže se, že separabilní isogenie $E$ se dělí na třídy isomorfismů jednoznačně určené svým jádrem.\\

\begin{veta}
Každá separabilní isogenie $\phi$ z $E$ je, až na isomorfismus, jednoznačně určena svým jádrem.
\end{veta}

Pokud je tak $G = \ker \phi$ grupa tvořená jádrem $\phi$, můžeme značit $E/G$ cílovou křivku $\phi$. Separabilní isogenie z $E \longrightarrow E^\prime$ je daná lomenou funkcí nad $K$ a známe-li její jádro, dokážeme ji explicitně spočíst, přičemž libovolná podgrupa $E(K)$ je jádrem isogenie. Vzorce udávající (až na isomorfismus) přesný tvar separabilní isogenie z $E \longrightarrow E^\prime$ s daným jádrem se nazývají \textit{Véluovy} po Jeanu Véluovy, který je první publikoval v [?]. Jejich zápis je obecně velmi nezáživný a~pro nás je nepodstatný, stačí nám mít v povědomí, že separabilní isogenie s daným jádrem můžeme explicitně vyjádřit. Jejich přesnou formu a důkaz správnosti je k nalezení v \cite[kap.~8.2]{DeFeo}.\\
 


\begin{veta}\label{isomor}
Buďte $\phi,\psi$ dvě isogenie z $E$. Pak:
\begin{equation*}
\phi (\psi (E)) \cong \psi (\phi (E)).
\end{equation*}
\end{veta}


Pro separabilní isogenie tak můžeme říci:
\begin{equation*}
(E/ \langle A \rangle)/ \langle B \rangle \cong (E/ \langle B \rangle)/ \langle A \rangle.
\end{equation*}
 





\chapter{Užití v kryptografii}

Přes Caesarovu šifru až po šifrování za pomocí Enigmy v období druhé světové války, po většinu lidské historie se využívaly kryptografické systémy založené na faktu, že obě komunikující partie si po domluvě vyberou způsob maskování zprávy a ten pro ostatní zůstává skrytý. Příkladem je právě o kolik písmen v Caesarově šifře transponujeme. Tento způsob nutně závisí na faktu, že se obě strany před výměnou mají možnost přes bezpečný kanál na tomto způsobu domluvit. S přibývajícím počtem účastníků a~frekvencí komunikace, na příklad našeho každodenního interagování na internetu, kde musí konverzace mezi všemi účastníky být bezpečná, je bohužel na úkor ceny přenosu třeba vyšší počet a velikost klíčů, a příbývá risk kompromitace.\\

Kvůli takovým obavám přišli Whitfield Diffie a Martin Hellman v~roce $1976$ s revolučním nápadem: asymetrickou kryptografií, kde každý z účastníků má svůj vlastní \textit{privátní klíč}, který s nikým nesdílí. Všechny strany, i potenciální útočník, znají několik informací, které jsou známé jako \textit{veřejné parametry}. Obě komunikující strany za pomocí veřejných informací tajně transformují svůj privátní klíč a výsledek, který budeme nazývat \textit{veřejným klíčem}, publikují. Oba účastnící vezmou veřejný klíč toho druhého a provedou s ním ty samé tajné kroky závisící na jejich privátním klíči. Podstatou takové výměny je, že na jejím konci získají obě původní strany netriviální informaci, tedy informaci takovou, že žádná třetí strana ji nedokáže snadno uhodnout, za pomocí níž poté mohou společnou komunikaci šifrovat a nikdo jiný již jejich zprávy neuvidí. Předpokládá se, že pouze ze znalosti veřejného klíče je pro každou další partii těžké replikovat klíč privátní a že pole možných sdílených informací je obrovské. Vyhneme se tak přímočarým řešením hrubou silou.\\

Pojďme se podívat na protokol, který Diffie a Hellman navrhli. Budeme o něm dále mluvit jako o \textit{Diffie-Hellmanově výměně}. Je založena na problému \textit{diskrétního logaritmu} prvku $a \in \mathbb{Z}_p^{*}$. Tento problém po nás ze znalosti primitivního kořene $g$ modulo $p$ žádá najít $k$, že $g^k = a$ v $\mathbb{Z}_p$. Obecně můžeme $\mathbb{Z}_p$ nahradit cyklickou grupou $G$ a mít $g$ její generátor. Protokol požaduje, aby nebyl diskrétní logaritmus spočitatelný efektivně, tj. v~polynomiálním čase vzhledem k velikosti grupy, jinak může útočník jednoduše privátní klíče obou stran spočíst, ale mocnění bylo. Umocnit číslo dokážeme v logaritmickém čase, a v konečné grupě nám stačí umocnit pouze na exponent modulo řádu grupy. \\

%\footnote{Funkce, která je na každém vstupu efektivně spočitatelná, ale není efektivně invertibilní, se nazývá \textit{jednosměrná}. Existence takové funkce by byla velkým pokrokem pro kryptografie, bohužel žádná taková nebyla nalezena}
\begin{figure}[h]
\begin{center} 
\makebox[1cm]{\rule{15cm}{0.4pt}}\\
\hspace{-1.35cm} \textbf{Veřejné parametry:} Grupa $G$ řádu $p$, kde $p$ je prvočíslo, s generátorem $g$.\\

\vspace{-0.25cm}
\makebox[\linewidth]{\rule{15cm}{0.4pt}}\\
\vspace{0.2cm}
\begin{tabular}{l l c l}
\cline{2-2} \cline{4-4} 
& Alfréd & & Blažena \\ 
\cline{2-2} \cline{4-4} 
& \textbf{Vstup:} $a \in G^{*}$ & & \textbf{Vstup:} $b \in G^{*}$ \\
 & & $\stackrel{g^a}{\longrightarrow} $ &  \\
&  & $\stackrel{g^b}{\longleftarrow} $ &  \\
& $G_{AB} := \big(g^b\big)^{a} = g^{ab}$ &  & $ G_{BA} := \big(g^a\big)^{b} = g^{ba}$ \\
& \textbf{Výstup:} $G_{AB}$ & & \textbf{Výstup:} $G_{BA}$
\end{tabular}
\caption*{Algoritmus 1: Diffie-Hellmanova výměna}
\vspace{-0.8cm}
\end{center}
\end{figure}

Díky předpokladu, že $G$ je cyklická, je i abelovská, tedy $G_{AB} = g^{ab} = g^{ba} = G_{BA}$. Na konci protokolu tak mají obě strany shodné tajemství $g^{ab}$.\\

Řád $G$ se prakticky bere prvočíslo $q = 2p+1$, že $p$ je prvočíslo, $p$ nazveme tzv. \textit{Sophie-Germainovým prvočíslem} a $q$ zase \textit{bezpečným prvočíslem}. V takovém případě má $G$ podgrupu prvočíselného řádu $p$, což je z kryptografického hlediska žádané, je tuto grupu totiž obtížnější spočíst. Navíc bezpečná prvočísla skýtají i výhody pro inicializování výměny, pro taková prvočísla dokážeme totiž snadno nalézt primitivní kořen. Konkrétně, $g$ je primitivní kořen modulo $2p+1$, tedy má řád $q-1 = 2p$ modulo $q$, právě pokud $g^{p} \equiv -1 \pmod{q}$. Stačí nám pak najít $g^{p} \pmod{q}$, což nám mohou usnadnit nástroje jako Eulerovo kritérium, díky kterému je postačující mít $g$ kvadratický nezbytek modulo $q$.\\

Veřejné klíče $g^a,g^b$, jsou nicméně, jak jejich název napovídá, veřejné, a má k nim přístup libovolná jiná osoba. Dejme tomu, že Eva, která má přístup pouze k veřejně dostupným informacím $G,g,g^a,g^b$, by chtěla též znát sdílené tajemství. Jeden způsob, jak by mohla tajnou informací získat, je pokud by spočítala diskrétní logaritmus $\log_g(g^a) = a$, nicméně předpokládáme, že to je obtížné. Na klasických počítačích jsou nejlepší známé útoky na problémy, jako diskrétní logaritmus a faktorizace čísla, na čemž jsou založené protokoly jako RSA, subexponenciální, nicméně na počítačích kvantových jsou už od poloviny $90$. let známé algoritmy polynomiální. V čem však takto podstatné zrychlení spočívá?

\section{Kvantové počítače}
\begin{center}
\begin{verse}
\setverselinenums{1}{3}
\qquad \textit{If computers that you build are quantum,}\\
\qquad \textit{Then spies of all factions will want 'em.}\\
\qquad \textit{Our codes will all fail,}\\
\qquad \textit{And they'll read our email,}\\
\qquad \textit{Till we've crypto that's quantum, and daunt 'em. }
\end{verse}
\hfill \textit{Jennifer a Peter Shorovi}
\end{center}

Při diskuzi moderní kryptografie se často zmiňuje, že kvantové počítače dokáží problémy, jako faktorizaci čísla či diskrétní logaritmus, vyřešit v polynomiálním čase, přičemž nejrychlejší známé algoritmy pro klasické počítače pracují v čase subexponenciálním.  V čem ale takto podstatné zrychlení spočívá?\\

Ve světě kvantových obvodů místo s klasickými bity pracuje s \textit{qubity}. V $n$ bitovém systémů máme $2^n$ různých stavů, které v $n$ qubitovém systému tvoří generátory našeho prostoru. Podstatou je, že před pozorováním nemá daný qubit jednu z těchto hodnot, ale jejich (komplexní) superpozici. Generátory systému s jedním qubitem jsou stavy $\vert 0 \rangle = \begin{bmatrix}
1 \\
0
\end{bmatrix}, \vert 1 \rangle = \begin{bmatrix}
0 \\
1
\end{bmatrix}$, systém je tedy:
\begin{equation*}
\alpha \vert 0 \rangle + \beta \vert 1 \rangle,
\end{equation*}
kde $\alpha, \beta$ jsou komplexní čísla $\vert \alpha \vert ^2 + \vert \beta \vert ^2 = 1$. Zápis $\vert \psi \rangle$ je tzv. \textit{ket} notace, kde $\psi$ je vektor.\\

V dvojqubitovém systému máme čtyři báze a stav takového systému je:
\begin{equation*}
\alpha \vert 00 \rangle + \beta \vert 01 \rangle + \gamma \vert 10 \rangle + \delta \vert 11 \rangle, 
\end{equation*}
kde $\alpha,\beta,\gamma,\delta$ jsou komplexní čísla s $\vert \alpha \vert ^2 + \vert \beta \vert^2 + \vert \gamma \vert^2 + \vert \delta \vert^2 = 1$. Qubity jsou značně nestabilní, musí být uchovány v izolované soustavě, nejčastěji v neutrinu, přičemž jakékoli narušení, i~pouhé pozorování hodnoty qubitu, ho kolapsuje na jednu hodnotu, kterou už pak zůstane. Při pozorování má qubit pravděpodobnost ukázat stav právě takovou, kolik je druhá mocnina absolutní hodnoty příslušného koeficientu, proto ona normalizační podmínka. Pokud bychom pozorovali náš jedno-qubitový systém, s pravděpodobností $\vert \alpha \vert ^2$ získáme výstup $0$, s pravděpodobností $\vert \beta \vert ^2$ získáme $1$.\\

Můžeme ale též náš qubit vyjádřit ve vektorovém zápisu:
\begin{equation*}
\vert \psi \rangle = \begin{bmatrix}
\alpha \\
\beta \\
\gamma \\
\delta
\end{bmatrix},
\end{equation*}
což samozřejmě zobecníme pro systémy více qubitů. Tento vektor je díky naší podmínce jednotkový. V klasických obvodech máme brány, které jsou lineární zobrazení našich stavů, příklady takových bran jsou $OR$ a $NOT$. V kvantových obvodech bereme jako brány právě unitární matice a jejich operaci násobení, neboť ty zachovávají normu vektoru, jejich výsledky jsou proto opět qubity.\\

Nedá moc práce ukázat, že všechny operace proveditelné na klasickém obvodu jsou replikovatelné kvantovými branami, model kvantového počítače, jakožto obvodu, je tak alespoň stejně silný jako počítač klasický.\\

Jedním z důvodů, proč se věří, že s veřejně dostupnými kvantovými počítači přijde nová éra výpočetní techniky je, že existují procesy, o kterých se dodnes neví, zda jsou v~polynomiálním čase proveditelné na počítači klasickém, a jejichž kvantové algoritmy již byly nalezené. Klasické násobení čísel ($n$ bitových), zabere $O(n^2)$ času, případně až $O(n^2 \log n)$ pro velká čísla, neboť poté násobení ani sčítání neprovedeme v konstantním čase. Násobení dvou čísel se dá redukovat na problém násobení dvou polynomů, přičemž diskrétní Fourierova transformace z našeho polynomu nám dá informaci o hodnotách polynomu v odmocninách z jednotky, což je vše, co potřebujeme k určení polynomu. Rychlá Fourierova transformace toto dokáže pouze v $O(n \log n)$ čase. Díky její multiplikativitě, linearitě a její inverzní funkci pak dokážeme zpětně v tomto čase získat součin dvou čísel.\\

Kvantová Fourierova transofrmace (QFT), která obdobnou operaci aplikuje na náš vektor, na klasickém počítači s $n$ qubity počítá s $2^n$ prvky a nejlepší známé algoritmy ji provádí v~$\Omega (n^2 2^n)$ čase, zatímco na kvantových počítačích pracuje v kvadratickém čase a s jistou přesností i v $O(n \log n)$, viz \cite[kap 4. a 5.]{Chuang} pro více informací.\\
 



proč to Shor rozbíjí.
--

\section{SIDH}

Nyní, když jsme již trochu obeznámeni s kvantovými algoritmy, pojďme se vrátit k eliptickým křivkám. Zjevnou adaptací Diffie-Hellmanova protokolu je protokol, který nese název ECDH (Elliptic Curve Diffie-Hellman):
\begin{figure}[h]
\begin{center} 
\makebox[1cm]{\rule{17.3cm}{0.4pt}}\\
\hspace{-1.35cm} \textbf{Veřejné parametry:} Prvočíslo $p$ a eliptická křivka $E$ nad $\mathbb{Z}_p$ s generátorem $G \in E(\mathbb{Z}_p)$.\\

\vspace{-0.25cm}
\makebox[\linewidth]{\rule{17.3cm}{0.4pt}}\\
\vspace{0.2cm}
\begin{tabular}{l l c l}
\cline{2-2} \cline{4-4} 
& Alfréd & & Blažena \\ 
\cline{2-2} \cline{4-4} 
& \textbf{Vstup:} $a \leqslant \# E(\mathbb{Z}_p)-1$ & & \textbf{Vstup:} $b \leqslant \# E(\mathbb{Z}_p)-1$ \\
 & & $\stackrel{[a]G}{\longrightarrow} $ &  \\
&  & $\stackrel{[b]G}{\longleftarrow} $ &  \\
& $G_{AB} := [a]([b]G) = [a][b]G$ &  & $ G_{BA} := [b]([a]G) = [b][a]G$ \\
& \textbf{Výstup:} $G_{AB}$ & & \textbf{Výstup:} $G_{BA}$
\end{tabular}
\caption*{Algoritmus 2: Protokol ECDH}

\end{center}
\end{figure}


Tento protokol je založen na předpokladu, že diskrétní logaritmus na eliptických křivkách, tedy ze znalosti $P$ a $[n]P$ spočíst $n$, je těžký problém. Není znám žádný algoritmus, který by nezískal společné tajemství výpočtem privátních klíčů obou stran. \\

??\\

Pojďme se nyní znovu podívat na větu \eqref{isomor}.  Vidíme, že křivky $\phi (\psi (E)), \psi (\phi (E))$ sdílí $j$-invariant, neboť jsou isomorfní, což by v potenciálním protokolu založeném na isogeniích mohlo být sdílené tajemství obou stran. Pokud tak mají obě strany danou křivku $E$ nad $\mathbb{Z}_p$, vyberou si tajné isogenie $\phi_A$, resp. $\phi_B$, pošlou druhé straně $\phi_A(E)$, resp. $\phi_B(E)$ a obě strany již snadno spočtou své tajemství. Takové myšlenky měli De Feo a Jao v [?], nicméně než se dostaneme přímo k jejich navrhovaném protokolu SIDH, musíme diskutovat několik důležitých detailů, které se vyhýbají známým útoků, případně usnadňují výměnu.\\

Jak napovídá název protokolu, požadujeme supersingularitu $E$. Pak totiž z věty \eqref{super} je $\#E(\mathbb{Z}_{p^2}) = p+1$ a $E(\mathbb{Z}_{p^2}) \cong \mathbb{Z}_{p+1} \times \mathbb{Z}_{p+1}$. Pro prvočíslo $p = \ell_A ^{e_A} \ell_B ^{e_B} - 1$, kde $\ell_A,\ell_B$ jsou prvočísla, proto existují dva body $G_1,G_2$ řádu $\ell_A ^{e_A} \ell_B ^{e_B}$, které generují $E(\mathbb{Z}_p^2)$. Speciálně dvojice $\langle P_A, Q_A \rangle := \langle [\ell_B ^{e_B}]G_1, [\ell_B ^{e_B}]G_2 \rangle$, resp. $\langle P_B, Q_B \rangle := \langle  [\ell_A ^{e_A}]G_1, [\ell_A ^{e_A}]G_2 \rangle$, generují po řadě $\ell_A ^{e_A}$, $\ell_B ^{e_B}$ torzi.\\

Uvažme bod $P \in E[\ell_A ^{e_A}]$ řádu $\ell_A  ^t$ a separabilní isogenii $\phi : E \mapsto E/\langle P \rangle$. Pokud bychom chtěli $E/\langle P \rangle$ spočíst, stačilo by spočítat celou $\langle P \rangle$ a za pomocí Véluových formulí spočíst výslednou křivku v exponenciálním čase $O(\ell_A  ^t)$, což zjevně není optimální.

\begin{figure}[h]
\begin{center} 
\makebox[1cm]{\rule{15cm}{0.4pt}}\\
\hspace{-1.35cm} \textbf{Veřejné parametry:} Grupa $G$ řádu $p$, kde $p$ je prvočíslo, s generátorem $g$.\\

\vspace{-0.25cm}
\makebox[\linewidth]{\rule{15cm}{0.4pt}}\\
\vspace{0.2cm}
\begin{tabular}{l l c l}
\cline{2-2} \cline{4-4} 
& Alfréd & & Blažena \\ 
\cline{2-2} \cline{4-4} 
& \textbf{Vstup:} $a \in G$ & & \textbf{Vstup:} $b \in G$ \\
 & & $\stackrel{g^a}{\longrightarrow} $ &  \\
&  & $\stackrel{g^b}{\longleftarrow} $ &  \\
& $G_{AB} := \big(g^b\big)^{a} = g^{ab}$ &  & $ G_{BA} := \big(g^a\big)^{b} = g^{ba}$ \\
& \textbf{Výstup:} $G_{AB}$ & & \textbf{Výstup:} $G_{BA}$
\end{tabular}
\caption*{Algoritmus 1: Diffie-Hellmanova výměna}
\vspace{-0.8cm}
\end{center}
\end{figure}

\chapter{Algebraická teorie čísel}

Ve snaze vybudovat teorii k hlubšímu studiu eliptických křivek a isogenií, natož diskuzi protokolu CSIDH, musíme samozřejmě někde začít, od čtenáře následujících sekcí se proto předpokládá znalost základu algebraické teorie čísel. Jako podrobné materiály ke studiu této krásné oblasti matematiky vřele doporučuji \cite{4}, ?. jako decentní stručný úvod motivovaný poznatky z elementární teorie čísel může posloužit má SOČ \cite[kap. 2]{Pezlar}.\\

Připomeňme si několik pár základních faktů ohledně tříd ideálů okruhu $\mathcal{O}$ 
 

\chapter*{Závěr}
\addcontentsline{toc}{chapter}{Závěr}
\markboth{Závěr}{}
zu ende


%\begin{algorithm}
%\caption{A}
%\begin{algorithmic}
%\REQUIRE $n \geq 0 \vee x \neq 0$
%\ENSURE $y = x^n$
%\STATE $y \leftarrow 1$
%\IF{$n < 0$}
%\STATE $X \leftarrow 1 / x$
%\STATE $N \leftarrow -n$
%\ELSE
%\STATE $X \leftarrow x$
%\STATE $N \leftarrow n$
%\ENDIF
%\WHILE{$N \neq 0$}
%\IF{$N$ is even}
%\STATE $X \leftarrow X \times X$
%\STATE $N \leftarrow N / 2$
%\ELSE[$N$ is odd]
%\STATE $y \leftarrow y \times X$
%\STATE $N \leftarrow N - 1$
%\ENDIF
%\ENDWHILE
%\end{algorithmic}
%\end{algorithm}



\begin{thebibliography}{97}

\bibitem{Chuang}
\textsc{Chuang}, Isaac L. a \textsc{Nielsen}, Michael A.: \textit{Quantum computation and Quantum Information}. Cambridge University Press, Cambridge, 2000. 


\bibitem{DeFeo}
\textsc{De Feo}, Luca.: \textit{Fast Algorithms for Towers of Finite Fields and Isogenies}. EcolePolytechnique X, 2010. 

\bibitem{DeFeo2}
\textsc{De Feo}, Luca.: \textit{Mathematics of Isogeny Based Cryptography}. Université de Versailles \& Inria Saclay, 2017. Dostupné z: \url{https://arxiv.org/abs/1711.04062}.

\bibitem{Marcus}
\textsc{Marcus}, Daniel A.: \textit{Number fields}. New York: Springer-Verlag, 1977.

\bibitem{Matushak}
\textsc{Matushak}, Andy a \textsc{Nielsen}, Michael A.: \textit{ Quantum computing for the very curious}. San Francisco, 2019. Dostupné z: \url{https://quantum.country/qcvc}.

\bibitem{Pezlar}
\textsc{Pezlar}, Zdeněk: \textit{Zajímavá využití algebraické teorie čísel}. Středoškolská odborná práce. Brno, 2020.

\bibitem{Raclavsky}
\textsc{Raclavský}, Marek: \textit{Racionální body na eliptických křivkách}. Diplomová práce. Praha, 2014.

\bibitem{Rosicky}
\textsc{Rosický}, Jiří: \textit{Algebra}. Brno: Masarykova univerzita, 2002.

\bibitem{SilvermanArithm}
\textsc{Silverman}, Joseph H.: \textit{The Arithmetic of Elliptic Curves}. New York: Springer-Verlag, 1992. 


\bibitem{Shor}
\textsc{Shor}, Peter W.: \textit{Polynomial-Time Algorithms for Prime Factorizationand Discrete Logarithms on a Quantum Computer}. New York: Springer-Verlag, 1994. Dostupné z: \url{https://arxiv.org/abs/quant-ph/9508027}.

\bibitem{Sutherland}
\textsc{Sutherland}, Andrew: \textit{Elliptic Curves}. Massachusetts Institute of Technology, 2017. Dostupné z: \url{https://math.mit.edu/classes/18.783/2017/lectures.html}. 


\bibitem{Washington}
\textsc{Washington}, Lawrence C.: \textit{Elliptic Curves: Number theory and cryptography}. Maryland, 2008. 

\end{thebibliography}
\end{document}



%\begin{veta} Nechť $p,q$ jsou různá lichá prvočísla. Potom 
%$$\left( \frac{p}{q} \right) = \left( \frac{q}{p} \right) \cdot (-1)^{\frac{(p-1)(q-1)}{4}}.$$

%Dále navíc Pro libovolná celá čísla $a,b$ a liché prvočíslo $p$ platí:
%\begin{enumerate}
%\item $\bigl( \frac{a}{p} \bigr)\cdot\bigl( \frac{b}{p} \bigr)=\bigl( %\frac{ab}{p} \bigr),$
%\item $\bigl( \frac{-1}{p} \bigr) = (-1)^{\frac{p-1}{2}},$
%\item $\bigl( \frac{2}{p} \bigr) = (-1)^{\frac{p^2-1}{8}}.$ 
%\end{enumerate}
%\end{veta}

%\ Vzhledem k důležitosti těchto tvrzení uvedeme ještě ekvivalentní formu, jíž je možné některé z nich vyjádřit -- a to pomocí kongruencí:

%\begin{veta} Nechť $p,q$ jsou různá lichá prvočísla. Potom 
%$$\left( \frac{p}{q} \right) = \begin{cases}
%\left( \frac{q}{p} \right) \qquad \text{pokud} \; p\;\text{nebo}\; q\equiv 1\pmod4;\\ 
%-\left( \frac{q}{p} \right) \qquad \mbox{pokud} \; p\equiv q\equiv 3 \pmod{4}. \end{cases} $$

%Dále navíc pro libovolná celá čísla $a,b$ a liché prvočíslo $p$ platí:
%\begin{enumerate}
%\item $\bigl( \frac{a}{p} \bigr)\cdot\bigl( \frac{b}{p} \bigr)=\bigl( \frac{ab}{p} \bigr),$
%\item $\bigl( \frac{-1}{p} \bigr) =
%\begin{cases}
%1 \quad \text{pokud} \; p\equiv1\pmod4\\
%-1\quad\text{pokud}\; p\equiv3\pmod4
%\end{cases}$
%\item $\bigl( \frac{2}{p} \bigr) = 
%\begin{cases}
%1 \quad \text{pokud} \; p\equiv\pm1\pmod8\\
%-1\quad\text{pokud}\; p\equiv\pm3\pmod8.
%\end{cases}$
%\end{enumerate}
%\end{veta


% Cvičení: falešný násobení PO_L


% \Zdroje: Dumit Foote, Zakony reciprocity, Rosicky, Cox, Marcus, clanek o Mihaelescau?, Ireland Rosen?, Pupik?

%\begin{definice} Množinu $G$ spolu s binární operací $\odot$ na ní definovanou nazveme grupou, pokud splňuje tyto podmínky:
%\begin{enumerate}
%\item operace je asociativní, tzn.\ pro každé $x,y,z \in G$ platí $(x\odot y)\odot z=x\odot(y\odot z)$,
%\item existuje tzv.\ neutrální prvek, tedy nějaké $e \in G$ takové, že pro každé $x \in G$ platí $e\odot x=x=x\odot e$,
%\item ke každému prvku můžeme nalézt prvek k němu inverzní, tedy pro každé $x\in G$ existuje $y\in G$ tak, že $x\odot y=e=y\odot x$. 
%\end{enumerate}
%\ Pokud je operace navíc komutativní, hovoříme o abelovské nebo komutativní grupě.
%\end{definice}

%\begin{definice} Nechť R je množina, $+$, $\cdot$ binární operace na ní definované. Pak $(R,+,\cdot)$ je okruh, pokud:
%\begin{enumerate}
%\item $(R,+)$ je komutativní grupa,
%\item operace $\cdot$ je asociativní a existuje vzhledem k ní neutrální prvek,
%\item platí oboustranná distributivita, tedy pro libovolné $a,b,c\in R$ platí $a\cdot(b+c)=a\cdot b+a\cdot c,$ $(b+c)\cdot a=b\cdot a+ c\cdot a.$
%\end{enumerate}
%\ Pokud je i operace $\cdot$ komutativní, hovoříme o komutativním okruhu.
%\end{definice}

%\ Operaci + běžně nazýváme sčítání a neutrální prvek vůči ní značíme symbolem 0, operaci $\cdot$ nazýváme násobení a neutrální prvek vůči ní značíme jako $1$.



%\begin{definice} Komutativní okruh $R$ nazýváme obor integrity, pokud pro libovolná $a,b\in R$ platí, že pokud $a\cdot b=0$, tak $a=0$ nebo $b=0$. \end{definice}


%\begin{definice} Nechť T je množina, $+$, $\cdot$ binární operace na ní definované. Pak $(T,+,\cdot)$ je těleso, pokud:
%\begin{enumerate}
%\item $(T,+)$ je komutativní grupa,
%\item $(T\smallsetminus\{0\},\cdot)$ je komutativní grupa,
%\item platí oboustranná distributivita, tedy pro libovolné $a,b,c\in T$ platí $a\cdot(b+c)=a\cdot b+a\cdot c,$ $(b+c)\cdot a=b\cdot a+ c\cdot a.$
%\end{enumerate}
%\end{definice}

%\ Jinak řečeno, těleso je takový obor integrity, jehož každý nenulový prvek je jednotkou, neboli je \textit{invertibilní} -- tedy má vůči operaci $\cdot$ inverzní prvek.






%V případě $p=2$ se nám situace ztíží tím, že pokud $m\equiv1\pmod4$, nemůžeme aplikovat větu \ref{polynomy}, jelikož 2 dělí $|\z[\frac{1+\sqrt m}2]/\z[\sqrt m]|$. Přesto ale dokážeme následující větu:

%\begin{veta} Nechť $K=\q(\sqrt m)$. Potom: $$2\o_K=
%\begin{cases}
%(2,\sqrt m)^2 \quad \text{pokud}\; m\equiv2\pmod4, \\
%(2,1+\sqrt m)^2 \quad \text{pokud} \;m\equiv3\pmod4, \\
%(2,\frac{1-\sqrt m}2)(2,\frac{1+\sqrt m}2) \quad \text{pokud}\; m\equiv1\pmod 8, \\
%2\o_K,\; \text{tj. je prvoideál} \quad \text{pokud}\; m\equiv 5\pmod 8.
%\end{cases}$$
%\end{veta}

%\begin{proof} Zamysleme se nejprve nad diskriminantem okruhu $\o_K$. Z věty \ref{tabulka} víme, že v případě $m\equiv 2,3\pmod4$ platí $d(\o_K)=4m$ a v případě $m\equiv1\pmod4$ platí $d(\o_K)=m$.

%\ Uvažujme nejprve $m\equiv2,3\pmod4$. V tomto případě můžeme aplikovat větu \ref{polynomy}. Jelikož $d(\o_K)=4m$, tak se 2 bude vždy větvit.

%\ Pokud $m\equiv2\pmod4$, tak $2|m$ a tedy $x^2-m\equiv(x)^2\pmod2$. Tudíž $2\o_K=(2,\sqrt m)^2$ (analogicky jsme postupovali v důkazu předchozí věty).

%\ Pokud $m\equiv3\pmod4$, tak jelikož se 2 větví a zároveň nedělí $m$, platí $x^2-m\equiv x^2+1\equiv x^2+2x+1\equiv (x+1)^2\pmod2$ a tedy $2\o_K=(2,1+\sqrt m)^2$.

%\ Nyní uvažujme $m\equiv1\pmod4$, tedy $\o_K=\z[\frac{1+\sqrt m}2]$. Víme již, že nemůžeme použít větu \ref{polynomy}, musíme tedy použít jiné argumenty.

%\ Pokud $m\equiv1\pmod 8$, tak $2\in (2,\frac{1-\sqrt m}2)(2,\frac{1+\sqrt m}2)=(4,1+\sqrt m,1-\sqrt m,\frac{1- m}4)$ protože $\nsd(4,\frac{1-m}4)=2$ (a díky Bezoutově rovnosti s každými dvěma celými čísly ležícími v daném ideálu v něm leží i jejich největší společný dělitel). Tudíž $2\o_K\s(2,\frac{1-\sqrt m}2)(2,\frac{1+\sqrt m}2)$ a proto $(2,\frac{1-\sqrt m}2)(2,\frac{1+\sqrt m}2)|2\o_K$. Aby platila věta \ref{eifi}, musí už platit přímo rovnost.

%\ Zbývá případ $m\equiv 5\pmod 8$. Nechť $\P$ je prvoideál $o\_K$, $\P|2\o_K$. Ukážeme $f(\P|2)=2$. To uděláme sporem: pokud $f(\P|2)=1$, tak $\o_K/\P\cong\z/2\z$. Uvažujme polynom $x^2-x+\frac{1-m}4$. Ten má v $\o_K$ kořen $\frac{1+\sqrt m}4$; má tedy kořen i v $\o_K/\P$. Na druhou stranu tento polynom v $\z/2\z$ žádný kořen nemá, jelikož $x^2-x+\frac{1-m}4\equiv x^2-x+1\pmod2$. To je spor s tím, že jsou tělesa $\o_K/\P$ a $\z/2\z$ izomorfní a $2\o_K$ je tedy opravdu prvoideál.

%\
%\end{proof}

%\ Případ $p=2$ nás zajímá spíše pro úplnost, případ $p$ je liché bude hrát ústřední roli v důkazu kvadratické recprocity a dalších tvrzení. 





\begin{poznamka} Často nastává situace, kdy $K$ je \uv{skoro} podtěleso $L$. Uvažujme například těleso $\R$ s klasickým sčítáním a násobením a těleso $\R^2=\{(a,b)\mid (a,b)\in\R\}$ s operacemi sčítání po složkách (tj. $(a,b)+(c,d)=(a+c,b+d)$) a s násobením definovaným jako $(a,b)\cdot(c,d)=(ac-bd,ad+bc)$ pro všechna $a,b,c,d\in\R$.Sice $\R$ není podtělesem tělesa $\R^2$, ale existuje vnoření (tj. injektivní homomorfismus) $f: \R\rightarrow\R^2$ (např. definované jako $f(a)=(a,0)$), tzn. $\R$ je izomorfní s nějakým podtělesem $f(\R)=\{f(a)\mid a\in\R\}$ tělesa $\R^2$. Sice tedy přísně vzato nemůžeme hovořit o rozšíření $\R\s\R^2$, ale jelikož $\R$ a $f(\R)$ jsou izomorfní, tudíž mají s algebraického hlediska identické vlastnosti, tak někdy nebudeme zcela korektní a např. v této situaci budeme mluvit o rozšíření $\R\s\R^2$ místo o $f(\R)\s\R^2$. \label{nejsmekorektni} \end{poznamka}

\ Zohledníme-li tuto poznámku, můžeme psát $[\R^2:\R]=2$.

\section{Základní poznatky}

\ V této části stručně připomeneme pojmy z algebry, které budeme v práci nejčastěji používat -- především hlavní větu o faktorgrupách, ideál okruhu a vlastnosti okruhu polynomů jedné proměnné.

\ Uveďme tedy nejprve hlavní větu o faktorgrupách:

\begin{veta} Nechť $f:G\rightarrow K$ je homomorfismus grup, $H$ normální podgrupa grupy $G$ splňující $H\s\ker f$. Nechť $\pi:G\rightarrow G/H$ je projekce grupy $G$ na faktorgrupu $G/H$. Pak existuje, a to jediné, zobrazení $\fii:G/H\rightarrow K$ splňující $\fii\odot\pi=f$. Navíc platí: \begin{enumerate}
\item $\fii$ je homomorfismus grup,
\item $\fii$ je injekce, právě když $H=\ker f$,
\item $\fii$ je surjekce, právě když $f$ je surjekce. \end{enumerate} \end{veta}
$$
\xymatrix{
G\ar[rr]^f\ar[dr]^\pi&&K \\
&G/H\ar@{-->}[ur]_\fii&
}
$$

\ Věta má podstatné důsledky:

\begin{dusledek} Nechť $f:G\rightarrow K$ je homomorfismus grup, $f(G)=\{f(g)\mid g\in G\}$. Pak $G/\ker f\cong f(G)$. \end{dusledek}

\begin{dusledek} Homomorfismus grup $f:G\rightarrow K$ je injektivní, právě když $\ker f$ je triviální grupa. \end{dusledek}

Nyní přejděme k pojmu ideál.
 

\begin{poznamka} V dalším textu budeme používat následující značení: pro těleso $K$ symbolem $K(a_1,...,a_n)$ míníme těleso generované množinou $K\cup\{a_1,...,a_n\}$. V případě okruhů používáme obdobné značení -- nejmenší okruh obsahující nějaký okruh $R$ a množinu $\{a_1,...,a_n\}$ značíme $R[a_1,...,a_n]$. Tedy např. $\mathbb{Q}(\sqrt 2)$ nejmenší těleso obsahující racionální čísla a odmocninu ze dvou a $\z[i]$ je nejmenší okruh obsahující celá čísla a imaginární jednotku~$i$. \end{poznamka}


\begin{definice} Nechť R je okruh. Neprázdnou množinu $\I\subseteq R$ nazveme ideálem okruhu R, pokud:
\begin{enumerate}
\item pro libovolné $a,b\in \I$ platí $a+b\in \I$,
\item pro libovolné $r\in R, a\in \I$ platí $ar\in \I, ra\in \I$.
\end{enumerate}
\end{definice}

\ Každý okruh má alespoň dva ideály, a to celý okruh a triviální ideál $\{0\}$ -- říkáme jim nevlastní ideály a ostatní ideály nazýváme vlastní. 

\begin{definice} Ideál $\I$ okruhu $R$ nazýváme hlavní, pokud je ve tvaru $aR=\{ar|r\in R\}$ pro nějaké $a\in R$. Danému oboru integrity říkáme okruh hlavních ideálů, pokud je každý jeho ideál hlavní. \end{definice}

\ Typickým okruhu hlavních ideálů jsou celá čísla: jediné ideály jsou zde tvaru $n\mathbb{Z}$, kde $n$ je libovolné nezáporné celé číslo.

\begin{poznamka} Hlavní ideál $aR$ někdy značíme $(a)$. Obecně je možné definovat ideál generovaný množinou a ideál generovaný konečnou množinou $\{a_1,a_2,...,a_n\}$ značíme $(a_1,a_2,...,a_n)$.\end{poznamka}

\ Všimněme si, že z definice ideálu přímo plyne důležitý poznatek:

\begin{veta} Nechť $\I$ je ideál okruhu R. Pak $(\I,+)$ je normální podgrupa grupy (R,+). \end{veta}

\ Ideály jsou úzce spjaté s homomorfismy okruhů. Platí totiž následující věta:

\begin{veta} Nechť $f:R\rightarrow S$ je homomorfismus okruhů. Pak platí:
\begin{enumerate}
\item je-li $\J$ ideál okruhu $S$, pak $f^{-1}(\J)=\{x\in R|f(x)\in \J\}$ je ideál okruhu $R$,
\item je-li $f$ surjekce a $\I$ ideál okruhu $R$, pak $f(\I)=\{f(x)|x\in \I\}$ je ideál okruhu $S$.
\end{enumerate}
\end{veta}

\ To mimo jiné znamená, že jádro libovolného homomorfismu $f:R\rightarrow S$ je ideál okruhu $R$, jelikož $\ker f=f^{-1}(0)$.

\ Podle ideálů můžeme faktorizovat. Jelikože je pro každý ideál $\I$ je $(\I,+)$ normální podgrupa $(R,+)$, můžeme sestrojit faktorgrupu $(R/\I,+)$, kde + je nyní sčítání tříd pomocí reprezentantů. Lze ukázat, že na této faktorgrupě je možné definovat i násobení pomocí reprezentantů tak, že $R/\I$ je s těmito operacemi okruh. Takto vzniklý okruh nazýváme faktorokruh. Existuje hlavní věta o faktoroktuzích analogická hlavní větě o faktorgrupách.

\ Existují dvě významné skupiny ideálů, které má smysl definovat:

\begin{definice} Nechť R je okruh, $\I$ jeho vlastní ideál. O ideálu $\I$ říkáme, že je:
\begin{enumerate}
\item prvoideál, pokud pro libovolné $a,b\in R$ platí implikace $ab\in \I\Rightarrow a\in \I nebo b\in \I$,
\item maximální ideál, pokud neexistuje žádný ideál $\J$ okruhu $R$ splňující $\I\subsetneq \J\subsetneq R$.
\end{enumerate}
\end{definice}

\ Pokud je okruh $R$ komutativní, můžeme tyto skupiny ideálů poznat podle toho, jak vypadá jimi určený faktorokruh:

\begin{veta} Nechť $R$ je komutativní okruh, $\I$ jeho vlastní ideál. Pak je $\I$ prvoideál, právě když faktorokruh $R/\I$ je obor integrity, a $\I$ je maximální ideál, právě když je $R/\I$ těleso. \label{prvmax} \end{veta}

\ Dále stručně připomeňme vlastnosti polynomů jedné proměnné.

\begin{veta} Nechť $R$ je okruh. Pak množina $R[x]$ polynomů jedné proměnné s koeficienty z $R$ tvoří rovněž okruh (s obvykle definovaným sčítáním a násobením polynomů). Navíc je-li $R$ komutativní (resp. obor integrity), tak i $R[x]$ je komutativní (resp. obor integrity). \end{veta}

\begin{veta} Nechť $R$ je obor integrity. Pak polynom $f\in R[x]$ má v $R$ nejvýše tolik kořenů, kolik je jeho stupeň (který značíme $\deg f$). \end{veta}

\begin{veta} Nechť $R$ je obor integrity. Pak je $R[x]$ euklidovský okruh, přesněji pro každé dva polynomy $f,g\in R[x]$ existují právě jedna dvojice polynomů $k,r\in R[x]$ taková, že $f=kg+r$ a $\deg r<\deg g$. \end{veta} 

\ Na závěr definujme podílové těleso a uveďme některé jeho příklady.

\begin{definice} Nechť $R$ je obor integrity. Nejmenší těleso obsahující $R$ nazveme podílové těleso okruhu R. \end{definice}

\begin{veta} Nechť $R$ je obor integrity. Pak jeho podílové těleso $F$ můžeme psát ve tvaru $$F=\left\{\frac ab\mid a,b\in R, b\ne0\right\},$$ kde $\frac ab=\frac cd$, právě když $ad=bc$ a operace sčítání a násobení provádíme následovně: $$\frac ab+\frac cd=\frac{ad+bc}{bd},$$ $$\frac ab\cdot \frac cd= \frac{ac}{bd}.$$ \end{veta}

\ Podílové těleso okruhu $\z$ je těleso racionálních čísel. Podílové těleso okruhu $R[x]$ nazýváme \textit{těleso racionálních funkcí} a značíme ho $R(x)$. Tyto pojmy můžeme zobecnit: množina polynomů více proměnných nad oborem integrity $R$ tvoří rovněž obor integrity a jeho podílové těleso také nazýváme těleso racionálních funkcí.
